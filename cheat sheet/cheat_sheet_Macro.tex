\documentclass[10pt,landscape,a4paper]{article}
\usepackage[utf8]{inputenc}
\usepackage[ngerman]{babel}
\usepackage{tikz}
\usetikzlibrary{shapes,positioning,arrows,fit,calc,graphs,graphs.standard}
\usepackage[nosf]{kpfonts}
\usepackage{comment}
\usepackage{graphicx}
\usepackage[t1]{sourcesanspro}
%\usepackage[lf]{MyriadPro}
%\usepackage[lf,minionint]{MinionPro}
\usepackage{multicol}
\usepackage{wrapfig}
\usepackage[top=1mm,bottom=2mm,left=1mm,right=2mm]{geometry}
\usepackage[framemethod=tikz]{mdframed}
\usepackage{microtype}
\usepackage{hyperref}

\usepackage{url}
\usepackage{multirow}
\usepackage{esint}
\usepackage{amsfonts}
\usetikzlibrary{decorations.pathmorphing}

\usepackage{colortbl}
\usepackage{xcolor}
\usepackage{mathtools}
\usepackage{amsmath,amssymb}
\usepackage{enumitem}
\makeatletter

\let\bar\overline

\setlist[itemize]{topsep=0pt,leftmargin=10pt,itemsep=-0.2em}
\definecolor{myblue}{cmyk}{1,.72,0,.38}
\definecolor{mypurple}{cmyk}{.57,1,0,.58}
\definecolor{myred}{cmyk}{0,.88,.88,.58}
\definecolor{mygreen}{cmyk}{1,0,.69,.66}
\definecolor{myorange}{cmyk}{0,.58,100,.20}

\def\firstcircle{(0,0) circle (1.5cm)}
\def\secondcircle{(0:2cm) circle (1.5cm)}

\pgfdeclarelayer{background}
\pgfsetlayers{background,main}

\renewcommand{\baselinestretch}{.8}
\pagestyle{empty}

\global\mdfdefinestyle{header}{%
linecolor=gray,linewidth=1pt,%
leftmargin=0mm,rightmargin=0mm,skipbelow=0mm,skipabove=0mm,
}

%\newcommand{\header}{
%\begin{mdframed}[style=header]
%\scriptsize
%\sffamily
%Cheat sheet\\
%by~Your~Name,~page~\thepage~of~2
%\end{mdframed}
%}

\makeatletter
\renewcommand{\section}{\@startsection{section}{1}{0mm}{1ex}{.2ex}{\normalsize\bfseries}}
\renewcommand{\subsection}{\@startsection{subsection}{1}{0mm}{.2ex}{.2ex}{\small\bfseries}}
\renewcommand{\subsubsection}{\@startsection{subsubsection}{1}{0mm}{.2ex}{.2ex}{\bfseries}}

\newcommand*\bigcdot{\mathpalette\bigcdot@{.5}}
\newcommand*\bigcdot@[2]{\mathbin{\vcenter{\hbox{\scalebox{#2}{$\m@th#1\bullet$}}}}}
\makeatother

\def\multi@column@out{%
   \ifnum\outputpenalty <-\@M
   \speci@ls \else
   \ifvoid\colbreak@box\else
     \mult@info\@ne{Re-adding forced
               break(s) for splitting}%
     \setbox\@cclv\vbox{%
        \unvbox\colbreak@box
        \penalty-\@Mv\unvbox\@cclv}%
   \fi
   \splittopskip\topskip
   \splitmaxdepth\maxdepth
   \dimen@\@colroom
   \divide\skip\footins\col@number
   \ifvoid\footins \else
      \leave@mult@footins
   \fi
   \let\ifshr@kingsaved\ifshr@king
   \ifvbox \@kludgeins
     \advance \dimen@ -\ht\@kludgeins
     \ifdim \wd\@kludgeins>\z@
        \shr@nkingtrue
     \fi
   \fi
   \process@cols\mult@gfirstbox{%
%%%%% START CHANGE
\ifnum\count@=\numexpr\mult@rightbox+2\relax
          \setbox\count@\vsplit\@cclv to \dimexpr \dimen@-1cm\relax
\setbox\count@\vbox to \dimen@{\vbox to 1cm{\header}\unvbox\count@\vss}%
\else
      \setbox\count@\vsplit\@cclv to \dimen@
\fi
%%%%% END CHANGE
            \set@keptmarks
            \setbox\count@
                 \vbox to\dimen@
                  {\unvbox\count@
                   \remove@discardable@items
                   \ifshr@nking\vfill\fi}%
           }%
   \setbox\mult@rightbox
       \vsplit\@cclv to\dimen@
   \set@keptmarks
   \setbox\mult@rightbox\vbox to\dimen@
          {\unvbox\mult@rightbox
           \remove@discardable@items
           \ifshr@nking\vfill\fi}%
   \let\ifshr@king\ifshr@kingsaved
   \ifvoid\@cclv \else
       \unvbox\@cclv
       \ifnum\outputpenalty=\@M
       \else
          \penalty\outputpenalty
       \fi
       \ifvoid\footins\else
         \PackageWarning{multicol}%
          {I moved some lines to
           the next page.\MessageBreak
           Footnotes on page
           \thepage\space might be wrong}%
       \fi
       \ifnum \c@tracingmulticols>\thr@@
                    \hrule\allowbreak \fi
   \fi
   \ifx\@empty\kept@firstmark
      \let\firstmark\kept@topmark
      \let\botmark\kept@topmark
   \else
      \let\firstmark\kept@firstmark
      \let\botmark\kept@botmark
   \fi
   \let\topmark\kept@topmark
   \mult@info\tw@
        {Use kept top mark:\MessageBreak
          \meaning\kept@topmark
         \MessageBreak
         Use kept first mark:\MessageBreak
          \meaning\kept@firstmark
        \MessageBreak
         Use kept bot mark:\MessageBreak
          \meaning\kept@botmark
        \MessageBreak
         Produce first mark:\MessageBreak
          \meaning\firstmark
        \MessageBreak
        Produce bot mark:\MessageBreak
          \meaning\botmark
         \@gobbletwo}%
   \setbox\@cclv\vbox{\unvbox\partial@page
                      \page@sofar}%
   \@makecol\@outputpage
     \global\let\kept@topmark\botmark
     \global\let\kept@firstmark\@empty
     \global\let\kept@botmark\@empty
     \mult@info\tw@
        {(Re)Init top mark:\MessageBreak
         \meaning\kept@topmark
         \@gobbletwo}%
   \global\@colroom\@colht
   \global \@mparbottom \z@
   \process@deferreds
   \@whilesw\if@fcolmade\fi{\@outputpage
      \global\@colroom\@colht
      \process@deferreds}%
   \mult@info\@ne
     {Colroom:\MessageBreak
      \the\@colht\space
              after float space removed
              = \the\@colroom \@gobble}%
    \set@mult@vsize \global
  \fi}

\hypersetup{
    colorlinks=true,
    linkcolor=myblue,
    filecolor=magenta,      
    urlcolor=myblue,
    pdfpagemode=FullScreen,
    }

\urlstyle{same}

\makeatother
\setlength{\parindent}{0pt}

% material references: Prof. Geert Ridder's lecture notes
% latex coding references: https://github.com/tim-st/latex-cheatsheet, https://www.overleaf.com/latex/templates/hoja-de-ecuaciones-electricidad-y-magnetismo/xwgjqkrjjgcb

\begin{document}
\begin{center}{\large{\textbf{Macroeconomic Model Cheat Sheet}}}\\
Author: Sai Zhang (\href{mailto:saizhang.econ@gmail.com}{email} me or check my \href{https://github.com/SaiChrisZHANG}{Github} page)
\end{center}

\scriptsize
\begin{multicols*}{3}

% set box styles: in this file, I only use red blue and purple boxes
%% blue boxes
\tikzstyle{bluebox} = [draw=myblue, fill=white, thick, rectangle, rounded corners, inner sep=5pt, inner ysep=10pt, text=myblue]
\tikzstyle{bluetitle} =[fill=myblue, text=white, font=\bfseries]
\tikzstyle{ibluebox} = [draw=myblue, fill=myblue, thick, rectangle, rounded corners, inner sep=5pt, inner ysep=10pt, text=white]
\tikzstyle{ibluetitle} =[draw=myblue, fill=white, text=myblue, font=\bfseries]
%% red boxes
\tikzstyle{redbox} = [draw=myred, fill=white, thick, rectangle, rounded corners, inner sep=5pt, inner ysep=10pt, text=myred]
\tikzstyle{redtitle} =[fill=myred, text=white, font=\bfseries]
\tikzstyle{iredbox} = [draw=myred, fill=myred, thick, rectangle, rounded corners, inner sep=5pt, inner ysep=10pt, text=white]
\tikzstyle{iredtitle} =[draw=myred, fill=white, text=myred, font=\bfseries]
%% purple boxes
\tikzstyle{purplebox} = [draw=mypurple, fill=white, thick, rectangle, rounded corners, inner sep=5pt, inner ysep=10pt, text=mypurple]
\tikzstyle{purpletitle} =[fill=mypurple, text=white, font=\bfseries]
\tikzstyle{ipurplebox} = [draw=mypurple, fill=mypurple, thick, rectangle, rounded corners, inner sep=5pt, inner ysep=10pt, text=white]
\tikzstyle{ipurpletitle} =[draw=mypurple, fill=white, text=mypurple, font=\bfseries]
%% orange boxes
\tikzstyle{orangebox} = [draw=myorange, fill=white, thick, rectangle, rounded corners, inner sep=5pt, inner ysep=10pt, text=myorange]
\tikzstyle{orangetitle} =[fill=myorange, text=white, font=\bfseries]
\tikzstyle{iorangebox} = [draw=myorange, fill=myorange, thick, rectangle, rounded corners, inner sep=5pt, inner ysep=10pt, text=white]
\tikzstyle{iorangetitle} =[draw=myorange, fill=white, text=myorange, font=\bfseries]

\vspace{2pt}
\begin{tikzpicture}
\node [orangebox] (box){%
    \begin{minipage}{0.315\textwidth}
    \color{myorange}
    \scriptsize
    \begin{itemize}
        \item[-] \textbf{Primitive assumptions}:
        \begin{itemize}
            \item[-] Who the \textbf{agents} are, what are their \textbf{preferences} and objective functions
            \item[-] What \textbf{technology} agents can access
            \item[-] What \textbf{endowment} agents have
        \end{itemize}
        \item[-] \textbf{Decision problems}: resource allocation problem (among agents, over time, etc.).
        \item[-] \textbf{Information sets}: what do agents know, how will their knowledge change, what is their \textbf{expectation}.
        \item[-] \textbf{Allocation mechanism}: how agents interact and achieve equilibrium. 2 main mechanisms are:
        \begin{itemize}
            \item[-] \textbf{price system} in competitive equilibrium
            \item[-] benevolent \textbf{central planner} maximizes a social welfare function.
        \end{itemize}
    \end{itemize}
    \end{minipage}
};
\node[orangetitle, right=4pt] at (box.north west) {Modelling essential: what we need to decide};
\end{tikzpicture}

\section*{Infinitely Lived Agent Model}
\begin{itemize}
    \item[-] \textbf{discrete} time, indexed by $t$
    \item[-] economy lives \textbf{infinitely}, $t=0,1,2,\cdots$
    \item[-] single commodity exogenously produced, indexed by $t$, pure \textbf{exchange/endowment} economy.
    \item[-] no firms/government, only \textbf{two types of households}.
    \item[-] each type of households is continuum of \textbf{identical} households of that type, they are \textbf{price takers}, can be represented by a \textbf{representative} household
\end{itemize}

\vspace{2pt}
\begin{tikzpicture}
\node [bluebox] (box){%
    \begin{minipage}{0.315\textwidth}
    \color{myblue}
    \scriptsize
    Utility of type $i$ household is
    $$U(c^i)=\sum^{\infty}_{t=0}\beta^t_i u(c_t^i)$$
    where $\left(c^i\right)=\left\{c_t^i\right\}^{\infty}_{t=0}$, $\beta_i\in(0,1)$.
    
    A deterministic endowment stream of the consumption good for type $i$ household is
    $$w^i=\left(w^i_0,w_1^i,\cdots\right)=\left\{ w^i_t \right\}^{\infty}_{t=0}$$
    \end{minipage}
};
\node[bluetitle, right=4pt] at (box.north west) {Agents' preferences and endowments};
\end{tikzpicture}

The utility function $u(c_t^i)$ is assumed to be:
\begin{itemize}
    \item[-] \textbf{continuously differentiable} of the second order
    \item[-] \textbf{monotonically increasing, strictly concave}: $u'(c_t^i)>0,u''(c_t^i)<0$
    \item[-] \textbf{satisfies Inada conditions} (never 0 or infinity consumption): $\lim_{c^i_t\rightarrow \infty}u'(c_t^i)=0, \lim_{c^i_t\rightarrow 0}u'(c_t^i)=\infty$
    \item[-] \textbf{time additivity}: $u(c_t^i)$ is independent of $c_{t+j}^i$, $c_{t-j}^i$.
    \item[-] \textbf{impatient discounting} $\beta_i<1$: households value today's consumption more than future's.
    \item[-] Constant relative risk aversion (\textbf{CRRA}): $$u(c_t^i)=\frac{c^{1-\sigma}-1}{1-\sigma}\ \  \left(\lim_{\sigma\rightarrow1}\frac{c^{1-\sigma}-1}{1-\sigma}=\lim_{\sigma\rightarrow1}\frac{e^{(1-\sigma)\ln(c)}-1}{1-\sigma}=\ln(c)\right)$$, 
    \scriptsize
    \textbf{The RRA coefficient} $\sigma(c)=$
    \tiny
    $\frac{-u''(c_t^i)c}{u'(c_t^i)}=\frac{-\left(-\sigma c^{-(1+\sigma)}c\right)}{c^{-\sigma}}$
    \scriptsize
    $=\sigma$. Higher RRA means higher risk aversion. 
    \item[-] Constant intertemporal elasticity of substitution \textbf{(IES)}: $$IES=-\frac{\mathrm{d}\ln\left(c_{t+1}/c_t\right)}{\mathrm{d}\ln\left(u'(c_{t+1}/u'(c_t)\right)}=\frac{1}{\sigma}$$
    hence higher RRA (more risk-averse), lower IES (consumption variation over time).
\end{itemize}

\subsection*{Arrow-Debreu Market (AD) approach}
\underline{\textbf{\color{myred}Market structure}}: Households trade just \textbf{once} in $t=0$ market, they trade all future consumption and deliver the promised amount in $t=1,2,\cdots$ market. 

Households have perfect information of the entire endowment sequence, all information is public.

\vspace{2pt}
\begin{tikzpicture}
\node [iredbox] (box){%
    \begin{minipage}{0.315\textwidth}
    \color{white}
    \scriptsize
    \begin{itemize}
        \item[-] allocation: $\left\{\hat{c}^1_t,\hat{c}^2_t\right\}^{\infty}_{t=0}$
        \item[-] regulating mechanism: $\left\{\hat{p}_t\right\}^{\infty}_{t=0}$, with numeraire $\hat{p}_0=1$
    \end{itemize}
    such that:
    \begin{itemize}
        \item[-] given $\left\{\hat{p}_t\right\}^{\infty}_{t=0}$, $\left\{\hat{c}^1_t,\hat{c}^2_t\right\}^{\infty}_{t=0}$ solves:
        $$\max \sum^{\infty}_{t=0}\beta^t u(c_t^i)\ \text{ s.t. }\sum^{\infty}_{t=0}\hat{p}_t c^i_t\leq \sum^{\infty}_{t=0}\hat{p}_t w^i_t,c^i_t\geq0$$
    \item[-] market clearing (disposal of unused goods is costly):
    $$\hat{c}^1_t+\hat{c}^2_t=w^1_t+w^2_t,\forall t$$
    \end{itemize}

    \end{minipage}
};
\node[iredtitle, right=4pt] at (box.north west) {Equilibrium};
\end{tikzpicture}

\vspace{2pt}
\underline{\textit{\textbf{How to solve}}}:
\begin{itemize}
    \item[-] \textbf{\textit{Step 1}}: solve the Lagrangian
    $$\sum^{\infty}_{t=0}\beta^t\ln(c_t^i)+\lambda^i\left(\sum^{\infty}_{t=0}p_t w_t^i-\sum^{\infty}_{t=0}p_t c_t^i\right)$$
    FOCs: $\beta^t/c_t^i=\lambda^i p_t\Rightarrow \frac{\beta^t}{c_t^i p_t}=\frac{\beta^{t+1}}{c_{t+1}^i p_{t+1}}\Rightarrow c^i_{t+1}=\beta\frac{p_t}{p_{t+1}}c^i_t$. FOC gives that price changes $p_t/p_{t+1}$ and subjective discounting $\beta$ determines consumption smoothing.
    \item[-] \textbf{\textit{Step 2}}: Use market clearing condition
    $$c_t^1 +c_t^2=w_t^1+w_t^2$$
    and FOC $p_{t+1}c_{t+1}^i=\beta p_t c_t^i$, get $\frac{p_{t+1}}{p_{t}}=\beta\frac{w_t^1+w_t^2}{w_{t+1}^1+w_{t+1}^2}$, combined with the numeraire assumption $p_0=1$, solve the price sequence $\left\{\hat{p}_t\right\}^{\infty}_{t=0}$.
\item[-] \textit{\textbf{Step 3}}: Plug $\left\{\hat{p}_t\right\}^{\infty}_{t=0}$ and $p_{t+1}c_{t+1}^i=\beta p_t c_t^i$ back to budget constraint
$$\sum^{\infty}_{t=0}p_t c_t^i=\sum^{\infty}_{t=0}\beta^t c_0^i=\frac{c_0^i}{1-\beta}= \sum^{\infty}_{t=0}p_t w_t^i$$
to solve allocation $\left\{\hat{c}^1_t,\hat{c}^2_t\right\}^{\infty}_{t=0}$
\end{itemize}

\vspace{2pt}
\begin{tikzpicture}
\node [redbox] (box){%
    \begin{minipage}{0.315\textwidth}
    \color{myred}
    \scriptsize
    An allocation $\left(c^1,c^2\right)=\left\{c_t^1,c_t^2\right\}^{\infty}_{t=0}$ is \textbf{Pareto efficient} if:
    \begin{itemize}
        \item[-] it is feasible: $\sum^{\infty}_{t=0}p_t c^i_t\leq \sum^{\infty}_{t=0}p_t w^i_t$
        \item[-] no other feasible allocation $(\tilde{c}^1,\tilde{c}^2)$ such that $\forall i, U(\tilde{c}^i)\geq U(c^i)$ and $\exists i, U(\tilde{c}^i)\geq U(c^i)$
    \end{itemize}
    
    \vspace{2pt}
    \begin{tikzpicture}
    \node [iredbox] (box){%
        \begin{minipage}{0.95\textwidth}
        \color{white}
        \scriptsize
        
        \textbf{An AD competitive equilibrium allocation $\left(c^1,c^2\right)=\left\{c^1_t,c^2_t\right\}^{\infty}_{t=0}$ is Pareto efficient.}
        \end{minipage}
    };
    \end{tikzpicture}
    \end{minipage}
};
\node[redtitle, right=4pt] at (box.north west) {Pareto efficiency};
\end{tikzpicture}

\vspace{2pt}
\underline{\textit{\textbf{Proof}}}:
Suppose there is an allocation $\left(\bar{c}^1,\bar{c}^2\right)=\left\{\bar{c}^1_t,\bar{c}^2_t\right\}^{\infty}_{t=0}$, Pareto-dominating AD allocation $\left(\hat{c}^1,\hat{c}^2\right)=\left\{\hat{c}^1_t,\hat{c}^2_t\right\}^{\infty}_{t=0}$.

If the Pareto dominating allocation $\left(\bar{c}^1,\bar{c}^2\right)$ exists, its utility $\bar{U}=U(\bar{c}^1,\bar{c}^2)$ must be bigger than the AD allocation utility $\tilde{U}=U(\tilde{c}^1,\tilde{c}^2)$, therefore, the only reason that it is not chosen as the AD allocation is that it is \textbf{infeasible}.

\vspace{2pt}
Formally, suppose $\bar{c}^1>\hat{c}^1$ and $\bar{c}^2\geq \hat{c}^2$,
\begin{itemize}
    \item[-] \textbf{Step 1}: for household 1 ($\hat{c}^1<\bar{c}^1$), if $\sum^{\infty}_{t=0}\hat{p}_t \bar{c}^1_t \leq \sum^{\infty}_{t=0} \hat{p}_t\hat{c}^1_t=\sum^{\infty}_{t=0}\hat{p}_t w_t^1$ (the Pareto-dominating allocation is also feasible), the AD equilibrium $(\hat{c}^1,\hat{c}^2)$ will NOT maximize HH1's utility, hence contradiction.
    \item[-] \textbf{Step 2}: for household 2 ($\hat{c}^2\leq\bar{c}^2$), if $\sum^{\infty}_{t=0}\hat{p}_t \bar{c}^2_t<\sum^{\infty}_{t=0}\hat{p}_t\hat{c}^2_t=\sum^{\infty}_{t=0}\hat{p}_t w_t^2$ (the Pareto-superior allocation cost less for HH2), then $\exists \delta>0$ s.t. $\sum^{\infty}_{t=0}\hat{p}_t\bar{c}^2_t+\delta\leq \sum^{\infty}_{t=0}\hat{p}_t\hat{c}^c_t=\sum^{\infty}_{t=0}\hat{p}_t w_t^2$, then there is always an allocation $\left\{\bar{c}_0^2+\delta,\bar{c}_t^2\right\}$, achieves a strictly higher utility than the AD allocation (which is utility maximizing), hence contradiction.
    \item[-] \textbf{Step 3}: from \textbf{Step 1}, $\sum^\infty_{t=0}\hat{p}_t\bar{c}^1_t>\sum^{\infty}_{t=0}\hat{p}_t\hat{c}^1_t=\sum^{\infty}_{t=0}\hat{p}_t w^1_t$; from \textbf{Step 2}, $\sum^{\infty}_{t=0}\hat{p}_t\bar{c}^2_t\geq \sum^{\infty}_{t=0}\hat{p}_t \hat{c}^2_t=\sum^{\infty}_{t=0}\hat{p}_t w^2_t$, then
    $$\sum_{i=1,2}\sum^{\infty}_{t=0}\hat{p}_t w_t^i<\sum_{i=1,2}\sum^{\infty}_{t=0}\hat{p}_t\bar{c}^i_t$$
\end{itemize}
Therefore, this Pareto allocation is actually infeasible. This proof requires \textbf{the value of the aggregate endowment is finite}, which is quite intuitive.

\subsubsection*{1st welfare theorem: social planner}
By \textbf{1st welfare theorem}, we can solve the competitive equilibrium allocation by solving Pareto efficient allocation. This is the social planner's problem: a \textbf{weighted} utility maximization problem.

\vspace{2pt}
\begin{tikzpicture}
\node [iredbox] (box){%
    \begin{minipage}{0.315\textwidth}
    \color{white}
    \scriptsize
    
    
    \begin{itemize}
        \item[-] allocation: $\left\{\hat{c}^1_t,\hat{c}^2_t\right\}^{\infty}_{t=0}$
        \item[-] utility weight: $\left\{\hat{\alpha}^1,\hat{\alpha}^2\right\}$
    \end{itemize}
    such that:
    \begin{itemize}
        \item[-] given $\left\{\hat{\alpha}^1,\hat{\alpha}^2\right\}$, $\left\{\hat{c}^1_t,\hat{c}^2_t\right\}^{\infty}_{t=0}$ solves:
        $$\max \alpha^1 \sum^{\infty}_{t=0}\beta^t u(c_t^1)+\alpha^2 \sum^{\infty}_{t=0}\beta^t u(c_t^2)$$
        s.t. $c^1_t+c^2_t\leq w^1_t+w^2_t,\alpha^1+\alpha^2=1,\alpha^i,c^i_t\geq0$
    \item[-] market clearing is the budget constraint.
    \end{itemize}
    \end{minipage}
};
\node[iredtitle, right=4pt] at (box.north west) {PE allocation: Social planner's problem};
\end{tikzpicture}

\vspace{2pt}
\underline{\textit{\textbf{How to solve}}}:
        
solve the Lagrangian
$$\alpha^1 \sum^{\infty}_{t=0}\beta^t u(c_t^1)+\alpha^2 \sum^{\infty}_{t=0}\beta^t u(c_t^2)+\sum^{\infty}_{t=0}\mu_t\left(w_t^1+w_t^2-c_t^1-c_t^2\right)$$
FOC gives $\alpha^1\beta^t u'(c_t^1)=\mu_t=\alpha^2\beta^t u'(c_t^2)\Rightarrow \alpha^1 u'(c_t^1)=\alpha^2 u'(c_t^2)$, plug them back to $c_t^1+c_t^2=w_t^1+w_t^2$, solve $(c_t^1,c_t^2)=\left(c_t^1(\alpha_1,\alpha_2),c_t^2(\alpha_1,\alpha_2)\right)$.

\textbf{The Lagrangian multiplier $\mu_t$ is the AD equilibrium prices, normalized by the total endowment each period}:
$\hat{\alpha}^i\beta^t u'(\hat{c}^i_t)=\hat{\mu}_t,\beta^t u'(\hat{c}^i_t)=\hat{\lambda}^i\hat{p}_t\Rightarrow \frac{\hat{\mu}_t}{\hat{\alpha}^i}=\hat{\lambda}^i\hat{p}_t$.

\subsubsection*{2nd welfare theorem: decentralization}
\textbf{All the PE allocations are Pareto efficient}, but only one is AD equilibrium allocation, that allocation needs to satisfy the AD budget constraint, achieved by \underline{\textit{\textbf{transfer}}}. This procedure, Negishi "trick", follows the second welfare theorem: every Pareto-efficient allocation can be decentralized as an equilibrium with transfers.

\vspace{2pt}
\begin{tikzpicture}
\node [iredbox] (box){%
    \begin{minipage}{0.315\textwidth}
    \color{white}
    \scriptsize
    The AD equilibrium with transfer is:
    \begin{itemize}
        \item[-] allocation: $\left\{\hat{c}^1_t,\hat{c}^2_t\right\}^{\infty}_{t=0}$
        \item[-] lifetime transfer: $\left\{\hat{t}^1,\hat{t}^2\right\}$
        \item[-] regulating mechanism: $\left\{\hat{p}_t\right\}^{\infty}_{t=0}$, with numeraire $\hat{p}_0=1$
    \end{itemize}
    such that:
    \begin{itemize}
        \item[-] given $\left\{\hat{p}_t\right\}^{\infty}_{t=0}$, $\left\{\hat{c}^1_t,\hat{c}^2_t\right\}^{\infty}_{t=0}$ solves:
        $$\max \sum^{\infty}_{t=0}\beta^t u(c_t^i)\ \text{ s.t. }\sum^{\infty}_{t=0}\hat{p}_t c^i_t\leq \sum^{\infty}_{t=0}\hat{p}_t w^i_t+\hat{t}^i_t,c^i_t\geq0$$
    \item[-] market clearing: $\hat{c}^1_t+\hat{c}^2_t=w^1_t+w^2_t,\forall t$
    \end{itemize}
    \end{minipage}
};
\node[iredtitle, right=4pt] at (box.north west) {PE allocation: AD equilibrium with transfer};
\end{tikzpicture}


\underline{\textbf{\textit{How to solve}}}: just solve the zero lifetime transfer condition:
$$t^i(\alpha)\equiv \sum^{\infty}_{t=0}\mu_t\left(c^i_t(\alpha)-w_t^i\right) =\sum^{\infty}_{t=0}\alpha^i\beta^t u'(c_t^i)\left(c^i_t(\mathbf{\alpha})-w_t^i\right)=0$$
In general, transfer function $t^i(\mathbf{\alpha})$ satisfies \textbf{zero sum}: $t^1(\mathbf{\alpha})+t^2(\mathbf{\alpha})=0$; and \textbf{homogeneous of degree 1} $t^i(k\mathbf{\alpha})=kt^i(\mathbf{\alpha})$.


\subsection*{Sequential Market (SM) approach}

\underline{\textbf{\color{myred}Market structure}}: Households trade in spot markets for immediate delivery of consumption goods at every $t$, \textbf{bond} is traded, (purchasing at $t$ denoted by $a^i_{t+1}$), they are traded at $t$, representing one unit of consumption at $t+1$. The interest rate of bonds $r_{t+1}$ regulates the market: bond of 1 unit of consumption at $t$ will be compensated by $(1+r_{t+1})$ units of consumption at $t+1$.
    
Households have perfect information of the entire endowment sequence, all information is public.


\vspace{2pt}
\begin{tikzpicture}
\node [iredbox] (box){%
    \begin{minipage}{0.315\textwidth}
    \color{white}
    \scriptsize
    \begin{itemize}
        \item[-] allocation: $\left\{\left\{\tilde{c}^1_t,\tilde{c}^2_t\right\},\left\{\tilde{a}^1_{t+1},\tilde{a}^2_{t+1}\right\}\right\}^{\infty}_{t=0}$
        \item[-] regulating mechanism: $\left\{\tilde{r}_{t+1}\right\}^{\infty}_{t=0}$
    \end{itemize}
    such that:
    \begin{itemize}
        \item[-] given $\left\{\tilde{r}_{t+1}\right\}^{\infty}_{t=0}$, $\left\{\left\{\tilde{c}^1_t,\tilde{c}^2_t\right\},\left\{\tilde{a}^1_{t+1},\tilde{a}^2_{t+1}\right\}\right\}^{\infty}_{t=0}$ solves:
        $$\max \sum^{\infty}_{t=0}\beta^t u(c_t^i)$$
        s.t. $c^i_t+\frac{a^i_{t+1}}{1+\tilde{r}_{t+1}}\leq w^i_t+a^i_t,\ c^i_t\geq0,\ a^i_{t+1}\geq-\bar{A}^i>-\infty$
    \item[-] market clearing (disposal of unused goods is costly):
    $$\tilde{c}^1_t+\tilde{c}^2_t=w^1_t+w^2_t,\  \tilde{a}^1_{t+1}+\tilde{a}^2_{t+1}=0,\ \forall t$$
    \end{itemize}
    \end{minipage}
};
\node[iredtitle, right=4pt] at (box.north west) {Equilibrium};
\end{tikzpicture}

\underline{\textit{\textbf{How to solve}}}:
\begin{itemize}
    \item[-] \textbf{\textit{Step 1}}: Take advantage the fact that the SM equilibrium allocation $\left\{\tilde{c}^1_t,\tilde{c}^2_t\right\}^{\infty}_{t=0}$ and the AD equilibrium $\left\{\hat{c}^1_t,\hat{c}^2_t\right\}^{\infty}_{t=0}$ are equivalent.
    \item[-] \textbf{\textit{Step 2}}: Solve the asset holdings $\left\{\tilde{a}^1_t,\tilde{a}^2_t\right\}^{\infty}_{t=0}$ with: $$\tilde{a}^i_{t+1}=\sum^{\infty}_{\tau=1}\frac{\hat{p}_{t+\tau}\left(\hat{c}^i_{t+\tau}-w^i_{t+\tau}\right)}{\hat{p}_{t+1}}$$
    i.e., the asset holding at $t$ is the sum of all future excess demands, discounted by the $t+1$ price.
\end{itemize}

The existence of a SM equilibrium requires \textbf{\color{myred}No-Ponzi scheme}: $\color{myred}\bar{A}^i<\infty$. 

\underline{\textit{\textbf{Proof by contradiction}}}: Suppose there is no debt limit.
        
Without debt limit, agent $i$ can consume more at $t=0$ and keep borrowing to keep the consumption level in the future, formally:
\begin{align*}
    c_0^i=\tilde{c}_0^i+\frac{\epsilon}{1+\tilde{r}_1},&\ c_t^i=\tilde{c}_t^i\\
    a_1^i=\tilde{a}_1^i+\epsilon,&\ a_{t+1}^i=\tilde{a}_{t+1}^i-\prod^t_{\tau=1}\left(1+\tilde{r}_{\tau+1}\right)\epsilon
\end{align*}
This allocation satisfies the budget constraint, and can achieve a strictly higher utility, hence contradicting utility maximization. At the same time, $\prod^t_{\tau=1}\left(1+\tilde{r}_{\tau+1}\epsilon\right)\xrightarrow{t\rightarrow\infty}\infty$, contradicting to the limited resource nature of the economy.

\subsection*{Link AD and SM equilibrium}
The link between AD equilibrium and SM equilibrium is built on 2 propositions:

\vspace{2pt}
\begin{tikzpicture}
\node [ibluebox] (box){%
    \begin{minipage}{0.315\textwidth}
    \color{white}
    \scriptsize
    \begin{itemize}
        \item[1] For an AD equilibrium allocation $\left\{\hat{c}^1_t,\hat{c}^2_t\right\}^{\infty}_{t=0}$ and prices $\left\{\hat{p}_t\right\}^{\infty}_{t=0}$ with $\frac{\hat{p_{t+1}}}{\hat{p}_{t}}=\xi<1,\forall t$, then there exists debt limits $\left(\bar{A}^1,\bar{A}^2\right)$ and a corresponding SM equilibrium, with allocation $\left\{\tilde{c}^1_t,\tilde{c}^2_t\right\}^{\infty}_{t=0}$ and interest rates $\left\{\tilde{r}_{t+1}\right\}^{\infty}_{t=0}$, such that $$\tilde{c}^i_t=\hat{c}^i_t,\forall i,t$$
        \item[2] For a SM equilibrium allocation $\left\{\left\{\tilde{c}^1_t,\tilde{c}^2_t\right\},\left\{\tilde{a}^1_{t+1},\tilde{a}^2_{t+1}\right\}\right\}^{\infty}_{t=0}$ and interest rates $\left\{\tilde{r}_{t+1}\right\}^{\infty}_{t=0}$ where $\tilde{a}^i_t\geq -\bar{A}^i$, $\tilde{r}_{t+1}>0$, there exists a corresponding AD equilibrium with allocation $\left\{\hat{c}^1_t,\hat{c}^2_t\right\}^{\infty}_{t=0}$ and prices $\left\{\hat{p}_t\right\}^{\infty}_{t=0}$ such that $$\hat{c}^i_t=\tilde{c}^i_t,\forall i,t$$
    \end{itemize}
    
    \end{minipage}
};
\node[ibluetitle, right=4pt] at (box.north west) {Propositions of AD$\equiv$SM};
\end{tikzpicture}

\vspace{2pt}
\underline{\color{myblue}\textbf{Proof of Position 1: AD$\Rightarrow$SM}}:
\begin{itemize}
    \item[-] \textbf{Step 1: Construct interest rate}
    
    Define the SM interest rate as $\frac{1}{1+\tilde{r}_{t+1}}=\frac{\hat{p}_{t+1}}{\hat{p}_t}$, then the SM budget will be $\hat{p}_t\tilde{c}_t^i+\hat{p}_{t+1}\tilde{a}^i_{t+1}=\hat{p}_tw_t^i+\hat{p}_t\tilde{a}^i_t$, iterate this, get $\sum^{\infty}_{t=0}\hat{p}_t\tilde{c}^i_t+\lim_{T\rightarrow\infty}\hat{p}_{T+1}\tilde{a}^i_{T+1}=\sum^{\infty}_{t=0}\hat{p}_t w^i_t$, since $\lim_{T\rightarrow\infty}\hat{p}_{T+1}\tilde{a}^i_{T+1}=\lim_{T\rightarrow\infty}\prod^{T+1}_{\tau=1}\frac{\tilde{a}^i_{T+1}}{1+\tilde{r}_{\tau}^i}\geq 0$, \textbf{SM budget satisfaction leads to AD budget satisfaction}.
    
    \item[-] \textbf{Step 2: Derive $\tilde{a}^i_{t+1}$}
    
    By plug in $\frac{1}{1+\tilde{r}_{t+1}}=\frac{\hat{p}_{t+1}}{\hat{p}_t}$ to SM budget constraint, get $\tilde{a}_t^i=c_t^i-w_t^i+\tilde{a}_{t+1}^i\frac{\hat{p}_{t+1}}{\hat{p}_t}$, do an iteration of this equation, derive asset holding as $$\tilde{a}^i_{t+1}=\sum^{\infty}_{\tau=1}\frac{\hat{p}_{t+\tau}\left(\hat{c}^i_{t+\tau}-w^i_{t+\tau}\right)}{\hat{p}_{t+1}}$$
    plug it in the SM equilibrium budget constraint $\hat{c}^i_t+\frac{\tilde{a}_{t+1}^i}{1+\tilde{r}_{t+1}}=w_t+\tilde{a}_t^i$, the constraint is satisfied.
    
    \item[-] \textbf{Step 3: Find debt limit $\bar{A}^i$}
    
    It is easy to show that the asset holding $\tilde{a}^i_{t+1}=\sum^{\infty}_{\tau=1}\frac{\hat{p}_{t+\tau}\left(\hat{c}^i_{t+\tau}-w^i_{t+\tau}\right)}{\hat{p}_{t+1}}\geq -\sum^{\infty}_{\tau=1}\frac{\hat{p}_{t+\tau}w^i_{t+\tau}}{\hat{p}_{t+1}}\geq -\sum^{\infty}_{\tau=1}\xi^{\tau-1}{\hat{p}_{t+1}}>-\infty$, therefore, households will \textbf{never} choose $\tilde{a}^i_{t+1}$ to exceed the debt limit $-\bar{A}^i=-\sup \sum^{\infty}_{\tau=1}\frac{\hat{p}_{t+\tau}}{\hat{p}_{t+1}}w^i_{t+\tau}$. This debt limit can also be derived by setting consumption $c^i_t$ as 0 in SM budget constraint and a backward iteration.
    
    \item[-] \textbf{Step 4: Check utility maximization}
    
    This constructed SM allocation must satisfy AD budget as well, and it is an AD equilibrium allocation, hence it is lifetime utility maximizing allocation.

\end{itemize}
   
\vspace{2pt}
\underline{\color{myblue}\textbf{Proof of Position 2: AD$\Leftarrow$SM}}:
\begin{itemize}
    \item[-] \textbf{Step 1: Construct price series}
    
    From the construction before, $$\hat{p}_{t+1}=\frac{\hat{p}_t}{1+\tilde{r}_{t+1}}$$
    SM equilibrium consumption satisfies AD budget constraint, market clearing and non-negative consumption condition.
    
    \item[-] \textbf{Step 2: SM equilibrium allocation is utility maximizing within AD budget}
    
    Prove by contradiction: suppose there exist an alternative satisfying AD budget and yields a higher utility. Optimization within SM budget has one more constraint than optimization within AD budget: \textbf{No Ponzi scheme}.  
    
    Since no Ponzi scheme constraints ever bind in the SM equilibrium, SM equilibrium allocation will be optimizer within AD budget set as well.

\end{itemize}

\section*{Overlapping Generation Model}
\vspace{2pt}
\begin{itemize}
\item[-] \textbf{discrete} time, indexed by $t$
\item[-] For each $t$, a new generation $t$ of identical individuals is born and live for two periods $t,t+1$.
\item[-] There is an initial old generation (generation 0), they are born before $t=1$, they can be endowed with $m$ units of \textbf{fiat money}.
\item[-] single commodity exogenously produced, indexed by $t$, pure \textbf{exchange/endowment} economy.
\item[-] There is exogenously given \textbf{net} asset/liability of the entire private sector
\end{itemize}

\vspace{2pt}
\begin{tikzpicture}
\node [bluebox] (box){%
    \begin{minipage}{0.315\textwidth}
    \color{myblue}
    \scriptsize
    utility of a generation $t$ agent is
    $$U_t(c^t)=u(c_t^t)+\beta u(c_{t+1}^t)$$
    utility of the initial old is $$U_0(c^0)=u(c_1^0)$$
    
    Again, $u(\cdot)$ is assumed to be strictly increasing, strictly concave, twice continuously differentiable, typically satisfying Inada condition and CRRA as well.
    \end{minipage}
};
\node[bluetitle, right=4pt] at (box.north west) {Agents' preferences};
\end{tikzpicture}

\vspace{2pt}
Again, we have two market structures: \textbf{Arrow-Debreu} Market and \textbf{Sequential} Market.

\vspace{2pt}
\begin{tikzpicture}
\node [iredbox] (box){%
    \begin{minipage}{0.315\textwidth}
    \color{white}
    \scriptsize
    \begin{itemize}
        \item[-] allocation: $\left\{\hat{c}^0_1,\left\{\hat{c}^t_t,\hat{c}^t_{t+1}\right\}^{\infty}_{t=1}\right\}$
        \item[-] regulating mechanism: $\left\{\hat{p}_t\right\}^{\infty}_{t=1}$, with $m$ or $p_1$ (when $m=0$) as the numeraire.
    \end{itemize}
    such that:
    \begin{itemize}
        \item[-] given $\left\{\hat{p}_t\right\}^{\infty}_{t=1}$, $\left\{\hat{c}^0_1,\left\{\hat{c}^t_t,\hat{c}^t_{t+1}\right\}^{\infty}_{t=1}\right\}$ solves:
        \begin{align*}
            \max_{c^t\geq 0} u(c_t^t)+\beta u(c_{t+1}^t)\ &\text{ s.t. } \hat{p}_t c^t_t+\hat{p}_{t+1}c^t_{t+1}\leq \hat{p}_t w^t_t+\hat{p}_{t+1}w^t_{t+1}\\
            \max_{c_1^0\geq 0} u(c_1^0)\ &\text{ s.t. } \hat{p}_1 c_1^0\leq \hat{p}_1 w_1^0+m
        \end{align*}
        
    \item[-] market clearing (disposal of unused goods is costly):
    $$\hat{c}^{t-1}_t+\hat{c}^t_t=w^{t-1}_t+w^t_t,\forall t\geq 1$$
    \end{itemize}

    \end{minipage}
};
\node[iredtitle, right=4pt] at (box.north west) {AD Equilibrium};
\end{tikzpicture}

\scriptsize
\underline{\textit{\textbf{How to solve}}}:
\begin{itemize}
    \item[-] \textbf{\textit{Step 1}}: solve the Lagrangian
    $$ u_t(c^t_t)+\beta u_{t+1}(c^t_{t+1})+\lambda^t\left( p_t w_t^t+p_{t+1}w_{t+1}^t-p_t c_t^t-p_{t+1}c^t_{t+1} \right)$$
    FOCs: $\frac{u'_t(c^t_t)}{p_t}=\lambda^t =\frac{\beta u'_{t+1}(c^t_{t+1})}{p_{t+1}}\Rightarrow \frac{p_{t+1}}{p_t}=\frac{\beta u'_{t+1}(c^t_{t+1})}{u'_t(c_t^t)}\Rightarrow c_t^t=f(c^t_{t+1},w^t_t,w^t_{t+1})$. FOC gives that price changes $p_t/p_{t+1}$ and subjective discounting $\beta$ determines consumption smoothing.
    
    \item[-] \textit{\textbf{Step 2}}: Solve the initial old's Lagrangian:
    $$u(c^0_1)+\lambda^0\left(w_1^0+m-c_1^0\right)$$ get the initial old's consumption $\hat{c}^0_1=w_1^0+m$.
    
    \item[-] \textbf{\textit{Step 3}}: Plug $\hat{c}_1^0$ into the market clearing condition $c_t^{t-1} +c_t^t=w_t^{t-1}+w_t^t$, get
    $$\hat{c}_1^1=w_1^0+w_1^1-\hat{c}_1^0=w_1^1-m$$
    
    
    \item[-] \textit{\textbf{Step 4}}: Plug $\hat{c}_1^1=w_1^1-m$ and the FOC back to the budget constraint
    $$\hat{p}_1 c^1_1+\hat{p}_2 c^1_2 = \hat{p}_1 w^1_1+\hat{p}_2 w^1_2$$
    get $\hat{c}^1_2$. Iterate this process forward, solve the allocation $\left\{\hat{c}^0_1,\left\{\hat{c}^t_t,\hat{c}^t_{t+1}\right\}^{\infty}_{t=1}\right\}$ and price stream $\left\{\hat{p}_t\right\}^{\infty}_{t=1}$.
\end{itemize}

\vspace{2pt}
\begin{tikzpicture}
\node [iredbox] (box){%
    \begin{minipage}{0.315\textwidth}
    \color{white}
    \scriptsize
    \begin{itemize}
        \item[-] allocation: $\left\{\tilde{c}^0_1,\left\{\tilde{c}^t_t,\tilde{c}^t_{t+1},\tilde{s}_t^t\right\}^{\infty}_{t=1}\right\}$
        \item[-] regulating mechanism: interest rates $\left\{\tilde{r}_{t}\right\}^{\infty}_{t=1}$
    \end{itemize}
    such that:
    \begin{itemize}
        \item[-] given $\left\{\tilde{r}_{t+1}\right\}^{\infty}_{t=1},\forall t>1$, $\left\{\tilde{c}^t_t,\tilde{c}^t_{t+1}\right\}^{\infty}_{t=1}$ solves:
        \begin{align*}
            \max_{c^t\geq 0} u(c_t^t)+\beta u(c_{t+1}^t)\ \text{ s.t. } & c^t_t+s^t_t\leq w_t^t,\\
            & c^t_{t+1}\leq w^t_{t+1}+(1+\tilde{r}_{t+1})s^t_t\\
            \max_{c_1^0\geq 0} u(c_1^0)\ \text{ s.t. } & c_1^0\leq w_1^0+m(1+\tilde{r}_1)
        \end{align*}
    \item[-] good market clearing (disposal of unused goods is costly):
    $$\tilde{c}^{t-1}_t+\tilde{c}^t_t=w^{t-1}_t+w^t_t,\forall t\geq 1$$
    \item[-] asset market clearing: the budget constraint gives $$\tilde{c}^{t+1}_{t+1}+\tilde{c}^t_{t+1}+s^{t+1}_{t+1}=w^{t+1}_{t+1}+w^t_{t+1}+(1+\tilde{r}_{t+1})s^t_t$$
    plug the good market clearing condition, get
    $$s^{t+1}_{t+1}=(1+\tilde{r}_{t+1})s^t_t $$
    iterate this backwards to $s^0_0=m$, get
    $$s^t_t=\prod^t_{\tau=1}(1+\tilde{r}_{\tau})m$$
    \end{itemize}
    \end{minipage}
};
\node[iredtitle, right=4pt] at (box.north west) {SM Equilibrium};
\end{tikzpicture}

\underline{\textit{\textbf{How to solve}}}:
\begin{itemize}
    \item[-] \textbf{\textit{Step 1}}: solve the Lagrangian
    \begin{align*}
        u_t(c^t_t)+\beta u_{t+1}(c^t_{t+1}) &+ \mu^t_t\left(w^t_t-c^t_t-s^t_t \right)\\
        &+ \mu^t_{t+1}\left(w^t_t+(1+\tilde{r}_{t+1})s^t_t-c^t_{t+1} \right)
    \end{align*}
    FOCs: $u'_t(c^t_t)=\mu^t =\beta u'_{t+1}(c^t_{t+1})(1+\tilde{r}_{t+1})\Rightarrow \tilde{r}_{t+1}=\frac{u'_{t}(c^t_t)}{\beta u'_{t+1}(c_{t+1}^t)}-1\Rightarrow c_t^t=f(c^t_{t+1},w^t_t,w^t_{t+1})$.
    
    \item[-] \textit{\textbf{Step 2}}: Solve the initial old's Lagrangian:
    $$u(c^0_1)+\lambda^0\left(w_1^0+m(1+r_1)-c_1^0\right)$$ get the initial old's consumption $\hat{c}^0_1=w_1^0+m(1+\tilde{r}_{t+1})$.
    
    \item[-] \textbf{\textit{Step 3}}: Plug $\hat{c}_1^0$ into the market clearing condition $c_t^{t-1} +c_t^t=w_t^{t-1}+w_t^t$, get
    $$\hat{c}_1^1=w_1^0+w_1^1-\hat{c}_1^0=w_1^1-m(1+\tilde{r}_{t+1})$$
    
    \item[-] \textit{\textbf{Step 4}}: Plug $\hat{c}_1^1=w_1^1-m(1+\tilde{r}_{t+1})$ and the FOC back to the budget constraint
    $$w_t^t+\frac{1}{1+\tilde{r}_{t+1}}w_{t+1}^t= c_t^t+\frac{1}{1+\tilde{r}_{t+1}}c^t_{t+1}$$
    get $\hat{c}^1_2$. Iterate this process forward, solve the allocation $\left\{\hat{c}^0_1,\left\{\hat{c}^t_t,\hat{c}^t_{t+1}\right\}^{\infty}_{t=1}\right\}$ and price stream $\left\{\tilde{r}_{t+1}\right\}^{\infty}_{t=1}$.
\end{itemize}

\vspace{2pt}
And again, the two market sturctures will lead to the same allocation:

\vspace{2pt}
\begin{tikzpicture}
\node [ibluebox] (box){%
    \begin{minipage}{0.315\textwidth}
    \color{white}
    \scriptsize
    \begin{itemize}
        \item[1] For an AD equilibrium allocation $\left\{\hat{c}^0_1,\left\{\hat{c}^t_t,\hat{c}^t_{t+1}\right\}^{\infty}_{t=1}\right\}$ and prices $\left\{\hat{p}_t\right\}^{\infty}_{t=1}$ with $\hat{p}_t>0$, then there exists a corresponding SM equilibrium, with allocation $\left\{\tilde{c}^0_1,\left\{\tilde{c}^t_t,\tilde{c}^t_{t+1}\right\}^{\infty}_{t=1}\right\}$ and interest rates $\left\{\tilde{r}_{t+1}\right\}^{\infty}_{t=0}$, such that $$\tilde{c}^{t-1}_t=\hat{c}^{t-1}_t,\tilde{c}^t_t=\hat{c}^t_t\forall t\geq 1$$
        \item[2] For a SM equilibrium allocation $\left\{\tilde{c}^0_1,\left\{\tilde{c}^t_t,\tilde{c}^t_{t+1}\right\}^{\infty}_{t=1}\right\}$ and interest rates $\left\{\tilde{r}_{t+1}\right\}^{\infty}_{t=0}$ where $\tilde{r}_{t+1}>-1$, there exists a corresponding AD equilibrium with allocation $\left\{\hat{c}^0_1,\left\{\hat{c}^t_t,\hat{c}^t_{t+1}\right\}^{\infty}_{t=1}\right\}$ and prices $\left\{\hat{p}_t\right\}^{\infty}_{t=1}$ such that $$\hat{c}^{t-1}_t=\tilde{c}^{t-1}_t,\hat{c}^t_t=\tilde{c}^t_t\forall t\geq 1$$
    \end{itemize}
    
    The interest rate and price stream are still inter-determined:
    $$\frac{1}{1+\tilde{r}_{t+1}}=\frac{\hat{p}_{t+1}}{\hat{p}_t},\ \frac{1}{1+\tilde{r}_1}=\hat{p}_1$$
    
    The two Euler equations are:
    \begin{align*}
        u'_t(c_t^t)=\beta u'_{t+1}(c_{t+1}^t)(\hat{p}_t/\hat{p}_{t+1}) &\ \cdots \text{ AD}\\
        u'_t(c_t^t)=\beta u'_{t+1}(c_{t+1}^t)(1+\tilde{r}_{t+1}) & \ \cdots\text{ SM}
    \end{align*}
    
    \end{minipage}
};
\node[ibluetitle, right=4pt] at (box.north west) {Propositions of AD$\equiv$SM};
\end{tikzpicture}

\vspace{2pt}
\underline{\textit{\textbf{An easy proof}}}:
\begin{itemize}
    \item[-] \textbf{\textit{Prop. 1}}: AD equilibrium allocation  satisfies SM FOC and the SM budget constraints: 
    $$\tilde{\mu}_t^t=\hat{\lambda}\hat{p}_t,\tilde{\mu}_{t+1}^t=\hat{\lambda}\hat{p}_{t+1},\forall t\geq 0;\  \tilde{s}^t_t=w^t_t-\hat{c}^t_t,\forall t\geq 1$$
    \item[-] \textbf{\textit{Prop. 2}}: SM equilibrium allocation satisfies AD FOC and the AD budget constraints:  $$\hat{\lambda}^t=\frac{\tilde{\mu}^t_t}{\prod^{t-1}_{\tau=0}(1+\tilde{r}_{t-\tau})}=\frac{\tilde{\mu}^t_{t+1}}{\prod^{t-1}_{\tau=0}(1+\tilde{r}_{t+1-\tau})},\forall t\geq 1$$
    
    $$\hat{\lambda}^0=\tilde{\mu}^0_1(1+\tilde{r}_1)$$
\end{itemize}

\subsection*{Offer curve}
\scriptsize
Assume the economy is stationary, the endowments at each period of life by successive generations are constant:

$$w^t_t=w_1,w^t_{t+1}=w_2,\forall t\geq 1;\ w^0_1=w_2$$

By the Lagrangian of AD equilibrium:
\begin{align*}
    u'(c^t_t) &=\lambda^t p_t\\
    \beta u'(c^t_{t+1}) &= \lambda^t p_{t+1}
\end{align*}

solve $c^t_t,c^t_{t+1}$ as functions of $p_t,p_{t+1}$, get $c^t_{t}(p_t,p_{t+1}),c^t_{t+1}(p_t,p_{t+1})$, then define $y$ as the excess demand at $t$, define $z$ as the excess demand at $t+1$:
\begin{align*}
    y(p_t,p_{t+1})=&c^t_t(p_t,p_{t+1})-w_1\\
    z(p_t,p_{t+1})=&c^t_{t+1}(p_t,p_{t+1})-w_2
\end{align*}

The excess demand at $t=0$ is
$$z(m,p_1)=c^0_1-w_2=\frac{m}{p_1}$$

\vspace{2pt}
\begin{tikzpicture}
\node [bluebox] (box){%
    \begin{minipage}{0.315\textwidth}
    \color{myblue}
    \scriptsize
    By the AD budget constraint: $p_t c_t^t+p_{t+1} c^t_{t+1}=p_t w_t+p_{t+1} w^t_{t+1}$, this gives
    $$ p_t y(p_t,p_{t+1}) + p_{t+1} z(p_t,p_{t+1})=0$$
    leading to
    $$\frac{z(p_t,p_{t+1})}{y(p_t,p_{t+1})}=-\frac{p_t}{p_{t+1}},\forall t\geq 1$$
    Notice that $y(p_t,p_{t+1})$ and $z(p_t,p_{t+1})$ are both functions of $\frac{p_t}{p_{t+1}}$, this way, we can replace $\frac{p_t}{p_{t+1}}$ as a function of $y$, plug this $\frac{p_t}{p_{t+1}}=g(y)$ into $z(p_t,p_{t+1})$, we will have the offer curve $$z=f(y)$$

    \vspace{2pt}
    \begin{tikzpicture}
    \node [ibluebox] (box){%
        \begin{minipage}{0.95\textwidth}
        \color{white}
        \scriptsize
        \underline{\textit{\textbf{Properties of offer curve}}}:
        \begin{itemize}
            \item[-] The offer curve is bounded by endowment:
            $$y(p_t,p_{t+1})\geq -w_1,z_(p_t,p_{t+1})\geq -w_2$$
            \item[-] The curve is in the $2^{nd}$ and $4^{th}$ Quadrant:
            $$y(p_t,p_{t+1})\cdot z(p_t,p_{t+1})<0$$
            \item[-] The origin $(0,0)$ is on the offer curve: $z^*(p_t,p_{t+1})=y^*(p_t,p_{t+1})=0$.
            \item[-] for $y(p_t,p_{t+1})\neq 0$, AD budget constraint is always satisfied:
            $$\frac{z(p_t,p_{t+1})}{y(p_t,p_{t+1})}=-\frac{p_t}{p_{t+1}},\forall t\geq 1$$
            hence the slope of the straight line connecting each point on the offer curve to the origin determines the price ratio $\frac{p_t}{p_{t+1}}$
        \end{itemize}
        \end{minipage}
    };
    \end{tikzpicture}
    
    \begin{center}
    \begin{tikzpicture}
    \draw[->] (-3.1, 0) -- (1.5, 0) node[above, yshift=-6pt] {$y(p_t,p_{t+1})$};
    \draw[->] (0, -1) -- (0, 3) node[left] {$z(p_t,p_{t+1})$};
    \draw[domain=-2.33:1.4, smooth, thick, variable=\x, black] plot ({\x}, { (2.7/(2.7+\x))^(2/3)-1 })
    node[right] {Offer curve};
    
    \draw[domain=-2.77:0.5, smooth, thick, variable=\x, myblue] plot ({\x}, { -\x})
    node[left] {Market clearing};
    
    \end{tikzpicture}
    \end{center}
    
    \vspace{2pt}
    Besides, we also have a market clearing curve:
    \begin{align*}
        y(p_t,p_{t+1})+z(p_{t-1},p_t)=0,&\ \forall t>1\\
        y(p_1,p_2)+z(m,p_1)=0,&\ t=1
    \end{align*}
    
    this is a 45-degree line through the $2^{nd}$ and $4^{th}$ quadrants.
    \end{minipage}
};
\node[ibluetitle, right=4pt] at (box.north west) {Features of offer curve};
\end{tikzpicture}

\vspace{2pt}
With offer curve and market clearing curve, we can determine the \textbf{entire sequence of excess demands} of young and old at every date as following:
\begin{itemize}
    \item[-] \textit{\textbf{Step 1}}: for a given $m$, at $t=1$, the excess demand of the initial old $z^0=z^0(p_1,m)=m/p_1$
    \item[-] \textit{\textbf{Step 2}}: the initial young's excess demand at $t=1$ is determined by the market clearing curve $y^1(p_1,p_2)=-z^0$
    \item[-] \textit{\textbf{Step 3}}: $y^1$ will determine $z^1(p_1,p_2)$ on the offer curve.
    \item[-] \textit{\textbf{Step 4}}: repeat this \textbf{market clearing curve}-\textbf{offer curve} procedure.
\end{itemize}
In this procedure, the initial price $p_1$ must be picked first. If $m\neq 0$, different $p_1$ can index a continuum of equilibria.

\vspace{2pt}
\begin{tikzpicture}
\node [ibluebox] (box){%
    \begin{minipage}{0.315\textwidth}
    \color{white}
    \scriptsize
    First, we need the slope of the offer curve to be convex:
    $$ \frac{\partial c^t_t}{\partial (p_t/p_{t+1})} =\frac{\partial c^t_t}{\partial (1+r_{t+1})} <0 $$
    \begin{itemize}
        \item[-] graphically, $-\frac{p_t}{p_{t+1}}$ is the slope of the line connecting a point on the offer curve to the origin, $c_t^t-w_t^t=y(p_t,p_{t+1})$ is the x-axis, hence $\frac{\partial c^t_t}{\partial (p_t/p_{t+1})}<0$ means that as $c^t_t$ increases, the slope of the connecting line is increasing (less negative, or shallower).
        \item[-]economically, if interest rates increasing, young period saving will increase, consumption will decrease (substitution effect).
    \end{itemize}

    \vspace{2pt}
    \begin{tikzpicture}
    \node [bluebox] (box){%
        \begin{minipage}{0.95\textwidth}
        \color{myblue}
        \scriptsize
        \textit{\textbf{A very important object: \underline{Autarkic interest rate $\bar{r}$}}}:
        
        If we assume stationary endowment: $w^t_t=w1,w^t_{t+1}=w_2,\forall t\geq 1; w^0_1=w_2$, we will the autarkic interest rate as:
        $$1+\bar{r} = \frac{\hat{p}_t}{\hat{p}_{t+1}}=\frac{u_t'(w_1)}{\beta u_{t+1}'(w_2)}$$
        where 
        \begin{itemize}
            \item[-] $\hat{p}_t,\hat{p}_{t+1}$ are the autarky equilibrium price.
            \item[-]$\bar{r}$ is determined by the relative size of endowment $w_1,w_2$
            \item[-] $\bar{r}$ determines the general shape of the offer curve: $1+\bar{r}$ it is the \textbf{\underline{negative} of the slop of the offer curve at the origin}.
        \end{itemize}
        \end{minipage}
    };
    \end{tikzpicture}
    \end{minipage}
};
\node[ibluetitle, right=4pt] at (box.north west) {Slope of offer curve};
\end{tikzpicture}

\vspace{4pt}
\subsection*{Different cases of OLG model}
In the overlapping generation economy:
\begin{itemize}
    \item[-] $m$ and $p_1$
    
    $m$ determines the equilibrium sequences, the initial price level $p_1$ determines the starting point of an equilibrium sequence hence indexes a continuum of equilibria. There is a special initial price $p^*_1$ such that $m/p^*_1$ is exactly \textbf{the intersection point} of the offer curve and the market clearing curve. This will be the \textbf{stationary monetary equilibrium}.
    
    \item[-] $1+\bar{r}=\hat{p}_t/\hat{p}_{t+1}=u_t'(w_1)/\beta u_{t+1}'(w_2)$
    
    relative size of endowments $w_1,w_2$ determines autarkic interest rate $1+\bar{r}$, which determines the shape of the offer curve, and the Pareto efficiency and the stationary of the equilibria.
\end{itemize}
Now we discuss the different cases:

\vspace{2pt}
\begin{tikzpicture}
\node [orangebox] (box){%
    \begin{minipage}{0.315\textwidth}
    \color{myorange}
    \scriptsize
    \begin{itemize}
        \item[-] shape of offer curve: the offer curve is tangent to the market clearing line \textbf{at the origin}
        \item[-] equilibrium: 
        \begin{itemize}
            \item[-] the only stationary equilibrium is the origin allocation, autarky. It is Pareto efficiency.
            \item[-] there is NO stationary monetary equilibria, i.e., money is not valued; But $\forall m<0$, there is a continuum of monetary equilibria, converging to autarky.
        \end{itemize}
    \end{itemize}
    \end{minipage}
};
\node[orangetitle, right=4pt] at (box.north west) {Knife-edge economy $1+\bar{r} = \hat{p}_t/\hat{p}_{t+1}=u_t'(w_1)/\beta u_{t+1}'(w_2)=1$};
\end{tikzpicture}

\vspace{2pt}
\begin{tikzpicture}
\node [orangebox] (box){%
    \begin{minipage}{0.315\textwidth}
    \color{myorange}
    \scriptsize
    In this economy, the \textbf{old} generation is richer: $w_2>w_1$, since $u'>0,u''<0$, $u'(w_1)>u'(w_2)>0$
    \begin{itemize}
        \item[-] shape of offer curve: the slope of the offer curve is \textbf{steeper} than the 45-degree market clearing line \textbf{at the origin}.
        \item[-] two intersections: the origin and a point in the \textbf{fourth} quadrant $m/p^*_1, m<0$
        \item[-] equilibrium: 
        \begin{itemize}
            \item[-] autarky is a stationary equilibrium, and it is \textbf{Pareto efficient}.
            \item[-] there is a stationary monetary equilibria at the other intersection point $m/p^*_1$. It is \textbf{Pareto efficient}. In this equilibrium, $m<0$, the initial old has a debt, the economy has a net debt.
            \item[-] there is a continuum of dynamic monetary equilibria with $p_1>p^*_1$. For all of them, $m<0$. They all converge to the monetary stationary equilibrium.
        \end{itemize}
        \item[-] AD prices: $\hat{p}_t$ decreases over time.
    \end{itemize}
    
    \end{minipage}
};
\node[orangetitle, right=4pt] at (box.north west) {Classical economy $1+\bar{r} = \hat{p}_t/\hat{p}_{t+1}=u_t'(w_1)/\beta u_{t+1}'(w_2)>1$};
\end{tikzpicture}

\vspace{2pt}
\begin{tikzpicture}
\node [orangebox] (box){%
    \begin{minipage}{0.315\textwidth}
    \color{myorange}
    \scriptsize
    In this economy, the \textbf{young} generation is richer: $w_1>w_2$, since $u'>0,u''<0$, $u'(w_2)>u'(w_1)>0$
    \begin{itemize}
        \item[-] shape of offer curve: the slope of the offer curve is \textbf{shallower} than the 45-degree market clearing line \textbf{at the origin}.
        \item[-] two intersections: the origin and a point in the \textbf{second} quadrant $m/p^*_1, m>0$
        \item[-] equilibrium: 
        \begin{itemize}
            \item[-] autarky is a stationary equilibrium, and it is \textbf{NOT} Pareto efficient.
            \item[-] there is a stationary monetary equilibria at the other intersection point $m/p^*_1$. In this equilibrium, $m>0$, the initial old has positive fiat money.
            \item[-] there are two continuum of dynamic monetary equilibria:
            \begin{itemize}
                \item[-] $m>0$: the continuum is between autarky and stationary monetary equilibrium, with $p_1>p^*_1$, they all converge to autarky.
                \item[-] $m<0$: every possible $(m,p_1)$ is a dynamic monetary equilibrium, they all converge to autarky.
            \end{itemize}
            
        \end{itemize}
        \item[-] AD prices: $\hat{p}_t$ increases over time.
    \end{itemize}
    
    \end{minipage}
};
\node[orangetitle, right=4pt] at (box.north west) {Samuelson economy $1+\bar{r} = \hat{p}_t/\hat{p}_{t+1}=u_t'(w_1)/\beta u_{t+1}'(w_2)<1$};
\end{tikzpicture}

\subsection*{Pareto efficiency of the equilibria}
\begin{itemize}
    \item[-] \underline{\textbf{Autarky equilibrium}} 
    \begin{itemize}
        \item[-] In \textit{knife-edge economy}, \textbf{Autarky is PE}: young and old have the same marginal utility. To compensate old by transferring from young, the utility loss of transferring in young cannot be compensated by the compensation in old since utility is concave.
        \item[-] In \textit{classical economy}, \textbf{Autarky is PE}: young are endowed less, have a higher marginal utility. The utility loss of young from transferring to old can not be compensated by receiving that amount when old, again, it is due to the concavity of utility.
        \item[-] In \textit{Samuelson economy}, \textbf{Autarky is NOT PE}: young are endowed more, have a lower marginal utility. The utility loss of young from transferring to old actually can be compensated by receiving that amount when old, it is also due to the concavity of utility.
    \end{itemize}
    
    \item[-] \underline{\textbf{Monetary stationary equilibrium}}
    \begin{itemize}
        \item[-] In \textit{Samuelson economy}, \textbf{MSE is PE} and \textbf{Pareto dominating Autarky}: the initial old, with positive fiat money, are strictly better off, and as well off as they can be in equilibrium for any $m>0$. For generation $t\geq 1$, MSE allocation is identical to the steady-state utility maximizing allocation, hence every generation is at least as good as in autarky.
        
        \item[-] In \textit{classical economy}, \textbf{MSE is PE} but \textbf{NOT Pareto dominating Autarky}, MSE allocation is also steady-state utility maximizing, hence, it is Pareto efficient, but Autarky can not be Pareto improved either.
    \end{itemize}
    \item[-] \underline{\textbf{Monetary non-stationary equilibria}}
    \begin{itemize}
        \item[-] In \textit{Samuelson economy}, \textbf{MNSE is not PE}. All the non-stationary equilibria converge to Autarky, and features \textbf{rising prices}, hence $\lim_{t\rightarrow\infty}p_t=\infty$. All of these non-stationary equilibria are arbitrarily close to each other, they all converge to Autarky, which has a positive, constant price inflation $\hat{p}_t/\hat{p}_{t+1}<1$, hence a negative autarky interest rate $\bar{r}<0$, this means NOT PE.
        
        \item[-] In \textit{classical economy}, \textbf{every MNSE is PE}. All the non-stationary equilibria converge to the monetary stationary equilibrium, which has zero inflation, zero interest rate, hence they are all PE.
        
        \item[-] In \textbf{knife-edge economy}, the conclusion is the same as the classical economy.
    \end{itemize}
\end{itemize}

A general theoretical result is given by Balasko and Shell:

\vspace{2pt}
\begin{tikzpicture}
\node [redbox] (box){%
    \begin{minipage}{0.315\textwidth}
    \color{myred}
    \scriptsize
    Assume:
    \begin{itemize}
        \item[-] stationary endowment $w^t_t=w_1>0,w^t_{t+1}=w^0_1=w_2>0$
        \item[-] allocation is bounded away from zero: $\left(\hat{c}^{t-1}_t,\hat{c}^t_t\right)\geq \delta >0$
    \end{itemize}
    
    Define $$\frac{1}{1+r_{t+1}}=\frac{\beta U'\left(\hat{c}^t_{t+1}\right)}{U'\left(\hat{c}^t_t\right)}=\frac{\hat{p}_{t+1}}{\hat{p}_t}$$
    
    Then the allocation is Pareto efficient if and only if
    $$\sum^{\infty}_{t=1}\prod_{\tau}^t\left(1+r_{\tau+1}\right) =+\infty$$
    
    This \underline{\textbf{includes}} two scenarios (PE):
    \begin{itemize}
        \item[-] AD prices falling, i.e., positive interest rate $r_{t+1}$
        \begin{itemize}
            \item[-] \underline{classic Autarky}
        \end{itemize}
        
        \item[-] AD prices constant, i.e., zero interest rate $r_{t+1}$
        \begin{itemize}
            \item[-] \underline{knife-edge Autarky}, \underline{all knife-edge MNSE} (converge to Autarky) 
            \item[-]\underline{classic MSE}, \underline{all classic MNSE} (converge to classic MSE)
            \item[-] \underline{Samuelson MSE}
        \end{itemize}
    \end{itemize}
    This \underline{\textbf{excludes}} one scenario (not PE):
    \begin{itemize}
        \item[-] AD prices increasing, i.e., negative interest rate $r_{t+1}$ 
        \begin{itemize}
            \item[-] \underline{Samuelson Autarky}, \underline{all Samuelson MNSE} (converge to Samuelson Autarky). 
        \end{itemize}
    \end{itemize}
    \end{minipage}
};
\node[redtitle, right=4pt] at (box.north west) {Balasko and Shell Pareto-efficiency condition};
\end{tikzpicture}

\section*{Neoclassical growth model}
\begin{itemize}
    \item[-] \textbf{discrete} time, indexed by $t$
    \item[-] economy, households, firms live \textbf{infinitely}
    \item[-] single commodity endogenously produced by firms, consumed and invested by households.
    \item[-] there is a continuum of identical \underline{\textbf{households}}, they are \textbf{price takers}, they maximize their lifetime utility, can be represented by a \textbf{representative} household.
    \item[-] there is a continuum of identical \underline{\textbf{firms}}, they are perfectly competitive, maximize their lifetime profits, can be represented by a representative household.
\end{itemize}

\vspace{2pt}
\begin{tikzpicture}
\node [bluebox] (box){%
    \begin{minipage}{0.315\textwidth}
    \color{myblue}
    \scriptsize
    household: $\max U(c)=\sum^{\infty}_{t=0}\beta^t [u(c_t)+\psi(l_t)]$
    
    firms: $\max F(k_t,n_t)-r_t k_t-w_t n_t$ where $F(k_t,n_t)=y_t =c_t+i_t$
    
    \vspace{-6pt}
\begin{center}
\tiny
       \begin{tabular}{rl}
       \multicolumn{2}{c}{assumptions of $u(\cdot)$ and $F(\cdot)$}\\
        \hline\hline
        $u(c_t):$ &continuously differentiable, strictly increasing and concave, \\
          &satisfying Inada condition, time separable, $\beta\in (0,1)$\\
          \hline
          $\psi(l_t):$ &value of leisure\\
          \hline
          $F(k_t,n_t):$ & continuously differentiable, strictly increasing and concave \\
          &both $F_k$ and $F_n$ satisfying Inada condition\\
          & homogeneous of degree 1, constant return to scale\\
          \hline\hline
        \end{tabular} 
\end{center}

Households and firms are perfectly competitive: they are price takers, earn zero economic profit, and achieve a perfectly \textbf{competitive equilibrium environment}.
    \end{minipage}
};
\node[bluetitle, right=4pt] at (box.north west) {Agents' preferences and equilibrium behavior};
\end{tikzpicture}

\vspace{2pt}
\begin{tikzpicture}
\node [bluebox] (box){%
    \begin{minipage}{0.315\textwidth}
    \color{myblue}
    \scriptsize
    households are endowed with
    \begin{itemize}
        \item[-] 1 unit of time, can be used for labor or leisure $n_t+l_t=1$
        \item[-] initial capital stock $\bar{k}_0>0$
        \item[-] \textbf{NO goods}: all final goods are produced endogenously.
    \end{itemize}
    \vspace{3pt}
    households and firms meet in markets and trade:
    \begin{itemize}
        \item[-] households sell capital and labor, earn rental and wage
        \item[-] firms sell goods, earn \textbf{zero} profit
    \end{itemize}
    \vspace{3pt}
    \textbf{And, Households are assumed to own firms}, hence own profit.
    \end{minipage}
};
\node[bluetitle, right=4pt] at (box.north west) {Agents' endowment};
\end{tikzpicture}

\vspace{2pt}
\begin{tikzpicture}
\node [bluebox] (box){%
    \begin{minipage}{0.315\textwidth}
    \color{myblue}
    \scriptsize
    \textbf{households}: households are endowed with initial capital $k_0$ and produce new capital via: $$k_{t+1}=(1-\delta)k_t+i_t$$
    where $\delta$ is the depreciation ratio $\delta\in [0,1]$
    
    \rule{\textwidth}{0.4pt}
    
    \vspace{3pt}
    \textbf{firms}: the most common production functions is the family of the constant elasticity of substitution (CES) function:
    $$y_t = F(k_t,n_t)=A\left(\alpha k_t^{1-\frac{1}{\gamma}}+(1-\alpha)n_t^{1-\frac{1}{\gamma}}\right)^\frac{1}{1-\frac{1}{\gamma}}$$
    it has the following properties:
    
    \vspace{2pt}
    \begin{tikzpicture}
    \node [ibluebox] (box){%
        \begin{minipage}{0.935\textwidth}
        \color{white}
        \scriptsize
        \underline{\textit{\textbf{Properties of CES function}}}:
        \begin{itemize}
            \item[-] Convergence:
            \begin{itemize}
                \item[-] $\gamma\rightarrow 1$: converges to Cobb-Douglas function 
                \item[-] $\gamma\rightarrow 0$: converges to Leontief function
            \end{itemize}
            \item[-] constant return to scale
            \item[-] positive and diminishing marginal productivity
            \item[-] inputs are complements (cross partial derivatives $>0$)
        \end{itemize}
        \end{minipage}
    };
    \end{tikzpicture}
    \end{minipage}
};
\node[bluetitle, right=4pt] at (box.north west) {Agents' technology};
\end{tikzpicture}

Here is the proof of the properties of CES functions:
\begin{itemize}
    \item[-] $\gamma\rightarrow 1$ leading to Cobb-Douglas:
    \begin{itemize}
        \item[-] \textbf{L'Hopital rule:} rewrite the function as 
        \begin{align*}
        \tiny
        \ln y_t &= \ln A +\frac{\ln\left[\alpha k_t^{1-\frac{1}{\gamma}}+(1-\alpha)n_t^{1-\frac{1}{\gamma}}\right]{\color{myorange}\xrightarrow{\gamma\rightarrow 1}0}}{1-\frac{1}{\gamma}{\color{myorange}\xrightarrow{\gamma\rightarrow 1}0}}\\
        {\color{myorange}(\text{L'Hopital})}&\rightarrow\ln A+\frac{\alpha \ln k_t\cdot k_t^{1-\frac{1}{\gamma}}+(1-\alpha) \ln n_t\cdot n_t^{1-\frac{1}{\gamma}}}{\left(\alpha k_t^{1-\frac{1}{\gamma}}+(1-\alpha)n_t^{1-\frac{1}{\gamma}}\right)}\\
        \Rightarrow\lim_{\gamma\rightarrow 1}\ln y_t & =\ln A+\alpha \ln k_t +(1-\alpha)\ln n_t\\
        \Rightarrow \lim_{\gamma\rightarrow 1} y_t & =e^{\ln A+\alpha \ln k_t +(1-\alpha)\ln n_t} = Ak_t^{\alpha}n_t^{1-\alpha}
        \end{align*}
        
        \item[-] \textbf{Total differentials:} rewrite the function as 
        $$
            y_t^{1-\frac{1}{\gamma}}=A^{1-\frac{1}{\gamma}}\left[\alpha k_t^{1-\frac{1}{\gamma}}+(1-\alpha)n_t^{1-\frac{1}{\gamma}}\right]
        $$
        take total differentials, get:
        \begin{align*}
            \left(1-\frac{1}{\gamma}\right)y_t^{-\frac{1}{\gamma}}\mathrm{d}y_t =& A^{1-\frac{1}{\gamma}}\left(\left(1-\frac{1}{\gamma}\right)\alpha k_t^{-\frac{1}{\gamma}}\mathrm{d}k_t\right.\\
            &+\left.\left(1-\frac{1}{\gamma}\right)(1-\alpha)n_t^{-\frac{1}{\gamma}}\mathrm{d}n_t\right)\\
            \Rightarrow y_t^{-\frac{1}{\gamma}}\mathrm{d}y_t =&A^{1-\frac{1}{\gamma}}\left(\alpha k_t^{-\frac{1}{\gamma}}\mathrm{d}k_t +(1-\alpha)n_t^{-\frac{1}{\gamma}}\mathrm{d}n_t\right)\\
            \xRightarrow{\gamma\rightarrow 1} \frac{1}{y_t}\mathrm{d}y_t=&\left(\alpha\frac{1}{k_t}\mathrm{d}k_t+(1-\alpha)\frac{1}{n_t}\mathrm{d}n_t \right)\\
            \xRightarrow{\int} \ln y_t +c_y =& \left[\alpha\left(\ln k_t+c_k\right)+(1-\alpha)\left( \ln n_t+c_n\right) \right]\\
            \Rightarrow y_t =& {\color{myorange}e^{\alpha c_k+(1-\alpha)c_n-c_y}}k_t^{\alpha}n_t^{1-\alpha}=\tilde{A}k_t^{\alpha}n_t^{1-\alpha}
        \end{align*}
        
        \item[-] \textbf{Taylor expansion:} expand $Q(k_t,n_t)\equiv\alpha k_t^{1-\frac{1}{\gamma}}+(1-\alpha)n_t^{1-\frac{1}{\gamma}}$ at $\gamma=1$, get:
        \begin{align*}
            Q(k_t,n_t)= & \left.\alpha k_t^{1-\frac{1}{\gamma}}\right\vert_{\gamma=1}+\left.(1-\alpha)n_t^{1-\frac{1}{\gamma}}\right\vert_{\gamma=1}\\
            &+ \left.\alpha\ln k_t\cdot k_t^{1-\frac{1}{\gamma}}\right\vert_{\gamma=1}\cdot(\gamma-1)\\
            &+ \left. (1-\alpha)\ln n_t\cdot n_t^{1-\frac{1}{\gamma}} \right\vert_{\gamma=1}\cdot (\gamma-1) + O\left((\gamma-1)^2\right)\\
            = & 1+(\gamma-1)\ln k_t^{\alpha}n_t^{1-\alpha} +O\left((\gamma-1)^2\right)
        \end{align*}
        plug $Q(k_t,n_t)$ into $y_t$, get
        \begin{align*}
            y_t &= A\cdot Q(k_t,n_t)^{\frac{1}{1-\frac{1}{\gamma}}}\\
             &= A \left[1+(\gamma-1)\ln k_t^{\alpha}n_t^{1-\alpha} +O\left((\gamma-1)^2\right)\right]^{\frac{1}{1-\frac{1}{\gamma}}}\\
            \xRightarrow{r=\frac{1}{\gamma-1}} &= A\left[ 1+\frac{1}{r}\ln k_t^{\alpha}n_t^{1-\alpha} +O(\frac{1}{r^2}) \right]^{\gamma r}
        \end{align*}
        since $\gamma\rightarrow 1\Rightarrow r\rightarrow \infty$, we have 
        \begin{align*}
            \lim_{\gamma\rightarrow 1} y_t &= \lim_{\gamma\rightarrow 1} A\left[ 1+\frac{\ln k_t^{\alpha}n_t^{1-\alpha}}{r}+O(\frac{1}{r^2}) \right]^{\gamma r}\\
            &=\lim_{\gamma\rightarrow 1} A\left[ 1+\frac{\ln k_t^{\alpha}n_t^{1-\alpha}}{r} \right]^{\gamma r}\\
            \xRightarrow{(1+\frac{x}{\infty})^{\infty}\rightarrow e^x} & = \left(e^{\ln k_t^{\alpha}n_t^{1-\alpha}}\right)^{\gamma} = k_t^{\alpha}n_t^{1-\alpha}
        \end{align*}
    \end{itemize}
    
    \item[-] $\gamma\rightarrow 0$ leading to Leontief:
    \begin{itemize}
        \item[-] \textbf{L'Hopital rule:} rewrite the function as 
        \begin{align*}
        \ln y_t &= \ln A +\frac{\ln\left[\alpha k_t^{1-\frac{1}{\gamma}}+(1-\alpha)n_t^{1-\frac{1}{\gamma}}\right]{\color{myorange}\xrightarrow{\gamma\rightarrow 0}-\infty}}{1-\frac{1}{\gamma}{\color{myorange}\xrightarrow{\gamma\rightarrow 0}-\infty}}\\
        {\color{myorange}(\text{L'Hopital})}&\rightarrow\ln A+\frac{\alpha \ln k_t\cdot k_t^{1-\frac{1}{\gamma}}+(1-\alpha) \ln n_t\cdot n_t^{1-\frac{1}{\gamma}}{\color{myorange}\rightarrow 0}}{\left(\alpha k_t^{1-\frac{1}{\gamma}}+(1-\alpha)n_t^{1-\frac{1}{\gamma}}\right){\color{myorange}\rightarrow 0}}
        \end{align*}
        
        Here, we have a $\frac{0}{0}$ limit. Using L'Hopital rule again yields NOTHING. To proceed, define $x_t=\min\left\{k_t,n_t\right\}$, then we have
        \begin{align*}
        \lim_{\gamma\rightarrow 0}\ln y_t &=\ln A+\frac{\alpha \ln k_t\cdot \left(\frac{k_t}{x_t}\right)^{1-\frac{1}{\gamma}}+(1-\alpha) \ln n_t\cdot \left(\frac{n_t}{x_t}\right)^{1-\frac{1}{\gamma}}}{\left(\alpha \left(\frac{k_t}{x_t}\right)^{1-\frac{1}{\gamma}}+(1-\alpha)\left(\frac{n_t}{x_t}\right)^{1-\frac{1}{\gamma}}\right)}\\
        &=\ln A +
        \begin{cases}
        \ln k_t, & k_t<n_t\\
        \ln n_t, & k_t>n_t\\
        \ln \alpha\ln k_t +(1-\alpha)\ln n_t, & k_t=n_t\\
        \end{cases}\\
        &=\ln A + \ln(\min\{n_t,k_t\})\Rightarrow \lim_{\gamma\rightarrow 0}y_t = A\min\{n_t,k_t\}
        \end{align*}
        
        \item[-] \textbf{Sandwich theorem:} without losing generality, assume $k_t\geq n_t>0$, consider this inequality:
        \begin{align*}
            & (1-\alpha)n_t^{1-\frac{1}{\gamma}} \leq \alpha k_t^{1-\frac{1}{\gamma}}+(1-\alpha)n_t^{1-\frac{1}{\gamma}} \leq n_t^{1-\frac{1}{\gamma}}\\
            \Rightarrow & (1-\alpha)^{\frac{1}{1-\frac{1}{\gamma}}}n_t \leq \left[\alpha k_t^{1-\frac{1}{\gamma}}+(1-\alpha)n_t^{1-\frac{1}{\gamma}}\right]^{\frac{1}{1-\frac{1}{\gamma}}} \leq n_t
        \end{align*}
        notice that $$ \lim_{\gamma\rightarrow 0}(1-\alpha)^{\frac{1}{1-\frac{1}{\gamma}}}n_t = n_t$$ then by the Sandwich Theorem, we have
        $$ \lim_{\gamma\rightarrow 0}\left[\alpha k_t^{1-\frac{1}{\gamma}}+(1-\alpha)n_t^{1-\frac{1}{\gamma}}\right]^{\frac{1}{1-\frac{1}{\gamma}}} =n_t$$
        therefore
        \begin{align*}
            \lim_{\gamma\rightarrow 0}y_t &= \lim_{\gamma\rightarrow 0}A\left[\alpha k_t^{1-\frac{1}{\gamma}}+(1-\alpha)n_t^{1-\frac{1}{\gamma}}\right]^{\frac{1}{1-\frac{1}{\gamma}}}\\
            & =A n_t = A\min\left\{k_t,n_t\right\}
        \end{align*}
    \end{itemize}
    
    \item[-] Constant return to scale: $F(\lambda k_t,\lambda n_t)=\lambda F(k_t,n_t)$
    
    It is very easy to verify directly. Another way to verify this is use Euler's theorem, constant return to scale means homogeneity of degree 1, hence Euler's theorem requires $F(k_t,n_t)=F_k(k_t,n_t)k_t+F_n(k_t,n_t)n_t$, that is:
    \begin{align*}
        & k_t F_k(k_t,n_t)+n_t F_n(k_t,n_t) \\
        =& k_t A \frac{1}{1-\frac{1}{\gamma}}\cdot \left(\alpha k_t^{1-\frac{1}{\gamma}}+(1-\alpha)n_t^{1-\frac{1}{\gamma}}\right)^{\frac{1}{\gamma-1}} \alpha\left(1-\frac{1}{\gamma}\right)k_t^{-\frac{1}{\gamma}}\\ &+n_t A \frac{1}{1-\frac{1}{\gamma}}\cdot \left(\alpha k_t^{1-\frac{1}{\gamma}}+(1-\alpha)n_t^{1-\frac{1}{\gamma}}\right)^{\frac{1}{\gamma-1}} (1-\alpha)\left(1-\frac{1}{\gamma}\right)n_t^{-\frac{1}{\gamma}}\\
        =& A\left( \alpha k_t^{1-\frac{1}{\gamma}}+(1-\alpha) n_t^{1-\frac{1}{\gamma}} \right)\cdot\left(\alpha k_t^{1-\frac{1}{\gamma}}+(1-\alpha)n_t^{1-\frac{1}{\gamma}}\right)^{\frac{1}{\gamma-1}}\\
        =& A\left(\alpha k_t^{1-\frac{1}{\gamma}}+(1-\alpha)n_t^{1-\frac{1}{\gamma}}\right)^{\frac{1}{1-\frac{1}{\gamma}}}= F(k_t,n_t)
    \end{align*}
    
    \item[-] positive and diminishing marginal productivity.
    
    Again, fairly straightforward, we verity $F_k(k_t,n_t)>0, F_n(k_t,n_t)>0$; $F_{kk}(k_t,n_t)<0,F_{nn}(k_t,n_t)<0$:
    \begin{itemize}
        \item[-] $F_k$ and $F_n$: take partial derivatives w.r.t. $k_t$, get
        $$
        F_k=A\left(\alpha k_t^{1-\frac{1}{\gamma}}+(1-\alpha)n_t^{1-\frac{1}{\gamma}}\right)^{\frac{1}{1-\frac{1}{\gamma}}-1}\cdot\alpha k_t^{-\frac{1}{\gamma}}\geq 0
        $$
        
        for $n_t$, similarly
        $$
        F_n=A\left(\alpha k_t^{1-\frac{1}{\gamma}}+(1-\alpha)n_t^{1-\frac{1}{\gamma}}\right)^{\frac{1}{1-\frac{1}{\gamma}}-1}\cdot(1-\alpha) n_t^{-\frac{1}{\gamma}}\geq 0
        $$
        
        \item[-] $F_{kk}$ and $F_{nn}$: take twice partial derivatives w.r.t. $k_t$, for simplicity, again let $Q_t\equiv \alpha k_t^{1-\frac{1}{\gamma}}+(1-\alpha)n_t^{1-\frac{1}{\gamma}}$, get
        \begin{align*}
            F_{kk} =& AQ_t^{\frac{1}{1-\frac{1}{\gamma}}-1-1}\cdot\left(\frac{1}{1-\frac{1}{\gamma}}-1\right)\cdot\left(1-\frac{1}{\gamma}\right)\alpha k_t^{-\frac{1}{\gamma}}\cdot \alpha k_t^{-\frac{1}{\gamma}}\\
            &+ \alpha\left(-\frac{1}{\gamma}\right)k_t^{-\frac{1}{\gamma}-1}\cdot AQ_t^{\frac{1}{1-\frac{1}{\gamma}}-1}\\
            =& AQ_t^{\frac{1}{1-\frac{1}{\gamma}}-1-1}\alpha \frac{1}{\gamma}\cdot \alpha k_t^{-\frac{2}{\gamma}}-\alpha\frac{1}{\gamma}k_t^{-\frac{1}{\gamma}-1}AQ_t^{\frac{1}{1-\frac{1}{\gamma}}-1}\\
            =& \alpha \frac{1}{\gamma}k_t^{-\frac{1}{\gamma}-1}AQ_t^{\frac{1}{1-\frac{1}{\gamma}}-1}\cdot \left( Q_t^{-1}\alpha k_t^{1-\frac{1}{\gamma}} -1\right)
        \end{align*}
        since $\alpha k_t^{1-\frac{1}{\gamma}}Q_t^{-1}=\frac{\alpha k_t^{1-\frac{1}{\gamma}}}{\alpha k_t^{1-\frac{1}{\gamma}}+(1-\alpha) n_t^{1-\frac{1}{\gamma}}}<1$, $F_{kk}<0$. Similarly for $F_{nn}$:
        $$
            F_{nn} = (1-\alpha) \frac{1}{\gamma}n_t^{-\frac{1}{\gamma}-1}AQ_t^{\frac{1}{1-\frac{1}{\gamma}}-1}\cdot \left( Q_t^{-1}(1-\alpha) n_t^{1-\frac{1}{\gamma}} -1\right)<0
        $$
        
        \item[-] $F_{kn}=F_{nk}>0$: by the symmetry of Hessian matrix, we only need to verifyt $F_{kn}$, that is
        \begin{align*}
            F_{kn} =& AQ_t^{\frac{1}{1-\frac{1}{\gamma}}-1-1}\left(\frac{1}{1-\frac{1}{\gamma}}-1\right)\alpha k_t^{-\frac{1}{\gamma}}\cdot (1-\alpha)(1-\frac{1}{\gamma})n_t^{-\frac{1}{\gamma}}\\
            =& AQ_t^{\frac{1}{1-\frac{1}{\gamma}}-2}\frac{1}{\gamma}\alpha(1-\alpha)(k_tn_t)^{-\frac{1}{\gamma}}>0
        \end{align*}
    \end{itemize}
\end{itemize}

\subsection*{How to solve the model: social planner's problem}

This model can be analytically solved only when certain assumptions are imposed on utility functions and production functions. However, there exists a solution: capital must be cleared eventually, that is $k_{T+1}=0$; and constraint set for $k_{t+1}$ is compact.

\vspace{2pt}
\begin{tikzpicture}
\node [iredbox] (box){%
    \begin{minipage}{0.315\textwidth}
    \color{white}
    \scriptsize
        $$\max_{\left\{c_t,i_t,k_{t+1},l_t,n_t,y_t\right\}^{\infty}_{t=0}} \sum^{\infty}_{t=0}\beta^t(u(c_t)+\psi(l_t))$$
    s.t.
    \begin{align*}
        y_t&=c_t+i_t &\text{market clearing}\\
        k_{t+1}&=(1-\delta)k_t+i_t &\text{capital accumulation}\\
        1&=n_t+l_t & \text{time endowment}\\
        y_t &= A\left(\alpha k_t^{1-\frac{1}{\gamma}}+(1-\alpha)n_t^{1-\frac{1}{\gamma}}\right)^{\frac{1}{1-\frac{1}{\gamma}}} & \text{production function}\\
        0&\leq c_t,i_t,k_t,n_t,l_t,y_t,\forall t&\text{non-negativity}
    \end{align*}
    \end{minipage}
};
\node[iredtitle, right=4pt] at (box.north west) {Social planner's problem};
\end{tikzpicture}


This problem can be rewritten as:
$$
\max_{\left\{c_t,n_t,k_{t+1}\right\}^{\infty}_{t=0}}\sum^{\infty}_{t=0}\beta^t\left(u(c_t)+\psi(1-n_t)\right)
$$
s.t. 
$$
k_{t+1}=(1-\delta)k_t + A\left(\alpha k_t^{1-\frac{1}{\gamma}}+(1-\alpha)n_t^{1-\frac{1}{\gamma}}\right)^{\frac{1}{1-\frac{1}{\gamma}}} - c_t 
$$
and $c_t,k_t,n_t\geq 0,\forall t$. Then, we can derive the Euler equation with Lagrange method:
$$\mathcal{L}= \sum^{\infty}_{t=0}\left[ \beta^t\left(u(c_t)+\psi(1-n_t)\right) +\lambda_t\left( (1-\delta)k_t + F(k_t,n_t)-k_{t+1}-c_t\right)  \right]$$
FOC gives:
\begin{align*}
    \frac{\partial \mathcal{L}}{\partial c_t}=0 \Rightarrow & \beta^t u'(c_t)=\lambda_t &(1)\\
    \frac{\partial \mathcal{L}}{\partial n_t}=0 \Rightarrow & \beta^t \psi'(1-n_t)= \lambda_t F_n(k_t,n_t) &(2)\\
    \frac{\partial \mathcal{L}}{\partial k_{t+1}}=0\Rightarrow & \lambda_{t+1}\left(F_k(k_{t+1},n_{t+1}) +1-\delta \right)=\lambda_t &(3)
\end{align*}

With the FOCS, we can characterize the equilibrium and steady state of this economy:

\vspace{2pt}
\begin{tikzpicture}
\node [redbox] (box){%
    \begin{minipage}{0.315\textwidth}
    \color{myred}
    \scriptsize
    
    The equilibrium in this economy is characterized by the following equations:
    \begin{align*}
        u'(c_t)&=\left(F_k(k_{t+1},n_{t+1})+1-\delta\right)\beta u'(c_{t+1}) & \text{Euler equation} \\
        u'(c_t)&=\frac{\psi'(1-n_t)}{F_n(k_t,n_t)} & \text{leisure condition}\\
        k_{t+1}&=F(k_t,n_t)+(1-\delta)k_t-c_t & \text{budget constraint}
    \end{align*}
    \end{minipage}
};
\node[redtitle, right=4pt] at (box.north west) {Social planner's problem: equilibrium};
\end{tikzpicture}

\vspace{4pt}
\underline{\textbf{{\color{myred}Euler equation} Interpretation}}: today's consumption lost must be compensated by discounted tomorrow's consumption gain.
\begin{itemize}
    \item[-] $u'(c_t)$: the utility loss of giving up 1 unit of $c_t$ for investment
    \item[-] $\beta u'(c_{t+1})(F_k(k_{t+1},n_{t+1})+1-\delta)$:
    \begin{itemize}
        \item[-] $F_k(k_{t+1},n_{t+1})+1-\delta$: the capital accumulation ($1-\delta$) and production $F_k(k_{t+1},n_{t+1})$ of an extra unit of capital
        \item[-] $\beta u'(c_{t+1})$: per-unit discounted utility compensation at $t+1$
    \end{itemize}
\end{itemize}
\underline{\textbf{{\color{myred}leisure condition} interpretation}}: the utility loss of giving up 1 unit of $l_t$ ($\psi'(1-n_t)$) must be compensated by the consumption utility gain due to the additional output $F_n(k_t,n_t)\cdot u'(c_t)$


\vspace{2pt}
\begin{tikzpicture}
\node [redbox] (box){%
    \begin{minipage}{0.315\textwidth}
    \color{myred}
    \scriptsize
    The steady state in this economy is achieved when $c_t\equiv c^*,k_{t+1}\equiv k^*,n_t\equiv n^*$, hence:
    \begin{align*}
        \frac{1}{\beta}-(1-\delta)&=F_k(k^*,n^*) & \text{Euler equation} \\
        u'(c^*)&=\frac{\psi'(1-n^*)}{F_n(k^*,n^*)} & \text{leisure condition}\\
        c^*&=F(k^*,n^*)-\delta k^* & \text{budget constraint}
    \end{align*}
    Plug the output level $y_t=F(k_t,n_t)$ and its partial derivatives back into these three equations, we have the steady state:
    \begin{align*}
        \frac{1}{\beta}-(1-\delta)&=\alpha A^{1-\frac{1}{\gamma}}\left(\frac{y^*}{k^*}\right)^{\frac{1}{\gamma}} & \text{Euler equation} \\
        u'(c^*)&=\frac{\psi'(1-n^*)}{(1-\alpha) A^{1-\frac{1}{\gamma}}\left(\frac{y^*}{n^*}\right)^{\frac{1}{\gamma}}} & \text{leisure condition}\\
        c^*&=y^*-\delta k^* & \text{budget constraint}
    \end{align*}
    
    And we can have the steady state capital-labor ratio:
    $$
    \frac{k^*}{n^*}= \left( \frac{\alpha}{1-\alpha}\cdot\frac{\psi'(1-n^*)/u'(c^*)}{\frac{1}{\beta}-(1-\delta)} \right)
    $$
    but it is generally impossible to solve analytically.
    \end{minipage}
};
\node[redtitle, right=4pt] at (box.north west) {Social planner's problem: steady state};
\end{tikzpicture}

%%%%%%%%%%%%%%%%%%%%%%%%%%%%%%%%%%%%%%%%%
\subsection*{Social planner's problem: a simple example}
In general, social planner's problem of NGM cannot be solved analytically, but for the simplest example, it actually is possible. Assume:
\begin{align*}
    U(c)&=\log(c) & F(k,n) = k^{\alpha}n^{1-\alpha}\\
    \delta &= 1 & f(k)=F(k,1)=k^{\alpha}
\end{align*}
The Euler equation
$$
\frac{\beta c_t}{c_{t+1}}=\frac{1}{F_k(k,n)+1-\delta}\Rightarrow \beta \alpha k_{t+1}^{\alpha-1} (k^{\alpha}_t-k_{t+1}) = (k^{\alpha}_{t+1}-k_{t+2})
$$
define saving rate $z_t = k_{t+1}/k^{\alpha}_t$, then rewrite the Euler equation as 
$$
\frac{\beta \alpha k_{t+1}^{\alpha-1} (k^{\alpha}_t-k_{t+1})}{k_{t+1}^{\alpha}} =\frac{k^{\alpha}_{t+1}-k_{t+2}}{k_{t+1}^{\alpha}}\Rightarrow \beta\alpha\left(\frac{1}{z_t}-1\right)=1-z_{t+1}
$$
For the last period $t=T$, $z_T=0$, hence we can backward-solve $z_t$ as
$$
z_t = \frac{\alpha\beta}{(1+\alpha\beta)-z_{t+1}}, z_T=0
$$
this gives
\begin{align*}
    z_T&=0\\
    z_{T-1}&=\frac{\alpha\beta}{1+\alpha\beta}\\
    z_{T-2} &=\frac{\alpha\beta(1+\alpha\beta)}{1+\alpha\beta+(\alpha\beta)^2}\cdots\\
    z_t &= \frac{k_{t+1}}{k^{\alpha}_t}=\frac{\alpha\beta\left(1-(\alpha\beta)^{T-t}\right)}{1-(\alpha\beta)^{T-t+1}}
\end{align*}
hence
\begin{align*}
    k_{t+1}&=\frac{\alpha\beta\left(1-(\alpha\beta)^{T-t}\right)}{1-(\alpha\beta)^{T-t+1}}k_t^{\alpha}\xrightarrow{T\rightarrow\infty}\alpha\beta k^{\alpha}_t\\
    c_t &= k_t^{\alpha}-k_{t+1} = \frac{1-\alpha\beta}{1-(\alpha\beta)^{T-t+1}}k_t^{\alpha}\xrightarrow{T\rightarrow\infty} (1-\alpha\beta)k^{\alpha}_t
\end{align*}

In the steady state, $z_{t+1}=z_t$, gives
$$
\beta\alpha\left(\frac{1}{z}-1\right)=1-z\Rightarrow z=1,z=\alpha\beta
$$
obvious, $z=1$ ($c=0$) is not optimal, hence optimal saving rate is $\alpha\beta$.

Next, back to the general problem, check Pareto efficiency.

\vspace{2pt}
\begin{tikzpicture}
\node [redbox] (box){%
    \begin{minipage}{0.315\textwidth}
    \color{myred}
    \scriptsize
    The Pareto efficient condition for an allocation $\{k_{t+1}\}_{t=0}^{\infty}$ is (by Stokey and Lucas):
    \begin{itemize}
        \item[-] satisfying the Euler equation:
        $$U'(c_t)=\beta u'(c_{t+1})\left[F_k(k_t,n_t)+(1-\delta)\right]$$
        \item[-]: Transversality condition:
        $$
        \lim_{t\rightarrow\infty}\lambda_t F_k(k_t,n_t)k_t = 0 
        $$
        where $\lambda_t$, the social planner's Lagrange multiplier, is just $\beta^t u'(c_t)$.
    \end{itemize}
    \end{minipage}
};
\node[redtitle, right=4pt] at (box.north west) {Social planner's problem: Pareto efficiency};
\end{tikzpicture}

We can check the two conditions of the simple example. The two conditions to be satisfied are:
\begin{itemize}
    \item[-] Euler equation: $$ \beta \alpha k_{t+1}^{\alpha-1} (k^{\alpha}_t-k_{t+1}) = (k^{\alpha}_{t+1}-k_{t+2}) \Rightarrow \beta\alpha\left(\frac{1}{z_t}-1\right)=1-z_{t+1} $$ 
    \item[-] TVC: $$ \lim_{t\rightarrow\infty} \frac{\beta^t}{c_t}\alpha k^{\alpha-1}_{t}k_t =0 \xRightarrow{c_t=k^{\alpha}_t-k_{t+1}}\lim_{t\rightarrow\infty} \frac{\alpha\beta^t}{1-z_t} =0$$
\end{itemize}

Instead solving it backwards, deduct forwards, to check whether TVC is violated:
\begin{itemize}
    \item[-] if $z_0<\alpha\beta$, $z_t$ will become negative in finite time, violating TVC.
    \item[-] if $z_0>\alpha\beta$, Euler equation gives $\lim_{t\rightarrow \infty}z_t=1$, TVC must be violated since in the neighborhood or $z_t=1$, we have:
    \begin{align*}
        z_{t+1}&=1+\alpha\beta +\frac{\alpha\beta}{z_t}\\
        &\simeq 1+\alpha\beta - \left[\left.\frac{\alpha\beta}{z_t}\right\vert_{z_t=1} - \left.\frac{\alpha\beta}{z_t^2}\right\vert_{z_t=1}(z_t-1)\right]\\
        & = 1+\alpha\beta(z_t-1)\Rightarrow 1-z_{t+1}=\alpha\beta(1-z_t) \\
        \Rightarrow & 1-z_t \simeq (\alpha\beta)^{t-k}(1-z_k)\\
        \Rightarrow & \lim_{t\rightarrow\infty}\frac{\alpha\beta^t}{(\alpha\beta)^{t-k}(1-z_k)}=\lim_{t\infty}\frac{\beta^k}{\alpha^{t-k-1}(1-z_k)}=\infty\neq 0
    \end{align*}
    i.e., TVC is violated.
    \item[-] if $z_0=\alpha\beta$, the saving rate sequence $z_t=\alpha\beta,\forall t$, hence TVC $\lim_{t\rightarrow\infty}\frac{\alpha\beta^t}{1-\alpha\beta}=0$ is satisfied.
\end{itemize}

Now, the simple example gives the steady state:
\begin{align*}
    k^*& = \alpha \beta {k^*}^{\alpha} \Rightarrow k^* = (\alpha\beta)^{\frac{1}{1-\alpha}}\\
    c^* &= (1-\alpha\beta)k^* = (1-\alpha\beta)(\alpha\beta)^{\frac{\alpha}{1-\alpha}}\\
    y^* &= {k^*}^{\alpha}=(\alpha\beta)^{\frac{\alpha}{1-\alpha}}\\
    \frac{1}{\beta} & =F_k(k^*,n^*)+1-\delta &\cdots\text{Modified golden rule}\\
\end{align*}

\subsection*{Kaldor's facts and national accounting}
\begin{itemize}
    \item[-] steady capital-output ratios over time, i.e., common growth rates of capital and output
    \item[-] steady shares of profit and wages in income
    \item[-] cross-country differences in  output per worker (living standard) and in growth rates of output and output per worker
\end{itemize}

To  account for these, check $Y_t = A_t K_t^{\alpha}N_t^{1-\alpha}$, then:
\begin{itemize}
    \item[-] constant growth in output per worker
    $$
    \frac{Y_t}{N_t}=A_t\left(\frac{K_t}{N_t}\right)^{\alpha}
    $$
    requires constant growth of $A_t$ and $K_t/N_t$.
    
    It can also be written as
    $$
    \frac{Y_t}{N_t} =A_t^{\frac{1}{1-\alpha}}\left(\frac{K_t}{Y_t}\right)^{\frac{\alpha}{1-\alpha}}\Rightarrow \frac{Y_t}{Pop_t}= A_t^{\frac{1}{1-\alpha}}\left(\frac{K_t}{Y_t}\right)^{\frac{\alpha}{1-\alpha}}\frac{N_t}{Pop_t}
    $$
    where
    \begin{itemize}
        \item[-] $A_t^{\frac{1}{1-\alpha}}$: TFP factor
        \item[-] $\left(\frac{K_t}{Y_t}\right)^{\frac{\alpha}{1-\alpha}}$: capital factor
        \item[-] $\frac{N_t}{Pop_t}$: hours worked factor
    \end{itemize}
    \item[-] Empirically, rewrite production function in log
    $$
    \ln Y=\ln A+\alpha\ln K +(1-\alpha)\ln N
    $$
    this implies empirically
    $$
    g_{y,t+1}=g_{A,t+1}+\alpha g_{K,t+1}+(1-\alpha)g_{N,t+1}
    $$
    where $g_{i,t+1}=\frac{i_{t+1}-i_t}{i_t}$
\end{itemize}

\section*{NGM with exogenous growth}
To add exogenous growth, consider \textbf{labor-augmenting technological change}: $F(k_t,\gamma^t n_t)$, the goal is to replicate Kaldor's facts.

We deduct backwards: if a balanced growth path exists, all variables increase at constant rates:
$$
y_t=y_0g_y^t,\ c_t=c_0g_c^t,\ i_t=i_0g_i^t,\ k_t=k_0g_k^t,\ n_t=n_0 g_n^t 
$$
by the budget constraint:
$$
y_t=c_t+i_t,\ i_t=k_{t+1}-(1-\delta)k_t
$$
it must be that all variables grow at the \textbf{same} rate
$$
g_c = g_y = g_k = g_i
$$
since $F(k_t,\gamma^t n_t)$ is constant return to scale, we have
$$
g_y = \gamma g_n
$$
if assume $g_n=1$, then all variables will grow at $g_y=g_c=g_i=g_k =\gamma$.

If add all types of exogenous technological change simultaneously, get
$$
c_t+\gamma_i^{-t}i_t = \gamma_z^t F(\gamma_k^tk_t,\gamma_n^tn_t)
$$
we have
\begin{itemize}
    \item[-] TFP change $\gamma_z^t$ is redundant: $\gamma_z^tF(\gamma_k^t k_t,\gamma_n^t n_t)=F((\gamma_z\gamma_k)^tk_t,(\gamma_z\gamma_n)^tn_t)$
    \item[-] If $F(\cdot,\cdot)$ is \textbf{NOT} Cobb-Douglas, only $\gamma_n$ can be larger than 1
    \item[-] Only Cobb-Douglas $F(\cdot,\cdot)$ can support $\gamma_k,\gamma_n,\gamma_i>1$
\end{itemize}

Assume Cobb-Douglas production and allow capital, labor and investment technological change, then
$$
c_t+\gamma_i^{-t}i_t = \left(\gamma_k^tk_t\right)^{\alpha}\left(\gamma_n^t n_t\right)^{1-\alpha}
$$
define $\hat{\gamma}_n=\gamma^{\frac{\alpha}{1-\alpha}}\gamma_n$, then the constraint can be rewritten as
$$
c_t+\gamma_i^{-t}i_t = k_t^{\alpha}\left(\hat{\gamma_n}^t n_t\right)^{1-\alpha}
$$
For investment
$$
k_{t+1}-(1-\delta)k_t=i_t\Rightarrow \frac{k_{t+1}}{\gamma_i^{t+1}}\gamma_i = \frac{k_t}{\gamma_i^t}(1-\delta)+\frac{i_t}{\gamma_i^t}
$$
hence, \textbf{detrend} $k_t$ and $i_t$ by $\gamma_i$, get
$$
\tilde{k}_{t+1}\gamma_i =\tilde{k}_t(1-\delta)+\tilde{i}_t
$$
where $\tilde{k}_t=k_t/\gamma_i^t, \tilde{i}_t=i_t/\gamma_i^t$, the detrended capital and investment. Then
\begin{align*}
    y_t & = k_t^{\alpha}\left(\hat{\gamma}_n^t n_t\right)^{1-\alpha} = (\tilde{k}_t\gamma_i^t)^{\alpha}\left(\hat{\gamma}_n^t n_t\right)^{1-\alpha}\\
    &= \tilde{k}_t^{\alpha}\left(\left(\gamma_i^{\frac{\alpha}{1-\alpha}}\hat{\gamma}_n\right)^tn_t\right)^{1-\alpha} = c_t+\tilde{i}_t
\end{align*}
if define $\gamma_i^{\frac{\alpha}{1-\alpha}}\hat{\gamma}_n$ as $\tilde{\gamma}_n$, this is reduced to labor technological change only, hence can achieve a steady state. 

As for the de-trended variables:
$$
\frac{\tilde{k}_{t+1}}{\tilde{k}_t}=\frac{1-\delta}{\gamma_i}+\frac{\tilde{i}}{\tilde{k}_t\gamma_i}
$$
on BGP, $\tilde{k}_{t+1}/\tilde{k}_t$ is constant, hence $\tilde{i}_t/\tilde{k}_t$ is also constant, therefore $\tilde{k}_t$ and $\tilde{i}_t$ grow at the same rate, that is
$$
\frac{\tilde{k}_{t+1}}{\tilde{k}_t}=\frac{\tilde{i}_{t+1}}{\tilde{i}_t}\Rightarrow \frac{k_{t+1}}{k_t}=\frac{i_{t+1}}{i_t}\Rightarrow g_{\tilde{k}}=g_{\tilde{i}}\Rightarrow g_k=g_i
$$
and by $y_t=c_t+i_t$, naturally we have $g_y=g_c=g_{\tilde{i}}$. $g_y=\frac{y_{t+1}}{y_t}$ being constant gives
$$
g_{\tilde{k}}= c\tilde{\gamma_n}
$$
where $c$ is a constant. $g_y = \tilde{\gamma}_n(c)^\alpha$, hence $c=1$, then on BGP, transformed variables follow
$$
\tilde{\gamma}_n=\gamma_i^{\frac{\alpha}{1-\alpha}}\gamma_k^{\frac{\alpha}{1-\alpha}}\gamma_n = g_{\tilde{k}}=g_{\tilde{i}}=g_y=g_c
$$
original variables follow
$$
g_k=g_i=\tilde{\gamma}_n\gamma_i =\gamma_i^{\frac{1}{1-\alpha}}\gamma_k^{\frac{\alpha}{1-\alpha}}\gamma_n
$$
in summary, with Cobb-Douglas production function, BGP can support capital, labor, invest technological changes at the same time. And transformed variables satisfy Kaldor's facts.

Another problem is whether BGP is optimal. The answer is:
\begin{itemize}
    \item[-] if and only if preference is CES
    \item[-] leisure is NOT valued
    \item[-] preference can NOT include subsistence level: $U(c)=\frac{(c-\bar{c})^{1-\sigma}-1}{1-\sigma}$
\end{itemize}

To check this, examine whether the Euler equation can support constant BGP growth:
\begin{align*}
    &u'(c_t)=\beta u'(c_{t+1})\left( F_k(k_{t+1},\gamma_n^{t+1}n_{t+1})+1-\delta \right)\\
    \Rightarrow & u'(c_t)=\beta u'(c_{t+1})\left( F_k\left(\frac{k_{t+1}}{\gamma_n^{t+1}n_{t+1}},1\right)+1-\delta \right)
\end{align*}
on BGP, $k_{t+1}/\gamma_n^{t+1}n_{t+1}$ is constant, with CES utility, we have
$$
\frac{1}{\beta}= \frac{ F_k\left(\frac{k_{t+1}}{\gamma_n^{t+1}n_{t+1}},1\right)+1-\delta}{\gamma_n^{\sigma}}= \frac{ F_k\left(\frac{k_0}{n_0},1\right)+1-\delta}{\gamma_n^{\sigma}}
$$
This pins down the initial capital $k_0$ to put the economy on BGP.

\subsection*{Recursive problem: social planner}
A general procedure to write recursive problem is:
\begin{itemize}
    \item[-] rewrite the utility maximization problem into the Bellman equation:
    \begin{align*}
        w\left(\bar{k}_0\right) = \max& \sum^{\infty}_{t=0}\beta^t u(c_t) \\
    = \max & \sum^{\infty}_{t=0}\beta^t u(F(k_t)-k_{t+1}+(1-\delta)k_t)\\
        = \max& \left\{ u(F(k_0)-k_1+(1-\delta)k_0) +\right.\\
        & \left. \beta\sum^{\infty}_{t=0}\beta^t u(F(k_{t+1})-k_{t+2}+(1-\delta)k_{t+2}) \right\}\\
        = \max& \left\{ u(F(k_0)-k_1+(1-\delta)k_0) + \beta w(k_1) \right\}\\
        \Rightarrow v(k)=& \max_{k'\leq f(k)+(1-\delta)k}\left\{ u(F(k)-k'+(1-\delta)k)+\beta v(k') \right\}
    \end{align*}
    \item[-] use
    \begin{itemize}
        \item[-] guess and verify
        \item[-] analytic iterations
        \item[-] numerical iterations
    \end{itemize}
    to solve the problem.
\end{itemize}

\section*{NGM: Competitive equilibrium}
Next, write NGM in competitive equilibrium, first in Arrow-Debreu setting, then in sequential market setting.

\subsection*{NGM: AD equilibrium}
An AD equilibrium is prices $\left\{p_t,r_t,w_t\right\}_{t=0}^{\infty}$, allocation for firm $\left\{k^d_t,n^d_t\right\}$ and for HH $\left\{c_t,k^s_t,n^s_t,k_{t+1}\right\}_{t=0}^{\infty}$ that solve
\begin{itemize}
    \item[-] HH's problem
    $$
    \max \sum^{\infty}_{t=0}\beta^tu(c_t)
    $$
    s.t.
    $$
    \sum^{\infty}_{t=0}p_t\left(c_t+(k_{t+1}-(1-\delta)k_t)\right) = \sum^{\infty}_{t=0}p_t\left( w_t n^s_t+r_t k^s_t \right)+\pi
    $$
    \item[-] firm's problem
    $$
    \max p_t(F(k^s_t,n^s_t)-r_tk^s_t-w_t n^s_t)
    $$
    \item[-] market clearing:
    \begin{align*}
        F(k^s_t,n^s_t)&=c_i + k_{t+1}-(1-\delta)k_t\\
        k^s_t&=k^d_t=k_t\\
        n^s_t&=n^d_t=n_t
    \end{align*}
\end{itemize}

Solve firm's problem, get
\begin{align*}
    F_k(k_t,n_t)&=r_t\\
    F_w(k_t,n_t)&=w_t
\end{align*}
hence $\pi=0$.

Solve HH's problem. Lagrange
$$
\mathcal{L}=\sum^{\infty}_{t=0}\beta^tu(c_t)+ \lambda\sum^{\infty}_{t=0}p_t\left[\left( w_t n_t+r_t k_t \right)-\left(c_t+(k_{t+1}-(1-\delta)k_t)\right)\right]
$$
FOC
\begin{align*}
    \frac{\partial{\mathcal{L}}}{c_t}=0\Rightarrow & \beta^t u'(c_t) = \lambda p_t\\
    \frac{\partial{\mathcal{L}}}{k_{t+1}}=0\Rightarrow & p_{t+1}\left[r_{t+1}+(1-\delta)\right] = p_t 
\end{align*}
Euler equation
$$
\frac{\beta u'(c_{t+1})}{\beta u'(c_t)}=\frac{1}{F_k(k_{t+1},1)+(1-\delta)}
$$
TVC condition for Pareto efficiency
$$
\lim_{t\rightarrow\infty}p_tk_{t+1}=0
$$
and final good price sequence
$$
\frac{p_{t+1}}{p_t}=\frac{1}{F_k(k_{t+1},1)+1-\delta}\Rightarrow p_{t+1}=\prod^{t}_{\tau=0}\frac{1}{F_k(k_{\tau+1},1)+1-\delta}
$$

\subsection*{NGM: SM equilibrium}
Again, an SM equilibrium is prices $\left\{r_t,w_t\right\}_{t=0}^{\infty}$ and allocation of firm $\left\{k^d_t,n^d_t\right\}_{t=0}^{\infty}$ and allocation of HH $\left\{c_t,k_{t+1}\right\}_{t=0}^{\infty}$ solves
\begin{itemize}
    \item[-] HH's problem
    $$
    \max\sum_{t=0}^{\infty}\beta^t u(c_t)
    $$
    s.t.
    $$
    c_t+k_{t+1}-(1-\delta)k_t = r_tk_t+w_t
    $$
    \item[-] firm's problem
    $$
    \max F(k^d_t,n^d_t) -r_tk^d_t - w_tn^d_t
    $$
    -\item[-] market clearing
\begin{align*}
    F(k_t,n_t) &= c_t+k_{t+1}-(1-\delta)k_t\\
    k_t &= k^d_t\\
    1 &=n^d_t
\end{align*}
\end{itemize}

Solve firm's problem, again, get
\begin{align*}
    F_k(k_t,1)&=r_t\\
    F_w(k_t,1)&=w_t
\end{align*}

Solve HH's problem. Lagrange
$$
\mathcal{L}=\sum^{\infty}_{t=0}\beta^tu(c_t)+ \mu_t\left[ c_t+k_{t+1}-(1-\delta)k_t-F_k(k_{t+1},1)  \right]
$$
FOC
\begin{align*}
    \frac{\partial{\mathcal{L}}}{c_t}=0\Rightarrow & \beta^t u'(c_t) = \mu_t\\
    \frac{\partial{\mathcal{L}}}{k_{t+1}}=0\Rightarrow & \mu_{t+1}\left[F_k(k_{t+1},1)+(1-\delta)\right] = \mu_t 
\end{align*}
Euler equation
$$
\frac{\beta u'(c_{t+1})}{\beta u'(c_t)}=\frac{1}{F_k(k_{t+1},1)+(1-\delta)}
$$
TVC condition for Pareto efficiency
$$
\lim_{t\rightarrow\infty}\mu_t k_{t+1}=0
$$
The equivalence between AD and SM is built on
$$
\mu_t = \lambda p_t
$$

\vspace{2pt}
\begin{tikzpicture}
\node [redbox] (box){%
    \begin{minipage}{0.315\textwidth}
    \color{myred}
    \scriptsize
    Euler equation: 
    $$
    \frac{\beta u'(c_{t+1})}{\beta u'(c_t)}=\frac{1}{F_k(k_{t+1},1)+(1-\delta)}
    $$
    TVC condition for Pareto efficiency
    $$
    \lim_{t\rightarrow\infty}\lambda p_tk_{t+1}=\lim_{t\rightarrow\infty}\lim_{t\rightarrow}\mu_t k_{t+1}=0
    $$
    \end{minipage}
};
\node[redtitle, right=4pt] at (box.north west) {NGM competitive equilibrium: summary};
\end{tikzpicture}

\subsection*{NGM: recursive competitive equilibrium}
Use SM equilibrium since it already has a recursive feature in it: it is built on the spot market. The HH's Bellman equation is
$$
v(k,K) = \max_{c,k'}\left\{ u(c)+\beta v(k',K') \right\}
$$
s.t.
$$
c+k'-(1-\delta)k = w(K)+kr(K),\ K'=H(K)
$$
and the recursive competitive equilibrium is
\begin{itemize}
    \item[-] value function: $v(k,K)$
    \item[-] policy function: $c=C(k,K)$, $k'=G(k,K)$
    \item[-] pricing function: $w(K),r(K)$
    \item[-] aggregate law of motion: $K'=H(K)$
\end{itemize}
s.t.
\begin{itemize}
    \item[-] given pricing functions, value function solves the Bellman equation, with $C,G$ the associated policy functions
    \item[-] pricing function maximize firm's profit
    \item[-] consistency: $H(K)=K'=G(K,K)$
    \item[-] market clearing: $C(K,K)+G(K,K)-(1-\delta)K=F(K,1)$
\end{itemize}

%%%%%%%%%%%%%%%%%%%%%%%%%%%%%%%%%%%%%%%%%
% Endogenous growth model:
\section*{Endogenous growth models}
To explain \textbf{sustained differences} in growth rates across countries. Instead of assuming exogenous technological changes, these models either alter specifications of the production function and capital accumulation or endogenize technological changes.

\subsection*{AK model}
The basic idea is to \textbf{NOT} let the production function cross the 45-degree line so that it can imply sustained growth.

\begin{tikzpicture}
\node [bluebox] (box){%
    \begin{minipage}{0.315\textwidth}
    \color{myblue}
    \scriptsize
    Start from a simple Solow world: savings and investment is constant fractions of output. The savings and consumption per capita are:
    $$s_t =sAk_t,\ c_t =(1-s)Ak_t$$
    Then the capital accumulation is 
    $$k_{t+1}=(1-\delta)k_t +sAk_t\Rightarrow k_{t+1}=(1-\delta+sA)k_t $$
    Hence, capital growth rate is constant:
    $$ g_k = \frac{k_{t+1}}{k_t}=1-\delta+sA $$
    All endogenous variables grow at this rate:
    $$
    g_y=\frac{Ak_{t+1}}{Ak_t}=g_k = \frac{(1-s)Ak_{t+1}}{(1-s)Ak_t}= g_c = \frac{k_{t+2}-(1-\delta)k_{t+1}}{k_{t+1}-(1-\delta)k_t}=g_i
    $$
    That is, growth rate of output per capita $g_y$ always equals the growth rate of capital per capita $g_k$, \textbf{on or off a balance growth path}. With Solow preferences, there is no transition to BGP: you either on the path, or not.
    \end{minipage}
};
\node[bluetitle, right=4pt] at (box.north west) {A simple example: Solow policy};
\end{tikzpicture}

\vspace{2pt}
Now, we move to a AK model with CES preferences, consider the social planner's problem:

\begin{tikzpicture}
\node [redbox] (box){%
    \begin{minipage}{0.315\textwidth}
    \color{myred}
    \scriptsize
$$\max \sum^{\infty}_{t=0}\beta^t\frac{c_t^{1-\sigma}-1}{1-\sigma}$$
s.t.
$$c_t+k_{t+1} - (1-\delta)k_t =Ak_t, k_0>0$$
Here the production function is $F(K,L)=AK$, the per capita production function is $f\left(\frac{K}{L}\right)=f(k)=Ak$.
    \end{minipage}
};
\node[redtitle, right=4pt] at (box.north west) {Social planner's problem};
\end{tikzpicture}

The Lagrange:
$$ \mathcal{L}\sum^{\infty}_{t=0}\beta^t\frac{c_t^{1-\sigma}-1}{1-\sigma} +\mu_t \left[Ak_t - (c_t+k_{t+1}-(1-\delta)k_t) \right] $$
FOCs are:
\begin{align*}
    \frac{\partial \mathcal{L}}{\partial c_t}=0 &\Rightarrow \beta^t c_t^{-\sigma}=\mu_t\\
    \frac{\partial \mathcal{L}}{\partial k_{t+1}}=0 &\Rightarrow \mu_t = \mu_{t+1}(A+1-\delta)
\end{align*}
which give the Euler equation:

\begin{tikzpicture}
\node [iredbox] (box){%
    \begin{minipage}{0.315\textwidth}
    \color{white}
    \scriptsize
    The Euler equation is
    $$ c_t^{-\sigma} = \beta c_{t+1}^{-\sigma}(A+1-\delta) $$ which gives a constant consumption growth rate $$g_c = \frac{c_{t+1}}{c_t}=\left[\beta(A+1-\delta)\right]^{1/\sigma}$$
    \end{minipage}
};
\node[iredtitle, right=4pt] at (box.north west) {AK model: Euler equation};
\end{tikzpicture}

This optimal, constant consumption growth rate will be equal to the constant growth rate of investment and output.

\begin{tikzpicture}
\node [redbox] (box){%
    \begin{minipage}{0.315\textwidth}
    \color{myred}
    \scriptsize
    Rearrange the constraint:
    $$c_t+k_{t+1} - (1-\delta)k_t =Ak_t \xRightarrow{\div k_t} \frac{c_t}{k_t}+\frac{k_{t+1}}{k_t}=A+1-\delta$$
    this will give that $g_k$ is constant:
    \begin{itemize}
        \item[-] \textbf{if $g_k$ increases}, $c_t/k_t$ must be decreasing over time, but this violates the transversality condition, as marginal utility would fall less quickly than capital grows, hence it is \textbf{NOT an optimal path}. 
        \item[-] \textbf{if $g_k$ decreases}, $c_t/k_t$ must be increasing over time, consumption would exhaust output in finite time and \textbf{violate non-negativity}.
    \end{itemize}
    
    \vspace{2pt}
    \begin{tikzpicture}
    \node [iredbox] (box){%
        \begin{minipage}{0.935\textwidth}
        \color{white}
        Hence, on BGP, capital must grow at a \textbf{constant rate}, the same growth rate of consumption $g_k=g_c$. Rewrite the consumption-capital ratio:
        \begin{align*}
            & \frac{c_t}{k_t} +g_c=A+1-\delta\\
            & \Rightarrow \frac{c_t}{k_t}=A+1-\delta - \left[\beta(A+1-\delta)\right]^{1/\sigma}>0\\
            & \Rightarrow (A+1-\delta)\cdot\left[1-\beta^{1/\sigma}(A+1-\delta)^{\frac{1-\sigma}{\sigma}}\right]>0\\
            & \Rightarrow 1- \beta^{1/\sigma}(A+1-\delta)^{\frac{1-\sigma}{\sigma}}>0
        \end{align*}
        \end{minipage}
    };
    \end{tikzpicture}
    
    \vspace{2pt}
    This condition $1- \beta^{1/\sigma}(A+1-\delta)^{\frac{1-\sigma}{\sigma}}>0$, combined with the positive growth of consumption: $g_c=\frac{c_{t+1}}{c_t}=\left(\beta (A+1-\delta)\right)^{1/\sigma}>1$, gives the conditions of $\beta, A,\delta$:
    $$ \left(\beta (A+1-\delta)\right)^{1/\sigma}>1 >\beta^{1/\sigma}(A+1-\delta)^{\frac{1-\sigma}{\sigma}}$$
    
    \vspace{2pt}
    \begin{tikzpicture}
    \node [iredbox] (box){%
        \begin{minipage}{0.935\textwidth}
        \color{white}
        For a given $k_0$, the unique value of initial consumption $c_0$ that puts the economy on the unique balanced growth path is given by:
        $$ c_0 +k_1 -(1-\delta)k_0 =Ak_0 $$
        rearrange this, get
        $$
        c_0 = Ak_0 - k_1 + (1-\delta)k_0 = (A-g_c +(1-\delta))k_0>0
        $$
        If and only if $c_0 = (A-g_c +(1-\delta))k_0>0$, the economy is on the balanced growth path.
        \end{minipage}
    };
    \end{tikzpicture}
    
    \vspace{2pt}
    The balanced growth rate of this economy is
    $$ g_c = g_k =g_i =g_y = \left(\beta (A+1-\delta)\right)^{1/\sigma}$$
    Finally, we need to check whether lifetime utility is bounded: 
    $$ \max\sum^\infty_{t=0}\beta^t\frac{c_t^{1-\sigma}-1}{1-\sigma} = \sum^\infty_{t=0}\beta^t \frac{\left[\left(\beta (A+1-\delta)\right)^{t/\sigma}c_0\right]^{1-\sigma}-1}{1-\sigma}$$
    For this to be bounded, we must have $\beta^t\cdot \left(\beta (A+1-\delta)\right)^{\frac{1-\sigma}{\sigma}\cdot t} $ is bounded, or 
    $$ \beta \cdot \left(\beta (A+1-\delta)\right)^{\frac{1-\sigma}{\sigma}} = \beta^{\frac{1}{\sigma}} (A+1-\delta)^{\frac{1-\sigma}{\sigma}} <1$$
    this is satisfied by the condition of $c_t/k_t>0$ already.
    \end{minipage}
};
\node[redtitle, right=4pt] at (box.north west) {Growth rate of $c_t$, $k_t$, $i_t$, $y_t$};
\end{tikzpicture}

The core feature of this model is that the balanced growth rate, always attained immediately after selecting the proper initial consumption, is a function of preferences ($\beta,\sigma$) and technology ($A$), which vary across countries.
\vspace{2pt}

\subsection*{Romer externality model}
This model incorporates production externalities in capital accumulation. Aggregate productivity is assumed to grow due to production spillovers, which arise through capital accumulation. There are two interpretations:
\begin{itemize}
    \item[-] Arrow: \textbf{learning by doing} in capital accumulation and used in production
    \item[-] Romer: \textbf{investment in knowledge} and its spillovers acquired in capital accumulation
\end{itemize}

\vspace{2pt}
\begin{tikzpicture}
\node [redbox] (box){%
    \begin{minipage}{0.315\textwidth}
    \color{myred}
    \scriptsize
    Individual firms act as if they cannot affect the \textbf{aggregate level of capital} and take it as a parameter in optimization. But individual capital accumulation decision indeed impacts the aggregate capital stock. They solve:
    $$
    \max F(L_{i,t},K_{i,t},\bar{K}_t)-w_tL_{i,t} - r_tK_{i,t}
    $$
    s.t.
    $$
    \sum^N_{i=1}K_{i,t}=\bar{K}_t,\ \sum^N_{i=1}L_{i,t} = 1
    $$
    where:
    \begin{itemize}
        \item[-] production function $F(L_{i,t},K_{i,t},\bar{K}_t)=AK_{i,t}^{\alpha}L_{i,t}^{1-\alpha}\bar{K}_t^{\gamma}$
        \begin{itemize}
            \item[-] $L_{i,t}$, the labor employed by firm $i(=1,\cdots,N)$ at time $t$
            \item[-] $K_{i,t}$, the capital rented by firm $i$ at time $t$
            \item[-] $\bar{K}_t$, the \textbf{aggregate} capital at $t$
            \item[-] $\gamma=1-\alpha$ (assumed), i.e., aggregate capital is a labor augmenting technology
        \end{itemize}
        \item[-] $w_t$ wage of labor at $t$; $r_t$, rent of capital at $t$
    \end{itemize}
    
    solve this get
    \begin{align*}
        w_t &= (1-\alpha)A \left(\frac{K_{i,t}}{L_{i,t}}\right)^{\alpha} \bar{K}_t^{1-\alpha}\\
        r_t &= \alpha A\left(\frac{K_{i,t}}{L_{i,t}}\right)^{\alpha-1}\bar{K}^{1-\alpha}_t
    \end{align*}
    this gives that
    $$
    \frac{K_{i,t}}{L_{i,t}}=\frac{K_{j,t}}{L_{j,t}},\ \forall i,j
    $$
    and sum across all $N$ firms, get
    $$
    \bar{K}_t = \sum^N_{i=1}K_{i,t}=K_{1,t}\left(1+ \frac{L_{2,t}}{L_{1,t}}+\cdots+\frac{L_{N,t}}{L_{1,t}}\right) = K_{1,t}\frac{\bar{L}_t}{L_{1,t}}=\frac{K_{1,t}}{L_{1,t}}
    $$
    plug this back in the optimal wage and capital rent, get
    \begin{align*}
        w_t &= (1-\alpha)A\left(\frac{K_{i,t}}{L_{i,t}}\right)^{\alpha} \bar{K}_t^{1-\alpha} = (1-\alpha)A\bar{K}_t^{\alpha}\bar{K}_t^{1-\alpha}=(1-\alpha)A\bar{K}_t\\
        r_t &= \alpha A\left(\frac{K_{i,t}}{L_{i,t}}\right)^{\alpha-1}\bar{K}^{1-\alpha}_t = \alpha A\bar{K}_t^{\alpha-1}\bar{K}_t^{1-\alpha}=\alpha A
    \end{align*}
    And the output of firm $i$ is
    $$
    Y_{i,t}= AK_{i,t}^{\alpha}L_{i,t}^{1-\alpha}\bar{K}_t^{1-\alpha} = A\left(\frac{K_{i,t}}{L_{i,t}}\right)^{\alpha}L_{i,t}\bar{K}_t^{1-\alpha} = A\bar{K}_tL_{i,t}
    $$
    plug all these above back to the profit, get
    $$
    \pi_{i,t} = A\bar{K}_tL_{i,t}-(1-\alpha)A\bar{K}_tL_{i,t}-\alpha AK_{i,t} = 0
    $$
    \end{minipage}
};
\node[redtitle, right=4pt] at (box.north west) {Competitive equilibrium: firm's problem};
\end{tikzpicture}

\vspace{2pt}
When we look at aggregate output, it's clear that this model is reduced to AK model:
$$
\bar{Y}_t = \sum^N_{i=1}A\bar{K}_tL_{i,t}=A\bar{K}_t
$$
and when add labor and capital income together, we would also get this aggregate output
$$
w_t\sum^N_{i=1}L_{i,t}+r_t \sum^N_{i=1}K_{i,t} = (1-\alpha)A\bar{K}_t + \alpha A \bar{K}_t = A\bar{K}_t = Y_t
$$
now we look at HH's side.


\vspace{2pt}
\begin{tikzpicture}
\node [redbox] (box){%
    \begin{minipage}{0.315\textwidth}
    \color{myred}
    \scriptsize
    The representative consumer solves
    $$
    \max \sum^{\infty}_{t=0}\beta^t \frac{c_t^{1-\sigma}-1}{1-\sigma}
    $$
    s.t.
    $$
    c_t + k_{t+1}-(1-\delta)k_t = r_tk_t + w_t
    $$
    solve this, get the Euler equation
    $$
    \beta \left(\frac{c_{t+1}}{c_t}\right)^{-\sigma} = \frac{1}{r_{t+1}+(1-\delta)} \Rightarrow \frac{c_{t+1}}{c_t} = \left[\beta \left(\alpha A+1-\delta\right)\right]^{\frac{1}{\sigma}}\equiv g_c^{CE}
    $$
    This \textbf{constant} consumption growth should be the BGP growth rate.
    
    Here $k_t$ is HH's capital holding, in equilibrium, $k_t = \bar{K}_t$. 
    
    \end{minipage}
};
\node[redtitle, right=4pt] at (box.north west) {Competitive equilibrium: consumer's problem};
\end{tikzpicture}

\vspace{2pt}
Next, solve social planner's problem
$$
\max \beta^t \frac{c^{1-\sigma}-1}{1-\sigma}
$$
s.t.
$$
c_t + \bar{K}_{t+1}-(1-\delta)\bar{K}_t = Y_t=A\bar{K}_t
$$
the Lagrange is
$$
\mathcal{L} = \beta^t \frac{c^{1-\sigma}-1}{1-\sigma}+\lambda_t\left(A\bar{K}_t- c_t - \bar{K}_{t+1}+(1-\delta)\bar{K}_t\right)
$$
FOC gives
$$
\beta\left(\frac{c_{t+1}}{c_t}\right)^{-\sigma} = \frac{1}{A + (1-\delta)} \Rightarrow g_c^{SP} = \left[\beta \left(A+1-\delta\right) \right]^{\frac{1}{\sigma}}
$$
comparing this growth rate with the competitive equilibrium growth rate
$$
g_c^{CE}=\left[\beta \left(\alpha A+1-\delta\right) \right]^{\frac{1}{\sigma}} < \left[\beta \left(A+1-\delta\right) \right]^{\frac{1}{\sigma}} =g_c^{SP}
$$
this is to say:
\begin{itemize}
    \item[-] both CE and SP generates a BGP growth rate, but the economy needs to be put on the path to be on the path
    \item[-] production externality from capital is accounted for by the social planner, but not individual agents. Hence, competitive equilibrium BGP growth rate is \textbf{lower}.
\end{itemize}

\rule{0.325\textwidth}{0.4pt}

\vspace{2pt}

\textbf{Pros}: This model adds labor back into the production, though strong assumptions made, that is, aggregate labor is constant.

\textbf{Cons}: But the production externalities from capital accumulation are NOT that big empirically.

\rule{0.325\textwidth}{0.4pt}

%%%%%%%%%%%
\subsection*{Lucas human capital model}
This model allow human capital accumulation. Instead of labor $L$, a \textbf{reproducible} human capital $H$ is used for production: the quality of labor matters.

\vspace{2pt}
\begin{tikzpicture}
\node [redbox] (box){%
    \begin{minipage}{0.315\textwidth}
    \color{myred}
    \scriptsize
    Social planner's problem is as before, with a new production function including human capital
$$\max \sum^{\infty}_{t=0}\beta^t\frac{c_t^{1-\sigma}-1}{1-\sigma}$$
s.t.
$$
    c_t+i_{K,t} + i_{H,t} = Y_t
$$
    where
    \begin{itemize}
        \item[-] production function $Y_t=K_t^{\alpha}H_t^{1-\alpha}$
        \item[-] physical capital accumulation: $i_{K,t}=K_{t+1}-(1-\delta_K)K_{t}$
        \item[-] human capital accumulation: $i_{H,t}=H_{t+1}-(1-\delta_H)H_t$
    \end{itemize}
    hence, the Lagrange is
    \begin{align*}
        \mathcal{L} =& \sum^{\infty}_{t=0}\beta^t\frac{c_t^{1-\sigma}-1}{1-\sigma}+\\
        & \lambda_t\left[ K_t^{\alpha}H_t^{1-\alpha} - c_t - \left(K_{t+1}-(1-\delta_K)K_t\right) - \left(H_{t+1}-(1-\delta_H)H_t\right) \right]\\
    \end{align*}
    get FOC
    \begin{align*}
        \frac{\partial \mathcal{L}}{\partial c_t}=0 \Rightarrow & \beta\left(\frac{c_{t+1}}{c_t}\right)^{-\sigma}=\frac{\lambda_{t+1}}{\lambda_t}\\
        \frac{\partial \mathcal{L}}{\partial K_{t+1}}=0 \Rightarrow & \lambda_{t+1}\left[\alpha \left(\frac{H_{t+1}}{K_{t+1}}\right)^{1-\alpha}+ (1-\delta_K)\right]=\lambda_t\\
        \frac{\partial \mathcal{L}}{\partial H_{t+1}}=0 \Rightarrow & \lambda_{t+1}\left[(1-\alpha) \left(\frac{H_{t+1}}{K_{t+1}}\right)^{-\alpha}+ (1-\delta_H)\right]=\lambda_t\\
    \end{align*}
    then we have
    $$
    \alpha \left(\frac{H_{t+1}}{K_{t+1}}\right)^{1-\alpha}+ (1-\delta_K) = \frac{1}{\beta}\left(\frac{c_{t+1}}{c_t}\right)^{\sigma}= (1-\alpha) \left(\frac{H_{t+1}}{K_{t+1}}\right)^{-\alpha}+ (1-\delta_H)
    $$
    this requires
    $$
    \left(\alpha\frac{H_{t+1}}{K_{t+1}}-(1-\alpha)\right)\left(\frac{H_{t+1}}{K_{t+1}}\right)^{-\alpha}=\delta_K-\delta_H
    $$
    hence $H_{t}/K_{t}$ must be constant \textbf{on or off BGP} and determined by parameters $\alpha,\delta_K,\delta_H$
    $$
    \frac{H_{t+1}}{K_{t+1}} = f(\alpha,\delta_K,\delta_H)=x_{t+1}=\bar{x}
    $$
    Notice that $$g(\bar{x})=(\alpha\bar{x}-(1-\alpha))(\bar{x})^{-\alpha}=\delta_K-\delta_H$$
    must have a unique solution since $g(\bar{x})$ is continuous and strictly increasing in $\bar{x}$, $\lim_{\bar{x}\rightarrow x}=-\infty,\lim_{\bar{x}\rightarrow\infty}=\infty$. This leads to 
    $$H_{t+1}/H_t=K_{t+1}/K_t$$
    i.e., physical and human capital growth rates are \textbf{equal}
    again, the production function is reduced to AK:
    $$
    Y_t = K_t^\alpha H_t^{1-\alpha} = K_t^{\alpha} \left(\bar{x}K_t\right)^{1-\alpha} = \bar{x}^{1-\alpha}K_t = AK_t
    $$
    \end{minipage}
};
\node[redtitle, right=4pt] at (box.north west) {Social planner's problem};
\end{tikzpicture}

And again, the Euler equation is 
$$
\frac{c_{t+1}}{c_t} = \left[\alpha \left(\bar{x}\right)^{1-\alpha}+ (1-\delta_K)\right]^{\frac{1}{\sigma}} = \left[(1-\alpha) \left(\bar{x}\right)^{-\alpha}+ (1-\delta_H)\right]^{\frac{1}{\sigma}}
$$
now, check GBP growth rate. Output, physical capital and human capital are already proven to grow at the same rate, now check consumption and physical capital. Rewrite the budget constraint
\begin{align*}
    & Y_t = c_t + \left(K_{t+1}-(1-\delta_K)K_t\right) + \left(H_{t+1}-(1-\delta_H)H_t\right)\\
    \xRightarrow[Y_t=AK_t]{\frac{K_t}{H_t}=\bar{x}} & A=\frac{c_t}{K_t}+\frac{K_{t+1}}{K_t}-(1-\delta_K) + \frac{\bar{x}K_{t+1}}{K_t}-(1-\delta_H)\bar{x}\\
    \Rightarrow & \frac{c_t}{K_t} + \frac{(1+\bar{x})K_{t+1}}{K_t} = A+ (1-\delta_K)+(1-\delta_H)\bar{x}
\end{align*}
this shows that $c_t$ and $K_t$ grows at the same rate as well. Hence the BGP growth rate is
\begin{align*}
    \gamma &= g_c =g_K=g_H=g_Y\\
    & =\left[\alpha \left(\bar{x}\right)^{1-\alpha}+ (1-\delta_K)\right]^{\frac{1}{\sigma}} = \left[(1-\alpha) \left(\bar{x}\right)^{-\alpha}+ (1-\delta_H)\right]^{\frac{1}{\sigma}}
\end{align*}

\rule{0.325\textwidth}{0.4pt}

\vspace{2pt}

\textbf{Pros}: Labor is treated seriously, not relying on externalities that are not empirically observed.

\textbf{Cons}: A lot of strong assumptions are imposed on human capital accumulation.

\rule{0.325\textwidth}{0.4pt}

%%%%%%%%%% lucas two-sector
\subsection*{Lucas two-sector model}
\vspace{2pt}
\begin{tikzpicture}
\node [redbox] (box){%
    \begin{minipage}{0.315\textwidth}
    \color{myred}
    \scriptsize
$$\max \sum^{\infty}_{t=0}\beta^t\frac{c_t^{1-\sigma}-1}{1-\sigma}$$
s.t.
\begin{align*}
    c_t+K_{t+1}-(1-\delta_K)K_t& = K_t^{\alpha}\left(\phi_t H_t \right)^{1-\alpha} & {\tiny\color{black}{\scriptscriptstyle\phi_tH_t}\textbf{ for output}} \\
    H_{t+1}-H_{t}&=A(1-\phi_t)H_t & {\tiny\color{black}{\scriptscriptstyle(1-\phi_t)H_t} \textbf{ for accumulation}}\\
    c_t, \phi_t, H_{t+1},K_{t+1} & \geq 0
\end{align*}
    \end{minipage}
};
\node[redtitle, right=4pt] at (box.north west) {Social planner's problem};
\end{tikzpicture}

\vspace{2pt}
The \textbf{Lagrangean} (choosing ${\color{myred}\left\{c_t,\phi_t,H_{t+1},K_{t+1}\right\}}$) is
\begin{align*}
    \mathcal{L} = &\sum^{\infty}_{t=0}\beta^t\frac{c_t^{1-\sigma}-1}{1-\sigma} \\
    &+\mu_{K,t}\left[ K_t^{\alpha}\left(\phi_t H_t \right)^{1-\alpha}-\left(c_t+K_{t+1}-\left(1-\delta_K\right)K_t\right) \right]\\
    & + \mu_{H,t}\left[A\left(1-\phi_t\right)H_t-\left(H_{t+1}-H_t\right) \right]
\end{align*}

We have two sets of \textbf{FOC}:

\begin{itemize}
    \item[-] w.r.t. control variables $\{c_t,\phi_t\}$:
    \begin{align*}
        \frac{\partial \mathcal{L}}{\partial c_t}=0 \Rightarrow & \beta^t c_t^{-\sigma}=\mu_{K,t}\\
        \frac{\partial \mathcal{L}}{\partial \phi_t}=0 \Rightarrow& (1-\alpha)\left(\frac{\phi_t H_t}{K_t}\right)^{-\alpha}\mu_{K,t}=A\mu_{H,t}
    \end{align*}
    
    \item[-] w.r.t. state variables $\{K_{t+1},H_{t+1}\}$:
    \begin{align*}
        \frac{\partial \mathcal{L}}{\partial K_{t+1}} =0\Rightarrow  \mu_{K,t}=&\mu_{K,t+1}\left[\alpha\left( \frac{\phi_{t+1}H_{t+1}}{K_{t+1}}\right)^{1-\alpha} + (1-\delta_K)  \right]\\
        \frac{\partial \mathcal{L}}{\partial H_{t+1}}=0\Rightarrow \mu_{H,t}=&\mu_{H,t+1}\left[A(1-\phi_{t+1})+1\right]+\\
         &\mu_{K,t+1}(1-\alpha)\phi_{t+1}\left(\frac{\phi_{t+1}H_{t+1}}{K_{t+1}}\right)^{-\alpha}\\
    \end{align*}
\end{itemize}

To solve this system of equations, first plug $(1-\alpha)\left(\frac{\phi_t H_t}{K_t}\right)^{-\alpha}\mu_{K,t}=A\mu_{H,t}$ into the second set of FOCs, to get the relation between $\mu_{K,t}$ and $\mu_{K,t+1}$:
\begin{align*}
    &(1-\alpha)\left(\frac{\phi_t H_t}{K_t}\right)^{-\alpha}\mu_{K,t}\\
    = & A\left\{ \mu_{H,t+1}\left[A(1-\phi_{t+1})+1\right]+\mu_{K,t+1}(1-\alpha)\phi_{t+1}\left(\frac{\phi_{t+1}H_{t+1}}{K_{t+1}}\right)^{-\alpha} \right\} \\
    = & (1-\alpha)\left(\frac{\phi_{t+1}H_{t+1}}{K_{t+1}}\right)^{-\alpha}\mu_{K,t+1}\left[A(1-\phi_{t+1})+1\right]\\
    &+ A \mu_{K,t+1}(1-\alpha)\phi_{t+1}\left(\frac{\phi_{t+1}H_{t+1}}{K_{t+1}}\right)^{-\alpha}\\
    \Rightarrow & \left(\frac{\phi_t H_t}{K_t}\right)^{-\alpha}\mu_{K,t} = \mu_{K,t+1}\left(\frac{\phi_{t+1}H_{t+1}}{K_{t+1}}\right)^{-\alpha}(1+A)
\end{align*}

Plug this equation back to the capital accumulation FOC ($\partial \mathcal{L}/\partial K_{t+1}=0$), ${\color{myred}\mu_{K,t}=\mu_{K,t+1}\left[\alpha\left( \frac{\phi_{t+1}H_{t+1}}{K_{t+1}}\right)^{1-\alpha} + (1-\delta_K)  \right]}$, get: 
\begin{align*}
    &\left(\frac{(\phi_{t+1}/\phi_t)(H_{t+1}/H_t)}{(K_{t+1}/K_t)}\right)^{-\alpha}(1+A) = \left[\alpha\left( \frac{\phi_{t+1}H_{t+1}}{K_{t+1}}\right)^{1-\alpha} + (1-\delta_K)  \right]\\
    \Rightarrow & {\color{myred}\left(\frac{(\phi_{t+1}/\phi_t)(H_{t+1}/H_t)}{(K_{t+1}/K_t)}\right)^{-\alpha}=\frac{\left[\alpha\left( \frac{\phi_{t+1}H_{t+1}}{K_{t+1}}\right)^{1-\alpha} + (1-\delta_K)  \right]}{1+A}}
\end{align*}

Combine this with FOC w.r.t. $c_t$, ${\color{myred}\mu_{K,t}=\beta^t c_t^{-\sigma}}$, get the Euler equation:

\vspace{2pt}
\begin{tikzpicture}
\node [iredbox] (box){%
    \begin{minipage}{0.315\textwidth}
    \color{white}
    \scriptsize
    The Euler equation of this problem is
    $$
    \left(\frac{\phi_t H_t}{K_t}\right)^{-\alpha}c_t^{-\sigma} = \beta c_{t+1}^{-\sigma}\left(\frac{\phi_{t+1}H_{t+1}}{K_{t+1}}\right)^{-\alpha}(1+A)
    $$
    This equation gives the expression of the consumption growth rate:
\begin{align*}
    &\left(\frac{\phi_t H_t}{K_t}\right)^{-\alpha}c_t^{-\sigma} = \beta c_{t+1}^{-\sigma}\left(\frac{\phi_{t+1}H_{t+1}}{K_{t+1}}\right)^{-\alpha}(1+A)\\
    \Rightarrow & g_c=\frac{c_{t+1}}{c_t}=\left\{\beta \cdot \frac{\left[\alpha\left( \frac{\phi_{t+1}H_{t+1}}{K_{t+1}}\right)^{1-\alpha} + (1-\delta_K)  \right]}{1+A}\cdot(1+A) \right\}^{1/\sigma}\\
    \Rightarrow & g_c = \left\{\beta \cdot \left[\alpha\left( \frac{\phi_{t+1}H_{t+1}}{K_{t+1}}\right)^{1-\alpha} + (1-\delta_K)  \right] \right\}^{1/\sigma}
\end{align*}
    \end{minipage}
};
\node[iredtitle, right=4pt] at (box.north west) {Euler equation};
\end{tikzpicture}

\vspace{2pt}
For a \textbf{balanced growth path} to satisfy this equation, that is, $g_c$ being constant, $H_{t+1}$ and $K_{t+1}$ must have the same growth rate, and this growth rate is also the growth rate of output ($\phi_t$ constant): 
$$ g_K = g_H\Rightarrow g_y =g_K^{\alpha}g_H^{1-\alpha}=g_K=g_H$$

How do we know this? By looking at the resource constraint:
\begin{align*}
    & c_t+K_{t+1}-(1-\delta_K)K_t =Y_t\\
    \xRightarrow{\div K_t} &  \frac{c_t}{K_t}+\frac{K_{t+1}}{K_t}=\frac{Y_t}{K_t}+1-\delta_K
\end{align*}
where $K_{t+1}/K_t=g_K =A(1-\phi)+1 $ is constant, $Y_t/K_t$ is also constant since $Y_{t+1}/Y_t=K_{t+1}/K_t$, hence $c_t/K_t$ must also be constant, therefore, $c_{t+1}/c_{t}=K_{t+1}/K_t$.

Next, calculate the growth rates:

\vspace{2pt}
\begin{tikzpicture}
\node [redbox] (box){%
    \begin{minipage}{0.315\textwidth}
    \color{myred}
    \scriptsize
    On BGP, the growth rate of all variables are constant:
    \begin{itemize}
        \item[-] $g_H$: growth rate of human capital
        $$g_H = \frac{H_{t+1}}{H_t}=A\left(1-\phi_t\right)+1 \xRightarrow{\text{constant }g_H} g_H=A(1-\phi)+1 $$
        \item[-] $g_K$: growth rate of physical capital and output:
        $$g_y = g_K = g_H = A(1-\phi)+1$$ 
        \item[-] $g_c$: growth rate of consumption
        
        By Euler equation: 
    $$
    \left(\frac{\phi_t H_t}{K_t}\right)^{-\alpha}c_t^{-\sigma} = \beta c_{t+1}^{-\sigma}\left(\frac{\phi_{t+1}H_{t+1}}{K_{t+1}}\right)^{-\alpha}(1+A)
    $$
    the consumption growth rate is
    \begin{align*}
        g_c & =\frac{c_{t+1}}{c_t}=\left\{\beta \cdot \left(\frac{\frac{\phi_{t+1}}{\phi_t}\frac{H_{t+1}}{H_t}}{\frac{K_{t+1}}{K_t}}\right)^{-\alpha}\cdot(1+A) \right\}^{1/\sigma}\\
        \xRightarrow[\text{constant }\phi]{g_H=g_K} g_c & =\left(\beta(1+A) \right)^{1/\sigma} = g_H=g_K=A(1-\phi)+1
    \end{align*}
    \end{itemize}
    
    In return, $\phi = 1-\frac{\left(\beta(1+A) \right)^{1/\sigma}-1}{A}$ can be used to determine the optimal, BGP allocation of human capital to human capital accumulation $\phi$
    \end{minipage}
};
\node[redtitle, right=4pt] at (box.north west) {Growth rate on balanced growth path (BGP)};
\end{tikzpicture}

\rule{0.325\textwidth}{0.4pt}

\vspace{2pt}

\textbf{Pros}: human capital is treated different from physical capital; there are transitional dynamics in this model, BGP growth rate is determined by preferences \textbf{and $A$}, the productivity of human capital in its own accumulation.

\textbf{Cons}: Human capital can not be observed, it needs to be proxied empirically. Assumptions on human capital are strong: rivalrous and excludable, decreasing returns to its accumulation.

\rule{0.325\textwidth}{0.4pt}

%%%%%%%%%%%%%%%%%%%%%%%%%%%%%%%%%%%%%%%%%%%%%%%%
%Romer's endogeneous imperfect competitive market

\section*{Romer's endogenous growth model}
\subsection*{Love of variety: Dixit and Stiglitz (1977)}
The \textbf{love of variety} (here in final goods production) rationalizes the growth of trade in similar intermediate inputs among similar countries: 

\begin{itemize}
    \item[-]\textbf{{\color{myblue}demand side}}: different varieties of the same product are produced in each country and very similar varieties are traded among countries because consumers or firms value variety.
    \item[-]\textbf{{\color{myblue}supply side}}: international trade in varieties of the same intermediate good is due to a search for larger markets by firms facing increasing returns to scale/falling average costs.
\end{itemize}

\vspace{2pt}
\begin{tikzpicture}
\node [bluebox] (box){%
    \begin{minipage}{0.315\textwidth}
    \color{myblue}
    \scriptsize
    Consumers value diversity, a mass of varieties of consumer goods, branded by $i$:
    $$
    U=\left(\int^n_0 q(i)^{1-\frac{1}{\sigma}}\mathrm{d}i\right)^{\frac{\sigma}{\sigma-1}}
    $$
    here $i$ is the index, $q(i)$ is quantity of demand, $\sigma$ is the elasticity of substitution among products, assumed to be $\sigma > 1$.
    
    A consumer's problem is then:
    $$
    \max U \text{ s.t. }\int^n_0q(i)p(i)\mathrm{d}i\leq I
    $$
    where $I$ is the income, $p(i)$ is the $i$th product's price.
    \end{minipage}
};
\node[bluetitle, right=4pt] at (box.north west) {Love of variety: model setting};
\end{tikzpicture}

\vspace{2pt}
To solve this model, write Lagrange:
$$
\mathcal{L} = \left(\int^n_0 q(i)^{1-\frac{1}{\sigma}}\mathrm{d}i\right)^{\frac{\sigma}{\sigma-1}}+\mu\left( I-\int^n_0q(i)p(i)\mathrm{d}i\right)
$$
FOC w.r.t. $q(i)$ is:
\begin{align*}
    \frac{\partial\mathcal{L}}{\partial q(i)}=0\Rightarrow & \frac{\sigma}{\sigma-1} \left(\int^n_0 q(i)^{1-\frac{1}{\sigma}}\mathrm{d}i\right)^{\frac{1}{\sigma-1}}\cdot\frac{\sigma-1}{\sigma}q(i)^{-\frac{1}{\sigma}}=\mu p(i)\\
    \Rightarrow & U^{\frac{1}{\sigma}}q(i)^{-\frac{1}{\sigma}}=\mu p(i)
\end{align*}

Two objects can be derived from the FOC:
\begin{itemize}
    \item[-] demand for variety zero $q(0)$: 
    
    for $i$th and $j$th good, the relative demands $q(i)/q(j)$ is:
$$ \frac{q(i)}{q(j)} = \left( \frac{p(i)}{p(j)} \right)^{-\sigma} $$

    Then if we take an arbitrary good, good 0, as reference, the demand for every variety $i$ can be written as:
    $$
    q(i) = q(0)\left(\frac{p(i)}{p(0)} \right)^{-\sigma}
    $$
    plug this expression of $q(i)$ back into the budget constraint, get
    \begin{align*}
        & I=\int^n_0 q(0)p(i)\left(\frac{p(i)}{p(0)} \right)^{-\sigma} \mathrm{d}i 
    =q(0)p(0)^{\sigma} \int^n_0 p(i)^{1-\sigma}\mathrm{d}i\\
    \Rightarrow & q(0)=\frac{I}{p(0)^{\sigma} \int^n_0 p(i)^{1-\sigma}\mathrm{d}i}\ {\color{myblue}\cdots \tiny{\text{ demand for variety zero}}}
    \end{align*}

    \item[-] price index for varieties $P$:
    
    Transform FOC, we can get:
    \begin{align*}
        &U^{\frac{1}{\sigma}}q(i)^{-\frac{1}{\sigma}}=\mu p(i)\ {\color{myblue}\cdots \tiny{\text{ FOC w.r.t. }q(i)}}\\
        \xRightarrow{{\color{myblue}(\cdot)^{1-\sigma}}}& U^{\frac{1-\sigma}{\sigma}}q(i)^{-\frac{1-\sigma}{\sigma}}=\mu ^{1-\sigma}p(i)^{1-\sigma}\\
        \xRightarrow{{\color{myblue}\int\mathrm{d}i}} & U^{\frac{1-\sigma}{\sigma}} \int^n_0 q(i)^{-\frac{1-\sigma}{\sigma}}\mathrm{d}i=\mu ^{1-\sigma} \int^n_0 p(i)^{1-\sigma}\mathrm{d}i
    \end{align*}
    and we know $U=\left(\int^n_0 q(i)^{\frac{\sigma-1}{\sigma}}\mathrm{d}i\right)^{\frac{\sigma}{\sigma-1}}$, hence
    $$
    \mu^{1-\sigma}\int^n_0 p(i)^{1-\sigma}\mathrm{d}i=1\Rightarrow \frac{1}{\mu}=P=\left( \int^n_0 p(i)^{1-\sigma}\mathrm{d}i \right)^{\frac{1}{1-\sigma}}
    $$
    here, $\mu$ can be interpreted as {\color{myblue}\textbf{the shadow/utility value}} of an additional unit of expenditure on the {\color{myblue}\textbf{consumption index}}; $\frac{1}{\mu}$ can be interpreted as the {\color{myblue}\textbf{expenditure}} required for one unit of the {\color{myblue}\textbf{consumption index}}.
\end{itemize}

Now with $q(0)=\frac{I}{p(0)^{\sigma} \int^n_0 p(i)^{1-\sigma}\mathrm{d}i}$ and $P=\left( \int^n_0 p(i)^{1-\sigma}\mathrm{d}i \right)^{\frac{1}{1-\sigma}}$, we have demand for good 0:
$$
q(0)=\frac{I}{p(0)^{\sigma}P^{1-\sigma}}=\frac{I}{P}\left( \frac{p(0)}{P} \right)^{-\sigma}
$$
and since $q(i) = q(0)\left(\frac{p(i)}{p(0)} \right)^{-\sigma}$, this will give a nice result:

\vspace{2pt}
\begin{tikzpicture}
\node [bluebox] (box){%
    \begin{minipage}{0.315\textwidth}
    \color{myblue}
    \scriptsize
    The demand for good $i$ is:
    $$
    q(i)=\frac{I}{P}\left( \frac{p(0)}{P} \right)^{-\sigma}\cdot \left(\frac{p(i)}{p(0)} \right)^{-\sigma}= \frac{I}{P}\left( \frac{p(i)}{P} \right)^{-\sigma}
    $$
    it is clear that $\frac{\partial q(i)}{\partial p(i)}=\frac{I}{P^{1-\sigma}}(-\sigma)p(i)^{-\sigma-1}<0$: each variety has a \textbf{downward sloping} demand function.
    
    \rule{\textwidth}{0.4pt}
    \vspace{1pt}
    
    put the demand function back to utility function, get consumers \textbf{indirect utility}:
    \begin{align*}
       U & =\left(\int^n_0 q(i)^{1-\frac{1}{\sigma}}\mathrm{d}i\right)^{\frac{\sigma}{\sigma-1}} = \left(\int^n_0 \left( \frac{I}{P}\left( \frac{p(i)}{P} \right)^{-\sigma} \right)^{1-\frac{1}{\sigma}}\mathrm{d}i\right)^{\frac{\sigma}{\sigma-1}}\\
       & = \frac{I}{P}\left(\int^n_0 \left( \frac{p(i)}{P} \right)^{1-\sigma}\mathrm{d}i\right)^{\frac{\sigma}{\sigma-1}}=\frac{I}{P}\cdot P^\sigma \cdot \left(\int^n_0 p(i)^{1-\sigma}\mathrm{d}i\right)^{\frac{\sigma}{\sigma-1}}
    \end{align*}
    since $P=\left( \int^n_0 p(i)^{1-\sigma}\mathrm{d}i \right)^{\frac{1}{1-\sigma}}$, the indirect utility is
    $$
    U = \frac{I}{P}P^{\sigma}P^{-\sigma} =\frac{I}{P}
    $$
    
    If all goods have the same price, all goods are consumed in equal amounts i.e. $q(i)=q(j)= q,\forall i,j$. The by the \textbf{symmetry} of the market, the budget constraint can be rewritten as $I=\int ^n_0 p(i)q(i)\mathrm{d}i=nqp$, which gives the quality consumed for each variety:
    $$ q=\frac{I}{np} $$
    and the utility is 
    \begin{align*}
        U & =\left(\int^n_0 q(i)^{1-\frac{1}{\sigma}}\mathrm{d}i\right)^{\frac{\sigma}{\sigma-1}} = \left(\int^n_0 \left(\frac{I}{np}\right)^{1-\frac{1}{\sigma}}\mathrm{d}i\right)^{\frac{\sigma}{\sigma-1}}\\
        & = \left[\left(\frac{I}{np}\right)^{\frac{\sigma-1}{\sigma}}\cdot n\right]^{\frac{\sigma}{\sigma-1}} = n^{\frac{1}{\sigma-1}}\frac{I}{P}
    \end{align*}
    here, $\partial U/\partial n = \frac{1}{\sigma-1}n^{\frac{-\sigma}{\sigma-1}}\cdot \frac{I}{P}>0$, the love for variety is there.
    \end{minipage}
};
\node[bluetitle, right=4pt] at (box.north west) {Love of variety: demand and indirect utility};
\end{tikzpicture}

\textbf{The role of $\sigma$}: $\sigma$ is the willingness to substitute among varieties, when $n\geq 1$ (at least one type) and $\sigma>1$ (assumed) higher $\sigma$ leads to lower $\partial U/\partial n$, or a more \textit{mildly} love for variety, as shown in the following picture:

\begin{center}
\begin{tikzpicture}
\draw[->] (-0.2, 0) -- (4, 0) node[below right] {$n$};
\draw[->] (0, -0.2) -- (0, 4) node[above] {$U$};
\draw[scale=0.5, domain=1:2.7, smooth, variable=\x, black] plot ({\x}, {(\x)^2})
node[right] {$\sigma=1.5$};
\draw[scale=0.5, domain=1:7, smooth, variable=\x, black] plot ({\x}, {\x})
node[above] {$\sigma=2$};
\draw[scale=0.5, domain=1:8, smooth, variable=\x, black] plot ({\x}, {(\x)^(2/3)})
node[above] {$\sigma=2.5$};
\draw[scale=0.5, domain=1:8, smooth, variable=\x, black] plot ({\x}, {(\x)^(1/2)})
node[above] {$\sigma=3$};
\draw[scale=0.5, domain=1:8, smooth, variable=\x, black] plot ({\x}, {(\x)^(2/5)})
node[below] {$\sigma=3.5$};
\draw[scale=0.5, domain=1:8, smooth, variable=\x, black] plot ({\x}, {1})
node[below] {$\sigma=\infty$};
\end{tikzpicture}
\end{center}

But the 1st order increasing, i.e., the love for variety, persists.

Now we analysis the cost and prices in this model:
\vspace{2pt}

\vspace{2pt}
\begin{tikzpicture}
\node [bluebox] (box){%
    \begin{minipage}{0.315\textwidth}
    \color{myblue}
    \scriptsize
    Assume the profit is
    $$ \pi(i)=p(i)q(i)- wcq(i)-c_f $$
    where
    \begin{itemize}
        \item[-] $w$: wage of one unit of labor
        \item[-] $c$: number of units of labor required to produce one unit of output
        \item[-] $c_f$: fixed labor cost
    \end{itemize}
    
    Since each firm produces a differentiated (but very similar) good, they have a monopoly power (\textbf{monopolistic} competition), they can either be \textbf{price-setting} or \textbf{quantity-setting}. Here, assume price setting, the FOC is:
    $$
    \frac{\partial \pi(i)}{p(i)}=q(i)+p(i)\frac{\partial q(i)}{\partial p(i)}-wc\frac{\partial q(i)}{\partial p(i)}=0
    \Rightarrow p(i)=wc - \frac{q(i)}{\partial q(i)/\partial p(i)}$$
    here, $\partial q(i)/\partial p(i)<0$ is a market-up over marginal cost, i.e., the price is set to be higher than the perfect competition price (marginal cost).
    
    
    \rule{\textwidth}{0.4pt}
    \vspace{1pt}
    
    From the demand side, demand for good $i$ is 
    $$
    q(i)=\frac{I}{P}\left(\frac{p(i)}{P}\right)^{-\sigma}
    $$
    which gives
    $$
    \frac{\partial q(i)}{\partial p(i)}=-\sigma \frac{I}{P}\left(\frac{p(i)}{P}\right)^{-\sigma-1}\frac{1}{P}
    $$
    leading to 
    $$
    \frac{q(i)}{\partial q(i)/\partial p(i)} = \frac{1}{-\sigma \left(\frac{p(i)}{P}\right)^{-1}\frac{1}{P}}=-\frac{p(i)}{\sigma}<0
    $$
    plug this back in the pricing function, get:
    $$
    p(i)=wc-\frac{q(i)}{\partial q(i)/\partial p(i)}=wc+\frac{p(i)}{\sigma}\Rightarrow p(i)=\frac{wc\sigma}{\sigma-1}
    $$
    Hence, the optimal pricing strategy is a \textbf{proportional markup} over marginal cost.
    
    \end{minipage}
};
\node[bluetitle, right=4pt] at (box.north west) {Love of variety: supply profit, cost, prices};
\end{tikzpicture}

A typical variant of this model, production of \textbf{each variety} is free-entry, i.e., for each variety $i$, there is a level of $q(i)$ leading to zero-profit, that is 
$$
\pi(i)=p(i)q(i)- wcq(i)-c_f=0\Rightarrow \left(\frac{\sigma}{\sigma-1}-1\right) q(i) = c_f\Rightarrow q(i)=\sigma c_f
$$

\subsection*{Romer's 3 sector model}
To incorporate the imperfect competition in real economy, Romer model constructed a \textbf{3-technology/sector} model:

\vspace{2pt}
\begin{tikzpicture}
\node [redbox] (box){%
    \begin{minipage}{0.315\textwidth}
    \color{myred}
    \scriptsize
    \begin{itemize}
        \item[1.] Final good sector: \textbf{\underline{perfect}} competition
        \item[2.] Variety/intermediate sector: \textbf{\underline{monopolistic}} competition
        \item[3.] R\&D sector: \textbf{\underline{perfect}} competition.
    \end{itemize}

    \end{minipage}
};
\node[redtitle, right=4pt] at (box.north west) {Market structure: 3 sectors};
\end{tikzpicture}

\begin{itemize}
    \item[-] \underline{\textbf{Final goods market}}:
    \begin{itemize}
        \item[-] production: constant-return-to-scale technology with intermediate goods and labor:
        $$
        Y_t = L_{1,t}^{1-\alpha}\int^{A_t}_0 x_t^{\alpha}(i)\mathrm{d}i
        $$
        where
        \begin{itemize}
            \item[-] $A_t$ is the measure of varieties, final goods production take $A_t$ as exogenously given, $Y_t \propto A_t$.
            \item[-] labor $L_{1,t}$ is purchased from HHs, assume to be a fraction of the total labor time $L$ (assumed to be constant): 
            $$L_{1,t}=\phi_t L$$
            \item[-] intermediate goods $x_t(i)$ are from variety sector
        \end{itemize}
        
        \item[-] Perfectly competitive, \textbf{zero profit}.
    \end{itemize}
    
    \item[-] \underline{\textbf{Intermediate goods market}}: 
    \begin{itemize}
        \item[-] production: linear in raw physical capital, requires a \textbf{fixed cost} (R\&D patent):
        $$
        \int^{A_t}_0 x_t \mathrm{d}i = K_t
        $$
        where
        \begin{itemize}
            \item[-] $x_t(i)=K_t(i)$: one unit of $K_t$ produces one unit of $x_t(i)$, \underline{$\forall i$}
            \item[-] capital is purchased from HHs, the accumulation follows: 
            $$K_{t+1}=(1-\delta)K_t+i_t$$
            
        \end{itemize}
        
        \item[-] Monopolistically competitive, with pricing power, prices \textbf{above marginal cost}, makes \textbf{positive profits} to cover the fixed cost.
    \end{itemize}
    
    \item[-] \underline{\textbf{Patent market}}:
    \begin{itemize}
        \item[-] production: using labor and \textbf{existing ideas}, generating more varieties:
        $$
        A_{t+1}-A_t = \kappa (1-\phi_t)L A_t
        $$
        where
        \begin{itemize}
            \item[-] $L_{2,t}=(1-\phi_t)L$: labor used to produce new varieties
            \item[-] $A_t$ are the existing ideas: non-rivalry, hence \textbf{externality}, representing ''standing on shoulders'' effect
            
        \end{itemize}
        
        \item[-] Perfectly competitive, free entry and \textbf{zero profit}.
    \end{itemize}
    
\end{itemize}

\vspace{2pt}
This model has:
\begin{itemize}
    \item[-] appeals: incorporating the imperfect competition and externality of ideas

    \item[-] shortfalls: the patent production assumption doesn't reflect empirical observations:
    $$ A_{t+1} = A_t+\kappa (1-\phi_t)L A_t\Rightarrow \frac{A_{t+1}}{A_t}= \kappa (1-\phi_t)L+1$$
    the growth rate of new ideas is linearly increasing in labor, which doesn't reflect
    \begin{itemize}
        \item[-] fishing-out effect: the easiest-to-discover ideas have already been discovered
        \item[-] stepping-on-toes effect: a bigger workforce does NOT simply imply more ideas
    \end{itemize}
\end{itemize}

\subsubsection*{Solve social planner's problem}
Next, solve social planner's problem of this model in 3 steps:

\vspace{2pt}
\underline{\textbf{Step 1}}: solve the \textbf{\color{myred}\underline{static choice of production inputs}}. The problem is 
$$\max L_{1,t}^{1-\alpha}\int^{A_t}_0 x_t^{\alpha}(i)\mathrm{d}i
$$
s.t.
$$
\int^{A_t}_0 x_t(i)\mathrm{d}i=K_t,\ L_{1,t}=\phi_tL
$$
and the solution is symmetric: $x_t(i)=x_t,\forall i$, that is, there is a preference for variety, the more the better, all varieties are \textbf{equally costly} in terms of capital, hence \textbf{equally valued}:
$$
\int^{A_t}_0 x_t\mathrm{d}i=K_t\Rightarrow x_t = \frac{K_t}{A_t}
$$
plug this back into the production inputs, get

\vspace{2pt}
\begin{tikzpicture}
\node [redbox] (box){%
    \begin{minipage}{0.315\textwidth}
    \color{myred}
    \scriptsize
    $$
    Y_t = L_{1,t}^{1-\alpha}\int^{A_t}_0 x_t^{\alpha}(i)\mathrm{d}i = \left(\phi_t L\right)^{1-\alpha} A_t\left(\frac{K_t}{A_t} \right)^{\alpha}=K_t^{\alpha}\left(\phi_t L A_t\right)^{1-\alpha}
    $$
    this function is still CRS, with $A_t$ given exogenously.
    
    \end{minipage}
};
\node[redtitle, right=4pt] at (box.north west) {Solve social planner's problem: Step 1 result};
\end{tikzpicture}

\vspace{2pt}
\underline{\textbf{Step 2}}, solve the familiar \textbf{\color{myred}\underline{intertemporal planning problem}}.
$$\max\sum^{\infty}_{t=0}\beta^t \frac{c_t^{1-\sigma}-1}{1-\sigma}
$$
s.t.
\begin{align*}
    c_t+K_{t+1}-(1-\delta)K_t &= K_t^{\alpha}\left(\phi_t L A_t\right)^{1-\alpha}\\
    A_{t+1}-A_t &= \kappa(1-\phi_t)LA_t
\end{align*}
Lagrange:
\begin{align*}
    \mathcal{L}=&\sum^{\infty}_{t=0}\beta^t\frac{c_t^{1-\sigma}-1}{1-\sigma} +\lambda_{A,t}\left[\kappa(1-\phi_t)LA_t-\left(A_{t+1}-A_t\right)\right]\\
    &+ \lambda_{K,t}\left[K_t^{\alpha}\left(\phi_t L A_t\right)^{1-\alpha}-\left(c_t+K_{t+1}-(1-\delta)K_t\right)\right]
\end{align*}
FOC gives
\begin{align*}
    \frac{\partial \mathcal{L}}{\partial c_t}=0 \Rightarrow & \beta^t c_t^{-\sigma}=\lambda_{K,t}\\
    \frac{\partial \mathcal{L}}{\partial \phi_t} =0\Rightarrow & \lambda_{A,t}\kappa LA_t = \lambda_{K,t}K_t^{\alpha}(1-\alpha)\left(\phi_tLA_t\right)^{-\alpha}LA_t \\
    \frac{\partial \mathcal{L}}{\partial K_{t+1}}=0\Rightarrow & \lambda_{K,t+1}\left[\alpha\left(\frac{\phi_{t+1}LA_{t+1}}{K_{t+1}}\right)^{1-\alpha}+(1-\delta)\right]=\lambda_{K,t}\\
    \frac{\partial \mathcal{L}}{\partial A_{t+1}}=0\Rightarrow & \lambda_{A,t+1}\left[\kappa(1-\phi_{t+1})L+1 \right] +\lambda_{K,t+1}(1-\alpha)\left(\frac{K_{t+1}}{\phi_{t+1}L A_{t+1}}\right)^{\alpha}L\phi_{t+1}=\lambda_{A,t}
\end{align*}

\vspace{2pt}
\begin{tikzpicture}
\node [redbox] (box){%
    \begin{minipage}{0.315\textwidth}
    \color{myred}
    \scriptsize
    Euler equation:
    $$
    \left(\frac{c_{t+1}}{c_t}\right)^{\sigma}=\beta\left[\alpha\left(\frac{\phi_{t+1}LA_{t+1}}{K_{t+1}}\right)^{1-\alpha}+(1-\delta)\right]
    $$
    
    \end{minipage}
};
\node[redtitle, right=4pt] at (box.north west) {Solve social planner's problem: Step 2 result};
\end{tikzpicture}

\vspace{2pt}
\underline{\textbf{Step 3}}: \textbf{\color{myred}\underline{Balanced growth}} path analysis.
From the Euler equation, we have the consumption growth rate:
$$
g_c \equiv \frac{c_{t+1}}{c_t}= \left\{\beta\left[\alpha\left(\frac{\phi_{t+1}LA_{t+1}}{K_{t+1}}\right)^{1-\alpha}+(1-\delta)\right]\right\}^{\frac{1}{\sigma}}
$$

From $\frac{\partial \mathcal{L}}{\partial \phi_t}=0$, get:
$$
    \lambda_{A,t}=\lambda_{K,t}\cdot \frac{1-\alpha}{\kappa}\left(\frac{K_t}{\phi_tLA_t}\right)^{\alpha}
$$
plug into the condition of $\frac{\partial \mathcal{L}}{\partial A_{t+1}}$, get:
\begin{align*}
    \lambda_{A,t}= & \lambda_{A,t+1}\left[\kappa(1-\phi_{t+1})L+1 \right]\\
    & +\lambda_{K,t+1}(1-\alpha)\left(\frac{K_{t+1}}{\phi_{t+1}L A_{t+1}}\right)^{\alpha}L\phi_{t+1}\\
    \Rightarrow & \lambda_{K,t}\cdot \frac{1-\alpha}{\kappa}\left(\frac{K_t}{\phi_tLA_t}\right)^{\alpha} \\
    =& \lambda_{K,t+1}\cdot \frac{1-\alpha}{\kappa}\left(\frac{K_{t+1}}{\phi_{t+1}LA_{t+1}}\right)^{\alpha}\left[\kappa(1-\phi_{t+1})L+1 \right]+\lambda_{K,t+1}(1-\alpha)\left(\frac{K_{t+1}}{\phi_{t+1}L A_{t+1}}\right)^{\alpha}L\phi_{t+1}
\end{align*}
Here, on a balanced growth path, $\phi_t$ is constant, hence we have
\begin{align*}
    \lambda_{K,t} =& \lambda_{K,t+1}\left(\frac{K_{t+1}/K_t}{A_{t+1}/A_t}\right)^{\alpha}(L \kappa+1)
\end{align*}
on a balanced growth path, $g_k\equiv\frac{K_{t+1}}{K_t}=\frac{A_{t+1}}{A_t}\equiv g_a$, hence
$$
\lambda_{K,t}=\lambda_{K,t+1}(1+L\kappa)
$$
From the FOC w.r.t. $c_t$: $\beta^tc_t^{-\sigma}=\lambda_{K,t}$, we have
$$
g_c \equiv \frac{c_{t+1}}{c_t}= \left\{\beta\left[\alpha\left(\frac{\phi_{t+1}LA_{t+1}}{K_{t+1}}\right)^{1-\alpha}+(1-\delta)\right]\right\}^{\frac{1}{\sigma}} = (\beta(1+L\kappa))^{\frac{1}{\sigma}}
$$

From the law of motion of varieties $A_{t+1}-A_t = \kappa(1-\phi_t)LA_t$, we have on balanced growth path:
$$
\frac{A_{t+1}}{A_t}=1+(1-\phi)L\kappa =  (\beta(1+L\kappa))^{\frac{1}{\sigma}}
\Rightarrow 1- \phi = \frac{ (\beta(1+L\kappa))^{\frac{1}{\sigma}}-1}{L\kappa}$$

\vspace{2pt}
\begin{tikzpicture}
\node [redbox] (box){%
    \begin{minipage}{0.315\textwidth}
    \color{myred}
    
    The balanced growth rate is
    $$ \gamma = (\beta(1+L\kappa))^{\frac{1}{\sigma}} $$
    The optimal labor share in R\&D
    $$
    1-\phi^* = \frac{ (\beta(1+L\kappa))^{\frac{1}{\sigma}}-1}{L\kappa} = \frac{\gamma-1}{L\kappa}
    $$
\vspace{2pt}

same problem emerges: $\gamma\propto L$, i.e., the socially optimal balanced growth rate of the economy increases in the number of workers available in R\&D, which is at odd with empirical evidence.
    
    \end{minipage}
};
\node[redtitle, right=4pt] at (box.north west) {Solve social planner's problem: Step 3 result};
\end{tikzpicture}

\vspace{2pt}
\subsubsection*{Decentralize the model}
Next, we decentralize this model. NOTICE that there is monopolistic competition (in the intermediate good sector) and externality of ideas (in R\&D sector), hence the decentralized problem will NOT generate the same results as in the social planner's problem: the welfare theorems don't hold.

\vspace{2pt}
\textbf{\color{myblue}\underline{Firm's problem}}:
\begin{itemize}
    \item[-] \textbf{\color{myblue}final good} \textbf{producer}:
    There is perfect competition in the final goods market. Firms solve the profit maximization problem:
    $$
    \max_{x_t(i),L_{1,t}}\left\{ L_{1,t}^{1-\alpha}\int^{A_t}_0 x_t^{\alpha}(i)\mathrm{d}i-w_tL_{1,t}-\int^{A_t}_0 p_t(i)x_t(i)\mathrm{d}i \right\}
    $$
    where
    \begin{itemize}
        \item[-] $w_t$: wage of labor in the final goods production
        \item[-] $p_t(i)$: price of variety $i$
        \item[-] price of final goods is normalized to 1
    \end{itemize}
    solve this, get
\end{itemize}
\vspace{2pt}
\begin{tikzpicture}
\node [bluebox] (box){%
    \begin{minipage}{0.315\textwidth}
    \color{myblue}
    \scriptsize
    \begin{align*}
        w_t &= (1-\alpha)L_{1,t}^{-\alpha}\int^{A_t}_0 x_t^{\alpha}(i)\mathrm{d}i &\cdots\text{marginal product of }L_{1,t} \\
        p_t(i) &= \alpha L_{1,t}^{1-\alpha}(x_t(i))^{\alpha-1} &\cdots\text{demand curve of }x_t(i)\\
    \end{align*}
    these results will be used in intermediate goods producer's problem.
    \end{minipage}
};
\node[bluetitle, right=4pt] at (box.north west) {Firm's problem: final goods producer};
\end{tikzpicture}

\begin{itemize}
    \item[-] \textbf{\color{myblue}intermediate good} \textbf{producer}:
    There is monopolistic competition in the intermediate goods market. Firms set prices higher than marginal cost to cover the fix cost.
    
    Firms solve the profit maximization problem:
    $$
    \max_{x_t(i)}\left\{ p_t(i)x_t(i)-r_tK_t(i) \right\}
    $$
    s.t. $K_t(i)=x_t(i)$
    where
    \begin{itemize}
        \item[-] $K_t(i)$: $K_t(i)=x_t(i)$ is a one-to-one technology transforming raw capital to variety
        \item[-] $r_t(i)$: rent of one unit of capital
        \item[-] $p_t(i)$: the demand for variety $i$ given by the final good producers' FOC, $p_t(i)=\alpha L_{1,t}^{1-\alpha}(x_t(i))^{\alpha-1}$
    \end{itemize}
    
    rewrite the maximization as:
    $$
    \max_{x_t(i)}\left\{ \alpha L_{1,t}^{1-\alpha}(x_t(i))^{\alpha}-r_t x_t(i) \right\}
    $$
    solve it, get
    $$
    r_t = \alpha^2L_{1,t}^{1-\alpha}(x_t(i))^{\alpha-1}
    $$
    hence
\end{itemize}

\vspace{2pt}
\begin{tikzpicture}
\node [bluebox] (box){%
    \begin{minipage}{0.315\textwidth}
    \color{myblue}
    \scriptsize
    Variety production is symmetric $x_t(i)=x_t$ (production scale of all varieties are the same)
    $$
    r_t = \alpha^2L_{1,t}^{1-\alpha}(x_t)^{\alpha-1}\Rightarrow x_t = L_{1,t}\left(\frac{\alpha^2}{r_t} \right)^{\frac{1}{1-\alpha}}
    $$
    and the profit of all variety producers is the same:
    \begin{align*}
        \pi_t(i)=\pi_t &= \alpha L_{1,t}^{1-\alpha}(x_t)^{\alpha} - r_t x_t= \alpha L_{1,t}^{1-\alpha}(x_t)^{\alpha} - \alpha^2 L_{1,t}^{1-\alpha}(x_t)^{\alpha-1}x_t\\
        &= \alpha(1-\alpha)L_{1,t}^{1-\alpha}(x_t)^{\alpha}\\
        & = \alpha(1-\alpha)L_{1,t}^{1-\alpha}\left(L_{1,t}\left(\frac{\alpha^2}{r_t} \right)^{\frac{1}{1-\alpha}}\right)^{\alpha} = \alpha(1-\alpha) L_{1,t}\left(\frac{\alpha^2}{r_t}\right)^{\frac{\alpha}{1-\alpha}}>0
    \end{align*}
    \end{minipage}
};
\node[bluetitle, right=4pt] at (box.north west) {Firm's problem: intermediate goods producer};
\end{tikzpicture}

\begin{itemize}
    \item[-] \textbf{\color{myblue}R\&D sector} \textbf{producer}: This is also a perfectly competitive market: free entry, zero profit. For idea producers, they solve the profit maximization problem:
    $$
    \max_{A_{t+1},L_{2,t}}\left\{ p_t^P(A_{t+1}-A_t)-w_tL_{2,t} \right\}
    $$
    s.t.
    $$
    A_{t+1}-A_t = \kappa A_tL_{2,t}
    $$
    where $p_t^P$ is the price of a new patent, then
\end{itemize}
\vspace{2pt}
\begin{tikzpicture}
\node [bluebox] (box){%
    \begin{minipage}{0.315\textwidth}
    \color{myblue}
    \scriptsize
    wage of labor is 
    $$
    w_t = \kappa A_t L_{2,t}p_t^P
    $$
    \end{minipage}
};
\node[bluetitle, right=4pt] at (box.north west) {Firm's problem: R\&D sector};
\end{tikzpicture}

\vspace{2pt}
Then we can close the model on firms' side:

\vspace{2pt}
\begin{tikzpicture}
\node [ibluebox] (box){%
    \begin{minipage}{0.315\textwidth}
    \color{white}
    \scriptsize
    The present value of rents of a patent holder equals the price paid for the patent in equilibrium, i.e.:
    $$
    p_t^P\cdot p_t = \sum^{\infty}_{\tau = t+1}\pi_{\tau}(i)p_{\tau}
    $$
    where 
    \begin{itemize}
        \item[-] price of patent $p_t^P$ is from R\&D sector: $w_t=\kappa A_t L_{2,t} p^P_t$ 
        \item[-] positive profit of intermediate good producer $\pi_t(i)=\alpha(1-\alpha)L_{1,t}(\alpha^2/r_t)^{\frac{\alpha}{1-\alpha}}$
        \item[-] $p_t$ is the consumption price (time $t$ price $p_0$ as numeraire), faced by \underline{\textbf{households}}
    \end{itemize}
    \end{minipage}
};
\node[ibluetitle, right=4pt] at (box.north west) {Firm's problem: closing the model};
\end{tikzpicture}

\vspace{4pt}
\textbf{\color{myblue}\underline{Household's problem}}:

\vspace{2pt}
Now move to the households. Households' problem is as always, maximizing life-time consumption:
    $$
    \max \sum^{\infty}_{t=0}\beta^t \frac{c_t^{1-\sigma}-1}{1-\sigma}
    $$
    s.t.
\begin{align*}
    \sum^{\infty}_{t=0}p_t(c_t+i_t) & =\sum^{\infty}_{t=0} p_t(w_t L_t+r_t K_t+\int^{A_t}_0\pi_t(i)\mathrm{d}i)\\
    %$\phi_t L_t & = L_{1,t} \\
    i_t &= K_{t+1}-(1-\delta)K_t
    %\pi_t(i)&=\alpha(1-\alpha)L_{1,t}\left(\frac{\alpha^2}{r_t}\right)^{\frac{\alpha}{1-\alpha}}
\end{align*}

write the Lagrange:
\begin{align*}
    \mathcal{L} =& \sum^{\infty}_{t=0}\beta^t\frac{c_t^{1-\sigma}-1}{1-\sigma} + \lambda\left[ \sum^\infty_{t=0}p_t\left(w_t L_t+r_t K_t + \int^{A_t}_0 \pi_t(i) \mathrm{d}i\right) \right.\\
    &- \left. \sum^{\infty}_{t=0}p_t\left(c_t+K_{t+1}-(1-\delta)K_t \right) \right]
\end{align*}

FOC gives
\begin{align*}
    \frac{\partial \mathcal{L}}{\partial c_t}=0\Rightarrow & \beta\left(\frac{c_{t+1}}{c_t}\right)^{-\sigma}=\frac{p_{t+1}}{p_t}\\
    \frac{\partial \mathcal{L}}{\partial K_{t+1}}=0\Rightarrow & \frac{p_{t+1}}{p_t} =  \frac{1}{r_{t+1}+(1-\delta)}
\end{align*}

leading to the Euler equation:
$$
\left(\frac{c_{t+1}}{c_t}\right)^{\sigma} = \beta\left[r_{t+1}+(1-\delta)\right]
$$

from intermediate goods sector, we know 
$$
r_t = \alpha^2 L_{1,t}^{1-\alpha}(x_t)^{\alpha-1} \xRightarrow[x_t(i)=K_t(i)]{\int^{A_t}_0 K_t(i)=K_t} \alpha^2 \left(\phi_t L\right)^{1-\alpha} \left(\frac{A_t}{K_t}\right)^{1-\alpha}
$$
hence,

\vspace{2pt}
\begin{tikzpicture}
\node [bluebox] (box){%
    \begin{minipage}{0.315\textwidth}
    \color{myblue}
    \scriptsize
    the decentralized Euler equation is
    $$
    \left(\frac{c_{t+1}}{c_t}\right)^{\sigma} = \beta\left[ \alpha^2 \left(\frac{\phi_{t+1}L A_{t+1}}{K_{t+1}}\right)^{1-\alpha}+(1-\delta) \right]
    $$
    leading to the growth rate of consumption in the decentralized equilibrium
    $$
    \hat{g}_c\equiv \frac{c_{t+1}}{c_t} = \left\{\beta\left[ \alpha^2 \left(\frac{\phi_{t+1}L A_{t+1}}{K_{t+1}}\right)^{1-\alpha}+(1-\delta) \right]\right\}^{\frac{1}{\sigma}}
    $$
    \end{minipage}
};
\node[bluetitle, right=4pt] at (box.north west) {Household's problem: decentralized Euler equation};
\end{tikzpicture}

\vspace{4pt}
\textbf{\color{myblue}\underline{Compare Social planner equilibrium with decentralized equilibrium}}:

The consumption growth rate is 
\begin{itemize}
    \item[-] \textbf{decentralized equilibrium}: 
    $$
    \hat{g}_c\equiv \frac{c_{t+1}}{c_t} = \left\{\beta\left[ \alpha^2 \left(\frac{\phi_{t+1}L A_{t+1}}{K_{t+1}}\right)^{1-\alpha}+(1-\delta) \right]\right\}^{\frac{1}{\sigma}}
    $$
    \item[-] \textbf{social planner equilibrium}:
    $$
    g_c =\left\{\beta\left[ \alpha \left(\frac{\phi_{t+1}L A_{t+1}}{K_{t+1}}\right)^{1-\alpha}+(1-\delta) \right]\right\}^{\frac{1}{\sigma}}
    $$
\end{itemize}
combining the two, get 
$$
\hat{g}_c^{\sigma} = \alpha g_c^{\sigma} +\beta(1-\alpha)(1-\delta)\Rightarrow \hat{g}_c = \left[ \alpha g_c^{\sigma} +\beta(1-\alpha)(1-\delta) \right]^{\frac{1}{\sigma}}
$$

Since the balanced growth rate in social planner's problem is $g_c = \gamma = \left(\beta(1+L\kappa)\right)^{1/\sigma}$, we have the decentralized equilibrium balanced growth rate:
\begin{align*}
    \hat{\gamma} &= \left[ \alpha \gamma^{\sigma} +\beta(1-\alpha)(1-\delta) \right]^{\frac{1}{\sigma}} = \left[ \alpha \beta(1+L\kappa) +\beta(1-\alpha)(1-\delta) \right]^{\frac{1}{\sigma}}
\end{align*}
It is very easy to show that:
\begin{align*}
    \gamma^{\sigma} -\hat{\gamma}^{\sigma} &= \beta(1+L\kappa) - \left[ \alpha \beta(1+L\kappa) +\beta(1-\alpha)(1-\delta) \right]\\
    & = \beta(1-\alpha)(L\kappa + \delta) >0 \Rightarrow \gamma>\hat{\gamma}
\end{align*}
i.e., \underline{the balanced growth rate is \textbf{lower} in the decentralized equilibrium}. This is due to the fact the intermediate variety producers experience monopolistic competition.

\vspace{2pt}
And again, from $A_{t+1}-A_t = \kappa(1-\phi_t)LA_t$, we have the optimal labor share
\begin{align*}
    & 1+(1-\phi)L\kappa = \left[ \alpha \beta(1+L\kappa) +\beta(1-\alpha)(1-\delta) \right]^{\frac{1}{\sigma}}\\
    \Rightarrow & \hat{\phi}^* =1- \frac{\left[ \alpha \beta(1+L\kappa) +\beta(1-\alpha)(1-\delta) \right]^{\frac{1}{\sigma}}-1}{L\kappa} >\phi^*
\end{align*}
i.e., in the decentralized equilibrium, relatively \textbf{more labor} are used in the production of final goods.

\vspace{2pt}
\begin{tikzpicture}
\node [ibluebox] (box){%
    \begin{minipage}{0.315\textwidth}
    \color{white}
    \scriptsize
    The balanced growth rate $\gamma$ is 
    \begin{itemize}
        \item[-] \textbf{decentralized equilibrium}: $$\gamma_{dec} = \left[ \alpha \gamma^{\sigma} +\beta(1-\alpha)(1-\delta) \right]^{\frac{1}{\sigma}} = \left[ \alpha \beta(1+L\kappa) +\beta(1-\alpha)(1-\delta) \right]^{\frac{1}{\sigma}}$$
        \item[-] \textbf{social planner equilibrium}: $$\gamma_{sp} =\left[\beta\left(1+L\kappa\right)\right]^{1/\sigma} $$
    \end{itemize}
    and decentralizing leads to \textbf{LOWER} growth rate, due to monopolistic intermediate good production
    $$
    \gamma_{dec}<\gamma_{sp}
    $$
    and for the optimal labor share 
    $$
    \phi^* = 1-\frac{\gamma-1}{L\kappa}
    $$
    we have
    $$
    \phi^*_{dec} > \phi^*_{sp}
    $$
    \textbf{MORE} labor is used, which is inefficient too.
    \end{minipage}
};
\node[ibluetitle, right=4pt] at (box.north west) {Summary: decentralized equilibrium vs social planner equilibrium};
\end{tikzpicture}

\section*{Structural change}
This is a multi-sector variant of neoclassical growth model, inspired by the following observations:
\begin{itemize}
    \item[-] \textbf{\color{myred}agricultural sector}:
    \begin{itemize}
        \item[(a)] as $y_t$ (GDP per capita) increases, employment share and nominal value-added share \textbf{\color{myred}FALL}
        \item[(b)] poor countries tend to have \textbf{most} of employment in agricultural sector, though it is most inefficient
    \end{itemize}
    
    \item[-] \textbf{\color{myred}manufacturing sector}:
    \begin{itemize}
        \item[(a)] as $y_t$ increases, employment share and nominal value-added share follow a \textbf{\color{myred}HUMP shape}
        \item[(b)] the nominal value-added share for manufacturing \textbf{peaks} around $\log(y_t)\simeq 9$, then decreases
    \end{itemize}

    \item[-] \textbf{\color{myred}service sector}: 
    \begin{itemize}
        \item[(a)] as $y_t$ increases, employment share and nominal value-added share \textbf{\color{myred}INCREASE}
        \item[(b)] both employment share and the nominal value-added share are \textbf{bounded away} from 0 even for very poor countries
        \item[(c)] the nominal value-added share for services \textbf{accelerates} also around $\log(y_t)\simeq 9$
    \end{itemize}
\end{itemize}

There are two central mechanisms for structural changes:
\begin{itemize}
    \item[1] Income effects, via \textit{Stone-Geary} preferences
    \item[2] Relative price effects, via \textit{sectoral productivity growth differentials}
\end{itemize}

For this model, we assume no friction, i.e. competitive equilibrium is Pareto-efficient, but it is better to solve the CE directly.

\vspace{2pt}
\begin{tikzpicture}
\node [redbox] (box){%
    \begin{minipage}{0.315\textwidth}
    \color{myred}
    \scriptsize
    Consumer still maximizes her lifetime utility, w.r.t. consumption of the 3 types of final goods
    $$
    \max\sum^{\infty}_{t=0}\beta^t \log C_t
    $$
    with
    \begin{align*}
        C_t =& \left(\omega_a^{\frac{1}{\epsilon}}\left(c_{a,t}-\bar{c}_a\right)^{1-\frac{1}{\epsilon}} + \right. &\cdots \text{ agriculture}\\
        & \omega_m^{\frac{1}{\epsilon}}\left(c_{m,t}\right)^{1-\frac{1}{\epsilon}} + &\cdots \text{ manufacture}\\
        & \left. \omega_s^{\frac{1}{\epsilon}}\left(c_{s,t}+\bar{c}_s\right)^{1-\frac{1}{\epsilon}} \right)^{\frac{\epsilon}{\epsilon-1}} &\cdots \text{ service}
    \end{align*}
    s.t.
    $$
        p_{a,t}c_{a,t}+p_{m,t}c_{m,t}+p_{s,t}c_{s,t} + p_{k,t}[K_{t+1}-(1-\delta)K_t] = K_tr_t +w_t
    $$
    where
    \begin{itemize}
        \item[-] $\epsilon>0$ is the elasticity of substitution between the 3 types of consumption goods
        \item[-] inter-temporal elasticity of substitution is 1
        \item[-] \textit{Stone-Geary} preferences
        \begin{itemize}
            \item[-] $\bar{c}_a>0$ is the minimum required level of consumption of agricultural goods (food), the survival amount
            \item[-] $\bar{c}_s>0$ non-marketed endowment of a public good, it does NOT enter budget constraints and is exogenously produced, e.g., home services
        \end{itemize}
        to ensure a BGP, $\bar{c}_a$ and $\bar{c}_s$ must cancel out in budget constraint
        \item[-] capital stock $K_t$ and labor ($L=1$, inelastically supplied) are \textbf{PERFECTLY mobile across sectors}.
    \end{itemize}
    
    The Lagrange is
    \begin{align*}
        \mathcal{L} =& \sum^{\infty}_{t=0}\beta^t \log C_t + \lambda_t\left\{ K_tr_t+w_t -p_{k,t}[K_{t+1}-(1-\delta)K_t]- \right.\\
        &\left. p_{a,t}c_{a,t}-p_{m,t}c_{m,t}-p_{s,t}c_{s,t} \right\}
    \end{align*}
    
    FOC gives
    \begin{align}
        \frac{\partial\mathcal{L}}{\partial c_{a,t}}=0\Rightarrow & \frac{\beta^t}{C_t}C_t^{\frac{1}{\epsilon}}w_a^{\frac{1}{\epsilon}}\left(c_{a,t}-\bar{c}_a\right)^{-\frac{1}{\epsilon}} =\lambda_t p_{a,t}\\
        \frac{\partial\mathcal{L}}{\partial c_{m,t}}=0\Rightarrow & \frac{\beta^t}{C_t}C_t^{\frac{1}{\epsilon}}w_m^{\frac{1}{\epsilon}}c_{m,t}^{-\frac{1}{\epsilon}} =\lambda_t p_{m,t}\\
        \frac{\partial\mathcal{L}}{\partial c_{s,t}}=0\Rightarrow &  \frac{\beta^t}{C_t}C_t^{\frac{1}{\epsilon}}w_s^{\frac{1}{\epsilon}}\left(c_{s,t}+\bar{c}_s\right)^{-\frac{1}{\epsilon}} =\lambda_t p_{s,t}\\
        \frac{\partial\mathcal{L}}{\partial K_{t+1}}=0\Rightarrow & \lambda_{t+1}\left[r_{t+1}+(1-\delta)\right]=\lambda_t p_{k,t}
    \end{align}
    
    \end{minipage}
};
\node[redtitle, right=4pt] at (box.north west) {HH's problem: 3 types of final goods to consume};
\end{tikzpicture}

Now, calculate $(1)^{1-\epsilon}$, $(2)^{1-\epsilon}$, $(3)^{1-\epsilon}$, get 
\begin{align*}
    \left(\frac{\beta^t}{C_t}\right)^{1-\epsilon}C_t^{\frac{1-\epsilon}{\epsilon}}w_a^{\frac{1}{\epsilon}}\left(c_{a,t}-\bar{c}_a\right)^{1-\frac{1}{\epsilon}} & =\lambda_t^{1-\epsilon} w_a p_{a,t}^{1-\epsilon}\\
    \left(\frac{\beta^t}{C_t}\right)^{1-\epsilon}C_t^{\frac{1-\epsilon}{\epsilon}}w_m^{\frac{1}{\epsilon}}c_{m,t}^{1-\frac{1}{\epsilon}} & =\lambda_t^{1-\epsilon} w_m p_{m,t}^{1-\epsilon}\\ \left(\frac{\beta^t}{C_t}\right)^{1-\epsilon}C_t^{\frac{1-\epsilon}{\epsilon}}w_s^{\frac{1}{\epsilon}}\left(c_{s,t}+\bar{c}_s\right)^{1-\frac{1}{\epsilon}} & =\lambda_t^{1-\epsilon} w_s p_{s,t}^{1-\epsilon}
\end{align*}
sum them together, get
\begin{align*}
    & \left(\frac{\beta^t}{C_t}\right)^{1-\epsilon} C_t^{\frac{1-\epsilon}{\epsilon}}C_t^{\frac{\epsilon-1}{\epsilon}} = \lambda_t^{1-\epsilon}\left( \omega_ap_{a,t}^{1-\epsilon} +\omega_m p_{m,t}^{1-\epsilon} +\omega_w p_{s,t}^{1-\epsilon} \right)\\
    \Rightarrow & \frac{\beta^t}{C_t} = \lambda_t \left( \omega_ap_{a,t}^{1-\epsilon} +\omega_m p_{m,t}^{1-\epsilon} +\omega_w p_{s,t}^{1-\epsilon} \right)^{\frac{1}{1-\epsilon}}
\end{align*}
here
\begin{itemize}
    \item[-] $\lambda_t$ is the present shadow/utility value of an additional unit of expenditure.
    \item[-] RHS: the present value of marginal cost of consumption composite expenditure
    \item[-] LHS: the marginal benefit of consumption composite
\end{itemize}

Define
$$\left(\omega_ap_{a,t}^{1-\epsilon} +\omega_m p_{m,t}^{1-\epsilon} +\omega_w p_{s,t}^{1-\epsilon}\right)^{\frac{1}{1-\epsilon}}\equiv P_t$$ 
as the price index of a unit of composite consumption, then 
$$
P_tC_t\lambda_t = \beta^t 
$$

Next, multiply FOCs (1,2,3) by sectoral consumption, get 
\begin{align*}
    \beta^t C_t^{\frac{1-\epsilon}{\epsilon}}w_a^{\frac{1}{\epsilon}}\left(c_{a,t}-\bar{c}_a\right)^{1-\frac{1}{\epsilon}} & =\lambda_t  p_{a,t}(c_{a,t}-\bar{c}_a)\\
    \beta^t C_t^{\frac{1-\epsilon}{\epsilon}}w_m^{\frac{1}{\epsilon}}c_{m,t}^{1-\frac{1}{\epsilon}} & =\lambda_t p_{m,t}c_{m,t}\\ 
    \beta^t C_t^{\frac{1-\epsilon}{\epsilon}}w_s^{\frac{1}{\epsilon}}\left(c_{s,t}+\bar{c}_s\right)^{1-\frac{1}{\epsilon}} & =\lambda_t p_{s,t}(c_{s,t}+\bar{c}_s)
\end{align*}
again, sum them all together, get
$$
\beta^t = \lambda_t \left[ p_{a,t}\left(c_{a,t}-\bar{c}_a\right)+p_{m,t}c_{m,t}+p_{s,t}\left(c_{s,t}+\bar{c}_s\right) \right]
$$
hence
$$
\frac{\beta^t}{\lambda_t} = p_{a,t}\left(c_{a,t}-\bar{c}_a\right)+p_{m,t}c_{m,t}+p_{s,t}\left(c_{s,t}+\bar{c}_s\right) = P_tC_t
$$
this gives another version of the budget constraint
$$
p_{a,t}c_{a,t}+p_{m,t}c_{m,t}+p_{s,t}c_{s,t} = P_tC_t +p_{a,t}\bar{c}_a -p_{s,t}\bar{c}_s
$$

Next, calculate (2)/(1), (3)/(1) of the FOCs
\begin{align*}
    \frac{\omega_a^{\frac{1}{\epsilon}}(c_{a,t}-\bar{c}_a)^{-\frac{1}{\epsilon}}}{\omega_m^{\frac{1}{\epsilon}}c_{m,t}^{-\frac{1}{\epsilon}}}=\frac{p_{a,t}}{p_{m,t}} & \Rightarrow \frac{p_{a,t}(c_{a,t}-\bar{c}_a)}{p_{m,t}c_{m,t}} = \frac{w_a}{w_m}\left(\frac{p_{a,t}}{p_{m,t}}\right)^{1-\epsilon}\\
    \frac{\omega_a^{\frac{1}{\epsilon}}(c_{a,t}-\bar{c}_a)^{-\frac{1}{\epsilon}}}{\omega_s^{\frac{1}{\epsilon}}(c_{s,t}+\bar{c}_s)^{-\frac{1}{\epsilon}}}=\frac{p_{a,t}}{p_{s,t}} & \Rightarrow \frac{p_{a,t}(c_{a,t}-\bar{c}_a)}{p_{s,t}(c_{s,t}+\bar{c}_s)} = \frac{w_a}{w_s}\left(\frac{p_{a,t}}{p_{s,t}}\right)^{1-\epsilon}
\end{align*}
then consumption composite expenditure $P_tC_t$ can be rewritten as
\begin{align*}
    P_tC_t & = p_{a,t}(c_{a,t}-\bar{c}_a)+p_{m,t}c_{m,t}+p_{s,t}(c_{s,t}+\bar{c}_s)\\
    & = p_{a,t}(c_{a,t}-\bar{c}_a)\left[ 1+\frac{\omega_m}{\omega_a}\left(\frac{p_{m,t}}{p_{a,t}}\right)^{1-\epsilon} + \frac{\omega_s}{\omega_a}\left(\frac{p_{s,t}}{p_{a,t}}\right)^{1-\epsilon}\right]
\end{align*}
this gives the consumption spending shares of sectors
\begin{align*}
    \frac{p_{a,t}c_{a,t}}{P_tC_t}&=\frac{1}{\Theta}+\frac{p_{a,t}\bar{c}_a}{P_tC_t}\\
    \frac{p_{m,t}c_{m,t}}{P_tC_t}&=\frac{1}{\Theta}\frac{\omega_m}{\omega_a} \left(\frac{p_{m,t}}{p_{a,t}}\right)^{1-\epsilon}\\
    \frac{p_{s,t}c_{s,t}}{P_tC_t}&=\frac{1}{\Theta}\frac{\omega_s}{\omega_a} \left(\frac{p_{s,t}}{p_{a,t}}\right)^{1-\epsilon}-\frac{p_{s,t}\bar{c}_s}{P_tC_t}
\end{align*}
where $\Theta = 1+\frac{\omega_m}{\omega_a}\left(\frac{p_{m,t}}{p_{a,t}}\right)^{1-\epsilon} + \frac{\omega_s}{\omega_a}\left(\frac{p_{s,t}}{p_{a,t}}\right)^{1-\epsilon}$. This gives that the shares of the three consumer goods sectors change with relative prices.

Finally, (4) of FOCs gives the Euler equation

\vspace{2pt}
\begin{tikzpicture}
\node [iredbox] (box){%
    \begin{minipage}{0.315\textwidth}
    \color{white}
    \scriptsize
    $$
    \lambda_t p_{k,t}= \lambda_{t+1}\left[r_{t+1}+(1-\delta)\right]
    $$
    since investment goods are set as numeraire, $p_{k,t}=0$, and plug in  $\lambda_t=\beta^t/P_tC_t$, get the Euler equation
    \begin{align*}
        &\frac{\beta^t}{P_tC_t} = \frac{\beta^{t+1}}{P_{t+1}C_{t+1}}\left[r_{t+1}+(1-\delta)\right] \\
        \Rightarrow & \frac{P_{t+1}C_{t+1}}{P_t C_t} = \beta\left[r_{t+1}+(1-\delta)\right]
    \end{align*}
    \end{minipage}
};
\node[iredtitle, right=4pt] at (box.north west) {Euler equation};
\end{tikzpicture}

\vspace{2pt}
To summarize, HH's problem can be solved in 2 stages:
\begin{itemize}
    \item[1] \textbf{\color{myred}Dynamic problem} of allocating income between consumption $C_t$ and investment $i_t$
    \item[2] \textbf{\color{myred}Static problem} of allocating consumption spending across the 3 sectors
\end{itemize}

\vspace{2pt}
\begin{tikzpicture}
\node [redbox] (box){%
    \begin{minipage}{0.315\textwidth}
    \color{myred}
    \scriptsize
    Firms are perfectly competitive, using Cobb-Douglas production functions to produce \textbf{value-added}
    \begin{itemize}
        \item[-] consumption goods: 
        $$
        y_{i,t}=K_{i,t}^{\alpha}(A_{i,t}L_{i,t})^{1-\alpha},\ \forall i=a,m,s
        $$
        \item[-] investment goods:
        $$
        y_{k,t}= i_t =K_{k,t}^{\alpha}(A_{k,t}L_{k,t})^{1-\alpha}
        $$
        
    Hence, for firms in sector $i=a,m,s,k$, solve
    $$
    \max p_{i,t}y_{i,t}-w_tL_{i,t}-r_t K_{i,t}
    $$
    FOC gives
    \begin{align*}
        r_t &= p_{i,t}\alpha\left( \frac{L_{i,t}}{K_{i,t}} \right)^{1-\alpha}A_{i,t}^{1-\alpha}\\
        w_r &= p_{i,t}(1-\alpha) \left(\frac{L_{i,t}}{K_{i,t}}\right)^{-\alpha}A_{i,t}^{1-\alpha}
    \end{align*}
    leading to $\forall i=a,m,s,k$
    $$
    \frac{K_{i,t}}{L_{i,t}} = \frac{w_t}{r_t}\cdot \frac{\alpha}{1-\alpha}\Rightarrow \frac{K_t}{L_t(\equiv1)}=\frac{K_{i,t}}{L_{i,t}}
    \Rightarrow K_t =\frac{w_t}{r_t}\cdot \frac{\alpha}{1-\alpha}$$
    \end{itemize}
    \end{minipage}
};
\node[redtitle, right=4pt] at (box.north west) {Firm's problem: 3 types of $c_{i,t}$ and $i_t$ to produce};
\end{tikzpicture}

\vspace{2pt}
After figuring out the aggregate capital stock $K_t$, plug it back into the FOCs, get $\forall i =a,m,s,k$
\begin{align*}
    r_t &= p_{i,t}\alpha(K_t)^{\alpha-1}A_{i,t}^{1-\alpha} \\
    &=p_{i,t}\alpha \left( \frac{r_t}{w_t}\frac{1-\alpha}{\alpha} \right)^{1-\alpha}A_{i,t}^{1-\alpha} \Rightarrow
    p_{i,t}A_{i,t}^{1-\alpha} = \frac{r_t}{\alpha} \left( \frac{w_t}{r_t}\frac{\alpha}{1-\alpha} \right)^{1-\alpha}
\end{align*}

hence $p_{i,t}A_{i,t}^{1-\alpha}$ is equal across all sectors, i.e. 
$$p_{i,t}A_{i,t}^{1-\alpha}=p_{j,t}A_{j,t}^{1-\alpha},\ \forall i$$

This gives that relative prices of goods are given solely by technological differences
$$
\frac{p_{k,t}}{p_{a,t}}=\left(\frac{A_{a,t}}{A_{k,t}}\right)^{1-\alpha},\ \frac{p_{k,t}}{p_{m,t}}=\left(\frac{A_{m,t}}{A_{k,t}}\right)^{1-\alpha},\ \frac{p_{k,t}}{p_{s,t}}=\left(\frac{A_{s,t}}{A_{k,t}}\right)^{1-\alpha}
$$
And if we set investment goods as the numeraire, i.e. let $p_{k,t}=1$, get
$$
p_{a,t} = \left(\frac{A_{k,t}}{A_{a,t}}\right)^{1-\alpha},\ p_{m,t} = \left(\frac{A_{k,t}}{A_{m,t}}\right)^{1-\alpha},\ p_{s,t} = \left(\frac{A_{k,t}}{A_{s,t}}\right)^{1-\alpha}
$$
If the relative labor productivity of one sector rises, its relative price of the value added will fall.

\subsection*{Competitive equilibrium}
With both sides done, we can characterize the equilibrium as the following

\vspace{2pt}
\begin{tikzpicture}
\node [redbox] (box){%
    \begin{minipage}{0.315\textwidth}
    \color{myred}
    \scriptsize
    From the problems of consumers and firms, we get the equilibrium prices, $\forall i=a,m,s$
    \begin{align*}
        r_t &= p_{i,t}\alpha (K_t)^{\alpha-1} A_{i,t}^{1-\alpha}=\alpha A_{k,t}^{1-\alpha}(K_t)^{\alpha-1} \\
        w_t &= p_{i,t}(1-\alpha) (K_t)^{\alpha}A_{i,t}^{1-\alpha} = (1-\alpha)A_{k,t}^{1-\alpha} (K_t)^{\alpha}\\
        p_{k,t} &=1\\
        p_{i,t} &= \left( A_{k,t}/A_{i,t} \right)^{1-\alpha}\\
        p_{i,t}/p_{j,t} & = \left( A_{j,t}/A_{i,t} \right)^{1-\alpha}
    \end{align*}
    \end{minipage}
};
\node[redtitle, right=4pt] at (box.north west) {Equilibrium prices};
\end{tikzpicture}

The sector relative consumption shares can be rewritten as:
\begin{align*}
    \frac{p_{a,t}c_{a,t}}{P_tC_t}&=\frac{1}{\Omega}+\frac{\bar{c}_a}{P_tC_t}\cdot \left( \frac{A_{k,t}}{A_{a,t}}\right)^{1-\alpha}\\
    \frac{p_{m,t}c_{m,t}}{P_tC_t}&=\frac{1}{\Omega}\frac{\omega_m}{\omega_a} \left(\left(\frac{A_{a,t}}{A_{m,t}}\right)^{1-\alpha}\right)^{1-\epsilon}\\
    \frac{p_{s,t}c_{s,t}}{P_tC_t}&=\frac{1}{\Omega}\frac{\omega_s}{\omega_a} \left(\left(\frac{A_{a,t}}{A_{s,t}}\right)^{1-\alpha}\right)^{1-\epsilon}-\frac{\bar{c}_s}{P_tC_t}\cdot \left( \frac{A_{k,t}}{A_{s,t}}\right)^{1-\alpha}
\end{align*}
where $\Omega = 1+\frac{\omega_m}{\omega_a}\left(\left(\frac{A_{a,t}}{A_{m,t}}\right)^{1-\alpha}\right)^{1-\epsilon} + \frac{\omega_s}{\omega_a}\left(\left(\frac{A_{a,t}}{A_{s,t}}\right)^{1-\alpha}\right)^{1-\epsilon}$. This gives the impact of $\epsilon$ (\textit{almost} the elasiticity of substitution between goods of the 3 sectors):
\begin{itemize}
    \item[-] $\epsilon <1$: consumption is difficult to substitute across sectors, $\partial\Omega /\partial A_{a,t}>0$, hence $\frac{\partial}{\partial A_{a,t}}\frac{p_{a,t}c_{a,t}}{P_tC_t}<0$, i.e., agricultural consumption expenditure share \textbf{falls} when productivity $A_{a,t}$ increases (i.e. $p_{a,t}$ decreases). Same conclusions apply to manufacture and service sectors as well: $\epsilon<1 \rightarrow \frac{\partial}{\partial A_{i,t}}\frac{p_{i,t}c_{i,t}}{P_tC_t}<0,\forall i=a,m,s$
    \item[-] $\epsilon =1$: consumption expenditure share of each sector is \textbf{independent} of relative prices and relative productivity
    \item[-] $\epsilon >1$: consumption is very easy to substitute across sectors, $\partial\Omega /\partial A_{a,t}<0$, hence $\frac{\partial}{\partial A_{a,t}}\frac{p_{a,t}c_{a,t}}{P_tC_t}>0$, i.e., agricultural consumption expenditure share \textbf{increases} when productivity $A_{a,t}$ increases (i.e. $p_{a,t}$ decreases). Same conclusions apply to manufacture and service sectors as well: $\epsilon>1 \rightarrow \frac{\partial}{\partial A_{i,t}}\frac{p_{i,t}c_{i,t}}{P_tC_t}>0,\forall i=a,m,s$ 
\end{itemize}

\vspace{2pt}
\begin{tikzpicture}
\node [redbox] (box){%
    \begin{minipage}{0.315\textwidth}
    \color{myred}
    \scriptsize
    Define GDP as
    $$
    Y_t = \sum_{i=a,m,s,k}p_{i,t}y_{i,t}=y_{k,t}+p_{a,t}y_{a,t}+p_{m,t}y_{m,t}+p_{s,t}y_{s,t}
    $$
    where $y_i = A_{i,t}^{1-\alpha}K_t^{\alpha}L_{i,t}^{1-\alpha}=A_{i,t}^{1-\alpha}\left(\frac{K_{i,t}}{L_{i,t}}\right)^{\alpha}L_{i,t}=A_{i,t}^{1-\alpha}K_t^{\alpha}L_{i,t}$, plug this in $Y_t$, get
    $$
    Y_t = \sum_{i}p_{i,t}A_{i,t}^{1-\alpha}K_t^{\alpha}L_{i,t}=A_{k,t}^{1-\alpha}K_t^{\alpha}
    $$
    \end{minipage}
};
\node[redtitle, right=4pt] at (box.north west) {Equilibrium value-added};
\end{tikzpicture}

With GDP derived, we can solve the GDP share of each sector as
$$
\frac{p_{i,t}y_{i,t}}{Y_t}=\frac{p_{i,t}A_{i,t}^{1-\alpha}K_t^{\alpha}L_{i,t}}{A_{k,t}^{1-\alpha}K_t^{\alpha}} = \frac{A_{k,t}^{1-\alpha}K_t^{\alpha}L_{i,t}}{A_{k,t}^{1-\alpha}K_t^{\alpha}} = L_{i,t}
$$
that is, the \textbf{value-added share} of each sector should \textbf{\color{myred}equal} its \textbf{labor share}. However, empirically this is not the case, one way to solve this problem is to allow different capital intensities across sectors. 

\vspace{2pt}
Another way of calculating GDP share of sectors is using the sector relative prices, and the market clearing condition 
\begin{align*}
    p_{i,t}y_{i,t} &= p_{i,t}c_{i,t},\forall i = a,m,s;&p_{k,t}y_{k,t} = K_{t+1}-(1-\delta)K_t,p_{k,t}=1
\end{align*}
get
\begin{align*}
    \frac{p_{a,t}y_{a,t}}{Y_t}=&\frac{1}{\Omega}\cdot\frac{P_tC_t}{Y_t}+\frac{\bar{c}_a}{Y_t}\cdot \left( \frac{A_{k,t}}{A_{a,t}}\right)^{1-\alpha}\\
    \frac{p_{m,t}y_{m,t}}{Y_t}=&\frac{1}{\Omega}\cdot\frac{P_tC_t}{Y_t}\frac{\omega_m}{\omega_a} \left(\left(\frac{A_{a,t}}{A_{m,t}}\right)^{1-\alpha}\right)^{1-\epsilon}\\
    \frac{p_{s,t}y_{s,t}}{Y_t}=&\frac{1}{\Omega}\cdot\frac{P_tC_t}{Y_t}\frac{\omega_s}{\omega_a} \left(\left(\frac{A_{a,t}}{A_{s,t}}\right)^{1-\alpha}\right)^{1-\epsilon}-\frac{\bar{c}_s}{Y_t}\cdot \left( \frac{A_{k,t}}{A_{s,t}}\right)^{1-\alpha}
\end{align*}
where $\Omega = 1+\frac{\omega_m}{\omega_a}\left(\left(\frac{A_{a,t}}{A_{m,t}}\right)^{1-\alpha}\right)^{1-\epsilon} + \frac{\omega_s}{\omega_a}\left(\left(\frac{A_{a,t}}{A_{s,t}}\right)^{1-\alpha}\right)^{1-\epsilon}$, and GDP $Y_t = A_{k,t}^{1-\alpha}K_t^{\alpha}$. Here, a new variable merge: \textbf{the aggregate nominal consumption as a fraction of GDP}, $P_tC_t/Y_t$. This can be solve with the Euler equation:
$$
\frac{P_{t+1}C_{t+1}}{P_tC_t}=\beta\left(r_{t+1}+1-\delta\right)
$$
where $r_{t+1}=\alpha A_{k,t+1}^{1-\alpha}K_{t+1}^{\alpha-1}$. Hence we have 
$$
P_tC_t = P_0C_0\beta^t\prod^{t}_{\tau=0}\left[\alpha A_{k,\tau}^{1-\alpha}K_{\tau}^{\alpha-1}+(1-\delta)\right]
$$

\subsection*{Balanced growth path study}
Assume sectorial productivity grow at constant, exogenous rates (could be different across sectors)
$$
\frac{A_{i,t+1}}{A_{i,t}}\equiv  1+\gamma_{A_i}>1
$$
then the relative price
$$
\frac{p_{i,t+1}}{p_{i_t}}=\frac{p_{i,t+1}/p_{k,t+1}}{p_{i,t}/p_{k,t}} = \frac{\left(A_{k,t+1}/A_{i,t+1}\right)^{1-\alpha}}{\left(A_{k,t}/A_{i,t}\right)^{1-\alpha}} = \left(\frac{1+\gamma_{A_k}}{1+\gamma_{A_i}}\right)^{1-\alpha}
$$
is also constant. Again, it can be different across sectors, and it could also be \textbf{less than 1}.

The aggregate consumption expenditure also grows in a constant rate
$$
1+\gamma_{PC}\equiv \frac{P_{t+1}C_{t+1}}{P_tC_t}=\beta(r_{t+1}+1-\delta)
$$
this requires $r_{t+1}=\alpha\left(A_{k,t}/K_t\right)^{1-\alpha}$ to be constant, hence $K_t$ grows at the same rate as $A_{k,t}$, i.e.
$$
1+\gamma_k\equiv \frac{K_{t+1}}{K_t}=1+
\gamma_{A_k}
$$
this gives the GDP growth rate
$$
\frac{Y_{t+1}}{Y_t}=\frac{\alpha A_{k,t+1}^{1-\alpha}K_{t+1}^{\alpha}}{\alpha A_{k,t}^{1-\alpha}K_t^{\alpha}} = (1+\gamma_k)^{\alpha}(1+\gamma_{A_k})^{1-\alpha}=1+\gamma_{A_k}
$$
investment growth rate is also
$$
\frac{i_{t+1}}{i_t} = \frac{K_{t+2}-K_{t+1}(1-\delta)}{K_{t+1}-K_t(1-\delta)} = 1+\gamma_{A_k} = \frac{y_{k,t+1}}{y_{k,t}}
$$
in the investment sector, we have
$$
\frac{y_{k,t+1}}{y_{k,t}}=\frac{A_{k,t+1}^{1-\alpha}L_{k,t+1}^{1-\alpha}K_{t+1}^{\alpha}}{A_{k,t}^{1-\alpha}L_{k,t}^{1-\alpha}K_t^{\alpha}} = (1+\gamma_{A_k})\left(\frac{L_{k,t+1}}{L_{k,t}}\right) = 1+\gamma_{A_k}
$$
this gives labor (hence GDP) share of investment sector $L_{k,t}=\frac{p_{k,t}y_{k,t}}{Y_t}$ is constant, hence the labor and GDP share of \textbf{aggregate consumption} value added is also constant.

Next, we analyze the sector consumption. First, rewrite the budget constraint as
\begin{align*}
    &P_tC_t +p_{a,t}\bar{c}_a -p_{s,t}\bar{c}_s+\left[K_{t+1}-(1-\delta)K_t\right]=K_tr_t +w_t=Y_t\\
    \Rightarrow & \frac{P_tC_t +p_{a,t}\bar{c}_a -p_{s,t}\bar{c}_s}{Y_t}+\frac{i_t}{Y_t}=1
\end{align*}
We already know that $i_t$ and $Y_t$ grows at the same rate, hence $i_t/Y_t$ is constant, therefore $(P_tC_t+p_{a,t}\bar{c}_a-p_{s,t}\bar{c}_s)/Y_t$ must be constant. i.e.
$$
\frac{Y_{t+1}}{Y_t}=\frac{P_{t+1}C_{t+1}+p_{a,t+1}\bar{c}_a-p_{s,t+1}\bar{c}_s)}{P_tC_t+p_{a,t}\bar{c}_a-p_{s,t}\bar{c}_s)}
$$

Plug in the growth rates of sectoral prices $\frac{p_{i,t+1}}{p_{i,t}}=\left(\frac{1+\gamma_{A_k}}{1+\gamma_{A_i}}\right)^{1-\alpha}$, get
$$
\frac{Y_{t+1}}{Y_t} = \frac{\left(1+\gamma_{PC}\right)P_tC_t + \left(\frac{1+\gamma_{A_k}}{1+\gamma_{A_a}}\right)^{1-\alpha}p_{a,t}\bar{c}_a-\left(\frac{1+\gamma_{A_k}}{1+\gamma_{A_s}}\right)^{1-\alpha}p_{s,t}\bar{c}_s}{P_tC_t +p_{a,t}\bar{c}_a-p_{s,t}\bar{c}_s}
$$
For this to hold, we have two different approach:
\begin{itemize}
    \item[I] \underline{\textbf{income-effect-only assumption}}, that is 
    $$
    \gamma_{A_a}=\gamma_{A_m}=\gamma_{A_s}=\gamma_{c}
    $$
    plug this back in the equation above, get
    $$
    1+\gamma_{A_k}= \frac{\left(1+\gamma_{PC}\right)P_tC_t + \left(\frac{1+\gamma_{A_k}}{1+\gamma_c}\right)^{1-\alpha}\left(p_{a,t}\bar{c}_a-p_{s,t}\bar{c}_s\right)}{P_tC_t +p_{a,t}\bar{c}_a-p_{s,t}\bar{c}_s}
    $$
    Since $\left(\frac{1+\gamma_{A_k}}{1+\gamma_c}\right)^{1-\alpha}\neq (1+\gamma_{A_k})$, for this equation to hold, it requires
    $$
    p_{a,t}\bar{c}_a=p_{s,t}\bar{c}_s,\forall t
    $$
    this is
    \begin{align*}
        & p_{a,t}\bar{c}_a=p_{s,t}\bar{c}_s\\
        \Rightarrow &  \frac{p_{a,t}}{p_{s,t}}=\frac{\left(\frac{1+\gamma_{A_k}}{1+\gamma_c}\right)^{t(1-\alpha)}p_{a,0}}{\left(\frac{1+\gamma_{A_k}}{1+\gamma_c}\right)^{t(1-\alpha)}p_{s,0}}=\frac{\bar{c}_s}{\bar{c}_a} = \left(\frac{A_{s,0}}{A_{a,0}}\right)^{1-\alpha}
    \end{align*}
    this is a fragile parameter condition, linking preferences and technologies:(
    But, the aggregate consumption expenditure $P_tC_t$ will grow at $\gamma_{A_k}$:)
    
    \item[II] \underline{\textbf{no-income-effect assumption}}, that is
    $$
    \bar{c}_s= \bar{c}_a =0
    $$
    same deductions apply, again $1+\gamma_{PC}=1+\gamma_{A_k}$
\end{itemize}

Hence, this model requires the $\bar{c}_s$ and $\bar{c}_a$ must be both 0 or both positive.

If either of the 2 approaches apply, we have the balanced growth path

\vspace{2pt}
\begin{tikzpicture}
\node [iredbox] (box){%
    \begin{minipage}{0.315\textwidth}
    \color{white}
    \scriptsize
    Aggregate economy (capital, aggregate consumption, GDP, investment) is on BGP, driven by capital investment
    $$
    \frac{Y_{t+1}}{Y_t}=\frac{P_{t+1}C_{t+1}}{P_tC_t}=\frac{K_{t+1}}{K_t}=1+\gamma_{A_k}
    $$
    Return to capital is given by the Euler equation
    $$
    \frac{P_{t+1}C_{t+1}}{P_tC_t}=1+\gamma_{A_k}=\beta(1-\delta+r)\Rightarrow r=\frac{1+\gamma_{A_k}}{\beta}-(1-\delta)
    $$
    with this return rate, we can pin down the initial value of capital stock $K_0$ that place the economy on a BGP
    \begin{align*}
        & r = \alpha \left(\frac{A_{k,t}}{K_t}\right)^{1-\alpha} = \alpha \left(\frac{A_{k,0}}{K_0} \right)^{1-\alpha}\\
        \Rightarrow & \alpha \left(\frac{A_{k,0}}{K_0} \right)^{1-\alpha} = \frac{1+\gamma_{A_k}}{\beta}-(1-\delta)\\
        \Rightarrow & K_0 = A_{k,0}\left( \frac{\alpha\beta}{\left(1+\gamma_{A_k}\right)-(1-\delta)\beta} \right)^{\frac{1}{1-\alpha}}
    \end{align*}
    \end{minipage}
};
\node[iredtitle, right=4pt] at (box.north west) {Balanced growth path: aggregate};
\end{tikzpicture}

\vspace{2pt}
Now we examine the structural change on the balanced growth path. Again, think about the two approaches:
\begin{itemize}
    \item[-] \underline{\textbf{income-effect-only assumption}}, that is $$\gamma_{A_a}=\gamma_{A_m}=\gamma_{A_s}=\gamma_c,\ \frac{p_{a,0}}{p_{s,0}}=\frac{\bar{c}_s}{\bar{c}_a}=\left(\frac{A_{s,0}}{A_{a,0}}\right)^{1-\alpha}$$
    here, relative prices of $c_a,c_m,c_s$ are constant, that is $p_{a,0}/p_{s,0}=p_{a,t}/p_{s,t},p_{m,0}/p_{s,0}=p_{m,t}/p_{s,t}$, then the \textit{Stone-Geary} \textbf{consumption spending share} can be rewritten as
    \begin{align*}
    \frac{p_{a,t}c_{a,t}}{P_tC_t}&=\frac{\omega_a}{\Theta'}+\frac{p_{a,t}\bar{c}_a}{P_tC_t}&\cdots \frac{\partial}{\partial C_t}\frac{p_{s,t}c_{s,t}}{P_tC_t}<0\\
    \frac{p_{m,t}c_{m,t}}{P_tC_t}&=\frac{\omega_a}{\Theta'}\frac{\omega_m}{\omega_a} \left(\frac{p_{m,0}}{p_{a,0}}\right)^{1-\epsilon}&\cdots \frac{\partial}{\partial C_t}\frac{p_{s,t}c_{s,t}}{P_tC_t}=0\\
    \frac{p_{s,t}c_{s,t}}{P_tC_t}&=\frac{\omega_a}{\Theta'}\frac{\omega_s}{\omega_a} \left(\frac{p_{s,0}}{p_{a,0}}\right)^{1-\epsilon}-\frac{p_{s,t}\bar{c}_s}{P_tC_t} &\cdots \frac{\partial}{\partial C_t}\frac{p_{s,t}c_{s,t}}{P_tC_t}>0
\end{align*}
where $\Theta' = \left( \omega_a+\omega_m\left(\frac{p_{m,0}}{p_{a,0}}\right)^{1-\epsilon} +\omega_s\left(\frac{p_{s,0}}{p_{a,0}}\right)^{1-\epsilon}\right)$.

The \textit{Stone-Geary} consumption is
\begin{align*}
   c_{a,t} &=\frac{\omega_a}{\Theta'}\frac{P_tC_t}{p_{a,t}}+\bar{c}_a\\
    c_{m,t} &=\frac{\omega_m}{\Theta'} \left(\frac{p_{m,0}}{p_{a,0}}\right)^{1-\epsilon}\frac{P_tC_t}{p_{m,t}}\\
    c_{s,t}&=\frac{\omega_s}{\Theta'} \left(\frac{p_{s,0}}{p_{a,0}}\right)^{1-\epsilon}\frac{P_tC_t}{p_{s,t}}-\bar{c}_s
\end{align*}
notice that here
\begin{align*}
    \frac{P_t}{p_{a,t}} &= \frac{\left(\omega_a p_{a,t}^{1-\epsilon}+\omega_m p_{m,t}^{1-\epsilon}+\omega_s p_{s,t}^{1-\epsilon}\right)^{\frac{1}{1-\epsilon}}}{p_{a,t}}\\
    %& = \left( \omega_a+\omega_m \left(\frac{p_{m,t}}{p_{a,t}}\right)^{1-\epsilon} +\omega_s \left(\frac{p_{s,t}}{p_{a,t}}\right)^{1-\epsilon} \right)^{\frac{1}{1-\epsilon}}\\
    & =  \left( \omega_a+\omega_m \left(\frac{p_{m,0}}{p_{a,0}}\right)^{1-\epsilon} +\omega_s \left(\frac{p_{s,0}}{p_{a,0}}\right)^{1-\epsilon} \right)^{\frac{1}{1-\epsilon}}
\end{align*}
which is also constant and same for $P_t/P_{m,t}$ and $P_t/P_{s,t}$. Hence, we have 
\begin{align*}
   c_{a,t} = \Xi_a C_t+\bar{c}_a &\Rightarrow c_{a,t+1}/c_{a,t} < C_{t+1}/C_t\\
    c_{m,t} = \Xi_m C_t &\Rightarrow c_{m,t+1}/c_{m,t} = C_{t+1}/C_t\\
    c_{s,t}=\Xi_s C_t -\bar{c}_s &\Rightarrow c_{s,t+1}/c_{s,t} > C_{t+1}/C_t
\end{align*}
here is why: $\frac{c_{a,t+1}}{c_{a,t}}=\frac{\Xi_aC_{t+1}+\bar{c}_a}{\Xi_aC_t+\bar{c}_a}<\frac{\Xi_a\frac{C_{t+1}}{C_t}C_t+\bar{c}_a\frac{C_{t+1}}{C_t}}{\Xi_aC_t+\bar{c}_a}=\frac{C_{t+1}}{C_t}$.


 % which is a \textbf{constant}. As consumption grows, agricultural share decreases, manufacturing share stays constant, service share increases. comparing to the aggregate consumption growth rate, agricultural consumption grows at a lower rate, manufacturing consumption grows at the same rate, service consumption grows at a higher rate
    
    \item[-] \underline{\textbf{no-income-effect assumption}}, that is 
    $$
    \bar{c}_a=\bar{c}_s= 0,\ \gamma_{A_a}\neq\gamma_{A_m}\neq \gamma_{A_s}
    $$
    now the \textbf{consumption spending share} can be rewritten as
    \begin{align*}
    \frac{p_{a,t}c_{a,t}}{P_tC_t}&=\frac{\omega_a}{\Theta_t} \\
    \frac{p_{m,t}c_{m,t}}{P_tC_t}&=\frac{\omega_a}{\Theta_t}\frac{\omega_m}{\omega_a} \left(\frac{p_{m,t}}{p_{a,t}}\right)^{1-\epsilon}\\
    \frac{p_{s,t}c_{s,t}}{P_tC_t}&=\frac{\omega_a}{\Theta_t}\frac{\omega_s}{\omega_a} \left(\frac{p_{s,t}}{p_{a,t}}\right)^{1-\epsilon}
\end{align*}
where $\Theta_t = \left( \omega_a+\omega_m\left(\frac{p_{m,t}}{p_{a,t}}\right)^{1-\epsilon} +\omega_s\left(\frac{p_{s,t}}{p_{a,t}}\right)^{1-\epsilon}\right)$.

Here, $\epsilon$ matter:
\begin{itemize}
    \item[-] If $\epsilon<1$, goods are gross complements, spending shares decline with a fall in relative prices
    \item[-] If $\epsilon>1$, goods are gross substitutes, spending shares rise with a fall in relative prices
\end{itemize}

Here, since $\gamma_{A_a}$, $\gamma_{A_m}$, $\gamma_{A_s}$ are not assumed to be equal, the relative prices grow at different rates, hence the price index
$$
P_t = \left( \omega_a p^{1-\epsilon}_{a,t}+\omega_m p^{1-\epsilon})_{m,t}+\omega_s p^{1-\epsilon}_{s,t} \right)^{\frac{1}{1-\epsilon}}
$$
does NOT grow at a constant rate, but at the same time, $P_tC_t$ grows at a constant rate, hence $C_t$ does NOT grow at a constant rate either. Therefore, there is no \textit{steady-state} welfare statements. The relative consumption are
\begin{align*}
    \frac{p_{a,t}}{p_{m,t}}\frac{c_{a,t}}{c_{m,t}} = \frac{\omega_a}{\omega_m}\left(\frac{p_{a,t}}{p_{m,t}}\right)^{1-\epsilon} \Rightarrow \frac{c_{a,t}}{c_{m,t}} &= \frac{\omega_a}{\omega_m}\left(\frac{p_{a,t}}{p_{m,t}}\right)^{-\epsilon}\\
    \frac{p_{a,t}}{p_{s,t}}\frac{c_{a,t}}{c_{s,t}} = \frac{\omega_a}{\omega_s}\left(\frac{p_{a,t}}{p_{s,t}}\right)^{1-\epsilon} \Rightarrow \frac{c_{a,t}}{c_{s,t}} &= \frac{\omega_a}{\omega_s}\left(\frac{p_{a,t}}{p_{s,t}}\right)^{-\epsilon}
\end{align*}
plug in the relative prices $p_{i,t}/p_{j,t}=\left(A_{j,t}/A_{i,t}\right)^{1-\alpha}$, get 
\begin{align*}
\frac{c_{a,t}}{c_{m,t}} & = \frac{\omega_a}{\omega_m}\left(\frac{A_{a,t}}{A_{m,t}}\right)^{\epsilon(1-\alpha)}\\
    \frac{c_{a,t}}{c_{s,t}} & = \frac{\omega_a}{\omega_s}\left(\frac{A_{a,t}}{A_{s,t}}\right)^{\epsilon(1-\alpha)}
\end{align*}
relative consumption is decreasing in relative prices and increasing in relative productivity.

Plug sectoral production $c_{i,t}=y_{i,t}=A_{i,t}^{1-\alpha}K_t^{\alpha}L_{i,t}^{1-\alpha}$ and get relative employment
\begin{align*}
\frac{L_{a,t}}{L_{m,t}} & = \frac{A_{m,t}}{A_{a,t}}\cdot \left(\frac{c_{a,t}}{c_{m,t}}\right)^{\frac{1}{1-\alpha}} = \left(\frac{\omega_a}{\omega_m}\right)^{\frac{1}{1-\alpha}}\left(\frac{A_{a,t}}{A_{m,t}}\right)^{\epsilon-1}\\
    \frac{L_{a,t}}{L_{s,t}} & = \frac{A_{s,t}}{A_{a,t}}\cdot \left(\frac{c_{a,t}}{c_{s,t}}\right)^{\frac{1}{1-\alpha}} = \left(\frac{\omega_a}{\omega_s}\right)^{\frac{1}{1-\alpha}}\left(\frac{A_{a,t}}{A_{s,t}}\right)^{\epsilon-1}
\end{align*}
here, long as $\epsilon \neq 1$, relative employment (hence GDP) also change with different rates of productivity growth. Manufacturing shares are NOT constant. To match the empirical observations, assume 
$$
\gamma_a > \gamma_m >\gamma_s,\ \epsilon<1
$$
This gives that
\begin{itemize}
    \item[-] agriculture: 
    \begin{itemize}
        \item[-]the relative price of agriculture (relative to manufacturing and services)  \textbf{falls}
        \item[-]the relative employment of agriculture (relative to manufacturing and services) \textbf{falls} ($\epsilon<1$ assumed)
    \end{itemize}
    \item[-] service:
    \begin{itemize}
        \item[-] the relative price of services (relative to agriculture and manufacturing) \textbf{rises}
        \item[-] the relative employment of services (relative to agriculture and manufacturing) \textbf{rises} ($\epsilon<1$ assumed)
    \end{itemize}
    \item[-] manufacturing:
    \begin{itemize}
        \item[-] the relative price of manufacturing \textbf{rises} relative to agriculture and \textbf{falls} relative to services
        \item[-] the changes relative employment of manufacturing is bounded by the changes in agriculture and service sectors
    \end{itemize}
\end{itemize}
\end{itemize}

\vspace{4pt}
To summarize:

\vspace{2pt}
\begin{tikzpicture}
\node [iredbox] (box){%
    \begin{minipage}{0.315\textwidth}
    \color{white}
    \scriptsize
    Aggregate economy (capital, aggregate consumption, GDP, investment) is on BGP, driven by capital investment
    $$
    \frac{Y_{t+1}}{Y_t}=\frac{P_{t+1}C_{t+1}}{P_tC_t}=\frac{K_{t+1}}{K_t}=1+\gamma_{A_k}
    $$
    Return to capital is given by the Euler equation
    $$
    \frac{P_{t+1}C_{t+1}}{P_tC_t}=1+\gamma_{A_k}=\beta(1-\delta+r)\Rightarrow r=\frac{1+\gamma_{A_k}}{\beta}-(1-\delta)
    $$
    with this return rate, we can pin down the initial value of capital stock $K_0$ that place the economy on a BGP
    \begin{align*}
        & r = \alpha \left(\frac{A_{k,t}}{K_t}\right)^{1-\alpha} = \alpha \left(\frac{A_{k,0}}{K_0} \right)^{1-\alpha}\\
        \Rightarrow & \alpha \left(\frac{A_{k,0}}{K_0} \right)^{1-\alpha} = \frac{1+\gamma_{A_k}}{\beta}-(1-\delta)\\
        \Rightarrow & K_0 = A_{k,0}\left( \frac{\alpha\beta}{\left(1+\gamma_{A_k}\right)-(1-\delta)\beta} \right)^{\frac{1}{1-\alpha}}
    \end{align*}
    \end{minipage}
};
\node[iredtitle, right=4pt] at (box.north west) {Balanced growth path: structural change};
\end{tikzpicture}


%%%%%%%%%%%%%%%%%%%%%%%%%%%%%%%%%
% wealth heterogeneity
%%%%%%%%%%%%%%%%%%%%%%%%%%%%%%%%%
\section*{Wealth heterogeneity}
Three sources of wealth heterogeneity:
\begin{itemize}
    \item[-] \textbf{\color{myred}fixed, ex-ante heterogeneity}: skills, labor time, ability, family wealth, etc.
    \item[-] \textbf{\color{myred}idiosyncratic risks}: cannot be perfect insured, causing ex-post heterogeneity
    \item[-] \textbf{\color{myred}differential access} to asset markets and investment opportunities 
\end{itemize}

\begin{center}
\begin{tikzpicture}[scale=4]
\draw[->] (-0.05, 0) -- (1.05, 0) node[below] {cum.share of HHs};
\draw[->] (0, -0.05) -- (0, 1.05) node[above] {cum.share of wealth};
\draw[domain=0:1, smooth, variable=\x, black] plot ({\x}, {\x});
\draw[domain=0:1, smooth, variable=\x, myblue] plot ({\x}, {(\x)^2});
\draw[domain=0:1, smooth, variable=\x, myred] plot ({\x}, {(\x)^4});

\draw[->, black] (0.5,0.8) node[above] {Equality} -- (0.5,0.5);
\draw[->, myblue] (0.3,0.36) node[left] {Income} -- (0.6,0.36);
\draw[->, myred] (1,0.7^4) node[right] {Wealth} -- (0.7,0.7^4);

\end{tikzpicture}
\end{center}


\textbf{\color{myred}Goal of the model}: 
\begin{itemize}
    \item[-] to capture the three sources: both \textbf{\color{myred}ex-ante} and \textbf{\color{myred}ex-post} heterogeneity
    \item[-] \textbf{\color{myred}aggregation} even with heterogeneity: for simplicity, to model the distribution of wealth \textbf{NOT} as part of the aggregate state vector that are sufficient to determine aggregate allocations and prices
    \item[-] account for the \textbf{\color{myred}evolution} of inequality
\end{itemize}

\subsection*{Aggregate demand with heterogeneity}
\subsubsection*{Gorman's (1961) aggregation}
Setting:
\begin{itemize}
    \item[-] static economy, no uncertainty
    \item[-] $I$ consumer, consumer $i$ has wealth $w_i$
    \item[-] partial equilibrium: consumers take prices as given. $L$ good, price $p_l,l=1,\cdots,L$
\end{itemize}
For consumer $i$, her maximization problem gives her demand for good $l$: $x_i(p,w_i)$, leading to aggregate demand
$$
x(p,w_1,\cdots,w_I)=\sum^I_{i=1}x_i(p,w_i)
$$
to write the aggregate demand as a function only of \textbf{aggregate wealth} $\sum^I_{i=1}w_i$, instead of $\Gamma(w_i)$, the distribution of $w_i$, i.e.,
$$
x\left(p,\sum^I_{i=1}w_i\right)=\sum^I_{i=1}x_i(p,w_i),\ \forall p,\Gamma(w_i)
$$
this requires marginal redistribution of wealth from agent $i$ to $j$ must \textbf{not} change the aggregate demand, in general, for $\sum^I_{i=1}\mathrm{d}w_i=0$, 
$$
\sum^I_{i=1}\frac{\partial x_i(p,w_i)}{\partial w_i}\mathrm{d}w_i = 0
$$
which holds only if the marginal propensity to consume out of wealth is the same for all consumers:
$$
\frac{\partial x_i(p,w_i)}{\partial w_i} = \frac{\partial x_j(p,w_j)}{\partial w_j},\ \forall i,j
$$
this implies the indirect utility function of every agent has the form (MU of wealth $b(p)$ is independent of $w_i$):
$$
V(p,w_i)=U\left(x_i(p,w_i)\right) = a_i(p)+b(p)w_i
$$
then aggregation can be done, for the representative agent:
$$
V(p,w) = a(p)+b(p) \equiv \sum^I_{i=1}a_i(p) + b(p)\left(\sum^I_{i=1}w_i\right)
$$

\subsubsection*{Rubinstein's aggregation (1974)}
It is an extension of Gorman's aggregation, in the sense that:
\begin{itemize}
    \item[-] It is a \textbf{\color{myred}multi-period} economy
    \item[-] agents hold a portfolio of \textbf{\color{myred}risky assets and a risk-free asset}
\end{itemize}
The key results requires the assumption that \textbf{all agents have {\color{myred}linear risk tolerance}}:
$$
\frac{-u'(c_i)}{u''(c_i)}=\rho+\gamma c_i,\ \forall i
$$
i.e., \textbf{\color{myred}HARA} (Hyperbolic absolute risk aversion) preference:
$$
u(c_i)=\frac{(\rho+\gamma c_i)^{\frac{\gamma-1}{\gamma}}}{\gamma-1}
$$
special cases of HARA: (a) $\rho=0$: CRRA; (b) $\gamma=-1$: quadratic utility $u(c_i)=-\frac{1}{2}\left(\rho-c_i\right)^2$; (c) $\gamma=0$: exponential utility.

Rubinstein's aggregation requires:
\begin{itemize}
    \item[$\ $] \textbf{all agents} have the \textbf{\color{myred}same discount factor $\beta$} and \textbf{\color{myred}taste parameters $\gamma\neq 0$} 
    \item[Or] \textbf{all agents} have the \textbf{\color{myred}the same taste parameters $\gamma=0$} 
\end{itemize}

\subsection*{Chatterjee's model}
\underline{\textbf{Setting}}:
\begin{itemize}
    \item[-] one-sector
    \item[-] neoclassical capital accumulation model \textbf{without uncertainty}
    \item[-] agents are differentiated by \textbf{\color{myred}initial wealth}
\end{itemize}
\underline{\textbf{Assumptions}}:
\begin{itemize}
    \item[-] \textbf{identical {\color{myred}quasi-homothetic} utility functions} and \textbf{Engel curves {\color{myred}linear} in lifetime wealth} for aggregation, can be solved from a social planner's problem (or representative agent CE).
\end{itemize}
\underline{\textbf{Results}}:
\begin{itemize}
    \item[-] The distribution of wealth is \textbf{\color{myred}irrelevant} for aggregate allocations
    \item[-] As capital accumulates, the wealth distribution \textbf{\color{myred}can change} along equilibrium path converging to the steady state, depending on preferences
    \item[-] the initial wealth distribution \textbf{\color{myred}persists}: it matters \textbf{forever}
    \item[-] wealth distribution at $t\geq 0$ can be \textbf{\color{myred}Lorenz ranked}, \textbf{\color{myred}Lorenz improving} re-distributions \textbf{increase} social welfare.
\end{itemize}

\vspace{2pt}
\underline{\textbf{The setting of the problem is}}

\vspace{2pt}
\begin{tikzpicture}
\node [redbox] (box){%
    \begin{minipage}{0.315\textwidth}
    \color{myred}
    \scriptsize
    \begin{itemize}
        \item[-] \underline{\textbf{\textit{consumer}}}
        \begin{itemize}
            \item[-] \textbf{N} types, indexed by $i$
            \item[-] measure of type $i$ consumers is $\mu^i$, and $\sum_i\mu^i =1$ (\textit{aggregate=per capita})
            \item[-] consumers are \textbf{infinitely lived}
            \item[-] lifetime utility functions $u(c^i)=\sum^\infty_{t=0}\beta^t u(c_t^i)$ can be aggregated
            
            some examples are: $u(c^i_t)=\omega \log\left(\bar{c}+c^i_t\right)$, $u(c^i_t)=\omega \left(\bar{c} +\eta c^i_t\right)^{1-\sigma}$, $u(c^i_t)=-\bar{c}\exp(-\eta c^i_t)$
        \end{itemize}
        \item[-] \underline{\textbf{\textit{firm}}}
        \begin{itemize}
            \item[-] produce single final good with aggregate(per capita) capital: $y_t=f(k_t)$
        \end{itemize}
        \item[-] \underline{\textbf{\textit{ownership}}}
        \begin{itemize}
            \item[-] \textbf{firm} owns capital, initial capital $k_0$, then produce, invest, generate \textbf{profits} and distribute profits to firm owners as \textbf{dividends}
            \item[-] \textbf{consumer} own the firm: type $i$ endowed with number of ownership shares $s_0^i$, which pay dividends and 
            $$
            \sum^i\mu^is_0^i=1
            $$
            that is, aggregate wealth normalized
        \end{itemize}
        \item[-] \underline{\textbf{\textit{market structure}}}:
        \begin{itemize}
            \item[-] AD market in consumption goods $c_t^i$, with AD price $p_t$ (normalize $p_0=1$)
            \item[-] Sequential market in consumption goods $c^i_t$ (price $p_t$) and ownership share $s^i_t$ (prcie $q_t$);
        \end{itemize}
    \end{itemize}
    \end{minipage}
};
\node[redtitle, right=4pt] at (box.north west) {Chatterjee: setting};
\end{tikzpicture}

Set up the problem and solve it as before

\vspace{2pt}
\begin{tikzpicture}
\node [bluebox] (box){%
    \begin{minipage}{0.315\textwidth}
    \color{myblue}
    \scriptsize
    The firm maximizes discounted present value of profits, given $k_0$:
    $$
    \max \sum^\infty_{t=0}p_t\left(f(k_t)+(1-\delta)k_t-k_{t+1}\right)
    $$
    the profit is simply the aggregate(per capita) dividends:
    $$
    d_t =f(k_t)+(1-\delta)k_t-k_{t+1}
    $$
    \end{minipage}
};
\node[bluetitle, right=4pt] at (box.north west) {Chatterjee: firm's problem};
\end{tikzpicture}

FOC w.r.t. $k_{t+1}$ gives: 
$$
-p_t+p_{t+1}\left(f'(k_{t+1})+1-\delta\right)=0\Rightarrow\frac{p_{t+1}}{p_t} = \frac{1}{f'(k_{t+1})+1-\delta}
$$
which gives the gross interest rate $r=\frac{p_t}{p_{t+1}}$. Since $p_0=1$, we have 
$$
p_t = \prod^{t}_{\tau=1}\frac{1}{f'(k_{\tau})+1-\delta}
$$
which is just a function of $k_t$.

Meanwhile, consumers' problem is
$$
\max_{c^i_t}\sum^{\infty}_{t=0}\beta^t u(c^i_t)
$$
s.t.
$$
\sum^\infty_{t=0}p_tc^i_t \leq s_0^i\sum^\infty_{t=0}p_t d_t
$$
the AD budget constraint is: the life-time consumption value does not exceed the life-time dividend value.

For convenience, set up the problem in sequential market:

\vspace{2pt}
\begin{tikzpicture}
\node [bluebox] (box){%
    \begin{minipage}{0.315\textwidth}
    \color{myblue}
    \scriptsize
    $$
    \max_{\{c^i_t,s_{t+1}^i\}}\sum^{\infty}_{t=0}\beta^t u(c^i_t)
    $$
    s.t.
    $$
    c^i_t+q_ts_{t+1}^i=s_t^i(d_t+q_t)
    $$
    here, consumption good is the numeraire at each period, and ownership sums to 1 at each $t$:
    $$
    \sum_i \mu^i s^i_t =1,\ \forall t
    $$
    \end{minipage}
};
\node[bluetitle, right=4pt] at (box.north west) {Chatterjee: consumer's problem in SM};
\end{tikzpicture}

In the setting of Chatterjee (\textbf{no uncertainty}), AD and SM allocations coincide:
\begin{itemize}
    \item[-] FOCs align:
    \begin{itemize}
        \item[$\cdot$]\textbf{AD}: $\frac{u'(c^i_t)}{p_t}=\beta \frac{u'(c^i_{t+1})}{p_{t+1}}$
        \item[$\cdot$]\textbf{SM}: $q_tu'(c^i_t)=\beta u^i(c^i_{t+1})(q_{t+1}+d_{t+1})$
    \end{itemize}
    let 
    $$
    \frac{p_{t+1}}{p_t} = \frac{q_t}{q_{t+1}+d_{t+1}} \Rightarrow q_t = \frac{p_{t+1}}{p_t}(q_{t+1}+d_{t+1})
    $$
    solve this forward, with $p_0=1$, get
    $$
    q_0 = \sum^T_{t=0}p_{t+1}d_{t+1} + p_{T+1}q_{T+1} \xrightarrow{T\rightarrow\infty} \sum^{\infty}_{t=1}p_{t+1}d_{t+1}+\lim_{T\rightarrow\infty}p_{T+1}q_{T+1}
    $$
    by imposing TVC on $\lim_{T\rightarrow\infty}p_{T+1}q_{T+1}$ such that consumer does not die with positive asset valuations, get the \textbf{\color{myred}fundamental value}
    \begin{align*}
        q_0 &= \sum^{\infty}_{t=0}p_{t+1}d_{t+1} & q_t &= \sum^\infty_{\tau=1}\frac{p_{t+\tau}}{p_t}d_{t+\tau}
    \end{align*}
    use this, establish the equivalence of budget constraints
\item[-] BCs align: starts from the $t=0$ BC of SM
\begin{align*}
   & & c_0^i+q_0 s_1^i &= s_0^i(d_0+q_0) =s_0^i\left( d_0 +\frac{p_1}{p_0}(d_1+q_1) \right)\\
    \text{\color{myred}iterate forward} & & &= s_0^i \sum^{\infty}_{t=0}p_td_t\\
    \text{\color{myred}adding $t=1$} & & c_0^i+q_0s_1^i+c_1^ip_1+q_1s_2^ip_1&=s_0^i\sum^{\infty}_{t=0}p_td_t+s_1^i(q_1+d_1)p_1\\
    p_1(q_1+d_1)=q_0 & & c_0^i+c_1^ip_1+q_1s_2^ip_1 &= s_0^i\sum^{\infty}_{t=0}p_td_t\\
    \text{\color{myred}iterate forward} & & \sum^{\infty}_{t=0}p_tc_t^i = s_0^i\sum^{\infty}_{t=0}p_td_t
\end{align*}
which is the AD BC.
\end{itemize}

If define wealth of agent $i$ at $t$ as
$$
w_t^i= s_t^i(d_t+p_t) = s_t^i\sum^{\infty}_{\tau=0}\frac{p_{t+\tau}}{p_t}d_{t+\tau}
$$
then the aggregate(per capita) wealth at $t$ is
$$
w_t = \sum_t\mu^i w_t^i=\sum^i\mu^is_t^i(d_t+q_t)=d_t+q_t
$$
here 
$$
\frac{w_t^i}{w_t} = \frac{s_t^i(d_t+q_t)}{\sum^i\mu^is_t^i(d_t+q_t)} = s_t^i \Rightarrow \frac{s_{t+1}^i}{s_t^i} = \frac{w_{t+1}^i/w_{t+1}}{w_t^i/w_t}
$$

We can set up the consumer's problem w.r.t. $w_t^i$

\vspace{2pt}
\begin{tikzpicture}
\node [bluebox] (box){%
    \begin{minipage}{0.315\textwidth}
    \color{myblue}
    \scriptsize
    $$
    \max\sum^{\infty}_{\tau=0}\beta^{\tau} u(c^i_{t+\tau})
    $$
    s.t.
    $$
    \sum^{\infty}_{\tau=0}\frac{p_{t+\tau}}{p_t}c_{t+\tau}^i=w_t^i
    $$
    \end{minipage}
};
\node[bluetitle, right=4pt] at (box.north west) {Chatterjee: consumer's problem, type $i$, date $t$};
\end{tikzpicture}
The Lagrange is
$$
\mathcal{L}= \sum^{\infty}_{\tau=0}\beta^{\tau} u(c^i_{t+\tau}) + \lambda^i\left[ w_t -\sum^{\infty}_{\tau=0}\frac{p_{t+\tau}}{p_t}c_{t+\tau}^i \right]
$$
FOC gives
\begin{align*}
    \frac{\partial \mathcal{L}}{\partial c_{t+\tau}}=0 & \Rightarrow \beta^{\tau} u'(c^i_{t+\tau}) = \lambda^i \frac{p_{t+\tau}}{p_t}\\
    \frac{\partial \mathcal{L}}{\partial c_t}=0 & \Rightarrow u'(c^i_t) = \lambda^i\ {\color{myblue}(\tau =0)}
\end{align*}
then get the familiar Euler equation
$$
\beta^{\tau}u'(c^i_{t+\tau}) = u'(c_t^i) \frac{p_{t+\tau}}{p_t}
$$
in addition,
\begin{align*}
    \sum^{\infty}_{\tau=0}\frac{p_{t+\tau}}{p_t}c_{t+\tau}^i = w_t^i &\Rightarrow & \frac{w_{t+1}^i}{w_t^i} &= \frac{\sum^{\infty}_{\tau=0}\frac{p_{t+1+\tau}}{p_{t+1}}c_{t+1+\tau}^i}{\sum^{\infty}_{\tau=0}\frac{p_{t+\tau}}{p_t}c_{t+\tau}^i} = \frac{p_t}{p_{t+1}}\frac{\sum^{\infty}_{\tau=0}p_{t+1+\tau}c_{t+1+\tau}^i}{\sum^{\infty}_{\tau=0}p_{t+\tau}c_{t+\tau}^i}\\
    & & & = \frac{p_t}{p_{t+1}}\frac{\sum^{\infty}_{\tau=0}p_{t+\tau}c_{t+\tau}^i-p_tc_t^i}{\sum^{\infty}_{\tau=0}p_{t+\tau}c_{t+\tau}^i} = \frac{p_t}{p_{t+1}}\left(1- \frac{p_tc_t^i}{\sum^{\infty}_{\tau=0}p_{t+\tau}c_{t+\tau}^i}\right)\\
    & \Rightarrow &  \frac{w_{t+1}^i}{w_t^i}& = \frac{p_t}{p_{t+1}}\left(1-\frac{c_t^i}{w_t^i}\right)
\end{align*}

And with First Welfare Theorem, set up the social planner's problem:

\vspace{2pt}
\begin{tikzpicture}
\node [bluebox] (box){%
    \begin{minipage}{0.315\textwidth}
    \color{myblue}
    \scriptsize
    $$
    \max\sum^{\infty}_{t=0}\beta^t u(c_t)
    $$
    s.t.
    $$
    c_t+k_{t+1}-(1-\delta) k_t= f(k_t)
    $$
    which, with CE FOCs, gives the \textbf{steady state} for the aggregate economy:
    \begin{align*}
        \frac{p_t}{p_{t+1}}&=\frac{1}{\beta} = 1-\delta+f'(k^*) & f'(k^*)&=\frac{1}{\beta}-1+\delta\\
        c^*=&d^*=f(k^*)-\delta k^* & q^* &=  \sum^{\infty}_{\tau=1}p_{t+\tau}d_{t+\tau} = d^*\frac{\beta}{1-\beta}\\
        w^* &= d^* + q^* = \frac{d^*}{1-\beta}
    \end{align*}
    \end{minipage}
};
\node[bluetitle, right=4pt] at (box.north west) {Chatterjee: social planner's problem};
\end{tikzpicture}

Next, we check some examples

\subsubsection*{Example 1: CRRA utility}
Given
$$
u(c)=\frac{c^{1-\sigma}-1}{1-\sigma}
$$
Then the Euler equation is
$$
\beta^{\tau}\left(c_{t+\tau}^i\right)^{-\sigma} = u'\left(c_t^i\right)^{-\sigma}\frac{p_{t+\tau}}{p_t}
\Rightarrow c_{t+\tau}^i = c_t^i\left( \frac{p_{t+\tau}}{p_t} \beta^{-\tau} \right)^{-1/\sigma}$$
plug it back in the AD budget constraint:
$$
\sum^{\infty}_{\tau=0}\frac{p_{t+\tau}}{p_t}c_{t+\tau}^i = w_t^i \Rightarrow \sum^{\infty}_{\tau=0}\frac{p_{t+\tau}}{p_t}\left( \frac{p_{t+\tau}}{p_t} \beta^{-\tau}\right)^{-1/\sigma} c_t^i = w_t^i\Rightarrow c_t^i = \frac{w_t^i}{\sum^{\infty}_{\tau=0} \left(\frac{p_{t+\tau}}{p_t}\right)^{1-1/\sigma} \beta^{\tau/\varsigma}}
$$
hence, for CRRA utility, we have

\vspace{2pt}
\begin{tikzpicture}
\node [bluebox] (box){%
    \begin{minipage}{0.315\textwidth}
    \color{myblue}
    \scriptsize
    For an agent of type $i$ at $t$, her consumption is linear of type $i$'s wealth at $t$:
    $$
    c_t^i = \frac{w_t^i}{\sum^{\infty}_{\tau=0} \left(\frac{p_{t+\tau}}{p_t}\right)^{1-1/\sigma} \beta^{\tau/\sigma}} = c_t^i = b(P^t)w_t^i
    $$
    where $P^t = g\left(\sum^{\infty}_{\tau=0}p_{t+\tau}\right)$, it is a function of $k$, hence $b(P^t)$ is \textbf{common} across all agents.
    
    Next, the aggregate(per capita) consumption is
    \begin{align*}
        c_t &=\sum_i\mu^N_i c_t^i = b(P^t)\sum^N_i \mu^iw^i_t = b(P^t)w_t = b(P^t)(d_t+p_t)\\
        &= b\left(P^t\right)\left[ \left(f(k_t)+(1-\delta)k_t-k_{t+1}\right) + \prod_{\tau=1}^t\frac{1}{f'(k_{\tau})+1-\delta} \right]
    \end{align*}
    Therefore, aggregate(per capita) consumption is a function of aggregate capital stock $k_t$.
    
    In steady state:
    \begin{align*}
        & c_{t+\tau}^i = c_t^i\left( \frac{p_{t+\tau}}{p_t} \beta^{-\tau} \right)^{-1/\sigma} \Rightarrow \frac{p_{t+\tau}}{p_t} = \beta^\tau \Rightarrow c_t^i = \frac{w_t^i}{\sum^{\infty}_{\tau=0}\beta^{\tau-\tau/\sigma}\beta^{\tau/\sigma}} = (1-\beta)w_t^i\\
         \Rightarrow & w^i_{ss} \equiv s^i(f(k^*)-\delta k^* +q^*) = s^i\left(f(k^*)-\delta k^* + \frac{1}{p_t}\sum^{\infty}_{\tau=1}p_{t+\tau}d_{t+\tau}\right)\\
         \Rightarrow & w_{ss}^i = s^i\left(\frac{f(k^*)-\delta k^*}{1-\beta}\right)
    \end{align*}
    In steady state, the wealth share is constant, and a function of steady-state aggregate capital. And individuals' shares of wealth are constant:
    \begin{align*}
        & \frac{w_{t+1}^i}{w_t^i} = \frac{p_t}{p_{t+1}}\left(1-\frac{c_t^i}{w_t^i}\right) = \frac{p_t}{p_{t+1}}\left(1-b(P^t)\right) = \frac{p_t}{p_{t+1}}\left(1-\frac{c_t}{w_t}\right) = \frac{w_{t+1}}{w_t}\\
        \Rightarrow & \frac{s_{t+1}^i}{s_t^i} = \frac{w^i_{t+1}/w_{t+1}}{w_t^i/w_t} = 1
    \end{align*}

    \end{minipage}
};
\node[bluetitle, right=4pt] at (box.north west) {Chatterjee: CRRA utility};
\end{tikzpicture}

\subsubsection*{Example 2: log HARA utility}
Given
$$
u(c) = \log(\bar{c}+c)
$$
where $\bar{c}<0$, representing the \textbf{subsistence} consumption. Then the Euler equation is 
$$
\beta^{\tau}\frac{1}{\bar{c}+c_{t+\tau}^i} = \frac{1}{\bar{c}+c_t^i}\frac{p_{t+\tau}}{p_t} \Rightarrow c_{t+\tau}^i = \beta^\tau \frac{p_t}{p_{t+\tau}}(c_t^i+\bar{c})-\bar{c}
$$
plug it back in the AD budget constraint:
\begin{align*}
    \sum^{\infty}_{\tau=0}\frac{p_{t+\tau}}{p_t}c_{t+\tau}^i &= \sum^{\infty}_{\tau=0}\frac{p_{t+\tau}}{p_t}\beta^{\tau}\frac{p_t}{p_{t+\tau}}(c_t^i+\bar{c})- \sum^{\infty}_{\tau=0}\frac{p_{t+\tau}}{p_t}\bar{c}\\
    &= (c_t^i+\bar{c})\sum^{\infty}_{\tau=0}\beta^{\tau} -\bar{c}\sum^{\infty}_{\tau=0}\frac{p_{t+\tau}}{p_t} = \frac{c_t^i+\bar{c}}{1-\beta} -\bar{c}\sum^{\infty}_{\tau=0}\frac{p_{t+\tau}}{p_t} = w_t^i\\
    \Rightarrow c_t^i &= \bar{c}\left[(1-\beta)\sum^{\infty}_{\tau=0}\frac{p_{t+\tau}}{p_t}-1\right]+(1-\beta)w_t^i
\end{align*}
if all agents have the same $\bar{c}$ and $\beta$, aggregation still holds, but $c_t/w_t$ differs across agents:
$$
\frac{c_t^i}{w_t^i} = \frac{\bar{c}\left[(1-\beta)\sum^{\infty}_{\tau=0}\frac{p_{t+\tau}}{p_t}-1\right]}{w_t^i}+(1-\beta)
$$
hence, the wealth share dynamics are more interesting than the CRRA case:

\vspace{2pt}
\begin{tikzpicture}
\node [bluebox] (box){%
    \begin{minipage}{0.315\textwidth}
    \color{myblue}
    \scriptsize
    In steady state:
    \begin{align*}
        & \beta^{\tau}\frac{1}{\bar{c}+c_{t+\tau}^i} = \frac{1}{\bar{c}+ c_t^i}\frac{p_{t+\tau}}{p_t} \Rightarrow c_t^i = c_{t+\tau}^i = c_{ss}^i\\
         \Rightarrow & c_{ss}^i = \bar{c}\left[(1-\beta)\sum^{\infty}_{\tau=0}\frac{p_{t+\tau}}{p_t}-1\right]+(1-\beta)w_{ss}^i = \bar{c}\left[(1-\beta)\sum^{\infty}_{\tau=0}\beta^{\tau}-1\right]+(1-\beta)s^i_t(d^+q^*)\\
         \Rightarrow & c_{ss}^i = (1-\beta)s_t^i(d^*+q^*)\Rightarrow s_t^i=s_{ss}^i,\forall t
    \end{align*}
    that is, the wealth share is constant, like in the CRRA case.
    
    However, individuals' shares of wealth are generally \textbf{NOT} constant:
    \begin{align*}
    & \frac{s_{t+1}^i}{s_t^i} \lesseqgtr 1\Leftrightarrow  \frac{w^i_{t+1}/w_{t+1}}{w_t^i/w_t} \lesseqgtr 1 \Leftrightarrow \frac{c_t^i}{w_t^i}\gtreqless \frac{c_t}{w_t} \\
    \Leftrightarrow & \frac{\bar{c}\left[(1-\beta)\sum^{\infty}_{\tau=0}\frac{p_{t+\tau}}{p_t}-1\right]}{w_t^i} \gtreqless \frac{\bar{c}\left[(1-\beta)\sum^{\infty}_{\tau=0}\frac{p_{t+\tau}}{p_t}-1\right]}{w_t}
    \end{align*}
    which depends on
    \begin{itemize}
        \item[-] sign of $\bar{c}$: assumed to be negative, representing subsistence consumption
        \item[-] sign of $\left[(1-\beta)\sum^{\infty}_{\tau=0}\frac{p_{t+\tau}}{p_t}-1\right]$: depends on whether the economy is \textbf{growing} or \textbf{decaying} towards the steady state
        \item[-] $w_t^i$
    \end{itemize}
    \end{minipage}
};
\node[bluetitle, right=4pt] at (box.north west) {Chatterjee: Log HARA utility};
\end{tikzpicture}

Specifically:
\begin{itemize}
    \item[-] if $k_0<k_{ss}$: economy is growing towards steady state, then $\forall t$
    \begin{align*}
        &k_{t}<k_{ss}\Leftrightarrow f'(k_t)>f'(k_{ss}) \Leftrightarrow \frac{p_{t+1}}{p_t} = \frac{1}{f'(k_{t+1})+1-\delta} <\frac{1}{f'(k_{ss})+1-\delta} =\beta\\
        \Rightarrow & (1-\beta)\sum^{\infty}_{\tau=0}\frac{p_{t+\tau}}{p_t}-1 < (1-\beta)\sum^{\infty}_{\tau=0}\beta^{\tau} - 1 = 0 \Rightarrow {\color{myred}\bar{c}\left[(1-\beta)\sum^{\infty}_{\tau=0}\frac{p_{t+\tau}}{p_t}-1\right]>0}
    \end{align*}
    this leads to 
    \begin{align*}
        \frac{s_{t+1}^i}{s_t^i} \lesseqgtr 1 \Leftrightarrow w_t^i \lesseqgtr w_t
    \end{align*}
    that is: in a \textbf{growing} economy, initially poorer agents have \textbf{\color{myred}lower} growth rates of wealth  (\textbf{\textit{\color{myred}falling more behind}}), and the distribution of wealth is \underline{\textbf{\color{myred}LESS equal}}.
    \item[-] if $k_0 > k_{ss}$: economy is decaying towards steady state, then $\forall t$
    \begin{align*}
        &k_{t}>k_{ss}\Leftrightarrow f'(k_t)<f'(k_{ss}) \Leftrightarrow \frac{p_{t+1}}{p_t} = \frac{1}{f'(k_{t+1})+1-\delta} >\frac{1}{f'(k_{ss})+1-\delta} =\beta\\
        \Rightarrow & (1-\beta)\sum^{\infty}_{\tau=0}\frac{p_{t+\tau}}{p_t}-1 > (1-\beta)\sum^{\infty}_{\tau=0}\beta^{\tau} - 1 = 0 \Rightarrow {\color{myred}\bar{c}\left[(1-\beta)\sum^{\infty}_{\tau=0}\frac{p_{t+\tau}}{p_t}-1\right]<0}
    \end{align*}
    this leads to
    \begin{align*}
        \frac{s_{t+1}^i}{s_t^i} \lesseqgtr 1 \Leftrightarrow w_t^i \gtreqless w_t
    \end{align*}
    that is: in a \textbf{growing} economy, initially poorer agents have \textbf{\color{myred}higher} growth rates of wealth (\textbf{\textit{\color{myred}catching up}}), and the distribution of wealth is \underline{\textbf{\color{myred}MORE equal}}.
    
\end{itemize}

\subsubsection*{Lorenz domination}
\vspace{2pt}
\begin{tikzpicture}
\node [bluebox] (box){%
    \begin{minipage}{0.315\textwidth}
    \color{myblue}
    \scriptsize
    \underline{\textbf{definition}}: a distribution of wealth ($\left\{\mu^is_t^i\right\}$) at $t$ \textbf{\textit{Lorenz-dominates}} a distribution of wealth ($\left\{\mu^is_{t+1}^i\right\}$) at $t+1$ if $\forall K\leq N$
    $$
    \sum^K_{i=1}\mu^is_t^i \geq \sum^K_{i=1}\mu^is_{t+1}^i
    $$
    and strict inequality holds for some $K$.
    \end{minipage}
};
\node[bluetitle, right=4pt] at (box.north west) {Lorenz domination};
\end{tikzpicture}

After defining Lorenz domination, we have the two general result of Chatterjee

\vspace{2pt}
\begin{tikzpicture}
\node [redbox] (box){%
    \begin{minipage}{0.315\textwidth}
    \color{myred}
    \scriptsize
    If
    \begin{itemize}
        \item[-] $\bar{c}\left(k_t-k_{ss}\right)>0$: $\left\{\mu^is_t^i\right\}$ Lorenz-\textbf{dominates} $\left\{\mu^is_{t+1}^i\right\}$
        \item[-] $\bar{c}\left(k_t-k_{ss}\right)<0$: $\left\{\mu^is_t^i\right\}$ is  Lorenz-\textbf{dominated} by $\left\{\mu^is_{t+1}^i\right\}$
        \item[-] $\bar{c}\left(k_t-k_{ss}\right)=0$: $\left\{\mu^is_t^i\right\} = \left\{\mu^is_{t+1}^i\right\}$
    \end{itemize}
    \end{minipage}
};
\node[redtitle, right=4pt] at (box.north west) {Chatterjee: theorem 1};
\end{tikzpicture}

An easy proof uses the example 2 discussion: suppose $\bar{c}(k_t-k_{ss})>0$, in example 2, this means subsistence consumption ($\bar{c}<0$) and growing economy towards steady state ($k_t<k_{ss}$), then $\frac{s_{t+1}^i}{s_t^i} \lesseqgtr 1 \Leftrightarrow w_t^i \lesseqgtr w_t$, ordering the $N$ agents and select $K$ s.t. $w_t^K\leq w_t,w_t^{K+1}>w_t$, then 
\begin{align*}
    \sum^J_{i=1}\mu^is_{t+1}^i &\leq \sum^J_{i=1}\mu^is_t^i, &\forall J\leq K\\
    \sum^N_{i=J+1}\mu^i s_{t+1}^i > \sum^N_{i=J+1}\mu^i s_{t}^i \Rightarrow \sum^J_{i=1}\mu^i s_{t+1}^i &< \sum^J_{i=1}\mu^i s_{t}^i, &\forall J>K
\end{align*}
which completes the proof. 

How $\bar{c}\neq 0$ works here? It is efficient for the \textbf{ratio of marginal utilities} across any two agents constant, when $\bar{c}\neq 0$, this means $\forall t$
$$
\psi^{i,j} u'(c_t^i+\bar{c}) = u'\left(c_t^j,\bar{c}\right) \xRightarrow{u(c)=\log(c)} c_t^i+\bar{c} = \psi^{i,j}\left(c_t^j+\bar{c}\right)
$$
then 
$$
\frac{c_t^i}{c_t^j} = \frac{\psi^{i,j}\left(c_t^j+\bar{c}\right)}{c_t^j}-\frac{\bar{c}}{c_t^j} = \psi^{i,j} + (\psi^{i,j}-1)\frac{\bar{c}}{c_t^j}
$$
if the economy grows; agent $i$ is poorer, that is $\psi^{i,j}<1$; and $\bar{c}<0$, then $c_t^i/c_t^i>\psi^{i,j}$ and diminishes towards it as economy grows: initially, poorer agents consume relatively more, leading to a slower growth of wealth.

\vspace{2pt}
\begin{tikzpicture}
\node [redbox] (box){%
    \begin{minipage}{0.315\textwidth}
    \color{myred}
    \scriptsize
    If two economies at $t$ are identical, except that the \textbf{distribution of wealth} in economy 1 \textbf{Lorenz-dominates} that in economy 2:
    $$
    \sum^K_{i=1}\mu^is_{1,t}^i \geq \sum^K_{i=1}\mu^is_{2,t}^i,\  \forall K \leq N
    $$
    and strict inequality holds for some $K$. Then the distribution of wealth in economy 1 will \textbf{Lorenz-dominates} that in economy 2 for \textbf{ALL} $\tau >t$, i.e., the inequality persists.
    \end{minipage}
};
\node[redtitle, right=4pt] at (box.north west) {Chatterjee: theorem 2};
\end{tikzpicture}

A straightforward proof is: since $w_{1,t}=w_{2,t}$, we have 
$$
\sum^K_{i=1}\mu^is_{1,t}^i \geq \sum^K_{i=1}\mu^is_{2,t}^i \Rightarrow \sum^K_{i=1}\mu^i\frac{w_{1,t}^i}{w_{1,t}} \geq \sum^K_{i=1}\mu^i\frac{w_{2,t}^i}{w_{2,t}} \Rightarrow \sum^K_{i=1}\mu^iw_{1,t}^i \geq \sum^K_{i=1}\mu^iw_{2,t}^i
$$
since $\frac{w^i_{t+1}}{w^i_t} =\frac{p_t}{p_{t+1}}(1-\frac{c^i_t}{w^i_t})$, then
\begin{align*}
    \sum^K_{i=1}\mu^i(s_{1,t+1}^i-s_{2,t+1}^i) &= \frac{1}{w_{t+1}}\sum^K_{i=1}\mu^i(w_{1,t+1}^i-w_{2,t+1}^i)\\ 
    &=\frac{1}{w_{t+1}}\frac{p_t}{p_{t+1}}\sum^{K}_{i=1}\mu^i\left( w_{1,t}^i -c_{1,t}^i -w_{2,t}^i+c_{2,t}^t \right)
\end{align*}
the optimal consumption is
$$
c_{t}^i = \bar{c}\left[ (1-\beta)\sum^{\infty}_{\tau=0}\frac{p_{t+\tau}}{p_t}-1 \right]+(1-\beta)w_t^i \Rightarrow {\color{myred}w_t^i-c_t^i = \beta w_t^i - \bar{c}\left[ (1-\beta)\sum^{\infty}_{\tau=0}\frac{p_{t+\tau}}{p_t}-1 \right]}
$$
then
\begin{align*}
    \sum^K_{i=1}\mu^i(s_{1,t+1}^i-s_{2,t+1}^i) &=\frac{1}{w_{t+1}}\frac{p_t}{p_{t+1}}\sum^{K}_{i=1}\mu^i\left( w_{1,t}^i -c_{1,t}^i -w_{2,t}^i+c_{2,t}^t \right)\\
    &= \beta \frac{1}{w_{t+1}}\frac{p_t}{p_{t+1}}\sum^{K}_{i=1}\mu^i\left( w_{1,t}^i-w_{2,t}^i \right)
\end{align*}
hence, the Lorenz-dominance is preserved for any $\tau>t$. And this result holds no matter the distributions in the two economies are improving or worsening over time.

\vspace{2pt}
\begin{tikzpicture}
\node [orangebox] (box){%
    \begin{minipage}{0.315\textwidth}
    \color{myorange}
    \scriptsize
    \begin{itemize}
        \item[-] \underline{\textbf{Obs 1}}: 
        
        The distribution of wealth is uniquely determined at all $t$ on equilibrium paths converging to the steady state, given preferences, technology and initial wealth.
        
        HOWEVER, if the economy starts and remains in a steady state equilibrium forever, only aggregate wealth is determined, the distribution is not
        \item[-] \underline{\textbf{Obs 2}}:
        
        The assumed market structure matters: if agents were autarkic (no trades in AD or stock market), any initial wealth distribution will converge to a perfectly equal one, since the capital stock of each agent converges to the steady state per capital stock, \textbf{independent} of initial conditions.
        
        \item[-] \underline{\textbf{Obs 3}}:
        
        If agents have different discount factor $\beta^i$, the aggregation is \textbf{not possible}.
        
        \item[-] \underline{\textbf{Obs 4}}:
        For a social welfare function of the form 
        $$
        \sum^{\infty}_{t=0}\theta^t W(c_t^1,\cdots,c_t^N)
        $$
        with continuous, symmetric, \textbf{strictly quasi-concave} $W(\cdot)$ and $0<\theta<1$, an economy with a Lorenz dominating initial distribution of wealth will be ranked higher.    
    \end{itemize}
    \end{minipage}
};
\node[orangetitle, right=4pt] at (box.north west) {Chatterjee: some observations};
\end{tikzpicture}

%%%%%%%%%%%%%%%%%%%%%%%%%%%%%%%
%%%%%%%%%%%%%%%%%%%%%%%%%%%%%%%
%%%%%%%%%%%%%%%%%%%%%%%%%%%%%%%
%%%%%%%%%%%%%%%%%%%%%%%%%%%%%%%
% MACRO 2: Pablo's part
%%%%%%%%%%%%%%%%%%%%%%%%%%%%%%%
\newpage
\section*{Real Business Cycle Model}
\subsection*{Empirical observations}
\subsubsection*{Hodrick-Prescott Filter}
The intuition of the Hodrick-Prescott Filter is: any time series can be decomposed into:
\begin{itemize}
    \item[-] trend: it follow the actual series \textbf{closely}, and moves \textbf{smoothly}
    \item[-] cyclical component
\end{itemize}

\vspace{2pt}
\begin{tikzpicture}
\node [redbox] (box){%
    \begin{minipage}{0.315\textwidth}
    \color{myred}
    \scriptsize
    For a variable $X$, define the trend $\hat{X}$ as 
    $$
    \hat{X}=\arg\max_{\hat{X}_t}\sum^T_{t=1} \left(X_t-\hat{X}_t\right)^2+\lambda\sum^{T-1}_{t=2}\left[\left(\hat{X}_{t+1}-\hat{X}_t\right)-\left(\hat{X}_t-\hat{X}_{t-1}\right)\right]^2
    $$
    where
    \begin{itemize}
        \item[-] $X_t-\hat{X}_t$: distance between trend and actual variable
        \item[-] $\left(\hat{X}_{t+1}-\hat{X}_t\right)-\left(\hat{X}_t-\hat{X}_{t-1}\right)$: change in the trend's growth rate
    \end{itemize}
    \end{minipage}
};
\node[redtitle, right=4pt] at (box.north west) {Hodrick-Prescott Filter};
\end{tikzpicture}

The core parameter of this decomposition is \underline{$\lambda$}, it governs how \textbf{smooth} the trend must be, $\lambda = 1600$ for quarterly data is a conventional choice.

\underline{\textbf{How to implement HP filter}}:
Solve $$\max_{\hat{X}_t}\sum^T_{t=1} \left(X_t-\hat{X}_t\right)^2+\lambda\sum^{T-1}_{t=2}\left[\left(\hat{X}_{t+1}-\hat{X}_t\right)-\left(\hat{X}_t-\hat{X}_{t-1}\right)\right]^2$$ get FOC
\begin{align*}
    & \left(X_t-\hat{X}_t\right)=\lambda \left( \hat{X}_{t+1}+ \hat{X}_{t-1}-2\hat{X}_t\right)\\
\Rightarrow & \hat{X}_t+\lambda \left(\hat{X}_{t+1}-2\hat{X}_t+ \hat{X}_{t-1}\right) =X_t
\end{align*}
In matrix form, it can be rewritten as 
$$
\hat{X}_t = \left( I+\lambda K'K \right)^{-1}X_t
$$
where $K$ is a $(T-2)\times T$ matrix:
$$
K= \begin{pmatrix}
1 &-2 & 1 \\
 & 1 &-2 & 1 \\
 & &\ddots & \ddots &  \ddots\\
 & & & 1 & -2 & 1 &\\
 & & & & 1 & -2 & 1\\
\end{pmatrix}
$$

\subsubsection*{Stylized facts}
Next, apply HP-filter to some variables of interest, of which one is total factor productivity, measured with Solow residual.

\vspace{2pt}
\begin{tikzpicture}
\node [redbox] (box){%
    \begin{minipage}{0.315\textwidth}
    \color{myred}
    \scriptsize
    For $Y=AF(K,L)$, differentiate it get
    \begin{align*}
        &\mathrm{d}F=\mathrm{d}A\cdot F(K,L)+AF_K(K,L)\cdot \mathrm{d}K +AF_L(K,L)\cdot \mathrm{d}L\\
        \xRightarrow{\div Y} & \frac{\mathrm{d}Y}{Y}=\frac{\mathrm{d}A}{A}+\frac{AF_K\cdot K}{AF(K,L)}\frac{\mathrm{d}K}{K}+\frac{AF_L\cdot L}{AF(K,L)}\frac{\mathrm{d}L}{L}\\
        \Rightarrow & \mathrm{d}\log Y= \mathrm{d}\log A + \frac{AF_K\cdot K}{AF(K,L)}\mathrm{d}\log K+\frac{AF_L\cdot L}{AF(K,L)}\mathrm{d}\log L
    \end{align*}
    assume perfect competition: $w=AF_L,r^K=AF_K$, then 
    \begin{align*}
        \mathrm{d}\log Y &=\mathrm{d}\log A+\frac{r^KK}{Y}\mathrm{d}\log K+\frac{wL}{Y}\mathrm{d}\log L\\
        & =\mathrm{d}\log A+\alpha_K\mathrm{d}\log K+\alpha_L \mathrm{d}\log L
    \end{align*}
    Then, the Solow residual is then 
    $$
    \mathrm{d}\log A =\mathrm{d}\log Y-\alpha_K\mathrm{d}\log K -\alpha_L \mathrm{d}\log L
    $$
    \end{minipage}
};
\node[redtitle, right=4pt] at (box.north west) {Solow residual};
\end{tikzpicture}
\vspace{1pt}

We then have the following stylized facts:
\begin{itemize}
    \item[-] standard deviation of GDP is 1.5\%~2\%
    \item[-] \underline{\textbf{pro-cyclical}} variables: $C$, $I$, $A$, working hours and imports are \textbf{highly positively correlated} with $Y$
    \item[-] \underline{\textbf{counter-cyclical}} variable: unemployment
    \item[-] \underline{\textbf{acyclical}} variables: real wages, inflation (very mildly pro-cyclical), government spending, exports (mildly pro-cyclical)
    \item[-] consumption of non-durables and services is less volatile than GDP, consumption of durables (especially investment) is more volatile than GDP
\end{itemize}

\subsection*{Model}
\underline{\textbf{Main idea}}: start from a two-period NGM, include only \textbf{real} variables instead of nominal variables, and add shocks to this model to \textit{mimic} the business cycle.

\vspace{2pt}
\begin{tikzpicture}
\node [bluebox] (box){%
    \begin{minipage}{0.315\textwidth}
    \color{myblue}
    \scriptsize
    Technology is given by
    \begin{align*}
        Y_1 &= F_1(L) & \text{1st period just uses labor}\\
        Y_2 &= F_2(K) &\text{2nd period just uses capital}\\
        K &= I_1 & \text{no capital in period 1}\\
        Y_1 & =c_1 +I &\text{no government, closed economy}\\
        c_2 &= Y_2 & \text{only 2 periods}
    \end{align*}
    \end{minipage}
};
\node[bluetitle, right=4pt] at (box.north west) {Technology};
\end{tikzpicture}

\underline{\textbf{Notice that}}:
\begin{itemize}
    \item[-] capital stock in 1st period is the result of \textbf{past decisions}, taken as given.
    \item[-] 2nd period production is increasing and concave in capital, that is $F_2'(K)>0,F_2''(K)<0$.
\end{itemize}

\vspace{2pt}
\begin{tikzpicture}
\node [bluebox] (box){%
    \begin{minipage}{0.315\textwidth}
    \color{myblue}
    \scriptsize
    Representative household has preferences:
    $$
    u(c_1)+v(l)+\beta u(c_2)
    $$
    where $l$ is leisure time, $\beta$ is the discount factor. Time endowment is normalized to 1.
    \end{minipage}
};
\node[bluetitle, right=4pt] at (box.north west) {Preferences};
\end{tikzpicture}

\underline{\textbf{Notice that}}: in the basic model, separable preferences is assumed. In general, we could have
\begin{itemize}
    \item[-] Non-separability of consumption consumption and leisure (for labor supply)
    \item[-] Non-time separability (for asset pricing)
\end{itemize}

\vspace{2pt}
\begin{tikzpicture}
\node [bluebox] (box){%
    \begin{minipage}{0.315\textwidth}
    \color{myblue}
    \scriptsize
    The market structure in this model is
    \begin{itemize}
        \item[-] competitive market for labor in period 1: priced with the real wage $w$ (period-1 good as numeraire)
        \item[-] competitive market for asset: priced with the real interest $r$ (exchange $1$ period-1 good for $1+r$ period-2 goods)
    \end{itemize}
    \end{minipage}
};
\node[bluetitle, right=4pt] at (box.north west) {Markets};
\end{tikzpicture}
\vspace{2pt}

Then, the equilibrium is characterized by the following problems:

\vspace{2pt}
\begin{tikzpicture}
\node [redbox] (box){%
    \begin{minipage}{0.315\textwidth}
    \color{myred}
    \scriptsize
    Household solves
    $$
    \max_{c_1,c_2,l,a}u(c_1)+v(l)+\beta u(c_2)
    $$
    s.t.
    \begin{align*}
        c_1 +a & = w(1-l)+\pi_1 +\pi^I\\
        c_2 & = (1+r)a+\pi_2
    \end{align*}
    where $a$ is the asset, $\pi_1,\pi_2$ are the profits of owning the representative firm, $\pi^I$ is the profit of investment.
    \end{minipage}
};
\node[redtitle, right=4pt] at (box.north west) {Household's problem};
\end{tikzpicture}

\underline{\textbf{Solve it}}: Lagrange is 
\begin{align*}
    \mathcal{L}= & u(c_1)+v(l)+\beta u(c_2)\\
    &+\lambda\left(w(1-l)+\pi_1+\pi^I-c_1-a\right) +\mu\left((1+r)a+\pi_2-c_2\right)
\end{align*}

FOC gives
\begin{align*}
    u'(c_1) -\lambda&=0\\
    \beta u'(c_2) -\mu&=0\\
    v'(l)-\lambda w&=0\\
    -\lambda + \mu(1+r)&=0
\end{align*}
rearrange, get
\begin{align*}
    u'(c_1) & =\beta(1+r)u'(c_2) & \text{intertemporal Euler equation}\\
    v'(l)&=u'(c_1)w & \text{intratemporal labor supply}
\end{align*}

\vspace{2pt}
\begin{tikzpicture}
\node [redbox] (box){%
    \begin{minipage}{0.315\textwidth}
    \color{myred}
    \scriptsize
    Firms solve
    \begin{itemize}
        \item[-] period 1:
        $$
        \pi_1 =\max_LF_1(L)-wL
        $$
        \item[-] period 2:
        $$
        \pi_2 =\max_K F_2(K)-r^KK
        $$
    \end{itemize}
    \end{minipage}
};
\node[redtitle, right=4pt] at (box.north west) {Production firm's problem};
\end{tikzpicture}

\underline{\textbf{Solve it}}: FOC gives
\begin{align*}
    w &=F'_1(L)\\
    r^K&= F'_2(K)
\end{align*}

\vspace{2pt}
\begin{tikzpicture}
\node [redbox] (box){%
    \begin{minipage}{0.315\textwidth}
    \color{myred}
    \scriptsize
    Investment firms solve
    $$
    \pi^I =\frac{r^K}{1+r}I-I
    $$
    in general, with depreciation rate $\delta$,
    $$
    \pi^I = \frac{1-\delta + r^K}{1+r}I-I
    $$
    \end{minipage}
};
\node[redtitle, right=4pt] at (box.north west) {Investment firm's problem};
\end{tikzpicture}

\underline{\textbf{Solve it}}: FOC gives
$$r^K=1+r$$
or in general, $r^K=r+\delta$.

\vspace{2pt}
Then the \underline{\textbf{\textit{equilibrium}}} is defined as

\vspace{2pt}
\begin{tikzpicture}
\node [bluebox] (box){%
    \begin{minipage}{0.315\textwidth}
    \color{myblue}
    \scriptsize
    \begin{itemize}
    \item[-] an allocation $\left\{ c_1,c_2,l,Y_1,Y_2,L,K,I \right\}$
    \item[-] a price vector $\left\{w,r,r^K\right\}$
\end{itemize}
such that
\begin{itemize}
    \item[-] $\{c_1,c_2,l\}$ solves HH's utility maximization problem, taking prices as given.
    \item[-] $\{Y_1,L\}$,$\{Y_2,K\}$,$I$ solves firms' problems, taking prices as given.
    \item[-] market clear
    \begin{itemize}
        \item[-] goods: $Y_1=c_1+I$, $Y_2=c_2$
        \item[-] labor: $l=1-L$
        \item[-] capital: $K=I$
    \end{itemize}
\end{itemize}
    \end{minipage}
};
\node[bluetitle, right=4pt] at (box.north west) {Equilibrium};
\end{tikzpicture}

And since there is no externality, the First Welfare Theorem holds, social planner's problem coincides with the competitive equilibrium:
$$
\max_{c_1,c_2,l,Y_1,Y_2,L,K,I}u(c_1)+v(l)+\beta u(c_2)
$$
s.t.
\begin{align*}
    Y_1 &=F_1(L)\\
    l&=1-L\\
    Y_2 &+F_2(K)\\
    K&I\\
    Y_1 &= c_1+I\\
    Y_2 &=c_2
\end{align*}

\subsubsection*{Solution}
The solution to this model is characterized by the production function $Y_1=F_1(L)$ and three equilibrium equations:
\begin{align*}
    v'(1-L)&=F_1'(L)u'(c_1) & \Rightarrow \frac{\mathrm{d}c_1}{\mathrm{d}L}<0\\
    u'(c_1)&=\beta F_2'(K)u'(F_2(K)) & \Rightarrow \frac{\mathrm{d}c_1}{\mathrm{d}K}>0\\
    Y_1 &= u'^{-1}\left[\beta F_2'(K)u'(F_2(K))\right]+K & \Rightarrow \frac{\mathrm{d}Y_1}{\mathrm{d}K}>0
\end{align*}

and by the graphs
\begin{center}
    \begin{tikzpicture}[scale=1.2]
    % basics
    \draw[color=black, thick] (0,0) -- (1,0) node[below] {$L$} -- (3,0) node[below] {$K$} -- (4,0);
    \draw[color=black, thick] (0,0) -- (0,1) node[left] {$c$} -- (0,3) node[left] {$Y$} -- (0,4);
    \draw[color=black, thick] (4,0) -- (4,4);
    \draw[color=black, thick] (0,4) -- (4,4);
    \draw[color=black, thick] (2,0) -- (2,4);
    \draw[color=black, thick] (0,2) -- (4,2);
    
    \draw[domain =0.2:1.8, blue!45!black, thick, smooth, variable=\x] plot ({\x},{(\x)^0.7+2});
    \draw[domain =0.2:1.8, blue!45!black, thick, smooth, variable=\x] plot ({\x},{0.7/(3*(\x))});
    \draw[domain =2.6:3.4, blue!45!black, thick, smooth, variable=\x] plot ({\x},{ ((0.98*1+1)*(\x-2.5)+2});
    \draw[domain =2.2:3.8, blue!45!black, thick, smooth, variable=\x] plot ({\x},{ (0.98*1)*(\x-2)});

\end{tikzpicture}
\end{center}

The model can incorporate several shocks:
\begin{itemize}
    \item[-] \underline{\textbf{short-lived productivity shock}}: $Y_1 =AF_1(L)$
    \begin{align*}
        Y_1&=AF_1(L) &\Rightarrow Y-L \text{ curve} \uparrow\\
        v'(1-L)&=AF_1'(L)u'(c_1) &\Rightarrow c-L \text{ curve} \uparrow\\
    \end{align*}
    there are income and substitution effects on labor supply, if \textbf{substitution effect} dominates, it looks like a business cycle.
    \item[-] \underline{\textbf{impatience}}: $\beta$ increases
    \begin{align*}
        u'(c_1)&=\beta F_2'(K)u'(F_2(K)) & \Rightarrow c-K \text{ curve} \leftarrow\\
    Y_1 &= u'^{-1}\left[\beta F_2'(K)u'(F_2(K))\right]+K & \Rightarrow Y-K \text{ curve} \leftarrow
    \end{align*}
    \item[-] \underline{\textbf{laziness or taxes}}:
    \begin{align*}
        v'(1-L)&=(1-\tau)F_1'(L)u'(c_1) &\Rightarrow c-L \text{ curve} \downarrow\\
        \theta v'(1-L)&=F_1'(L)u'(c_1) &\Rightarrow c-L \text{ curve} \downarrow
    \end{align*}
    \item[-] \underline{\textbf{optimism about the future}}: one example is positive productivity shock of $F_2(K)$
    \begin{align*}
        u'(c_1) &= \beta AF_2'(K)u'(AF_2(K)) &\Rightarrow c-K \text{ curve} \downarrow\\
        Y_1 &= u'^{-1}\left[\beta AF_2'(K)u'(AF_2(K))\right]+K & \Rightarrow Y-K \text{ curve} \downarrow
    \end{align*}
\end{itemize}

\subsection*{Infinite horizon version}
\vspace{2pt}
\begin{tikzpicture}
\node [redbox] (box){%
    \begin{minipage}{0.315\textwidth}
    \color{myred}
    \scriptsize
    Preferences (specially, separable utility)
    $$
    u(c_t,L_t)
    $$
    Production function
    $$
    Y_t = A_t F(K_t,(1+g)^t L_t)
    $$
    where $g$ is the trend of productivity growth, $A_t$ is an exogenous stochastic process, can be Markov chain ($A_t=A(s_t)$, $s_t$ is the realized state at $t$), or even AR(1): $\log A_t=\rho\log A_{t-1}+\varepsilon_t$.
    \end{minipage}
};
\node[redtitle, right=4pt] at (box.north west) {Basic settings of Infinite horizon RBC model};
\end{tikzpicture}

Here, the First Welfare Theorem still holds, hence we have the social planner's problem (sequential market) as 
$$
\max_{c(s^t),L(s^t),K(s^t)}\sum^{\infty}_{t=0}\beta^t\sum_{s^t}\Pr (s^t)u\left( c(s^t),L(s^t) \right)
$$
s.t.
$$
c(s^t)+K(s^{t+1})\leq A(s_t)F(K(s^t),L(s^t))+(1-\delta)K(s^t),\ K_0\text{ given}
$$
where $s^t=\{s_0,s_1,\cdots,s_t\}$ is the history of states up to $t$. FOC gives:
\begin{align*}
    \beta^t\Pr(s^t)u_c(c(s^t),L(s^t))-\lambda(s^t)&=0\\
    \beta^t\Pr(s^t)u_L(c(s^t),L(s^t))+\lambda(s^t)A(s^t)F_L(K(s^t),L(s^t))&=0\\
    -\lambda(s^t)+\sum_{s_{t+1}}\left[\lambda(s^{t+1})A(s_{t+1})F_K(K(s^{t+1},L(s^{t+1}))+(1-\delta)\right]&=0 
\end{align*}
leading to 

\vspace{2pt}
\begin{tikzpicture}
\node [redbox] (box){%
    \begin{minipage}{0.315\textwidth}
    \color{myred}
    \scriptsize
    the intra-temporal consumption-labor condition:
$$
-u_L(c(s^t),L(s^t))=u_c(c(s^t),L(s^t))A(s^t)F_L(K(s^t),L(s^t))
$$
and the intertemporal Euler equation:
\begin{align*}
    u_c(c(s^t),L(s^t))= \beta \mathbb{E}_t &\left\{u_c(c(s^{t+1}),L(s^{t+1}))\right.\\
    &\left.\cdot [A(s_{t+1})F_K(K(s^{t+1}),L(s^{t+1}))+(1-\delta)]\right\}
\end{align*}
and the transversality condition
$$
\lim_{t\rightarrow\infty}\beta^t\mathbb{E}_t\left[u_c\left(c(s^t),L(s^t)\right)\cdot A(s_t)\cdot F_K\left(K(s^t),L(s^t)\right)K(s^t)\right]=0
$$
    \end{minipage}
};
\node[redtitle, right=4pt] at (box.north west) {Character equations};
\end{tikzpicture}

\vspace{2pt}
\subsubsection*{Recursive form}
If rewrite this problem in the recursive form, we would have:
\begin{itemize}
    \item[-] state variables: $s,K$
    \item[-] Bellman equation:
    $$
    V(K,s)=\max_{c,L,K'}u(c,L)+\beta \sum\Pr(s'\mid s)V(K',s')
    $$
    s.t.
    $$
    c+K'\leq A(s)F(K,L)+(1-\delta)K
    $$
    \item[-] FOC w.r.t. $K'$ is
    $$
    \beta \sum \Pr(s'\mid s)V_K(K'(K,s),s')-\lambda(K,s)=0
    $$
    with envelope theorem
    $$
    V_K(K,s)=\lambda(K,s)\left[ A(s)F_K(K,L(K,s))+(1-\delta) \right]
    $$
    get Euler equation:
    \begin{align*}
        u_c(c(K,s),L(K,s)) =& \beta\sum\Pr(s'\mid s)u_c(c(K'(K,s),s'),L(K'(K,s),s'))\\
        &\cdot \left[ A(s')F_K\left(K'(K,s),L(K'(K,s),s')\right)+(1-\delta) \right]
    \end{align*}
    
\end{itemize}

\rule{0.238\textwidth}{0.4pt}
\vspace{1pt}

\textbf{In summary}, RBC model has 3 fundamental equations:

\begin{tikzpicture}
\node [bluebox] (box){%
    \begin{minipage}{0.315\textwidth}
    \color{myblue}
    \scriptsize
    \begin{itemize}
        \item[-] Intertemporal Euler equation:
        $$
        u_c(c_t,L_t)=\beta \mathbb{E}_t\left( u_c(c_{t+1},L_{t+1})\left[ A_{t+1}F_K(K_{t+1},L_{t+1})+(1-\delta)\right] \right)
        $$
        \item[-] Intratemporal consumption-labor equation:
        $$
        -u_L(c_t,L_t)=u_c(c_t,L_t)A_tF_L(K_t,L_t)
        $$
        \item[-] Resource constraint:
        $$
        c_t+K_{t+1}-(1-\delta)K_t\leq A_tF(K_t,L_t)
        $$
    \end{itemize}
    \end{minipage}
};
\node[bluetitle, right=4pt] at (box.north west) {RBC equations};
\end{tikzpicture}

\subsubsection*{Tune functions to fit long-term facts}
Both the production function and the preference function would need to be tuned to fit the long-term facts:

\begin{itemize}
    \item[-] constant factor shares of production:
    $$
    Y_t=A_tF(K_t,L_t,X_t)
    $$
    where $X_t=(1+g)^t$ must have labor-augmenting long-term growth:
    $$
    Y_t=A_tF(K_t,(1+g)^tL_t)
    $$
    or Cobb-Douglas
    $$
    Y_t=A_t(1+g)^tK_t^{\alpha}L_t^{1-\alpha}
    $$
    \item[-] constant labor: with the intratemporal equation
    $$
    -u_c(c_t,L_t)=u_c(c_t,L_t)A_tF_L(K_t,X_tL_t)
    $$
    we need $A=1$, $X$ fits any value, $\frac{K}{X}=\bar{k},\frac{C}{X}=\bar{c},L=\bar{L}$ are all constant, that is 
    $$
    -u_L(\bar{c}X,\bar{L})=u_c(\bar{c}X,\bar{L})XF_L(\bar{k}X,\bar{L}X)=u_c(\bar{cX},\bar{L})XF_L(\bar{k},\bar{L}),\ \forall X
    $$
    hence we need $-\frac{u_L(\bar{c}X,\bar{L})}{u_c(\bar{c}X,\bar{L})X}$ to be constant for all $X$. If
    \begin{itemize}
        \item[-] utility function is separable: suppose $u(c,L)=\frac{c^{1-\gamma}}{1-\gamma}-v(L)$, then 
        $$
        -\frac{u_L(\bar{c}X,\bar{L})}{u_c(\bar{c}X,\bar{L})X}=\frac{v'(\bar{L})}{(\bar{c}X)^{-\gamma}X}=\frac{v'(\bar{L})}{\bar{c}^{-\gamma}X^{1-\gamma}}
        $$
        for it to be constant for all $X$, $\gamma=1$, which is just log preferences.
        \item[-] utility function is non-separable: suppose
        $u(c,L)=\frac{(cv(L))^{1-\eta}}{1-\eta}$, then
        $$
        -\frac{u_L(\bar{c}X,\bar{L})}{u_c(\bar{c}X,\bar{L})X} = \frac{(\bar{c}Xv(\bar{L}))^{-\nu}\bar{c}Xv'(\bar{L})}{(\bar{c}Xv(\bar{L}))^{-\nu}Xv(\bar{L})}=\frac{\bar{c}v'(\bar{L})}{v(\bar{L})}
        $$
        which is constant for all $X$, regardless of the value of $\nu$.
    \end{itemize}
\end{itemize}

\subsubsection*{Tune parameters to fit empirical observations}
If assume:
\begin{itemize}
    \item[-] production: Cobb-Douglas $F(K,L)=AK^{\alpha}L^{1-\alpha}$, with AR(1) productivity $\log A_t=\rho \log A_{t-1}+\epsilon_t,\epsilon\sim \mathcal{N}(0,\sigma^2_{\epsilon})$
    \item[-] preferences: two specifications
    \begin{itemize}
        \item[-] easier: $u(c,L)=\log(c)+\psi\log(1-L)$
        \item[-] more complicated: $u(c,L)=\log(c)-\psi\frac{L^{1+\phi}}{1+\phi}$
    \end{itemize}
    \item[-] capital depreciation: $K_{t+1}=K_t(1-\delta)+I_t$
    \item[-] intertemporal discounting: $u_c(c_t,L_t)=\beta (1+r_{t+1})u_c(c_{t+1},L_{t+1})$
\end{itemize}

Then, the parameters are tuned to be:
\begin{itemize}
    \item[-] $\alpha=\frac{1}{3}$, recently $\alpha =0.4$ 
    \item[-] $\psi=3$: use intratemporal equation $-u_L(c,L)=u_c(c,L)F_L(K,L)$ and the easier preference, get
    $$
    \frac{\psi}{1-L} = \frac{1}{c}F_K(K,L)
    $$
    since $\frac{F_K(K,L)L}{Y}=1-\alpha$, we get
    $$
    \frac{\psi}{1-L} = \frac{1}{c}(1-\alpha)\frac{Y}{L}=(1-\alpha)\left(\frac{c}{Y}\right)^{-1}\frac{1-L}{L}
    $$
    here $\frac{c}{Y}$ comes from the national accounting data, normally it is around 0.8; $L=\text{employment rate}(65\%)\times\frac{\text{working hours}(35)}{\text{waking hours}(112)}$ is around 0.2. Hence, get $\psi=3$.
    \item[-] Frisch elasticity $\frac{1}{\phi} \in [0.4,1]$: again, use interatemporal equation, get 
    $$
    \psi L^{\phi}=\frac{1}{c}w\Rightarrow L^{\phi}=\frac{w}{c\psi}
    $$
    then take nature log:
    $$
    \log L=\frac{1}{\phi}\left[ \log w-\log c -\log \psi \right]
    $$
    and for the log-log preferences, we have
    $$
    \frac{\psi}{1-L}=\frac{w}{c}\Rightarrow\log(1-L)=\log(\psi)+\log(c)-\log(w)
    $$
    this gives 
    $$
    \frac{\partial \log L}{\partial \log w}=\frac{\partial \log L}{\partial \log(1-L)}\frac{\partial\log(1-L)}{\partial\log w}=\frac{1-L}{L}=4
    $$
    \item[-] $\rho$ and $\sigma_{\epsilon}$: compute time series of Solow residual, detrend using HP filter, and estimate $\rho=0.979$, $\sigma_{\epsilon}=0.0072$.
    \item[-] $\delta$: using $\frac{K}{Y}$ ratio to estimate:
    \begin{align*}
        &\frac{K_{t+1}}{Y_{t+1}}=\frac{K_t(1-\delta)+I_t}{(1+g)Y_t}=\frac{K_t}{Y_t}\frac{(1-\delta)+\frac{I_t}{K_t}}{1+g}\\
        \Rightarrow & g=\frac{I}{K}-\delta\Rightarrow \delta = \frac{I}{Y}\frac{Y}{K}-g
    \end{align*}
    \item[-] $\beta$: in steady state, Euler equation gives $c_t^{-\gamma}=\beta (1+r_{t+1})c_{t+1}^{-\gamma}$, hence
    $$
    (1+g)\gamma =\beta(1+r)\Rightarrow \beta =\frac{(1+g)^{\gamma}}{1+r}
    $$
    let $r=r^K-\delta=F_K(K,L)-\delta=\frac{1-\alpha}{K/Y}-\delta$
\end{itemize}

\subsubsection*{Detrend}
To rule out trend growth in business cycle models, i.e. $g=0$, do detrending, $\tilde{c}_t =\frac{c_t}{X_t}$, the (non-separable) preference will be 
\begin{align*}
    \sum^{\infty}_{t=0}\beta^t\frac{\left[c_t v(L_t)\right]^{1-\nu}}{1-\nu} &= \sum^{\infty}_{t=0}\beta^t \frac{\left[ \tilde{c}_t X_t v(L_t) \right]^{1-\nu}}{1-\nu}\\
   \xRightarrow{X_t=(1+g)^t} & = \sum^{\infty}_{t=0}\beta^t \left((1+g)^t\right)^{1-\nu}\frac{\left[ \tilde{c}_t v(L_t) \right]^{1-\nu}}{1-\nu}\\
   \xRightarrow{\tilde{\beta}\equiv\beta(1+g)^{1-\nu}} &= \sum^{\infty}_{t=0}\tilde{\beta}^t\frac{\left[ \tilde{c}_t v(L_t) \right]^{1-\nu}}{1-\nu}
\end{align*}
and the detrended budget constraint is
\begin{align*}
    &c_t+K_{t+1}\leq A_tF(K_t,X_tL_t)+(1-\delta)K_t\\
    \Rightarrow &\tilde{c}_tX_t+\tilde{K}_{t+1}X_{t+1}\leq A_tX_tF(\tilde{K}_t,L_t)+(1-\delta)\tilde{K}_tX_t\\
    \Rightarrow & \tilde{c}_t+\tilde{K}_{t+1}(1+g) \leq A_tF(\tilde{K}_t,L_t)+(1-\delta)\tilde{K}_t
\end{align*}

\subsubsection*{Q theory of investment}
Assume $K_{t+1}=g(K_t,I_t)$, that is, the transformation of goods into capital is NOT one-for-one:
\begin{itemize}
    \item[-] Hayashi's capital adjustment costs:
    $$
    g(K_t,I_t)=(1-\delta)K_t+I_t -\frac{\psi}{2}\left(\frac{I_t}{K_t}-\delta\right)^2K_t
    $$
    \item[-] Christinao, Eichenbaum and Evans' capital adjustment costs:
    $$
    g(K_t,I_t,I_{t-1})=(1-\delta)K_t +I_t-\frac{\psi}{2}\left(\frac{I_t}{I_{t-1}}-1\right)^2I_t
    $$
\end{itemize}

The social planner's problem will be:
$$
\max_{c_t,K_t,I_t,L_t} \mathbb{E}\sum\beta^t u(c_t,L_t)
$$
s.t.
$$
c_t+I_t\leq A_tF(K_t,L_t),\ K_{t+1}\leq g(K_t,I_t)
$$
Lagrange:
$$
\mathcal{L}=\mathbb{E}\sum\beta^t u(c_t,L_t)+\lambda_t\left(A_tF(K_t,L_t)-c_t-I_t\right)+\mu_t\left(g(K_t,I_t)-K_{t+1}\right)
$$
FOC:
\begin{align*}
    \frac{\partial\mathcal{L}}{\partial c_t}=0 \Rightarrow& \beta^t u_c(c_t,L_t)-\lambda_t=0\\
    \frac{\partial\mathcal{L}}{\partial L_t}=0 \Rightarrow & \beta^t u_L(c_t,L_t)-\lambda_t A_tF_L(K_t,L_t)=0\\
    \frac{\partial\mathcal{L}}{\partial I_t}=0 \Rightarrow & -\lambda_t+\mu_tg_I(K_t,I_t)=0\\
    \frac{\partial\mathcal{L}}{ \partial K_{t+1}}=0\Rightarrow & -\mu_t +\\
    &\mathbb{E}_t\left(\lambda_{t+1}A_{t+1}F_K(K_{t+1},L_{t+1})+\mu_{t+1}g_K(K_{t+1},I_{t+1})\right)=0
\end{align*}
here,
\begin{itemize}
    \item[-] $\mu_t$ is the marginal value of a unit of installed capital for investment
    \item[-] $\lambda_t$ is the marginal value of a unit of output for consumption
\end{itemize}
Define $q_t=\frac{\mu_t}{\lambda_t}$ as the value of installed capital w.r.t. consumption, get 
$$
\frac{1}{g_I(K_t,I_t)}=q_t
$$
$q_t$ is therefore the price of installed capital. When the relative price of installed capital is high, \textbf{investment} is high.

%%%%%%%%%%%%%%%%%%%%%%%%%%%%%%%%%%%%%%%5
% log linear
\section*{Log-linearization}
\begin{tikzpicture}
\node [redbox] (box){%
    \begin{minipage}{0.315\textwidth}
    \color{myred}
    \scriptsize
    For a variable $X_t$, denote its steady state value $X_{ss}$, its log value $x_t$
    \begin{itemize}
        \item[-] One variable: $X_t = X_{ss}(1+x_t)$
        \item[-] Product of two variables: $X_tY_t = X_{ss}Y_{ss}(1+x_t+y_t)$
        \item[-] function of a variable: $f(X_t) = f(X_{ss})\left(1+\frac{f'(X_{ss})}{f(X_{ss})}X_{ss} x_t\right)$
        \item[-] Multiplicative equation: $X_t^{\theta}Y_t = \alpha Z_t\Rightarrow \theta x_t y_t = z_t $
    \end{itemize}
    \end{minipage}
};
\node[redtitle, right=4pt] at (box.north west) {Log linearization formula};
\end{tikzpicture}

\vspace{2pt}
\underline{\textbf{\color{myred}Example}}: Log linearization of RBC model with 
\begin{align*}
    u(c,L)&=\frac{c^{1-\gamma}}{1-\gamma}-\psi\frac{L^{1+\phi}}{1+\phi} & F(K,L) &=AK^{\alpha}L^{1-\alpha}
\end{align*}
get,
\begin{itemize}
    \item[-] Intertemporal Euler equation $C_t^{-\gamma}=\beta \mathbb{E}_t\left( C_{t+1}^{\gamma}\left[ A_{t+1}K_{t+1}^{\alpha-1}L_{t+1}^{1-\alpha}+(1-\delta)\right] \right)$:
    {\color{myred}\begin{align*}
        -\gamma c_t = \mathbb{E}_t\left[ -\gamma c_{t+1} + (1-\beta(1-\delta)) \left(a_{t+1}+(\alpha-1)k_{t+1}+(1-\alpha)l_{t+1}\right) \right]
    \end{align*}}
    
    \item[-] Intratemporal consumption-labor equation $\psi L^{\phi}_t=C_t^{-\gamma}A_tK_t^{\alpha} L_t^{-\alpha}$:
    \begin{align*}
        \phi l_t = -\gamma c_t + a_t +\alpha k_t -\alpha l_t &\Rightarrow {\color{myred}l_t  = -\frac{1}{\phi+\alpha}\left(-\gamma c_t+a_t+\alpha k_t\right)}
    \end{align*}
    \item[-] Resource constraint $ C_t+K_{t+1}-(1-\delta)K_t = Y_t$
    \begin{align*}
        &C_{ss}(1+c_t)+K_{ss}(1+k_{t+1})-(1-\delta)K_{ss}(1+k_t)= Y_{ss}(1+y_t)\\
        \Rightarrow & C_{ss}c_t +K_{ss}k_{t+1} - (1-\delta)K_{ss}k_t = Y_{ss}y_t\Rightarrow{\color{myred}\frac{C_{ss}}{Y_{ss}}c_t +\frac{K_{ss}}{Y_{ss}}k_{t+1} - (1-\delta)\frac{K_{ss}}{Y_{ss}}k_t = y_t}
    \end{align*}
    \item[-] Production $Y_t=A_tK_t^{\alpha}L_t^{1-\alpha}$:
    $$
    y_t = a_t +\alpha k_t +(1-\alpha)l_t
    $$
\end{itemize}

\section*{Asset pricing}
Notation:
\begin{itemize}
    \item[-] history: $s^t = \left\{s_0,s_1,\cdots,s_t\right\}$
    \item[-] history that begins with $s^t$ and continues with $s_{t+1}$: $\left\{s^t,s_{t+1}\right\}$
    \item[-] income: $y(s^t)$
    \item[-] asset information: price $q(s^t)$, dividends $d(s^t)$
\end{itemize}
The budget constraint is then
\begin{align*}
    c(s^t)+q(s^t)\cdot a_{t+1}(s^t) &\leq W(s^t)\\
    W(s^{t+1}) &= y(s^{t+1})+\left( q(s^{s+1}) + d(s^{t+1}) \right)\cdot a_{t+1}(s^t)
\end{align*}
if assume \textbf{\color{myred}complete market}, \textbf{the asset (AD security) pays $d=1$ and has price $q=0$}, then the BC for complete market is $c(s^t)+q(s^t)\cdot a_{t+1}(s^t) \leq y(s^t)+a_t(s^{t-1})$.

The price of asset is 
\begin{itemize}
    \item[-] at $t=0$:$p_0 = \sum_{t,s^t}q_0(s^t)d(s^t)$
    \item[-] in history $s^t$: $p(s^t) = \frac{\sum_{j,s^j}q_0(s^t,s^j)d(s^t,s^j)}{q_0(s^t)}$
\end{itemize}
where $q_0(s^t) \equiv q(s_0,s_1)q(s^1,s_2)\cdots q(s^{t-1},s_t)$

Household's problem is
$$
\max_{c(s^t),a(s^{t+1})} \sum_{t,s^t}\beta^t \Pr(s^t)u(c(s^t))
$$
s.t.
$$
c(s^t)+ \sum_{s_{t+1}}q(s^t,s_{t+1})\cdot a(s^{t+1}) \leq y(s^t)+\left( q(s^t) + d(s^t) \right)\cdot a(s^{t-1})
$$
FOC gives
\begin{align*}
    \beta^t\Pr(s^t)u'(c(s^t))&=\lambda(s^t)\\
    \lambda(s^t)q(s^t,s_{t+1}) &= \sum_{s_{t+1}}\lambda(s^t,s_{t+1})(q(s^{t+1})+d(s^{t+1}))
\end{align*}
hence 
$$
u'(c(s^t)) = \beta \sum_{s_{t+1}}\Pr(s_{t+1}\mid s^t)u'(c(s^t,s_{t+1}))\frac{q(s^{t+1})+d(s^{t+1})}{q(s^t,s_{t+1})}
$$
define $R(s^t,s_{t+1})\equiv \frac{q(s^{t+1})+d(s^{t+1})}{q(s^t,s_{t+1})}$, then get the Euler equation
$$
1 = \beta \mathbb{E}_t\left[\frac{u'(c_{t+1})}{u'(c_t)} R_{t+1} \right]
$$
this should hold for \textbf{all} assets. For an Arrow security:
$$
u'(c(s^t)) = \beta \Pr(s_{t+1}\mid  s^t)u'(c(s^t,s_{t+1}))\frac{1}{q(s^t,s_{t+1})}\Rightarrow \frac{q(s^t,s_{t+1})}{\Pr(s_{t+1}\mid s^t)} = \beta \frac{u'(c(s^t,s_{t+1}))}{u'(c(s^t))} \equiv m(s^{t+1})
$$
which is the pricing kernel (SDF). Plug it in the Euler equation, get 
\begin{align*}
    &1 = \mathbb{E}_t\left[m(s^{t+1})R_{t+1}\right] = \mathbb{E}_t\left[m(s^{t+1})\right]\mathbb{E}_t\left[R_{t+1}\right] + \mathrm{Cov}(m_{t+1},R_{t+1}) \\
    \Rightarrow & \frac{1}{\mathbb{E}_t\left[m(s^{t+1})\right]} = \mathbb{E}_t\left[R_{t+1}\right] + \frac{\mathrm{Cov}(m_{t+1},R_{t+1})}{\mathbb{E}_t\left[m(s^{t+1})\right]}
\end{align*}
this holds for all assets, \textbf{including risk-free asset}, which means
$$
R^f_{t+1} = \frac{1}{\mathbb{E}_t\left[m_{t+1}\right]}
$$
this \textbf{risk-free} return can always be defined, even without actual risk free assets. Then, get
$$
\mathbb{E}_t(R_{t+1}) = R^f_{t+1} - R^f_{t+1}\mathrm{Cov}(m_{t+1},R_{t+1})
$$
We can also define the \textbf{risk-neutral probabilities}
$$
Q(s_{t+1}\mid s^t) = \frac{\Pr(s_{t+1}\mid s^t)u'\left(c(s^t,s_{t+1})\right)}{\sum_{s_{t+1}}\Pr(s_{t+1}\mid s^t)u'\left(c(s^t,s_{t+1})\right)}
$$
which is just the actual probabilities ($\Pr(s_{t+1}\mid s^t)$), re-weighted by marginal utility, then the Euler equation can be rewritten as
\begin{align*}
    u'(c(s^t)) &= \beta \sum_{s_{t+1}}\Pr(s^t,s_{t+1})u'(c(s^t,s_{t+1}))R(s^{t+1})\\
    &= \sum_{s_{t+1}}\Pr(s^t,s_{t+1})u'(c(s^t,s_{t+1})) \beta\sum_{s_{t+1}}Q(s^t,s_{t+1}) R(s^{t+1})\\
    \frac{1}{\beta}\frac{u'(c(s^t))}{\mathbb{E}_t\left(u'(c(s^t,c_{t+1}))\right)} &= \mathbb{E}^Q_t\left[R(s^{t+1})\right]
\end{align*}
since the LHS is the \textbf{same} for all assets, then $\mathbb{E}^Q_t\left[R(s^{t+1})\right]$ is the \textbf{\color{myred}same} for all assets, including the risk-free asset: $R^f = \mathbb{E}^Q_t \left[R(s^{t+1})\right] = \frac{1}{\mathbb{E}_t[m_{t+1}]}$.

\begin{tikzpicture}
\node [redbox] (box){%
    \begin{minipage}{0.315\textwidth}
    \color{myred}
    \scriptsize
    \begin{itemize}
        \item[-] Euler equation: $u'(c_t) = \beta \mathbb{E}_t\left[u'(c_{t+1})R_{t+1}\right]$
        \item[-] Pricing kernel: $m_{t+1} = \beta \frac{u'(c(s^t,s_{t+1}))}{u'(c(s^t))}$
        \item[-] risk-free rate: $R^f_{t+1}=\frac{1}{\mathbb{E}_t\left[m_{t+1}\right]}$
        \item[-] risk-neutral probabilities: $Q(s_{t+1}\mid s^t)=\frac{\Pr(s_{t+1}\mid s^t)u'\left(c'(s_{t+1},s^t)\right)}{\sum_{s_{t+1}}\Pr(s_{t+1}\mid s^t)u'\left(c'(s_{t+1},s^t)\right)}$
        \item[-] Sharpe ratio: $s^j = \frac{\mathbb{E}(R^j_{t+1})-R^f_{t+1}}{\sigma(R^j_{t+1})}$ and $\lvert s_j \rvert\leq \frac{\sigma(m_{t+1})}{\mathbb{E}(m_{t+1})}$
    \end{itemize}
    \end{minipage}
};
\node[redtitle, right=4pt] at (box.north west) {Summary: asset pricing};
\end{tikzpicture}

\section*{Welfare cost of business cycles}
\subsection*{Lucas' calculation}
Representing the business-cycle consumption path as $c_t = c_0e^{gt}\exp(\epsilon_t)$ where $\mathbb{E}(\exp(\epsilon_t))=1$, then the cost of business cycle is the number $\lambda$ that solves:
$$
\mathbb{E}\left(\sum^{\infty}_{t=0}\beta^t u\left(c_0(1+\lambda)e^{gt}\exp(\epsilon_t)\right)\right) = \mathbb{E}\left(\sum^{\infty}_{t=0}\beta^t u\left(c_0e^{gt}\right)\right)
$$
\subsection*{Alvarez and Jermann's calculation}
This calculation requires less assumptions. Consider
\begin{itemize}
    \item[-] $\left\{c_t\right\}$: business-cycle consumption process
    \item[-] $\left\{C_t\right\}$: an alternative consumption process
\end{itemize}
consider the utility of the consumption process $\left\{c_t\right\}$: $U(\left\{c_t\right\})$ (standard utility is $U(\left\{c_t\right\})=\sum_{s_t}\beta^t \Pr(s^t)u(c(s^t))$). Define $\lambda(\alpha)$ as the solution to
$$
U\left[ (1+\lambda(\alpha))\left\{c_t\right\} \right] = U\left[ (1-\alpha)\left\{c_t\right\} +\alpha \left\{C_t\right\} \right]
$$
which is just the welfare cost, but the alternative is the weighted average of the two processes. Consider $\lambda'(0)$, which is the \textbf{marginal value} of increasing the weight of $\left\{C_t\right\}$.

take derivative w.r.t. $\alpha$:
\begin{align*}
    &\lambda'(\alpha)\sum_{s^t}\left[U_{c(s^t)}\left((1+\lambda(\alpha))\left\{c(s^t)\right\}\right)\cdot c(s^t) \right]\\
    =& \sum_{s^t} \left[ U_{c(s^t)}\left( (1-\alpha)\left\{c(s^t)\right\}+\alpha\left\{C(s^t)\right\} \right) \cdot\left(C(s^t)-c(s^t)\right) \right]
\end{align*}
evaluate it at $\alpha =0$, get
\begin{align*}
   & \lambda'(0)\sum_{s^t}\left[U_{c(s^t)}\left(\left\{c(s^t)\right\}\right)\cdot c(s^t) \right] = \sum_{s^t} \left[ U_{c(s^t)}\left( \left\{c(s^t)\right\} \right) \cdot\left(C(s^t)-c(s^t)\right) \right] \\
\Rightarrow & \lambda'(0) = \frac{\sum_{s^t}\left[ U_{c(s^t)}\left( \left\{c(s^t)\right\} \right) \cdot C(s^t) \right]}{\sum_{s^t}\left[U_{c(s^t)}\left(\left\{c(s^t)\right\}\right)\cdot c(s^t) \right]}-1
\end{align*}
plug in the asset pricing $p_0=\sum_{t,s^t}q_0(s^t)d(s^t)$ and Euler equation $\frac{U_{c(s^t)}\left(\left\{c(s^t)\right\} \right)}{U_{c_0}\left( \left\{c(s^t)\right\} \right)} = q_0\left(s^t\right)$, get
\begin{align*}
    \lambda'(0) &= \frac{\sum_{s^t}\left[ U_{c(s^t)}\left( \left\{c(s^t)\right\} \right) \cdot C(s^t) \right]}{\sum_{s^t}\left[U_{c(s^t)}\left(\left\{c(s^t)\right\}\right)\cdot c(s^t) \right]}-1 = \frac{\sum_{s^t}\left[ q_0(s^t) \cdot C(s^t) \right]}{\sum_{s^t}\left[q_0(s^t) \cdot c(s^t) \right]}-1
\end{align*}
hence, we need an asset that pays dividends \textbf{equal} to the consumption process, which obviously does not exist. Instead of looking at \textbf{price}, looking at \textbf{expected return}:
\begin{align*}
    p\left( \left\{C_t\right\} \right)&= c_0\frac{1+g}{r^f-g} & p\left( \left\{c_t\right\} \right) &= c_0\frac{1+g}{r^c -g} &\Rightarrow \lambda'(0) = \frac{r^c-g}{r^f-g}
\end{align*}
where $g=\mathbb{E}(c_{t+1}/c_t)-1$. How to empirically get $r^c$? Regress $\frac{c_{t+1}}{c_t}=\beta R_{t+1}+\epsilon_t$ where $R_{t+1}$ is the return of a portfolio, $r^c = \hat{\beta}\mathbb{E}(R_{t+1})$

\section*{Money and inflation}
LM equation: 
$$\frac{M^S}{p} = m^D(Y,i)$$
that is, money demand equals money supply.

For households, if real money balance enters the utility function, the problem is
$$
\max \mathbb{E}\sum^{\infty}_{t=0}\beta^t u\left(c_t,L_t,\frac{M_t}{P_t}\right)
$$
s.t.
\begin{align*}
    P_tc_t+P_tk_{t+1}+B_{t+1}+M_{t+1}+T_t &\leq W_tL_t+(P_t(1-\delta)+R_t^K)k_t+(1+i_t)B_t+M_t\\
    P_tc_t& \leq M_t
\end{align*}

where $B_t$ are bonds, $T_t$ is nominal lump-sum tax, $i_t$ is \textbf{nominal} interest rate. FOCs are
\begin{align*}
    \frac{\partial\mathcal{L}}{\partial c_t}=0&\Rightarrow \beta^t u_{c,t}=\lambda_tP_t \\
    \frac{\partial\mathcal{L}}{\partial L_t}=0&\Rightarrow \beta^t u_{L,t} =- \lambda_tW_t \\
    \frac{\partial\mathcal{L}}{\partial k_{t+1}}=0 &\Rightarrow (P_{t+1}(1-\delta)+R_{t+1}^K)\lambda_{t+1}=\lambda_tP_t\\
    \frac{\partial\mathcal{L}}{\partial B_t}=0 &\Rightarrow \lambda_{t+1}(1+i_{t+1}) = \lambda_t\\
    \frac{\partial\mathcal{L}}{\partial M_t}=0 &\Rightarrow \beta^{t+1} \frac{u_{M,t+1}}{P_{t+1}} +\lambda_{t+1}-\lambda_t = 0
\end{align*}
get the characteristic equations of this economy:
\begin{align*}
    u_{L,t} &= -\frac{W_t}{P_t}u_{c,t}\equiv -w_t u_{c,t} & \text{intratemporal}\\
    u_{c,t} &= \beta u_{c,t+1}\left( 1-\delta + \frac{R^K_{t+1}}{P_{t+1}} \right)\equiv \beta u_{c,t+1}(1-\delta+r_{t+1}^K) & \text{intratemporal}\\
    u_{c,t} &= \beta u_{c,t+1} \frac{1+i_{t+1}}{P_{t+1}/P_t} \equiv \beta u_{c,t+1}(1+r_{t+1}) & \text{bond} \\
    i_{t+1} &= \frac{u_{M,t+1}}{u_{c,t+1}} &\text{money}
\end{align*}
the last equation $i_{t+1} = \frac{u_{M,t+1}}{u_{c,t+1}}$ means that as always, the \textbf{relative price equals the {\color{myred}marginal rate of substitution}}. This gives the money demand function of $i$, and it can re-written as
$$
i_{t+1} = \frac{P_{t+1}}{P_t}(1+r) -1 \xRightarrow{\text{steady state}} \frac{u_{M}\left(c_{ss},L_{ss},\frac{M_{t}}{P_{t}}\right)}{u_{c}\left(c_{ss},L_{ss},\frac{M_{t}}{P_{t}}\right)} = \frac{P_{t+1}}{P_t}\frac{1}{\beta}-1
$$
which gives a differential equation $\frac{M_t}{P_t}=L\left(\frac{P_{t+1}}{P_t}\right)$, a nice form is
$$
\frac{M_t}{P_t}=\left(\frac{P_{t+1}}{P_t}\right)^{-\theta} \Rightarrow \log P_t = \frac{1}{1+\theta}\left[ \log M_t + \theta \log P_{t+1}\right]
$$
iterate forward, get
\begin{align*}
     &\log P_t = \frac{1}{1+\theta}\sum^{T-1}_{s=0}\left(\frac{\theta}{1+\theta}\right)^s \log M_{t+s} +\left(\frac{\theta}{1+\theta}\right)^T p_{t+T}\\
     \xrightarrow{T\rightarrow\infty} & \log P_t = \frac{1}{1+\theta}\sum^{\infty}_{s=0}\left(\frac{\theta}{1+\theta}\right)^s \log M_{t+s}
\end{align*}

\section*{New Keynesian Model}
\subsection*{CES model with market power}
Consider a cost-minimization problem:
$$
\min_{C_j}\int_0^1 P_jC_j\mathrm{d}j
$$
s.t.
$$
\left(\int_0^1 C_j^{\frac{\epsilon-1}{\epsilon}}\right)^{\frac{\epsilon}{\epsilon-1}} = C
$$
FOC gives
\begin{align*}
    \frac{P_j}{P_k} &= \left(\frac{C_j}{C_k}\right)^{-\frac{1}{\epsilon}} &\Rightarrow \frac{C_j}{C_k} = \left(\frac{P_j}{P_k}\right)^{-\epsilon}
\end{align*}
this gives $P = \left(\int_0^1 P_j^{1-\epsilon}\mathrm{d}j\right)^{\frac{1}{1-\epsilon}}$ and $C_j = \left(\frac{P_j}{P}\right)^{-\epsilon}C$, the elasticity is $\partial \log C_j/\partial \log P_j = -\epsilon$, hence firm will set the markup as {\color{myred}$\frac{\epsilon}{\epsilon-1}$} over marginal cost.

\subsection*{RBC with market power and sticky price}
The real marginal cost of labor:
$$
\frac{v'(1-L)}{F_L'(L)u'(c)}
$$
where $\frac{1}{F_L'(L)}$ is the unit of labor required to produce one marginal unit of product, $v'(1-L)$ is the disutility of one unit of labor, and evaluate it by consumption by dividing $u'(c)$. Then firm $i$ set a nominal price
$$
p_i = p\frac{\epsilon}{\epsilon-1}\frac{v'(1-L)}{F_L'(L)u'(c)} \xRightarrow{p_i=p} v'(1-L) = {\color{myred}\frac{\epsilon-1}{\epsilon}}F_L'(L)u'(c)
$$
different from the basic RBC equation $v'(1-L)=F_L'(L)u'(c)$ due to the markup.

\subsubsection*{Two-period version}
With 
\begin{align*}
    u'(c_1) &= \beta(1+r)u'(c_2) &\text{Euler equation}\\
    F_2'(K)&= 1+r & \text{investment FOC}\\
    Y_1 &=c_1+K & t=1 \text{ market clearing}\\
    Y_2 &=c_2 = F_2(K) & t=2 \text{ market clearing}
\end{align*}
get \textbf{IS} equation (\textbf{I}nvestment = \textbf{S}avings)
$$
u'(Y_1-K(r)) = \beta(1+r)u'(F(K(r)))
$$
where $K(r) = (F_2')^{-1}(1+r)$. In nominal term
$$
\Delta = u'(Y_1-K(i-\pi))- \beta(1+i-\pi)u'(F(K(i-\pi)))
$$
then
\begin{align*}
    \frac{\mathrm{d}i}{\mathrm{d}Y} = -\frac{\partial \Delta/\partial Y_1}{\partial \Delta/\partial i} <0
\end{align*}
recall the \textbf{LM} equation (\textbf{L}iquidity = \textbf{M}oney supply):
$$
\frac{M^S}{p}=m^D(i,Y) \xRightarrow[p \text{ sticky}]{M^S\text{ fixed}} \frac{\mathrm{d}i}{\mathrm{d}Y} = -\frac{\partial m^D/\partial Y_1{\color{myred}>0}}{\partial m^D/\partial i{\color{myred}<0}} >0
$$

The system of 2 equations (IS-LM) determine the equilibrium $\{i,Y\}$:
\begin{align*}
    u'(Y_1-K(i-\pi)) &= \beta(1+i-\pi)u'(F(K(i-\pi))) & \text{IS}\\
    \frac{M^S}{p} &= m^D(i,Y) & \text{LM}\\
\end{align*}

\begin{center}
\begin{tikzpicture}
\draw[->] (-0.2, 0) -- (4, 0) node[below right] {$Y$};
\draw[->] (0, -0.2) -- (0, 4) node[above] {$i$};
\draw[domain=0.5:3.5, smooth, variable=\x, myred] plot ({\x}, {(\x)})
node[right] {LM: $\frac{M^S}{p}=m^D(i,Y)$};
\draw[domain=0.3:3.6, smooth, variable=\x, myblue] plot ({\x}, {4-\x})
node[right] {IS: $u'(Y_1-K(i-\pi)) =  \beta(1+i-\pi)u'(F(K(i-\pi)))$};
\end{tikzpicture}
\end{center}

If incorporating production shock of $t=2$, $AF_2(K)\Rightarrow K(r,A)=(F_2')^{-1}\left(\frac{1+r}{A}\right)$; and government spending $G\rightarrow c_1=Y_1-K(r)-G$ (both enter IS curve), get 
\begin{align*}
    u'(Y_1-G-K(i-\pi,A)) &= \beta(1+i-\pi)u'(AF(K(i-\pi,A))) & \text{IS}\\
    \frac{M^S}{p} &= m^D(i,Y) & \text{LM}\\
\end{align*}
we can analyze the following shocks:
\begin{itemize}
    \item[-] productivity shock in $t=1$, laziness, tax: \textbf{NO} effect
    \item[-] impatience $\beta \downarrow$: IS curve $\rightarrow$
    \item[-] higher expected inflation $\pi \uparrow$: IS curve $\uparrow$
    \item[-] higher government spending $G\uparrow$: IS curve $\rightarrow$
    \item[-] money supply increase $M^S \uparrow$: LM curve $\downarrow$
\end{itemize}

\subsubsection*{Partially sticky prices}
Suppose $\mu$ measure of firms are sticky-price setters, the price index:
\begin{align*}
    P_1 &= \left(\int_0^1 p_{1,j}^{1-\epsilon}\mathrm{d}j\right)^{\frac{1}{1-\epsilon}} = \left( \int_0^{\mu}(p^s_1)^{1-\epsilon}\mathrm{d}j + \int_{\mu}^1(p^f_1)^{1-\epsilon}\mathrm{d}j \right)^{\frac{1}{1-\epsilon}} \\
    &= \left(\mu (p_1^s)^{1-\epsilon} + (1-\mu)(p_1^f)^{1-\epsilon}\right)^{\frac{1}{1-\epsilon}}
\end{align*}
and flexible price is the marked-up marginal cost $p_1^f = \frac{v'(1-L)}{F_1'(L)u'(c_1)}\frac{\eta}{\eta-1}$ where $\eta = -\frac{q'(p)p}{q(p)}$, then
$$
p_1 = \left(\frac{\mu}{1-(1-\mu)\left(\frac{v'(1-L)}{F_1'(L)u'(c_1)}\frac{\eta}{\eta-1}\right)^{1-\epsilon}}\right)^{\frac{1}{1-\epsilon}}p_1^s
$$
suppose there is \textbf{NO} capital, then $c_1=F_1(L)$, the marginal cost is just a function of $L$: $\chi(L)=\frac{v'(1-L)}{F_1'(L)u'(F_1(L))}$, this function is increasing in $L$, and 
$$
p_1 = \left(\frac{\mu}{1-(1-\mu)\left(\chi(L)\frac{\eta}{\eta-1}\right)^{1-\epsilon}}\right)^{\frac{1}{1-\epsilon}}p_1^s
$$
$p_1$ is also increasing in $L$, hence in $Y=F(L)$ since flexible price setters would raise price. This gives the Phillips Curve. Again, analyze the following shocks:
\begin{itemize}
    \item[-] productivity shock in $t=1$ $AF_1(L)$: Phillips curve $\rightarrow$, $p_1\downarrow$
    \item[-] impatience $\beta \downarrow$: IS curve $\rightarrow\Rightarrow Y\uparrow \Rightarrow p_1\uparrow$
    \item[-] higher expected inflation $\pi \uparrow$: IS curve $\uparrow \Rightarrow Y\uparrow \Rightarrow p_1\uparrow$
    \item[-] higher government spending $G\uparrow$: IS curve $\rightarrow \Rightarrow Y\uparrow \Rightarrow p_1\uparrow$
    \item[-] money supply increase $M^S \uparrow$: LM curve $\downarrow \Rightarrow Y\uparrow \Rightarrow p_1\uparrow$
\end{itemize}

However the sticky-price setters set $p_1^s$? One way is $p_1^s = \mathbb{E}(p_1^f)$, specially, if the stick price setters have perfect expectation, then
\begin{align*}
   & 1 = \left(\frac{\mu}{1-(1-\mu)\left(\chi(L)\frac{\eta}{\eta-1}\right)^{1-\epsilon}}\right)^{\frac{1}{1-\epsilon}} \Rightarrow \mu = 1-(1-\mu)\left(\chi(L)\frac{\eta}{\eta-1}\right)^{1-\epsilon}\\
    \Rightarrow &\chi(L)\frac{\eta}{\eta-1} = \frac{v'(1-L)}{F_1'(L)u'(F_1(L))}\frac{\eta}{\eta-1}=1
\end{align*}
which is just a RBC-with-market-power model (without sticky prices) result, it is a \textbf{vertical} long-run Phillips curve.

\subsubsection*{Derive New Keynesian model}
\begin{tikzpicture}
\node [redbox] (box){%
    \begin{minipage}{0.315\textwidth}
    \color{myred}
    \scriptsize
    assume $U(C_t,L_t)=u(C_t)-v(L_t)$ where $u(C_t) = \frac{C_t^{1-\gamma}-1}{1-\gamma}$, and $v(L)=\psi\frac{L_t^{1+\phi}}{1+\phi}$
    \begin{itemize}
        \item[-] Euler equation: $C_t^{-\gamma} = \beta\mathbb{E}\left[C_{t+1}^{-\gamma}\frac{1+i_{t+1}}{1+\pi_{t+1}}\right]$
        \item[-] labor condition: $\psi L_t^{\phi} = \frac{W_t}{P_t}C_t^{-\gamma}$
        \item[-] market clearing: $Y_t= C_t$
        \item[-] GDP with misallocation: $Y_t=\frac{A_tL_t}{\Delta_t}$ where the productivity/misallocation $\Delta_t = (1-\mu)\left(\frac{1-\mu(1+\pi_t)^{\epsilon-1}}{1-\mu}\right)^{\frac{\epsilon}{\epsilon-1}}+(1+\pi_t)^{\epsilon}\mu\Delta_{t-1}$
        \item[-] firm pricing: $\frac{P_{j,t}}{P_t}=\frac{\epsilon}{\epsilon-1}\frac{X_t}{Z_t}$ where $X_t\equiv \chi_t Y_t^{1-\gamma}+\beta\mu\mathbb{E}_t\left[X_{t+1}\left(\frac{P_{t+1}}{P_t}\right)^{\epsilon}\right]$, $Z_t\equiv Y_t^{1-\gamma}+\beta\mu\mathbb{E}_t\left[Z_{t+1}\left(\frac{P_{t+1}}{P_t}\right)^{\epsilon-1}\right]$
        \item[-] real marginal cost: $\chi_t = \frac{W_t}{P_tA_t}$
        \item[-] inflation: $\frac{1-\mu(1+\pi_t)^{\epsilon-1}}{1-\mu} = \left(\frac{\epsilon}{\epsilon-1}\frac{X_t}{Z_t}\right)^{1-\epsilon}$
    \end{itemize}
    \end{minipage}
};
\node[redtitle, right=4pt] at (box.north west) {NK model Summary: characteristic equations};
\end{tikzpicture}

\vspace{2pt}
And the steady state is 
\begin{itemize}
    \item[-] constant inflation $\pi_{ss}$ (near 0)
    \begin{itemize}
        \item[-] Euler equation: $1+i_{ss} = \frac{1+\pi_{ss}}{\beta}$ 
        \item[-] labor condition: $\psi L_{ss}^{\phi} = w_{ss} Y_{ss}^{-\gamma}$
        \item[-] GDP: $Y_{ss}=\frac{A_{ss}L_{ss}}{\Delta_{ss}}$, where $\Delta_{ss}=\frac{(1-\mu)\left(\frac{1-\mu(1+\pi_{ss})^{\epsilon-1}}{1-\mu}\right)^{\frac{\epsilon}{\epsilon-1}}}{1-\mu(1+\pi_{ss})^{\epsilon}}$
        \item[-] inflation: $\frac{1-\mu(1+\pi_{ss})^{\epsilon-1}}{1-\mu} = \left( \frac{\epsilon}{\epsilon-1}\frac{X_{ss}}{Z_{ss}} \right)^{1-\epsilon}$, this gives steady-state wage $w_{ss} = \frac{\epsilon-1}{\epsilon} \frac{1-\beta\mu(1+\pi_{ss})^{\epsilon}}{1-\beta\mu(1+\pi_{ss})^{\epsilon-1}}\left( \frac{1-\mu}{1-\mu(1+\pi_{ss})^{\epsilon-1}} \right)^{\frac{1}{\epsilon-1}}$
    \end{itemize}
    \item[-] 0 inflation $\pi_{ss}=0$
    \begin{itemize}
        \item[-] Euler equation: $1+i_{ss} = \frac{1}{\beta}$ 
        \item[-] labor condition: $\psi L_{ss}^{\phi} = w_{ss} Y_{ss}^{-\gamma}$
        \item[-] GDP: $Y_{ss}=\frac{A_{ss}L_{ss}}{\Delta_{ss}}$, where $\Delta_{ss}=1$
        \item[-] inflation: $1 = \left( \frac{\epsilon}{\epsilon-1}\frac{X_{ss}}{Z_{ss}} \right)^{1-\epsilon}$, this gives steady-state wage $w_{ss} = \frac{\epsilon-1}{\epsilon}$
    \end{itemize}
\end{itemize}

Next, \textbf{\color{myred}log linearize} the equations around the $\pi_{ss}=0$:
\begin{itemize}
    \item[-] Euler equation: 
    $$-\gamma c_t = \mathbb{E}\left[ -\gamma c_{t+1} + (i_{t+1}-i_{ss}-(\pi_{t+1}-\pi_{ss})) \right] \Rightarrow \mathbb{E}\left[c_{t+1}-\frac{1}{\gamma}(i_{t+1}-\pi_{t+1}-\rho)\right]= c_t$$
    \item[-] labor condition: $\phi l_t = w_t-p_t-\gamma c_t$
    \item[-] GDP with misallocation: $y_t=a_t+l_t-\hat{\Delta}_t$ where the $\left.\frac{\partial \Delta_t}{\partial \pi_t}\right\vert_{\pi=0} = \epsilon\mu(\Delta_{t-1}-1) =0 \Rightarrow \hat{\Delta}_t=0$
    \item[-] real marginal cost: $\hat{\chi}_t = w_t-a_t-p_t$
    \item[-] flexible inflation and overall inflation: $(1+\pi_t)^{1-\epsilon} = (1-\mu)(1+\pi_t^f)^{1-\epsilon}+\mu \Rightarrow \pi_t \simeq (1-\mu)\pi_t^f$
    \item[-] inflation: $\frac{1-\mu(1+\pi_t)^{\epsilon-1}}{1-\mu} = \left(\frac{\epsilon}{\epsilon-1}\frac{X_t}{Z_t}\right)^{1-\epsilon}$
    \begin{itemize}
        \item[-] $x_t \simeq (1-\beta\mu)\left(\hat{\chi}_t +(1-\gamma)y_t\right)+\beta\mu\mathbb{E}_t\left[x_{t+1}+\epsilon\pi_{t+1}\right]$
        \item[-] $z_t \simeq (1-\beta\mu)(1-\gamma)y_t +\beta\mu\mathbb{E}_t\left[z_{t+1}+(\epsilon-1)\pi_{t+1}\right]$
        \item[-] $\pi_t = \frac{1-\mu}{\mu}(x_t-z_t) = \frac{(1-\mu)(1-\beta\mu)}{\mu}\hat{\chi}_t+\beta\mathbb{E}_t(\pi_{t+1})$: Phillips curve
    \end{itemize}
\end{itemize}

Hence, in NK model, we have:
\begin{align*}
    y_t &= \mathbb{E}\left[y_{t+1}-\frac{1}{\gamma} (i_{t+1}-\pi_{t+1}-\rho)\right] & \text{IS equation}\\
    \pi_t &= \frac{(1-\mu)(1-\beta\mu)}{\mu}\hat{\chi}_t+\beta\mathbb{E}_t(\pi_{t+1}) & \text{Phillips curve}\\
    w_t-p_t &= (\gamma+\phi)y_t-\phi a_t &\text{real wage}\\
    \hat{\chi}_t &= w_t-p_t -a_t = (\gamma+\phi)y_t-(1+\phi) a_t &\text{marginal cost}
\end{align*}
If prices are perfectly flexible ($w_t-p_t=a_t$), have
\begin{align*}
    y_t^n &= \frac{1+\phi}{\gamma+\phi}a_t & \text{nature level of output} \\
    l^n_t &= \frac{1-\gamma}{\gamma+\phi}a_t & \text{nature level of employment}\\
    r^n_{t+1} &= \rho + \gamma\frac{1+\phi}{\gamma+\phi}\left(\mathbb{E}[a_{t+1}]-a_t\right) &\text{natural real interest rate}
\end{align*}
then define $x_t = y_{t}-y^n_t$ as the \textbf{\color{myred}output gap}, we have 
\begin{align*}
    x_t &= \mathbb{E}\left[x_{t+1}-\frac{1}{\gamma} (i_{t+1}-\pi_{t+1}-r^n_{t+1})\right] & \text{IS equation}\\
    \pi_t &= \frac{(1-\mu)(1-\beta\mu)}{\mu} (\gamma+\phi)x_t +\beta\mathbb{E}_t(\pi_{t+1})  \equiv \kappa x_t +\beta\mathbb{E}_t (\pi_{t+1}) & \text{Phillips curve}
\end{align*}
where 
\begin{itemize}
    \item[-] $\kappa'(\mu)<0$: less flexible (higher $\mu$), lower $\kappa$
    \item[-] $\kappa'(\gamma),\kappa'(\phi)>0$: $\hat{\chi}_t = (\gamma+\phi)x_t$, higher $\gamma,\phi$ means that marginal cost is sensitive to output gap
\end{itemize}

\begin{tikzpicture}
\node [redbox] (box){%
    \begin{minipage}{0.315\textwidth}
    \color{myred}
    \scriptsize
    \begin{itemize}
        \item[-] Iterate forward IS curve:
        \begin{align*}
            x_t = \mathbb{E}\left[x_{t+1}-\frac{1}{\gamma}(r_{t+1}-r^n_{t+1})\right] = \cdots = -\frac{1}{\gamma}\sum^T_{\tau=1}(r_{t+\tau}-r^n_{t+\tau})+\mathbb{E}_t(x_{t+T})
        \end{align*}
        if $x_t$ does not blow up, get $x_t = -\frac{1}{\gamma}\sum^{\infty}_{\tau=1}(r_{t+\tau}-r^n_{t+\tau})$
        \item[-] iterate forward Phillips curve
        \begin{align*}
            \pi_t = \kappa x_t + \beta\mathbb{E}(\pi_{t+1}) = \kappa \sum^T_{\tau=0}\beta^{\tau}\mathbb{E}(x_{t+\tau}) + \beta^T\mathbb{E}(\pi_{t+T})
        \end{align*}
        if $\pi_{t}$ does not blow up, get $\pi_t = \kappa\sum^{\infty}_{\tau=0}\beta^{\tau}\mathbb{E}(x_{t+\tau})$
        \item[-] Taylor rule (monetary policy): $i_{t+1}=\rho_{t+1}+\lambda_{\pi}\pi_{t}+\lambda_{x}x_t$, where $\rho_{t+1}$ is the target real rate, $\lambda_{\pi}$ is the responsiveness to inflation, $\lambda_x$ is the responsiveness to output gap. Plug in TR into IS, get 
        \begin{align*}
            & x_t = \mathbb{E}\left[x_{t+1}-\frac{1}{\gamma}(\rho_{t+1}+\lambda_{\pi}\pi_{t}+\lambda_{x}x_t-\pi_{t+1}-r^n_{t+1})\right]
        \end{align*}
    \end{itemize}
    
    Then the 3-equation model is reduced to a 2-equation model, which can be solved:
    \begin{align*}
        \pi_t &= \kappa x_t +\mathbb{E}(\pi_{t+1})\\
        (\gamma+\lambda_x)x_t +\lambda_{\pi}\pi_t &= \mathbb{E}[\gamma x_{t+1}+\pi_{t+1}+r^n_{t+1}]-\rho_{t+1}
    \end{align*}
    \end{minipage}
};
\node[redtitle, right=4pt] at (box.north west) {3-equation NK model: properties};
\end{tikzpicture}

In matrix form,
\begin{align*}
    \begin{pmatrix}
    \gamma+\lambda_x & \lambda_{\pi} \\
    -\kappa & 1
    \end{pmatrix}
    \begin{pmatrix}x_t\\ \pi_t \end{pmatrix} = \begin{pmatrix}
    \gamma & 1\\
    0 & \beta 
    \end{pmatrix}\mathbb{E}_t\begin{pmatrix}x_{t+1}\\ \pi_{t+1}\end{pmatrix} - \begin{pmatrix}
    1\\ 0
    \end{pmatrix}\mathbb{E}_t(\rho_{t+1}-r^n_{t+1})
\end{align*}
then let $\mathbf{M} \equiv \begin{pmatrix}
    \gamma+\lambda_x & \lambda_{\pi} \\
    -\kappa & 1
    \end{pmatrix}^{-1}\begin{pmatrix}
    \gamma & 1\\
    0 & \beta 
    \end{pmatrix}$, $\mathbf{B}\equiv\begin{pmatrix}
    \gamma+\lambda_x & \lambda_{\pi} \\
    -\kappa & 1
    \end{pmatrix}^{-1}\begin{pmatrix}
    1\\ 0
    \end{pmatrix}$, and solve forward
\begin{align*}
    \begin{pmatrix} 
    x_t\\ \pi_t 
    \end{pmatrix} &= 
    \mathbf{M} \mathbb{E}_t\begin{pmatrix}
    x_{t+1}\\ \pi_{t+1}\end{pmatrix} 
    - \mathbf{B}\mathbb{E}_t(\rho_{t+1}-r^n_{t+1}) \\
    &= -\mathbf{B}\sum^{T}_{\tau=1}\mathbf{M}^{\tau}\mathbb{E}_t(\rho_{t+1+\tau}-r^n_{t+1+\tau}) + \mathbf{M}^T\mathbb{E}_t\begin{pmatrix}x_{t+1}\\ \pi_{t+1}\end{pmatrix}
\end{align*}
then if both eigenvalues of $\mathbf{M}$ are smaller than 1, that is $\kappa(\lambda_{\pi}-1)+(1-\beta)\lambda_x>0$, the second term vanishes,
$$
\begin{pmatrix} 
    x_t\\ \pi_t 
    \end{pmatrix}= -\mathbf{B}\sum^{\infty}_{\tau=0}\mathbf{M}^{\tau}\mathbb{E}_t(\rho_{t+1+\tau}-r^n_{t+1+\tau})
$$
hence, it the target real rate is the nature rate (\textbf{\color{myred}Divine coincidence}) $\rho_{t+\tau}=r^n_{t+\tau}$, then
$$
\begin{pmatrix} 
    x_t\\ \pi_t 
    \end{pmatrix}= \begin{pmatrix} 
    0\\ 0 
    \end{pmatrix}
$$

\subsubsection*{Adding shocks}
\begin{itemize}
    \item[-] monetary policy shock: $\rho_{t+1}=r^n_{t+1}+\nu_{t+1}$, the shock follows $\nu_{t+1}=\rho_{\nu}\nu_t+\epsilon_t^{\nu}$ 
    \begin{align*}
        \begin{pmatrix} 
        x_t\\ \pi_t 
        \end{pmatrix} &= -\mathbf{B}\sum^{\infty}_{\tau=0}\mathbf{M}^{\tau}\mathbb{E}_t(\rho_{t+1+\tau}-r^n_{t+1+\tau}) = -\mathbf{B}\sum^{\infty}_{\tau=0}\mathbf{M}^{\tau}\mathbb{E}_t(\nu_{t+1+\tau}) = -\mathbf{B}\sum^{\infty}_{\tau=0}\mathbf{M}^{\tau}\rho_{\nu}^{\tau}v_t \\
        &= -\mathbf{B}(I-\rho_v\mathbf{M})^{-1}v_t
    \end{align*}
    $-\mathbf{B}(I-\rho_v\mathbf{M})^{-1}$ is negative, meaning the an expansionary monetary shock \textbf{\color{myred}lower} $x_t$ and $\pi_t$
    \item[-] real-variable shocks: it reflects on the \textbf{reaction} of monetary shocks
    \begin{itemize}
        \item[-] productivity shock: $a_{t+1}=\rho_a a_t +\epsilon^a_{t+1}$, a positive productivity shock \textbf{decreases} the natural interest rate $r^n_{t+1} = \rho- \gamma(1-\rho_a)\left(\frac{1+\phi}{\gamma+\phi}\right)a_t$, to even-out this shock, $i_{t+1}$ needs to be \textbf{lower}, which is a shrinking monetary policy. Hence, a \textbf{\color{myred}positive} productivity shock \textbf{acts like} an \textbf{\color{myred}expansionary} monetary policy shock.
        \item[-] impatience shock: if households become more impatient, the real interest rate will increase, to even-out this shock, $i_{t+1}$ needs to be higher, which is a expansionary monetary policy. Hence, an \textbf{\color{myred}impatience} shock \textbf{acts like} a \textbf{\color{myred}shrinking} monetary policy shock
    \end{itemize}
    \item[-] cost-push shock: $\pi_t = \kappa x_t +\beta\mathbb{E}(\pi_{t+1})+{\color{myred}u_t}$, no divine coincidence anymore, $\begin{pmatrix}
    \gamma+\lambda_x & \lambda_{\pi} \\
    -\kappa & 1
    \end{pmatrix}
    \begin{pmatrix}x_t\\ \pi_t \end{pmatrix} = \begin{pmatrix}
    \gamma & 1\\
    0 & \beta 
    \end{pmatrix}\mathbb{E}_t\begin{pmatrix}x_{t+1}\\ \pi_{t+1}\end{pmatrix} - \begin{pmatrix}
    1 & 0\\ 0 & 1
    \end{pmatrix} \mathbb{E}_t\begin{pmatrix}\rho_{t+1}-r^n_{t+1}\\ u_t \end{pmatrix}$
\end{itemize}

\subsubsection*{Optimal monetary policy}
First, welfare evaluation requires second order approximation (first-order doesn't matter due to the envelope theorem). Two things are disliked: \textbf{deviation of output from natural level}, \textbf{misallocation}. The optimal monetary policy problem is 
$$
\max_{y_t,\pi_t}-\frac{1}{2}\mathbb{E}_t\sum^{\infty}_{t=0}\beta^t\left[ x_t^2 + \frac{\epsilon}{\kappa} \pi_t^2 \right]
$$
s.t. Phillips curve.

Assume cost-push shock $\pi_t = \kappa x_t +\beta \mathbb{E}_t(\pi_{t+1})+u_t$, there are several questions to be considered:
\begin{itemize}
    \item[-] \textbf{\color{myred}discretion}: no commitment, take $\mathbb{E}_t[x_{t+\tau}]$ as given, hence no intertemporal decision:
    $$
    \max_{x_t,\pi_t}x_t^2+\frac{\epsilon}{\kappa}\pi_t^2 + \mathbb{E}_0\sum^{\infty}_{\tau=1}\beta^{\tau} \left[x_{\tau}^2+\frac{\epsilon}{\kappa}\pi_{\tau}^2\right]
    $$
    s.t.
    $$
    \pi_t = \kappa x_t +\beta\mathbb{E}_t[\pi_{t+1}]+u_t
    $$
    FOCs give
    \begin{align*}
        x_t &= \epsilon \pi_t & \pi_t = \frac{1}{1+\kappa\epsilon}[\beta\mathbb{E}_t(\pi_{t+1})+u_t]
    \end{align*}
    and assume rational expectation $\mathbb{E}_t(\pi_{t+1})=\pi_{t+1}$, iterate forward to solve.
    
    \item[-] \textbf{\color{myred}discretion with inflation bias, no cost-push shock}: natural level output is NOT efficient, policy maker wants to raise output:
    $$
    \max_{x_t,\pi_t}(x_t-\theta)^2+\chi \pi_t^2
    $$
    s.t. 
    $$
    \pi_t = \kappa x_t + \beta \mathbb{E}(\pi_{t+1})
    $$
    FOC gives $x_t +\chi \kappa \pi_t =\theta$, combining with the Phillips curve, gives
    \begin{align*}
        x_t &= \frac{\theta-\chi\kappa \beta\mathbb{E}\pi_{t+1}}{1+\chi\kappa^2} \\
        \pi_t &= \frac{\kappa\theta}{1+\chi\kappa^2}+\frac{1}{1+\chi\kappa^2}\beta\mathbb{E}\pi_{t+1}
    \end{align*}
    again, assume $\mathbb{E}\pi_{t+1}=\pi_{t+1}$, solve forward get
    \begin{align*}
        \pi_t &= \frac{\kappa\theta}{1+\chi\kappa^2-\beta} & x_t &= \frac{(1-\beta)\theta}{1+\chi\kappa^2-\beta}
    \end{align*}

\item[-] \textbf{\color{myred}commitment}
This is an intertemporal question:
$$
\max_{x_t,\pi_t}-\frac{1}{2}\mathbb{E}_t\sum^{\infty}_{t=0}\beta^t\left[x_t^2 + \frac{\epsilon}{\kappa}\pi^2_t\right]
$$
s.t.
$$
\pi_t= \kappa x_t +\beta\mathbb{E}[\pi_{t+1}]+u_t
$$
FOCs are 
\begin{align*}
    \beta^t x_t &=\lambda_t \kappa\\
    \beta^t\frac{\epsilon}{\kappa}\pi_t +\lambda_t -\beta\lambda_{t-1} &=0
\end{align*}
then get
$$
 x_t =x_{t-1} - \epsilon \pi_t \Rightarrow x_t = -\epsilon\sum^t_{s=0}\pi_s \equiv -\epsilon p_t
$$
where $p_t$ is the deviation of the price level from steady state. Plug it back into the Phillips curve
$$
p_t - p_{t-1} = -\kappa \epsilon p_t +\beta\mathbb{E}_t(p_{t+1})-\beta p_t+u_t
$$
solve this, get
$$
p_t-\delta p_{t-1} =\delta \mathbb{E}_t\left[\sum^{\infty}_{k=0}(\beta\delta)^ku_{t+k}\right]
$$
where $\delta<1$ is the inverse of the larger of the two roots of the lag polynominal of the 2-order differential equation. Since $\delta<1$, prices revert back $p_t-p_{t-1}= (\delta-1)p_{t-1}$ if shocks go away.

\subsubsection*{Continuous time version: liquidity trap}
Rewrite two characteristic equations in continuous time:
\begin{align*}
    x_t &= \mathbb{E}\left[x_{t+1}-\frac{1}{\gamma}(i_{t+1}-\pi_{t+1}-r^n_{t+1})\right] & \Rightarrow \dot{x}_t &= \frac{1}{\gamma}(i_t-\pi_t -r^n_t) \\
    \pi_t &= \kappa x_t + \beta\mathbb{E}_t(\pi_{t+1}) & \Rightarrow \dot{\pi} &= \rho \pi_t -\kappa x_t
\end{align*}
the welfare problem is 
$$
\max -\int_0^{\infty}e^{-\rho t}(x_t^2+\chi \pi_t^2)
$$
s.t.
\begin{align*}
    \dot{x}_t &= \frac{1}{\gamma}(i_t-\pi_t -r^n_t) & \text{IS equation}\\
    \dot{\pi}_t &= \rho \pi_t -\kappa x_t & \text{Phillips curve}\\
    i_t &\geq 0 & \text{ZLB}
\end{align*}
consider a shock that makes the natural real rate \textbf{negative}: $r^n_t = \begin{cases}
\underline{r}<0 & t<T\\ \bar{r}>0 & t\geq T
\end{cases}$. Again, we have discretion versus commitment:

\begin{itemize}
    \item[-] \textbf{\color{myred}no commitment}:
    \begin{itemize}
        \item[-] after $T$, $\pi_t=0,x_t=0$ is the first best choice ($i_t = r^n_t$, divine coincidence)
        \item[-] before $T$, $r^n_t<0$, but $i_t\geq 0$, hence $i_t=0$, then
        \begin{align*}
            \dot{x}_t &= -\frac{1}{\gamma}(\pi_t +\underline{r})\\
            \dot{\pi}_t &= \rho \pi_t -\kappa x_t
        \end{align*}
    \end{itemize}
    this gives (graphically) that before $t=T$, $\pi_t<0,x_t<0$. Monetary is powerless.
    \item[-] \textbf{\color{myred}commitment}: with commitment, there are 3 phases:
    \begin{itemize}
        \item[-] \textbf{phase 1}: liquidity trap: same as no-commitment case, but with different terminal condition.
        \item[-] \textbf{phase 2}: just after the liquidity trap. This is where the commitment comes to play: $i_t=0$ \textbf{still holds}, then
        \begin{align*}
            \dot{x}_t &= -\frac{1}{\gamma}(\pi_t +\bar{r})\\
            \dot{\pi}_t &= \rho \pi_t -\kappa x_t
        \end{align*}
        \item[-] \textbf{phase 3}: back to $x_t=0,\pi_t=0$.
    \end{itemize}
    If \textbf{irresponsible promise} is credible, then the situation is better than the no-commitment case.
\end{itemize}
\end{itemize}

\section*{General equilibrium with incomplete market}
\subsection*{Aiyagari 1994}
Household solves
$$
\max_{c(s^t),A(s^{t+1})}\sum^{\infty}_{t=0}\sum_{s^t}\beta^t \Pr(s^t)u(c(s^t))
$$
s.t.
\begin{align*}
    A(s^{t+1}) &= y(s_t)-c(s^t) +RA(s^t) \\
    A(s^t) &\geq -b & \text{borrowing bound}
\end{align*}
a natural borrowing constraint is: the lowest-income state $\underline{s}$ forever $b = -\frac{1}{1-R}y(\underline{s})$. 

Recursively, state variables are $A,s$, then
$$
V(A,s) = \max_{c,A'} u(c) + \beta\sum_{s'}\Pr(s'\mid s)V(A',s')
$$
s.t.
$$A'\leq y(s)-c+RA,\ A\geq -b$$
define \textbf{cash on hand} $x\equiv RA+y$, rewrite the problem with $x,s$ as state variables:
$$
V(x,s) = \max_{c,x'(s')} u(c) + \beta\sum_{s'}\Pr(s'\mid s)V(x'(s'),s')
$$
s.t.
$$x'(s')\leq R(x-c)+y(s'),\ c\leq x+b$$
FOCs are
$$
\begin{aligned}
u'(c)-\sum_{s}\lambda(s)R-\mu &= 0 \\
\beta \Pr(s'\mid s)\frac{\partial V(x'(s'),s')}{\partial x} - \lambda(s) &= 0
\end{aligned} \Rightarrow u'(c) = \beta R\sum_{s'} \Pr(s'\mid s)\frac{\partial V(x'(s'),s')}{\partial x} + \mu
$$
and envelope theorem
$$
\frac{\partial V(x,s)}{\partial x} = u'(c(x,s))
$$
gives Euler equation
$$
u'(c) = \beta R\sum_{s'}\Pr(s'\mid s)u'(c(x'(s'),s'))+\mu {\color{myred}\geq \beta R\sum_{s'}\Pr(s'\mid s)u'(c(x'(s'),s'))}
$$
where $\mu \begin{cases}
=0 & c<x+b\\
>0 & c=x+b
\end{cases}$.
Then if
\begin{itemize}
    \item[-] $\beta R=1$: $u'(c)$ is a supermartingale, converge to 0, hence \textbf{consumption {\color{myred}diverge}} $c_t\rightarrow \infty$ 
    \item[-] $\beta R<1$ and $\lim_{c\rightarrow\infty}-\frac{u''(c)}{u'(c)}=0$: let $c(x)$ be the solution to the recursive problem, and $x'_{max}(x) = R[x-c(x)]+y(\bar{s})$ where $y(\bar{s})$ is the income from the best state $\bar{s}$, then $\exists x^*$ s.t. $x'_{max}\leq x,\forall x\geq x^*$, this means:
    \begin{itemize}
        \item[-] consumption is bounded
        \item[-] there is an upper bound on wealth
        \item[-] any given individual has an invariant distribution
        \item[-] the collection of i.i.d. individuals have a steady state wealth distribution
    \end{itemize}
\end{itemize}

Now, assume $\beta R<1$ and $\lim_{c\rightarrow\infty}-\frac{u''(c)}{u'(c)}=0$, analyze the steady state: the aggregate wealth is a function of $R$, and 
\begin{align*}
    \lim_{R\rightarrow\frac{1}{\beta}^-}A(R) &=\infty  \\
    \lim_{R\rightarrow 0^+}A(R) &= -b
\end{align*}
graphically, see Pablo's note.

The whole point of Aiyagari's model is 
\begin{align*}
    A(R,w) &= K\\
    \text{\color{myred}capital supply }{\color{myred}K^S} &={\color{myblue}K^D} \text{\color{myblue} capital demand}\\
    A(R,w(R)) &= F_K^{-1}(R-1+\delta)\\
    {\color{myred}A(R,F_L(}{\color{myblue}K^D(R)}{\color{myred},1))} &= {\color{myblue}F_K^{-1}(R-1+\delta)}
\end{align*}
$K^D(R)=F_K^{-1}(R-1+\delta)$ is monotonically decreasing, but $A(R,w(R))$ is NOT necessarily monotonic, hence there could be multiple equilibria.

\subsection*{Incomplete market and welfare: constrained efficiency}
Idea: 1st welfare theorem does \textbf{NOT} hold. The social planner is not that powerful (cannot create new markets), so the planning can only be done with the incomplete market. If welfare can be improved, then \textbf{constrained efficiency} is achieved.

This is very much case by case, but social planner's constraint is the market clearing condition, this will generally be the key to solve the problem.

\subsection*{Adding aggregate shocks}
The state variables are $\left\{\Gamma,\theta\right\}$:
\begin{itemize}
    \item[-] $\Gamma$ is the joint distribution of asset holding $A_i$ and idiosyncratic shock $s_i$. Marginal distribution: $\Gamma_A,\Gamma_s$
    \item[-] $\theta$: aggregate shock
\end{itemize}
then 
\begin{itemize}
    \item[-] capital stock is $K(\Gamma)=\int A\mathrm{d}\Gamma_A(A)$
    \item[-] factor prices:
    \begin{align*}
        w(\Gamma,\theta)&=F_L(K(\Gamma),\theta,L)\\
        r^K(\Gamma,\theta)&=F_K(K(\Gamma),\theta,L)\\
        R(\Gamma,\theta)&=r^K(\Gamma,\theta)+1-\delta
    \end{align*}
\end{itemize}
and the household's recursive problem is
$$
V(A,s,\Gamma,\theta) = \max_{c,A'}u(c)+\beta\mathbb{E}\left[ V(A',s',H(\Gamma,\theta),\theta')\mid s,\Gamma,\theta \right]
$$
s.t.
\begin{align*}
    A' &\leq l(s)w(\Gamma,\theta)-c +R(\Gamma,\theta)A\\
    A'\geq -b
\end{align*}
where $H(\Gamma,\theta)$ is the aggregate law of motion, which households take as given.

\subsection*{Micro-foundations of incomplete market: recursive contract}
Settings:
\begin{itemize}
    \item[-] one \textbf{risk-neutral} planner, one \textbf{risk-averse} household 
    \item[-] discount at $\beta$
    \item[-] output: $y(s)$, effort $e_t$
    \item[-] state: $s\in\left\{1,\cdots,S\right\}$, effort affects the probability distribution of states: $\Pr(s_t\mid e_t)$
\end{itemize}
the idea: social planner wants to maximize household's utility s.t. budget constraint, \textbf{and}, we consider its dual problem, planner chooses effort $\{e_t\}$ and consumption $\{c_t\}$ to \textbf{maximize profits s.t. to household's utility}

\subsubsection*{Frictionless benchmark}
Planner's problem
$$
V(w_0) =\max_{c,e}\sum_{s^t}\Pr(s^t\mid e)\beta^t (y(s_t)-c(s^t))
$$
s.t. 
$$
\sum_{s^t}\Pr(s^t\mid e)\beta^t\left[ u(c(s^t))-e(s^{t-1}) \right] = w_0
$$
FOC is 
\begin{align*}
    \mu \Pr(s^t\mid e)\beta^t u'(c(s^t)) &= \Pr(s^t\mid e)\beta^t &\Rightarrow u'(c(s^t)) = \frac{1}{\mu}
\end{align*}
which means full insurance, since planner is risk-neutral.

The recursive form is 
$$
V(w) = \max_{e,c(s),w'(s)}\sum_s\Pr(s\mid e)\left[y(s)-c(s)+\beta V(w'(s))\right]
$$
s.t.
$$
\sum_s \Pr(s\mid e)\left[u(c(s))-e+\beta w'(s)\right] = w
$$
FOCs are 
\begin{align*}
    -\Pr(s\mid e) +\mu\Pr(s\mid e)u'(c(s))=0 &\Rightarrow u'(c(s))=\frac{1}{\mu} \\
    \Pr(s\mid e)\beta V'(w'(s)) + \mu\Pr(s\mid e)\beta =0 &\Rightarrow V'(w'(s)) = -\mu
\end{align*}
plus envelope condition 
{\color{myblue}$$
V'(w(s))=-\mu
$$}
then \textbf{continuation utility is the {\color{myblue}same} as current utility}.

\subsubsection*{Limited commitment}
Planner threaten the household by no long insuring it if a payment is missed (autarky). For this to work, at every history the allocation has to be better than autarky $v_{aut}=\sum^{\infty}_{t=0}\beta^t \left(\sum_s \Pr(s)u(c(s))\right) = \frac{\sum_s\Pr(s)u(c(s))}{1-\beta}$.

The recursive problem is 
$$
V(w) = \max_{c(s),w'(s)}\sum_s \Pr(s) \left[ y(s)-c(s) + \beta V(w'(s)) \right]
$$
s.t.
\begin{align*}
    \sum_s \Pr(s)\left[ u(c(s))+\beta w'(s) \right] &= w \\
    u(c(s)) + \beta w'(s) &\geq u(y(s))+\beta v_{aut}, \forall s
\end{align*}
FOC gives
\begin{align*}
    -\Pr(s) + \mu\Pr(s)u'(c(s)) + \lambda(s)u'(c(s)) = 0 & \Rightarrow u'(c(s)) = \frac{\Pr(s)}{\mu\Pr(s)+\lambda(s)} \\
    \Pr(s)\beta V'(w'(s)) + \mu\Pr(s)\beta +\lambda(s)\beta = 0  &\Rightarrow V'(w'(s)) = -\frac{\mu \Pr(s)+\lambda(s)}{\Pr(s)}
\end{align*}
which gives
{\color{myred}$$
u'(c(s)) = -\frac{1}{V'(w'(s))}
$$}
envelope condition $V'(w)=-\mu$ still holds, then 
$$
V'(w'(s)) = V'(w) -\frac{\lambda(s)}{\Pr(s)} \begin{cases}
= V'(w) & \lambda(s)=0 \Rightarrow \text{ autarky constraint \textbf{NOT} bind}\\
< V'(w) & \lambda(s)>0 \Rightarrow \text{ autarky constraint \textbf{bind}}
\end{cases} 
$$
then if
\begin{itemize}
    \item[-] autarky constraint bind, $V'(w'(s))<V'(w)\Rightarrow w'(s)>w$, then $\left\{c(s),w'(s)\right\}$ solves 
    \begin{align*}
        u(c(s)) + \beta w'(s) &= u(y(s)) + \beta v_{aut} & u'(c(s)) &= -\frac{1}{V'(w'(s))}
    \end{align*}
    here, allocations are reset: \textbf{no memories}.
    \item[-] autarky constraint not bind, $V'(w'(s))=V'(w)\Rightarrow w'(s) = w$, hence the promised utility is constant, leading to consumption constant.
\end{itemize}
\underline{\textbf{\color{myred}Dynamic}}:
\begin{itemize}
    \item[1] when autarky constraint is not binding, consumption is constant
    \item[2] when autarky constraint is binding, consumption $c(s)$ and promised utility $w'(s)$ are both reset to a higher level
    \item[3] the consumption is constant again till the autarky constraint is binding again
    \item[4] consumption is constant forever after the best state $S$ is realized.
\end{itemize}

\subsubsection*{two-side limited commitment}
Planner needs an extra constraint of \textbf{positive profit}. The recursive problem is 
$$
V(w) = \max_{c(s),w'(s)}\sum_s \Pr(s) \left[ y(s)-c(s) + \beta V(w'(s)) \right]
$$
s.t.
\begin{align*}
    \sum_s \Pr(s)\left[ u(c(s))+\beta w'(s) \right] &= w \\
    u(c(s)) + \beta w'(s) &\geq u(y(s))+\beta v_{aut}, \forall s \\
    y(s)-c(s) +\beta V(w'(s)) & \geq 0, \forall s
\end{align*}
FOC gives
\begin{align*}
    -\Pr(s) + \mu\Pr(s)u'(c(s)) + \lambda(s)u'(c(s)) -\eta(s)= 0 & \Rightarrow u'(c(s)) = \frac{\Pr(s)+\eta(s)}{\mu\Pr(s)+\lambda(s)} \\
    \Pr(s)\beta V'(w'(s)) + \mu\Pr(s)\beta +\lambda(s)\beta +\eta(s)\beta V'(w'(s))= 0  &\Rightarrow V'(w'(s)) = -\frac{\mu \Pr(s)+\lambda(s)}{\Pr(s)+\eta(s)}
\end{align*}
which, again, gives
{\color{myred}$$
u'(c(s)) = -\frac{1}{V'(w'(s))}
$$}
envelope condition $V'(w)=-\mu$ still holds, then 
$$
V'(w'(s)) = \frac{V'(w)\Pr(s)-\lambda(s)}{\Pr(s)+\eta(s)}
$$
general pattern is
\begin{itemize}
    \item[-] if autarky constraint binds: promised utility $w'(s)$ and consumption $c(s)$ \textbf{increase}
    \item[-] if positive-profit constraint binds: promised utility $w'(s)$ and consumption $c(s)$ \textbf{decrease}
    \item[-] promised utility $w'(s)$ and consumption $c(s)$ \textbf{stay constant} if no constraints bind.
    \item[-] whenever the constraints bind, contract rest, again \textbf{no memories}.
\end{itemize}

\subsubsection*{Moral hazard}
Both parties commit, but household exerts unobserved effort, the recursive problem is
$$
V(w) = \max_{e, c(s),w'(s)}\sum_s \Pr(s\mid e) \left[ y(s)-c(s) + \beta V(w'(s)) \right]
$$
s.t.
\begin{align*}
    \sum_s \Pr(s\mid e)\left[ u(c(s))-e+\beta w'(s) \right] &= w \\
    \sum_s \Pr(s\mid e)\left[ u(c(s))-e+\beta w'(s) \right] &\geq \sum_s \Pr(s\mid \tilde{e})\left[ u(c(s))-\tilde{e}+\beta w'(s) \right], \forall \tilde{e}
\end{align*}
assume the special case: $e\in\left\{0,1\right\}$ and the planner wants to implement $e=1$, the budget constraints are then
\begin{align*}
    \sum_s \Pr(s\mid 1)\left[ u(c(s))-1+\beta w'(s) \right] &= w \\
    \sum_s \Pr(s\mid 1)\left[ u(c(s))-1+\beta w'(s) \right] &\geq \sum_s \Pr(s\mid 0)\left[ u(c(s))+\beta w'(s) \right]
\end{align*}
FOC gives
\begin{align*}
   & -\Pr(s\mid 1) + \mu\Pr(s\mid 1)u'(c(s)) + \lambda \left[ \Pr(s\mid 1)-\Pr(s\mid 0) \right]u'(c(s))= 0 \\
   \Rightarrow &  \frac{1}{u'(c(s))} = \mu + \lambda\left[ 1-\frac{\Pr(s\mid 0)}{\Pr(s\mid 1)} \right]\\
    & \Pr(s\mid 1)\beta V'(w'(s)) + \mu\Pr(s\mid 1)\beta + \lambda \left[ \Pr(s\mid 1)-\Pr(s\mid 0) \right]\beta= 0 \\
   \Rightarrow & V'(w'(s)) = -\mu - \lambda\left[ 1-\frac{\Pr(s\mid 0)}{\Pr(s\mid 1)} \right]
\end{align*}
which, again, gives
{\color{myred}$$
u'(c(s)) = -\frac{1}{V'(w'(s))}
$$}
envelope condition $V'(w)=-\mu$ still holds, then
\begin{align*}
    & V'(w'(s)) = V'(w) - \lambda\left[ 1-\frac{\Pr(s\mid 0)}{\Pr(s\mid 1)} \right] \\
    \Rightarrow& \mathbb{E}\left[\frac{1}{u'(c_{t+1})}\right] = \frac{1}{u'(c_t)} & \text{\textbf{inverse} Euler equation}
\end{align*}
since
${\color{myblue} \mathbb{E}\left(1-\frac{Pr(s\mid 0)}{\Pr(s\mid 1)}\mid e=1\right)=\sum_s\Pr(s\mid 1)\left(1-\frac{\Pr(s\mid 0)}{\Pr(s\mid 1)}\right)}$ and $u'(c(s))=-\frac{1}{V'(w'(s))}$.

How to understand the inverse Euler equation? It equates \textbf{expected marginal cost} of one unit of utility over time: to provide $\Delta$ utils, then
\begin{align*}
    {\color{myred}\text{today's cost of providing }\Delta} &= {\color{myblue}\text{tomorrow's cost of providing }\Delta\text{ in all states}}\\
    {\color{myred}\frac{\Delta}{u'(c_t)}} &= {\color{myblue}\beta \sum_{s+1}\Pr(s_{t+1})\frac{\Delta}{\beta u'(c_{t+1})}=\Delta\mathbb{E}\left[\frac{1}{u'(c_{t+1})}\right]}
\end{align*}
with Jensen's inequality ($1/x$ convex and $u'(c)>0$):
$$
\mathbb{E}\left[\frac{1}{u'(c_{t+1})}\right] > \frac{1}{\mathbb{E}\left[u'(c_{t+1})\right]} \Rightarrow u'(c_t) > \frac{1}{\mathbb{E}\left[u'(c_{t+1})\right]} \Rightarrow u'(c_t) < \mathbb{E}\left[u'(c_{t+1})\right]
$$
hence, if the household can save at $R=\frac{1}{\beta}$, they will save. This relies on the assumption that the incentive constraint binds ({\color{myred}$\lambda >0$}).

Inverse Euler equation implies that $\frac{1}{u'(c_t)}$ is a Martingale, then $\frac{1}{u'(c_t)}\xrightarrow{a.s.} C$, and $C\not>0$ since there is not full insurance and violates incentive constraints, hence 
$$
\frac{1}{u'(c_t)}\xrightarrow{a.s.} 0 \Rightarrow u'(c_t)\xrightarrow{a.s.}\infty \Rightarrow c_t\xrightarrow{a.s.} 0
$$
household would prefer to \textbf{quit} the scheme eventually to stay away from 0 consumption.

\subsubsection*{Hidden income}
Planner \textbf{cannot} observe the realization of $y(s)$, hence the incentive constraint would be to encourage households to report income truthfully (given $y(s)$, households would not report another state $\tilde{s}$). Define transfer $\tau(s)\equiv c(s)-y(s)$, then the planner's problem is
$$
V(w) = \max_{\tau(s),w'(s)}\sum_s \Pr(s) \left[ -\tau(s) + \beta V(w'(s)) \right]
$$
s.t.
\begin{align*}
    \sum_s \Pr(s)\left[ u(\tau(s)+y(s))+\beta w'(s) \right] &= w \\
    u(\tau(s)+y(s)) + \beta w'(s) &\geq u(\tau(\tilde{s})+y(s)) + \beta w'(\tilde{s}), \forall s,\tilde{s}
\end{align*}
with mechanism design theory, we have the following lemma (assume states are ranked ascendingly):
\begin{itemize}
    \item[\textbf{L1}] $\tau(s)\leq \tau(s-1)$, $w'(s) \geq w'(s-1)$
    \item[\textbf{L2}] local incentive constraints imply global incentive constraints
    \item[\textbf{L3}] downwards local incentive constraints always bind: people only have incentive to under-report their income
\end{itemize}
and the incentive constraint can be rewritten as
$$
V(w) = \max_{\tau(s),w'(s)}\sum_s \Pr(s) \left[ -\tau(s) + \beta V(w'(s)) \right]
$$
s.t.
\begin{align*}
    \sum_s \Pr(s)\left[ u(\tau(s)+y(s))+\beta w'(s) \right] &= w \\
    u(\tau(s)+y(s)) + \beta w'(s) &\geq u(\tau(s-1)+y(s)) + \beta w'(s-1), \forall s>1
\end{align*}

FOC gives
\begin{align*}
    &-\Pr(s) + \mu\Pr(s)u'(\tau(s)+y(s)) + \lambda(s)u'(\tau(s)+y(s)) -\lambda(s+1)u'(\tau(s)+y(s+1))= 0 \\
    \Rightarrow & \Pr(s)\left[1-\mu u'(y(s)+\tau(s))\right] = \lambda(s)u'(\tau(s)+y(s)) -\lambda(s+1)u'(\tau(s)+y(s+1)) \\
    &\Pr(s)\beta V'(w'(s)) + \mu\Pr(s)\beta +\lambda(s)\beta -\lambda(s+1)\beta= 0 \\ \Rightarrow& \Pr(s)\left[V'(w'(s))+\mu\right] = \lambda(s+1)-\lambda(s)
\end{align*}
since $\lambda(1)=0$ (lowest state, no misreport) and $\lambda(S+1)=0$ (no state $S+1$), sum over all $S$:
\begin{align*}
    \sum_s \Pr(s)\left[V'(w'(s))+\mu\right] &= \sum_s \left[\lambda(s+1)-\lambda(s)\right]\\
    \Rightarrow \mu + \sum_s\Pr(s)V'(w'(s)) &= \lambda(S+1)-\lambda(1)=0\\
    \Rightarrow \sum_s\Pr(s)V'(w'(s)) &= -\mu
\end{align*}

envelope condition $V'(w)=-\mu$ still holds, then 
$$
\mathbb{E}[V'(w'(s))] = V'(w)
$$
hence, $V'(w)$ is a Martingale. This will pin down the range of $w$ and $V(w)$:
\begin{itemize}
    \item[-] $w'(1) < w < w'(S)$
    \item[-] $-\frac{\bar{\tau}(w)}{1-\beta}\leq V(w)\leq \sum_s\Pr(s)\frac{y(s)-\bar{c}(w)}{1-\beta}$, where 
    \begin{itemize}
        \item[-] $\bar{\tau}(w)$ solves $\frac{\sum_s \Pr(s)u(y(s)+\bar{\tau}(w))}{1-\beta}\equiv w$
        \item[-] $\bar{c}(w)$ solves $\frac{\sum_s \Pr(s)u(\bar{c}(w))}{1-\beta}\equiv w$
    \end{itemize}
\end{itemize}

And since $V'(w)$ is a Martingale, and $w$ keeps spreading out,t $V'(w)\xrightarrow{a.s.} 0$.

\section*{Optimal tax}
Setting:
\begin{itemize}
    \item[-] linear tax on labor: $\tau_t^l$
    \item[-] linear tax on capital (on the net rental of capital): $\tau_t^k$. No arbitrage between bond and capital gives:
    $$
    R_t = 1+(1-\tau_t^k)(r_t^k-\delta)
    $$
    \item[-] no lump-sum tax
\end{itemize}

\subsection*{Primal approach}
Let the price series be
$$
p_t \equiv \begin{cases}
    \prod^{t-1}_{s=0}\frac{1}{R_{s+1}} & t\geq 1\\
    1 & t=0
\end{cases}
$$
government has an initial debt with interest $B_0R_0$, competitive wage $w_t$ and capital rent $r^k_t$, then 
\begin{itemize}
    \item[-] government budget:
    $$
     B_0R_0 + \sum_t p_tg_t \leq \sum_t p_t[\tau^l_tw_tL_t + \tau^k_t (r_t^k-\delta)K_t]
    $$
    \item[-] household budget:
    $$
    \sum_t p_tc_t \leq R_0A_0 + \sum_t p_tw_t(1-\tau^l_t)L_t
    $$
    \item[-] capital market clearing:
    $$
    A_t = B_t + K_t
    $$
\end{itemize}
\subsubsection*{implementability condition}
In competitive equilibrium, we have for households
\begin{itemize}
    \item[-] Euler equation: $p_t =\beta^t \frac{u_c(c_t,L_t)}{u_c(c_0,L_0)}$
    \item[-] labor condition: $w_t(1-\tau^l_t)=-\frac{u_L(c_t,L_t)}{u_c(c_t,L_t)}$
\end{itemize}
plug them and the market clearing condition in household's budget constraint, get
$$
\sum_t \beta^t\frac{u_c(c_t,L_t)}{u_c(c_0,L_0)}\left[c_t +\frac{u_L(c_t,L_t)}{u_c(c_t,L_t)}L_t\right] \leq R_0(B_0+K_0)
$$
rewrite, get the \textbf{\color{myred}implementability} condition:
$$
\sum_t \beta^t [u_c(c_t,L_t)c_t +u_L(c_t,L_t)L_t] \leq R_0(B_0+K_0)u_c(c_0,L_0)
$$

\subsubsection*{government's problem}
and the government's problem is then
$$
\max \sum^{\infty}_{t=0}\beta^t u(c_t,L_t)
$$
s.t.
\begin{align*}
    c_t+ K_{t+1} + g_t &= F(K_t,L_t) + (1-\delta)K_t\\
    \sum_t \beta^t \left[u_c(c_t,L_t)c_t + u_L(c_t,L_t)L_t\right] &= R_0(B_0+K_0)u_c(c_0,L_0)
\end{align*}
define $W(c,L) = u(c,L)+\mu[u_c(c,L)c+u_L(c,L)L]$, where $\mu$ is the Lagrange multiplier for the implementability constraint, then government's problem can be rewritten as:

$$
\max \sum^{\infty}_{t=0}\beta^t W(c_t,L_t) - \mu u_c(c_0,L_0)R_0(B_0+K_0)
$$
s.t.
$$
c_t+ K_{t+1} + g_t = F(K_t,L_t) + (1-\delta)K_t
$$
then like a neoclassical problem, FOCs are
\begin{align*}
    W_c(c_t,L_t) &= \beta (F_K(K_{t+1},L_{t+1})+1-\delta) W_c(c_{t+1},L_{t+1}) \\
    W_L(c_t,L_t) &= -F_L(K_t,L_t)W_c(c_t,L_t)
\end{align*}
then the optimal labor tax is given by:
$$
\begin{aligned}
    F_L(K_t,L_t)(1-\tau^l_t) &=-\frac{u_L(c_t,L_t)}{u_c(c_t,L_t)}\\
    F_L(K_t,L_t) &= -\frac{W_L(c_t,L_t)}{W_c(c_t,L_t)}
\end{aligned} \Rightarrow \tau_t^{l*} = 1-\frac{u_L(c_t,L_t)}{W_L(c_t,L_t)}\frac{W_c(c_t,L_t)}{u_c(c_t,L_t)}
$$
and the capital rent is
$$
\begin{aligned}
    \frac{u_c(c_t,L_t)}{u_c(c_{t+1},L_{t+1})} &= \beta R_{t+1}\\
    \frac{W_c(c_t,L_t)}{W_c(c_{t+1},L_{t+1})} &= \beta[F_K(K_{t+1},L_{t+1})+1-\delta]
\end{aligned} \Rightarrow R_{t+1} = \frac{W_c(c_{t+1},L_{t+1})}{u_c(c_{t+1},L_{t+1})}\frac{u_c(c_t,L_t)}{W_c(c_t,L_t)}R^*_{t+1}
$$
where $\beta[F_K(K_{t+1},L_{t+1})+1-\delta]\equiv \beta R^*_{t+1}$. This gives that in \textbf{steady state}, 
$$R_{t+1}=R^*_{t+1}\Rightarrow \tau^k=0$$

\subsubsection*{initial period}
Now check the initial period, if $\tau^k_0$ can be chosen freely:
$$
\max \sum^{\infty}_{t=0}W(c_t,L_t)-\mu u_c(c_0,L_0)[R_0B_0+ (1+(1-\tau^k_0)(r^k_0-\delta))K_0]
$$
s.t.
$$
c_t+g_t+K_{t+1}\leq F(c_t,L_t)+(1-\delta)K_t
$$
\begin{itemize}
    \item[-] \textbf{FOC w.r.t. $\tau_0$} is
\begin{align*}
    -\mu u_c(c_0,L_0)(r^k_0-\delta)K_0 = 0 & \Rightarrow \mu = 0
\end{align*}
hence, if there is no limit on the initial-period tax $\tau^k_0$, first-best allocation can be achieved, no distortion, no time-inconsistency; if $\tau^k_0\leq \bar{\tau}$, the the optimal tax rate is the upper bound $\bar{\tau}$, but it will not work as it could.
    \item[-] \textbf{FOC w.r.t. $c_0$} is
    $$
    W_c(c_0,L_0) = \mu u_{cc}(c_0,L_0)[R_0B_0+ (1+(1-\tau^k_0)(r^k_0-\delta))K_0]+ \lambda_0
    $$
\end{itemize}
if there is non-zero initial wealth $R_0B_0+ (1+(1-\tau^k_0)(r^k_0-\delta))K_0\neq 0$, this FOC differs from the FOCs of $t>0$.

\subsection*{Ramsey approach}
For household $i$, its preference
$$
u_i(c_i,y_i) = u(c_i) - v\left(\frac{y_i}{\theta_i}\right)
$$
where $l_i = \frac{y_i}{\theta_i}$, $y_i$ is the output, $\theta_i$ is the productivity, and tax on output $T(y_i)$.

\subsubsection*{First-best allocation}
Setting: weighted utilitarian welfare function, government choose $c,y,T$ for each $i$.
$$
\max_{c_i,y_i,T(y_i)} \sum_i \pi_i\lambda_i \left[u(c_i)-v\left(\frac{y_i}{\theta_i}\right)\right]
$$
s.t.
\begin{align*}
    c_i &\leq y_i -T(y_i) & \forall i\\
    G&\leq \sum_i\pi_i T(y_i)
\end{align*}
rewrite the problem 
$$
\max_{y_i,T(y_i)} \sum_i \pi_i\lambda_i \left[u(y_i-T(y_i))-v\left(\frac{y_i}{\theta_i}\right)\right]
$$
s.t.
$$
G\leq \sum_i\pi_iT(y_i)
$$
FOC w.r.t. $T(y_i)$ gives
\begin{align*}
    -\pi_i\lambda_i u'(c_i) +\mu \pi_i &=0 & \Rightarrow u'(c_i) = \frac{\mu}{\lambda_i}
\end{align*}
which indicates that \textbf{lower marginal utility for favored households}.

FOC w.r.t. $y_i$ gives
\begin{align*}
    \pi_i\lambda_i\left[u'(c_i)-\frac{1}{\theta_i}v'(l_i)\right] &=0 &\Rightarrow v'(l_i)=\theta_i \frac{\mu}{\lambda_i}
\end{align*}
which indicates that \textbf{more productive households work harder}.

This is obvious problematic due to \textbf{incentives}.

\subsubsection*{Mirrlees Taxation}
This taxation scheme relies on the revelation principle: ask households to report their \textbf{type} $i$, assign them $c_i$ and $y_i$, back out the tax function that can implement this mechanism. 

$$
\max_{c_i,y_i}u(c_i) - v\left(\frac{y_i}{\theta_i}\right)
$$
s.t. $c_i\leq y_i-T(y_i)$. FOC gives
$$
\begin{aligned}
    u'(c_i)-\mu&=0\\
    -\frac{1}{\theta_i}v'\left(\frac{y_i}{\theta_i}\right)+\mu(1-T'(y_i)) &=0
\end{aligned}\Rightarrow {\color{myred}v'(l_i) = u'(c_i)\theta_i(1-T'(y_i))} \Rightarrow {\color{myred}T'(y_i) = 1-\frac{v'(l_i)}{u'(c_i)\theta_i}}
$$

This is just as the \textbf{hidden income} of insurance modelling, \textbf{only downward constraints are necessary}, and \textbf{local constraints imply global constraints}. The planner's problem is
$$
\max \sum_i \pi_i\lambda_i \left[u(c_i)-v\left(\frac{y_i}{\theta_i}\right)\right]
$$
s.t.
\begin{align*}
    \sum_i\pi_i c_i +G &\leq \sum_i\pi_i y_i \\
    u(c_i)- v\left(\frac{y_i}{\theta_i}\right) &\geq u(c_{i-1})-v\left(\frac{y_{i-1}}{\theta_i}\right) &\forall i >1
\end{align*}
FOCs are 
\begin{align*}
    \lambda_i \pi_i u'(c_i)-\eta \pi_i+\left[\mu_i\pi_i-\mu_{i+1}\pi_{i+1}\right]u'(c_i) &=0 & i=1,\cdots,I-1\\
    -\lambda_i\pi_i \frac{1}{\theta_i}v'(\frac{y_i}{\theta_i}) + \eta\pi_i - \left[ \frac{\mu_i\pi_i}{\theta_i}v'\left(\frac{y_i}{\theta_i}\right) - \frac{\mu_{i+1}\pi_{i+1}}{\theta_{i+1}}v'\left(\frac{y_i}{\theta_{i+1}}\right) \right] &=0 & i=1,\cdots,I-1\\
    \lambda_i\pi_iu'(c_i)-\eta\pi_i +\mu_i\pi_iu'(c_i)&=0 &i=I\\
    -\lambda_i\pi_i \frac{1}{\theta_i}v'\left(\frac{y_i}{\theta_i}\right) + \eta\pi_i - \frac{\mu_i\pi_i}{\theta_i}v'\left(\frac{y_i}{\theta_i}\right) &=0 & i=I
\end{align*}

since
\begin{align*}
    T'(y_i) &= 1-\frac{v'(l_i)}{u'(c_i)\theta_i} & T(y_i) &= y_i-c_i
\end{align*}
this pins down the marginal tax rate $T'(y_i)$ (slope) and the tax rate $T(y_i)$ (value). This will pin down the tax function.

From the FOCs
\begin{align*}
    \lambda_i \pi_i u'(c_i)+\left[\mu_i\pi_i-\mu_{i+1}\pi_{i+1}\right]u'(c_i) &=\eta\pi_i\\
    \lambda_i\pi_i \frac{1}{\theta_i}v'\left(\frac{y_i}{\theta_i}\right) + \frac{\mu_i\pi_i}{\theta_i}v'\left(\frac{y_i}{\theta_i}\right) - \frac{\mu_{i+1}\pi_{i+1}}{\theta_{i+1}}v'\left(\frac{y_i}{\theta_{i+1}}\right) &=\eta\pi_i
\end{align*}
we have
\begin{align*}
    u'(c_i)[\lambda_i\pi_i+\mu_i\pi_i-\mu_{i+1}\pi_{i+1}] &= \frac{1}{\theta_i}v'\left(\frac{y_i}{\theta_i}\right)\left[\lambda_i\pi_i+\mu_i\pi_i -\mu_{i+1}\pi_{i+1}\frac{\theta_i}{\theta_{i+1}}\frac{v'\left(\frac{y_i}{\theta_{i+1}}\right)}{v'\left(\frac{y_i}{\theta_i}\right)}\right]\\
    \Rightarrow T'(y_i) &= 1-\frac{\lambda_i\pi_i+\mu_i\pi_i-\mu_{i+1}\pi_{i+1}}{\lambda_i\pi_i+\mu_i\pi_i -\mu_{i+1}\pi_{i+1}\frac{\theta_i}{\theta_{i+1}}\frac{v'\left(\frac{y_i}{\theta_{i+1}}\right)}{v'\left(\frac{y_i}{\theta_i}\right)}}
\end{align*}
which is the marginal tax rate. Two results:
\begin{itemize}
    \item[-] $T'(y_I)=0$: highest type's marginal tax rate is 0.
    \item[-] $T'(y_i)>0$ if $i=1,\cdot,I-1$: positive marginal tax rate other than the highest type. It's clear that $\theta_{i+1}>\theta_i \Rightarrow \frac{\theta_i}{\theta_{i+1}}\frac{v'(y_i\theta_{i+1})}{v'(y_i/\theta_i)}<1$.
\end{itemize}

\underline{\textbf{\color{myblue}An example: $v(l)=\frac{1}{1+\phi}l^{1+\phi}$}}

The marginal tax rate is 
\begin{align*}
    T'(y_i) &= 1-\frac{\lambda_i\pi_i+\mu_i\pi_i-\mu_{i+1}\pi_{i+1}}{\lambda_i\pi_i+\mu_i\pi_i -\mu_{i+1}\pi_{i+1}\left(\frac{\theta_i}{\theta_{i+1}}\right)^{1+\phi}} = \frac{\mu_{i+1}\pi_{i+1}\left(1- \left(\frac{\theta_i}{\theta_{i+1}}\right)^{1+\phi}\right)}{\lambda_i\pi_i+\mu_i\pi_i -\mu_{i+1}\pi_{i+1}\left(\frac{\theta_i}{\theta_{i+1}}\right)^{1+\phi}}\\
    {\color{myblue}\text{FOC}\Rightarrow}&= \frac{\mu_{i+1}\pi_{i+1}\left(1- \left(\frac{\theta_i}{\theta_{i+1}}\right)^{1+\phi}\right)\left(\frac{y_i}{\theta_i}\right)^{\phi}}{\eta\pi_i\theta_i} 
\end{align*}

and from FOC:
\begin{align*}
    \lambda_i\pi_i +\mu_i\pi_i -\mu_{i+1}\pi_{i+1}&=\frac{\eta\pi_i}{u'(c_i)}\\
    \mu_i\pi_i &= \frac{\eta\pi_i}{u'(c_i)}-\lambda_i\pi_i +\mu_{i+1}\pi_{i+1}\\
    \text{\color{myblue}iterate forward} &= \sum^I_{j=i}\pi_j\left[\frac{\eta}{u'(c_j)}-\lambda_j\right]
\end{align*}
by household's FOC $\left(\frac{y_i}{\theta_i}\right)^{\phi} = u'(c_i)\theta_i(1-T'(y_i))$, get the marginal tax rate:
\begin{align*}
    T'(y_i) &= \frac{\sum^I_{j=i+1}\pi_j\left[\frac{\eta}{u'(c_j)}-\lambda_j\right] \left(1- \left(\frac{\theta_i}{\theta_{i+1}}\right)^{1+\phi}\right) u'(c_i)\theta_i(1-T'(y_i))}{\eta\pi_i\theta_i}\\
    \frac{T'(y_i)}{1-T'(y_i)} &= \sum^I_{j=i+1}\pi_j\left[\frac{1}{u'(c_j)}-\frac{\lambda_j}{\eta}\right] \left(1- \left(\frac{\theta_i}{\theta_{i+1}}\right)^{1+\phi}\right) \frac{u'(c_i)}{\pi_i}
\end{align*}
Again, from the FOC $u'(c_i)[\lambda_i\pi_i+\mu_i\pi_i-\mu_{t+1}\pi_{t+1}]=\eta \pi_i$, get
$$
\sum_i u'(c_i)[\lambda_i\pi_i+\mu_i\pi_i-\mu_{t+1}\pi_{t+1}]=\sum_i \eta \pi_i \Rightarrow \sum_i u'(c_i)\lambda_i\pi_i = \eta
$$
then, 
\begin{align*}
    \frac{T'(y_i)}{1-T'(y_i)} &= \sum^I_{j=i+1}\pi_j\left[\frac{u'(c_i)}{u'(c_j)}-\frac{u'(c_i)\lambda_j}{\sum_iu'(c_i)\lambda_i\pi_i}\right] \left(1- \left(\frac{\theta_i}{\theta_{i+1}}\right)^{1+\phi}\right) \frac{1}{\pi_i}\\
   {\color{myblue}f(\theta_i)=\frac{\pi_i}{\theta_{i+1}-\theta_i}} \xRightarrow{\theta{i+1}\rightarrow\theta_i} &= \int^{\bar{\theta}}_{\theta}\left[\frac{u'(c(\theta))}{u'(c(s))} - \frac{u'(c(\theta))\lambda(s)}{\int u'(c(z))\lambda(z)f(z)\mathrm{d}z}\right]f(s)\mathrm{d}s\frac{1+\phi}{\theta f(\theta)}
\end{align*}

a special case is $u(c)=c$, then 
$$
\frac{T'(y_i)}{1-T'(y_i)}=\int^{\bar{\theta}}_{\theta}\left[1 - \frac{\lambda(s)}{\int \lambda(z)f(z)\mathrm{d}z}\right]f(s)\mathrm{d}s\frac{1+\phi}{\theta f(\theta)}
$$
even more special, for utilitarian (constant $\lambda$), we have
$$
\frac{T'(y_i)}{1-T'(y_i)}=\int^{\bar{\theta}}_{\theta}\left[1 - \frac{\lambda}{\int \lambda f(z)\mathrm{d}z}\right]f(s)\mathrm{d}s\frac{1+\phi}{\theta f(\theta)} =0
$$

\section*{Optimal monetary policy}
Again, we have the competitive equilibrium to derive the \textbf{implementability condition}, and planner's problem to solve the optimal monetary policy.
\subsection*{Implementability condition}
households solve ($c^1_t$ cash good, $c^2_t$ credit good)
$$
\max \sum_t \beta^t u(c^1_t,c^2_t,L_t)
$$
s.t.
\begin{align*}
    p^1_t c^1_t +p^2_t c^2_t + M_{t+1}+B_{t+1} &\leq w_t(1-\tau_t)L_t + M_t + (1+i_t)B_t\\
    p^1_t c^1_t \leq M_t
\end{align*}

firm side $p^1_t=p^2_t=p_t = \frac{w_t}{A}$, FOC gives
$$
\begin{aligned}
    \frac{\partial \mathcal{L}}{\partial c^1_t}=0 &\Rightarrow \beta_t u_{c^1}(c_t^1,c_t^2,L_t)=(\lambda_t+\eta_t) p_t \\
    \frac{\partial \mathcal{L}}{\partial c^2_t}=0 &\Rightarrow \beta_t u_{c^2}(c_t^1,c_t^2,L_t)=\lambda_t p_t \\
    \frac{\partial \mathcal{L}}{\partial L_t}=0 &\Rightarrow \beta_t u_{L}(c_t^1,c_t^2,L_t)=-\lambda_t w_t(1-\tau_t) \\
    \frac{\partial \mathcal{L}}{\partial B_{t+1}}=0 &\Rightarrow \lambda_{t+1}(1+i_{t+1}) = \lambda_t \\
    \frac{\partial \mathcal{L}}{\partial M_t}=0 &\Rightarrow \lambda_t = \lambda_{t+1}+\eta_{t+1}
\end{aligned}
$$
again, define the price series as $q_t = \begin{cases}
\prod^{t-1}_{s=0}\frac{1}{1+i_{s+1}} & t\geq 1\\
1 & t=0
\end{cases}$, then FOCs give
\begin{align*}
    q_tp_t &= \frac{\beta^t u_{c^2}(c_t^1,c_t^2,L_t)}{u_{c^2}(c_0^1,c_0^2,L_0)}  &\text{\color{myred}intertemporal}\\
    \frac{w_t}{p_t}(1-\tau_t) &= -\frac{u_L(c_t^1,c_t^2,L_t)}{u_{c^2}(c_t^1,c_t^2,L_t)} &\text{\color{myred}labor condition}\\
    1+i_t &= \frac{u_{c^1}(c_t^1,c_t^2,L_t)}{u_{c^2}(c_t^1,c_t^2,L_t)} &\text{\color{myred}money condition}
\end{align*}
Rewrite the budget constraint:
\begin{align*}
    \sum_t q_t \left[ p_t(c_t^1+c_t^2) +i_tM_t -w_t(1-\tau_t)L_t \right] &=0\\
    \sum_t q_t \left[ p_t(1+i_t)c_t^1+p_tc_t^2 -w_t(1-\tau_t)L_t \right] &=0\\
    \sum_t \frac{\beta^t u_{c^2}(c_t^1,c_t^2,L_t)}{u_{c^2}(c_0^1,c_0^2,L_0)} \left[ \left( \frac{u_{c^1}(c_t^1,c_t^2,L_t)}{u_{c^2}(c_t^1,c_t^2,L_t)}c_t^1 +c_t^2 \right) +\frac{u_{L}(c_t^1,c_t^2,L_t)}{u_{c^2}(c_t^1,c_t^2,L_t)} \right] &=0\\
    {\color{myred}\sum_t\beta^t \left[ u_{c^1}(c_t^1,c_t^2,L_t)c_t^1+ u_{c^2}(c_t^1,c_t^2,L_t)c_t^2 + u_{L}(c_t^1,c_t^2,L_t)L_t \right]} &{\color{myred}= 0}
\end{align*}

and $i_t>0$, hence $\frac{u_{c^1}(c_t^1,c_t^2,L_t)}{u_{c^2}(c_t^1,c_t^2,L_t)}>1$.

\subsection*{Planner's problem}
The planner's problem is 
$$
\max \sum_t\beta^t u(c_t^1,c_t^2,L_t)
$$
s.t.
\begin{align*}
    c^1_t + c_t^2 + g_t &= AL_t\\
    \sum_t\beta^t \left[ u_{c^1}(c_t^1,c_t^2,L_t)c_t^1+ u_{c^2}(c_t^1,c_t^2,L_t)c_t^2 + u_{L}(c_t^1,c_t^2,L_t)L_t \right]&=0
\end{align*}
again, define pseudo-preference as
\begin{align*}
    W(c^1,c^2,L) \equiv u(c^1,c^2,L)+\mu\left[u_{c^1}(c^1,c^2,L)+u_{c^2}(c^1,c^2,L)+u_L(c^1,c^2,L)\right]
\end{align*}
then rewrite the problem as 
$$
\max \sum_t \beta^ W(c^1_t,c^2_t,L_t)
$$
s.t. 
$$
c_t^1+c_t^2+g_t = AL_t
$$
FOC gives:
\begin{align*}
    \frac{W_{c^1}(c^1_t,c^2_t,L_t)}{W_{c^2}(c^1_t,c^2_t,L_t)} &= 1
\end{align*}
which gives the optimal nominal interest rate:
\begin{align*}
    1+i_t &= \frac{u_{c^1}(c^1_t,c^2_t,L_t)}{u_{c^2}(c^1_t,c^2_t,L_t)} = \frac{u_{c^1}(c^1_t,c^2_t,L_t)}{W_{c^1}(c^1_t,c^2_t,L_t)}\frac{W_{c^2}(c^1_t,c^2_t,L_t)}{u_{c^2}(c^1_t,c^2_t,L_t)}\\
    &= \frac{1+\mu\left[\frac{u_{c_1c_2}c_1+u_{c_2}+u_{c_2c_2}c_2+u_{Lc_2}L}{u_{c_2}}\right]}{1+\mu\left[\frac{u_{c_1c_1}c_1+u_{c_1}+u_{c_2c_1}c_2+u_{Lc_1}L}{u_{c_1}}\right]}
\end{align*}
one conclusion (\textbf{\color{myred}Friedman rule}) is if $u(c^1,c^2,L)=h(c_1,c_2)-v(L)$ and $h(\cdot)$ is homothetic: $\frac{h_1(\alpha c_1,\alpha c_2)}{h_2(\alpha c_1,\alpha c_2)}=\frac{h_1(c_1,c_2)}{h_2(c_1,c_2)}$, then optimal $i_t = 0$.

\section*{Job search}
Setting:
\begin{itemize}
    \item[-] \textbf{pool of unemployed}: $u$
    \item[-] \textbf{output when employed}: $y$
    \item[-] \textbf{utility of unemployed}: $b$
    \item[-] \textbf{number of vacancies}: $v$
    \item[-] \textbf{number of matches}: $h=m(u,v)$
    \item[-] \textbf{market tightness}: $\theta= \frac{v}{u}$
    \item[-] \textbf{vacancy filling rate}: $\frac{h}{v}=\frac{m(u,v)}{v}=m\left(\frac{1}{\theta},1\right)\equiv q(\theta)$, {\color{myred}$q'(\theta)<0$: tighter market (relatively more vacancies to fill) is harder to fill}
    \item[-] \textbf{job finding rate}: $\frac{h}{u} = \frac{h}{v}\frac{v}{u} = q(\theta)\theta = m(1,\theta)$, {\color{myred}$\partial\theta q(\theta)/\partial \theta>0$: tighter market (relatively more vacancies to fill) is better for job searchers}.
    \item[-] \text{cost of vacancy}: $c$
    \item[-] Nash bargaining: outside option for worker $b$, outside option for firm $0$
\end{itemize}

\subsection*{Static version}
\subsubsection*{bargaining problem}
Firms and job searchers do the bargaining
$$
\max_{w} (w-b)^{\beta}(y-w)^{1-\beta} \Rightarrow w=\beta y+(1-\beta)b
$$
where $\beta$ is the bargaining power of worker. 

Surplus the worker obtain is
$$
w = {\color{myblue}b}+{\color{myred}\beta(y-b)} \equiv {\color{myblue}\text{outside option}}+{\color{myred}\text{surplus}}
$$
Firm's profit is 
$$
-c +\frac{h}{v}(y-w) = -c + q(\theta)(1-\beta)(y-b)
$$
since it's competitive market, firms earn 0 profit:
$$
q(\theta) = \frac{c}{(1-\beta)(y-b)}
$$
\subsubsection*{planner's problem}
Social planner chooses the number of vacancy to create:
$$
\max_v m(u,v)(y-b)-cv
$$
FOC gives
$$
\frac{\partial m(u,v)}{\partial v} = \frac{c}{y-b}
$$
\subsubsection*{efficiency}
For social planner's choice and bargaining result to coincide, we need:
$$
\frac{\partial m(u,v)}{\partial v} = (1-\beta)q(\theta) \Rightarrow 1-\beta = \frac{\partial m(u,v)}{\partial v}\frac{v}{m(u,v)}
$$
that is, the {\color{myred}bargaining power of \textbf{firms} $1-\beta$} equals the {\color{myblue}\textbf{elasticity} of matches w.r.t. vacancies $\frac{\partial m(u,v)}{\partial v}\frac{v}{m(u,v)}$}. If $\beta$ is too high (workers have more power), then the vacancies are insufficient, higher level of unemployment.

\subsection*{Dynamic version}
let $s$ be the Poisson intensity of separation, it is constant and exogenous, then in \textbf{continuous time}:
\begin{itemize}
    \item[-] \textbf{flow of hires}: $h=m(u,v)$
    \item[-] \textbf{dynamic of unemployment}: $\dot{u}=s(1-u)-m(u,v)$
    \item[-] value:
    \begin{itemize}
        \item[-] $E$: value of being employed
        \item[-] $U$: value of being unemployed
        \item[-] $J$: firm's value of having an employee
        \item[-] $V$: firm's value of having an open vacancy
    \end{itemize}
    \item[-] bargaining: continuous, \textbf{separation} is the outside option
\end{itemize}
\textbf{Workers}' value
\begin{align*}
    rU &= b+\frac{h}{u}(E-U) +\dot{U}\\
    rE &= w+s(U-E) + \dot{E}
\end{align*}
\textbf{Firms}' value
\begin{align*}
    rV &= -c+\frac{h}{v}(J-V)+\dot{V}\\
    rJ &= y-w + s(V-J) + \dot{J}
\end{align*}
\textbf{surplus}
$$
S = E-U+J-V
$$
Nash bargaining gives:
\begin{align*}
    E &= U + \beta S & \text{workers}\\
    J &= V + (1-\beta) S & \text{firms}
\end{align*}
then for workers:
\begin{align*}
    r(E-U) &= w-b - \left(s+\frac{h}{u}\right)(E-U)+\dot{E}-\dot{U}\\
    \left(r+s+\frac{h}{u}\right)\beta S &= w-b + \beta\dot{S}
\end{align*}
and for firms:
\begin{align*}
    r(J-V) &= y-w+c - \left(s+\frac{h}{v}\right)(J-V)+\dot{J}-\dot{V}\\
    \left(r+s+\frac{h}{v}\right)(1-\beta) S &= y-w+c + (1-\beta)\dot{S}
\end{align*}
Free entry condition for firms $V=0,\dot{V}=0$ gives:
\begin{align*}
    rV=-c+\frac{h}{v}(J-V)+\dot{V}\xRightarrow{V=0,\dot{V}=0} &\frac{h}{v} \equiv q(\theta) = \frac{c}{J}\\
    J=V+(1-\beta)S \xRightarrow{V=0,\dot{V}=0} & S=\frac{J}{1-\beta} = \frac{c}{(1-\beta)q(\theta)}
\end{align*}

and
$$
\begin{aligned}
    \left(r+s+\frac{h}{u}\right)\beta S &= w-b + \beta\dot{S}\\
    \left(r+s+\frac{h}{v}\right)(1-\beta) S &= y-w+c + (1-\beta)\dot{S}
\end{aligned} \Rightarrow \left(\frac{h}{u}-\frac{h}{v}\right)\beta(1-\beta)S+(1-\beta)b+\beta(y+c)=w
$$
which gives the wage $w$:
\begin{align*}
    w&=\left(\frac{h}{u}-\frac{h}{v}\right)\beta(1-\beta)S+(1-\beta)b+\beta(y+c)\\
    &=(\theta q(\theta)-q(\theta))\beta(1-\beta)\frac{c}{(1-\beta)q(\theta)} + (1-\beta)b+\beta(y+c)\\
    &= (1-\beta)b+\beta y+\beta\theta c
\end{align*}

as for the market tightness:
$$
\begin{aligned}
    rJ=y-w+s(V-J)+\dot{J}\xRightarrow{V=0}& (r+s)J=y-w+\dot{J}\\
    J = \frac{c}{q(\theta)}\Rightarrow& \dot{J}=-\frac{c}{q(\theta)^2}q'(\theta)\dot{\theta}
\end{aligned}\Rightarrow (r+s)\frac{c}{q(\theta)}=y-w-\frac{c}{q(\theta)^2}q'(\theta)\dot{\theta}
$$
plug in $w=(1-\beta)b+\beta y+\beta\theta c$, get
\begin{align*}
    & (r+s)\frac{c}{q(\theta)} = y- (1-\beta)b-\beta y-\beta\theta c-\frac{c}{q(\theta)^2}q'(\theta)\dot{\theta}\\
    \text{stable requires }\dot{\theta}=0 \Rightarrow & \frac{r+s}{q(\theta)}+\beta\theta = \frac{(1-\beta)(y-b)}{c}
\end{align*}

this gives the elasticity of $\theta$ w.r.t. $y-b$:
\begin{align*}
    \epsilon &={\color{myred}\frac{\mathrm{d}\theta}{\mathrm{d}(y-b)}}\frac{y-b}{\theta} \\
    &= {\color{myred} \frac{1-\beta}{c}\frac{1}{\beta-(r+s)\frac{q'(\theta)}{q(\theta)^2}} }\frac{y-b}{\theta} = \frac{1-\beta}{c}\frac{1}{\beta-(r+s)\frac{q'(\theta)}{q(\theta)^2}} \frac{\left(\frac{r+s}{q(\theta)}+\beta\theta\right)\frac{c}{1-\beta}}{\theta}\\
    &= \frac{\beta\theta q(\theta)+r+s}{\beta\theta q(\theta)+(r+s)\left(-\theta \frac{q'(\theta)}{q(\theta)}\right)}
\end{align*}

and unemployment $u$ can be derived from:
\begin{align*}
    &\dot{u} = s(1-u)-m(u,v) = s(1-u) - u\theta q(\theta)\\
    \xRightarrow{\text{steady state }\dot{u}=0}& u=\frac{s}{s+\theta q(\theta)}
\end{align*}

some conclusions can be drawn:
\begin{itemize}
    \item[-] $q(\theta)=\frac{c}{J}$: \textbf{more valuable} employess $J\uparrow$ leads to \textbf{tighter} market $\theta \uparrow$ 
    \item[-] $w=(1-\beta)b+\beta y+\beta\theta c$: \textbf{tighter} market $\theta\uparrow$ leads to \textbf{higher} wage $w\uparrow$
    \item[-] $\frac{r+s}{q(\theta)}+\beta\theta = \frac{(1-\beta)(y-b)}{c}$: \textbf{tightness} $\theta$ is \textbf{constant} over time.
    \item[-] the elasticity of $\theta$ w.r.t $y-b$, $\epsilon$ is smaller than empirical observations
\end{itemize}

\subsection*{Directed search with posted wages}
\begin{itemize}
    \item[-] \textbf{\color{myred}search procedure}:
    \begin{itemize}
        \item[(a)] firms announce and commit to a wage
        \item[(b)] workers choose what wages to apply to, each separate wage is like an individual market
        \item[(c)] in each market, bargaining happens.
    \end{itemize}
\end{itemize}
Then a \textbf{competitive search equilibrium} is 
\begin{itemize}
    \item[-] a value for workers $U\in [b,y]$
    \item[-] a set of wages that firms actually offer $W\subseteq [b,y]$
    \item[-] market tightness function $\theta:[b,y\rightarrow[0,\infty]$
\end{itemize}
s.t.
\begin{itemize}
    \item[(1)] Free entry: $q(\theta(w))[y-w]\leq c,\forall w\in[b,y] $, with equality if $w\in W$
    \item[(2)] worker optimality:
    \begin{itemize}
        \item[(a)] $\theta(w)q(\theta(w))w+[1-\theta(w)q(\theta(w))]b = U, \theta(w)<\infty$
        \item[(b)] $U=\max_w \theta(w)q(\theta(w))w+[1-\theta(w)q(\theta(w))]b$
    \end{itemize}
\end{itemize}

\vspace{2pt}
\begin{tikzpicture}
\node [redbox] (box){%
    \begin{minipage}{0.315\textwidth}
    \color{myred}
    \scriptsize
    The existence, uniqueness and efficiency is established as:
\begin{itemize}
    \item[A.] suppose $\left\{U,W,\theta\right\}$ is a CSE, then if $w\in W$, $(w,\theta(w))$ solves
    $$
    U = \max_{\theta,w}\theta q(\theta) w+[1-\theta q(\theta)]b
    $$
    s.t.
    $q(\theta)[y-w]\geq c$
    \item[B.] Conversely, suppose $\left\{\theta^*,w^*,U^*\right\}$ solves the optimization problem above, then exists a CSE $\left\{U,W,\theta\right\}$ where 
    \begin{itemize}
        \item[-] $W=\{w^*\}$
        \item[-] $\theta(w)=\theta^*$
        \item[-] $U=U^*$
    \end{itemize}
    \end{itemize}
    \end{minipage}
};
\node[redtitle, right=4pt] at (box.north west) {Existence and efficiency of CSE};
\end{tikzpicture}

\underline{\textbf{Proof}}:
\begin{itemize}
    \item[A] proof by contradiction: suppose $\exists w',\theta'$ s.t.
    \begin{align*}
        \theta'q(\theta')w'+[1-\theta'q(\theta')]b &>U \\
        q(\theta')[y-w']&\geq c
    \end{align*}
    then the utility by search for wage $w'$ is
    \begin{align*}
        &\theta(w') q(\theta(w'))w'+[1-\theta(w') q(\theta(w'))]b\\
        =& b+ \theta(w') q(\theta(w'))(w'-b)\\
        \text{\color{myblue}by (1) \& (2)} \geq & b+ q^{-1}\left(\frac{c}{y-w}\right)\frac{c}{y-w}(w'-b)\\
        \text{\color{myblue}by (3) \& (2)} \geq & b+\theta'q(\theta')(w'-b)\\
        >&U
    \end{align*}
    where
    \begin{itemize}
        \item[(1)] free entry condition: $q(\theta(w))\leq \frac{c}{y-w}\Rightarrow\theta(w)\geq q^{-1}\left(\frac{c}{y-w}\right)$ 
        \item[(2)] $\theta q(\theta)$ increases in $\theta$.
        \item[(3)] non-negative profit (incentive) constraint of planner: $q(\theta')\geq \frac{c}{y-w'}\Rightarrow \leq q^{-1}\left(\frac{c}{y-w'}\right)\geq \theta'$
    \end{itemize}
    this contradicts the worker's optimality.
    
    \item[B] Let the $\theta$ function be 
    $$
    \begin{cases}
        \theta(w)q(\theta(w))w+[1-\theta(w)q(\theta(w))]b = U^* & w>U^*\\
        \infty & \text{otherwise}
    \end{cases}
    $$
    worker optimality holds by construction, check free entry condition. Again, proof by contradiction: assume $\exists w'$ s.t.
    $$
    q(\theta(w'))[y-w']> c
    $$
    let $\theta'$ be $\theta' = q^{-1}\left(\frac{c}{y-w'}\right)$, then $\theta'>\theta(w')$, therefore
    $$
    \theta' q(\theta')w' + [1-\theta' q(\theta')]b > U^*
    $$
    since $\theta',w'$ satisfies the constraints, then $\theta^*,w^*$ cannot be the solution to the optimization problem.
    
\end{itemize}

\end{multicols*}
\end{document}