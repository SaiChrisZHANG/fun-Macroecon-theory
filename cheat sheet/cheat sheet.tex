\documentclass[10pt,landscape,a4paper]{article}
\usepackage[utf8]{inputenc}
\usepackage[ngerman]{babel}
\usepackage{tikz}
\usetikzlibrary{shapes,positioning,arrows,fit,calc,graphs,graphs.standard}
\usepackage[nosf]{kpfonts}
\usepackage{comment}
\usepackage{graphicx}
\usepackage[t1]{sourcesanspro}
%\usepackage[lf]{MyriadPro}
%\usepackage[lf,minionint]{MinionPro}
\usepackage{multicol}
\usepackage{wrapfig}
\usepackage[top=1mm,bottom=2mm,left=1mm,right=2mm]{geometry}
\usepackage[framemethod=tikz]{mdframed}
\usepackage{microtype}
\usepackage{hyperref}

\usepackage{url}
\usepackage{multirow}
\usepackage{esint}
\usepackage{amsfonts}
\usetikzlibrary{decorations.pathmorphing}

\usepackage{colortbl}
\usepackage{xcolor}
\usepackage{mathtools}
\usepackage{amsmath,amssymb}
\usepackage{enumitem}
\makeatletter

\let\bar\overline

\setlist[itemize]{topsep=0pt,leftmargin=10pt,itemsep=-0.2em}
\definecolor{myblue}{cmyk}{1,.72,0,.38}
\definecolor{mypurple}{cmyk}{.57,1,0,.58}
\definecolor{myred}{cmyk}{0,.88,.88,.58}
\definecolor{mygreen}{cmyk}{1,0,.69,.66}
\definecolor{myorange}{cmyk}{0,.58,100,.20}

\def\firstcircle{(0,0) circle (1.5cm)}
\def\secondcircle{(0:2cm) circle (1.5cm)}

\pgfdeclarelayer{background}
\pgfsetlayers{background,main}

\renewcommand{\baselinestretch}{.8}
\pagestyle{empty}

\global\mdfdefinestyle{header}{%
linecolor=gray,linewidth=1pt,%
leftmargin=0mm,rightmargin=0mm,skipbelow=0mm,skipabove=0mm,
}

%\newcommand{\header}{
%\begin{mdframed}[style=header]
%\scriptsize
%\sffamily
%Cheat sheet\\
%by~Your~Name,~page~\thepage~of~2
%\end{mdframed}
%}

\makeatletter
\renewcommand{\section}{\@startsection{section}{1}{0mm}{1ex}{.2ex}{\normalsize\bfseries}}
\renewcommand{\subsection}{\@startsection{subsection}{1}{0mm}{.2ex}{.2ex}{\bfseries}}

\newcommand*\bigcdot{\mathpalette\bigcdot@{.5}}
\newcommand*\bigcdot@[2]{\mathbin{\vcenter{\hbox{\scalebox{#2}{$\m@th#1\bullet$}}}}}
\makeatother

\def\multi@column@out{%
   \ifnum\outputpenalty <-\@M
   \speci@ls \else
   \ifvoid\colbreak@box\else
     \mult@info\@ne{Re-adding forced
               break(s) for splitting}%
     \setbox\@cclv\vbox{%
        \unvbox\colbreak@box
        \penalty-\@Mv\unvbox\@cclv}%
   \fi
   \splittopskip\topskip
   \splitmaxdepth\maxdepth
   \dimen@\@colroom
   \divide\skip\footins\col@number
   \ifvoid\footins \else
      \leave@mult@footins
   \fi
   \let\ifshr@kingsaved\ifshr@king
   \ifvbox \@kludgeins
     \advance \dimen@ -\ht\@kludgeins
     \ifdim \wd\@kludgeins>\z@
        \shr@nkingtrue
     \fi
   \fi
   \process@cols\mult@gfirstbox{%
%%%%% START CHANGE
\ifnum\count@=\numexpr\mult@rightbox+2\relax
          \setbox\count@\vsplit\@cclv to \dimexpr \dimen@-1cm\relax
\setbox\count@\vbox to \dimen@{\vbox to 1cm{\header}\unvbox\count@\vss}%
\else
      \setbox\count@\vsplit\@cclv to \dimen@
\fi
%%%%% END CHANGE
            \set@keptmarks
            \setbox\count@
                 \vbox to\dimen@
                  {\unvbox\count@
                   \remove@discardable@items
                   \ifshr@nking\vfill\fi}%
           }%
   \setbox\mult@rightbox
       \vsplit\@cclv to\dimen@
   \set@keptmarks
   \setbox\mult@rightbox\vbox to\dimen@
          {\unvbox\mult@rightbox
           \remove@discardable@items
           \ifshr@nking\vfill\fi}%
   \let\ifshr@king\ifshr@kingsaved
   \ifvoid\@cclv \else
       \unvbox\@cclv
       \ifnum\outputpenalty=\@M
       \else
          \penalty\outputpenalty
       \fi
       \ifvoid\footins\else
         \PackageWarning{multicol}%
          {I moved some lines to
           the next page.\MessageBreak
           Footnotes on page
           \thepage\space might be wrong}%
       \fi
       \ifnum \c@tracingmulticols>\thr@@
                    \hrule\allowbreak \fi
   \fi
   \ifx\@empty\kept@firstmark
      \let\firstmark\kept@topmark
      \let\botmark\kept@topmark
   \else
      \let\firstmark\kept@firstmark
      \let\botmark\kept@botmark
   \fi
   \let\topmark\kept@topmark
   \mult@info\tw@
        {Use kept top mark:\MessageBreak
          \meaning\kept@topmark
         \MessageBreak
         Use kept first mark:\MessageBreak
          \meaning\kept@firstmark
        \MessageBreak
         Use kept bot mark:\MessageBreak
          \meaning\kept@botmark
        \MessageBreak
         Produce first mark:\MessageBreak
          \meaning\firstmark
        \MessageBreak
        Produce bot mark:\MessageBreak
          \meaning\botmark
         \@gobbletwo}%
   \setbox\@cclv\vbox{\unvbox\partial@page
                      \page@sofar}%
   \@makecol\@outputpage
     \global\let\kept@topmark\botmark
     \global\let\kept@firstmark\@empty
     \global\let\kept@botmark\@empty
     \mult@info\tw@
        {(Re)Init top mark:\MessageBreak
         \meaning\kept@topmark
         \@gobbletwo}%
   \global\@colroom\@colht
   \global \@mparbottom \z@
   \process@deferreds
   \@whilesw\if@fcolmade\fi{\@outputpage
      \global\@colroom\@colht
      \process@deferreds}%
   \mult@info\@ne
     {Colroom:\MessageBreak
      \the\@colht\space
              after float space removed
              = \the\@colroom \@gobble}%
    \set@mult@vsize \global
  \fi}

\hypersetup{
    colorlinks=true,
    linkcolor=myblue,
    filecolor=magenta,      
    urlcolor=myblue,
    pdfpagemode=FullScreen,
    }

\urlstyle{same}

\makeatother
\setlength{\parindent}{0pt}

% material references: Prof. Geert Ridder's lecture notes
% latex coding references: https://github.com/tim-st/latex-cheatsheet, https://www.overleaf.com/latex/templates/hoja-de-ecuaciones-electricidad-y-magnetismo/xwgjqkrjjgcb

\begin{document}
\begin{center}{\large{\textbf{Macroeconomic Model Cheat Sheet}}}\\
Author: Sai Zhang (\href{mailto:saizhang.econ@gmail.com}{email} me or check my \href{https://github.com/SaiChrisZHANG}{Github} page)
\end{center}

\small
\begin{multicols*}{4}

% set box styles: in this file, I only use red blue and purple boxes
%% blue boxes
\tikzstyle{bluebox} = [draw=myblue, fill=white, thick, rectangle, rounded corners, inner sep=5pt, inner ysep=10pt, text=myblue]
\tikzstyle{bluetitle} =[fill=myblue, text=white, font=\bfseries]
\tikzstyle{ibluebox} = [draw=myblue, fill=myblue, thick, rectangle, rounded corners, inner sep=5pt, inner ysep=10pt, text=white]
\tikzstyle{ibluetitle} =[draw=myblue, fill=white, text=myblue, font=\bfseries]
%% red boxes
\tikzstyle{redbox} = [draw=myred, fill=white, thick, rectangle, rounded corners, inner sep=5pt, inner ysep=10pt, text=myred]
\tikzstyle{redtitle} =[fill=myred, text=white, font=\bfseries]
\tikzstyle{iredbox} = [draw=myred, fill=myred, thick, rectangle, rounded corners, inner sep=5pt, inner ysep=10pt, text=white]
\tikzstyle{iredtitle} =[draw=myred, fill=white, text=myred, font=\bfseries]
%% purple boxes
\tikzstyle{purplebox} = [draw=mypurple, fill=white, thick, rectangle, rounded corners, inner sep=5pt, inner ysep=10pt, text=mypurple]
\tikzstyle{purpletitle} =[fill=mypurple, text=white, font=\bfseries]
\tikzstyle{ipurplebox} = [draw=mypurple, fill=mypurple, thick, rectangle, rounded corners, inner sep=5pt, inner ysep=10pt, text=white]
\tikzstyle{ipurpletitle} =[draw=mypurple, fill=white, text=mypurple, font=\bfseries]
%% orange boxes
\tikzstyle{orangebox} = [draw=myorange, fill=white, thick, rectangle, rounded corners, inner sep=5pt, inner ysep=10pt, text=myorange]
\tikzstyle{orangetitle} =[fill=myorange, text=white, font=\bfseries]
\tikzstyle{iorangebox} = [draw=myorange, fill=myorange, thick, rectangle, rounded corners, inner sep=5pt, inner ysep=10pt, text=white]
\tikzstyle{iorangetitle} =[draw=myorange, fill=white, text=myorange, font=\bfseries]

\vspace{2pt}
\begin{tikzpicture}
\node [orangebox] (box){%
    \begin{minipage}{0.23\textwidth}
    \color{myorange}
    \scriptsize
    \begin{itemize}
        \item[-] \textbf{Primitive assumptions}:
        \begin{itemize}
            \item[-] Who the \textbf{agents} are, what are their \textbf{preferences} and objective functions
            \item[-] What \textbf{technology} agents can access
            \item[-] What \textbf{endowment} agents have
        \end{itemize}
        \item[-] \textbf{Decision problems}: resource allocation problem (among agents, over time, etc.).
        \item[-] \textbf{Information sets}: what do agents know, how will their knowledge change, what is their \textbf{expectation}.
        \item[-] \textbf{Allocation mechanism}: how agents interact and achieve equilibrium. 2 main mechanisms are:
        \begin{itemize}
            \item[-] \textbf{price system} in competitive equilibrium
            \item[-] benevolent \textbf{central planner} maximizes a social welfare function.
        \end{itemize}
    \end{itemize}
    \end{minipage}
};
\node[orangetitle, right=4pt] at (box.north west) {Modelling essential: what we need to decide};
\end{tikzpicture}

\section*{Infinitely Lived Agent Model}

\vspace{2pt}
\begin{tikzpicture}
\node [ibluebox] (box){%
    \begin{minipage}{0.23\textwidth}
    \color{white}
    \scriptsize
    \begin{itemize}
        \item[-] \textbf{discrete} time, indexed by $t$
        \item[-] economy lives \textbf{infinitely}, $t=0,1,2,\cdots$
        \item[-] single commodity exogenously produced, indexed by $t$, pure \textbf{exchange/endowment} economy.
        \item[-] no firms/government, only \textbf{two types of households}.
        \item[-] each type of households is continuum of \textbf{identical} households of that type, they are \textbf{price takers}, can be represented by a \textbf{representative} household
    \end{itemize}
    \end{minipage}
};
\node[ibluetitle, right=4pt] at (box.north west) {Features};
\end{tikzpicture}

\vspace{2pt}
\begin{tikzpicture}
\node [bluebox] (box){%
    \begin{minipage}{0.23\textwidth}
    \color{myblue}
    \scriptsize
    Utility of type $i$ household is
    $$U(c^i)=\sum^{\infty}_{t=0}\beta^t_i u(c_t^i)$$
    where $\left(c^i\right)=\left\{c_t^i\right\}^{\infty}_{t=0}$, $\beta_i\in(0,1)$,
    
    The utility function $u(c_t^i)$ is assumed to be:
    \begin{itemize}
        \item[-] \textbf{continuously differentiable} of the second order
        \item[-] \textbf{monotonically increasing, strictly concave}: $u'(c_t^i)>0,u''(c_t^i)<0$
        \item[-] \textbf{satisfies Inada conditions} (never 0 or infinity consumption): $\lim_{c^i_t\rightarrow \infty}u'(c_t^i)=0, \lim_{c^i_t\rightarrow 0}u'(c_t^i)=\infty$
        \item[-] \textbf{time additivity}: $u(c_t^i)$ is independent of $c_{t+j}^i$, $c_{t-j}^i$.
        \item[-] \textbf{impatient discounting} $\beta_i<1$: households value today's consumption more than future's.
        \item[-] Constant relative risk aversion (\textbf{CRRA}): $u(c_t^i)=\frac{c^{1-\sigma}-1}{1-\sigma}$    
        \tiny
        $\left(\lim_{\sigma\rightarrow1}\frac{c^{1-\sigma}-1}{1-\sigma}=\lim_{\sigma\rightarrow1}\frac{e^{(1-\sigma)\ln(c)}-1}{1-\sigma}=\ln(c)\right)$, 
        \scriptsize
        \textbf{The RRA coefficient} $\sigma(c)=$
        \tiny
        $\frac{-u''(c_t^i)c}{u'(c_t^i)}=\frac{-\left(-\sigma c^{-(1+\sigma)}c\right)}{c^{-\sigma}}$
        \scriptsize
        $=\sigma$. Higher RRA means higher risk aversion. 
        \item[-] Constant intertemporal elasticity of substitution \textbf{(IES)}: $$IES=-\frac{\mathrm{d}\ln\left(c_{t+1}/c_t\right)}{\mathrm{d}\ln\left(u'(c_{t+1}/u'(c_t)\right)}=\frac{1}{\sigma}$$
        hence higher RRA (more risk-averse), lower IES (consumption variation over time).
    \end{itemize}
    \end{minipage}
};
\node[bluetitle, right=4pt] at (box.north west) {Agents' preferences};
\end{tikzpicture}

\vspace{2pt}
\begin{tikzpicture}
\node [bluebox] (box){%
    \begin{minipage}{0.23\textwidth}
    \color{myblue}
    \scriptsize
    A deterministic endowment stream of the consumption good for type $i$ household is
    $$w^i=\left(w^i_0,w_1^i,\cdots\right)=\left\{ w^i_t \right\}^{\infty}_{t=0}$$
    
    \end{minipage}
};
\node[bluetitle, right=4pt] at (box.north west) {Agents' endowment};
\end{tikzpicture}

\subsection*{Arrow-Debreu Market (AD) approach}

\vspace{2pt}
\begin{tikzpicture}
\node [redbox] (box){%
    \begin{minipage}{0.23\textwidth}
    \color{myred}
    \scriptsize
    Households trade just \textbf{once} in $t=0$ market, they trade all future consumption and deliver the promised amount in $t=1,2,\cdots$ market. 
    
    Households have perfect information of the entire endowment sequence, all information is public.
    
    \end{minipage}
};
\node[redtitle, right=4pt] at (box.north west) {Market structure: Basic Case};
\end{tikzpicture}

\vspace{2pt}
\begin{tikzpicture}
\node [iredbox] (box){%
    \begin{minipage}{0.23\textwidth}
    \color{white}
    \scriptsize
    \begin{itemize}
        \item[-] allocation: $\left\{\hat{c}^1_t,\hat{c}^2_t\right\}^{\infty}_{t=0}$
        \item[-] regulating mechanism: $\left\{\hat{p}_t\right\}^{\infty}_{t=0}$, with numeraire $\hat{p}_0=1$
    \end{itemize}
    such that:
    \begin{itemize}
        \item[-] given $\left\{\hat{p}_t\right\}^{\infty}_{t=0}$, $\left\{\hat{c}^1_t,\hat{c}^2_t\right\}^{\infty}_{t=0}$ solves:
        $$\max \sum^{\infty}_{t=0}\beta^t u(c_t^i)\ \text{ s.t. }\sum^{\infty}_{t=0}\hat{p}_t c^i_t\leq \sum^{\infty}_{t=0}\hat{p}_t w^i_t,c^i_t\geq0$$
    \item[-] market clearing (disposal of unused goods is costly):
    $$\hat{c}^1_t+\hat{c}^2_t=w^1_t+w^2_t,\forall t$$
    \end{itemize}
    
    \vspace{2pt}
    \begin{tikzpicture}
    \node [redbox] (box){%
        \begin{minipage}{0.935\textwidth}
        \color{myred}
        \scriptsize
        \underline{\textit{\textbf{How to solve}}}:
        \begin{itemize}
            \item[-] \textbf{\textit{Step 1}}: solve the Lagrangian
            $$\sum^{\infty}_{t=0}\beta^t\ln(c_t^i)+\lambda^i\left(\sum^{\infty}_{t=0}p_t w_t^i-\sum^{\infty}_{t=0}p_t c_t^i\right)$$
            FOCs: $\beta^t/c_t^i=\lambda^i p_t\Rightarrow \frac{\beta^t}{c_t^i p_t}=\frac{\beta^{t+1}}{c_{t+1}^i p_{t+1}}\Rightarrow c^i_{t+1}=\beta\frac{p_t}{p_{t+1}}c^i_t$. FOC gives that price changes $p_t/p_{t+1}$ and subjective discounting $\beta$ determines consumption smoothing.
            \item[-] \textbf{\textit{Step 2}}: Use market clearing condition
            $$c_t^1 +c_t^2=w_t^1+w_t^2$$
            and FOC $p_{t+1}c_{t+1}^i=\beta p_t c_t^i$, get $\frac{p_{t+1}}{p_{t}}=\beta\frac{w_t^1+w_t^2}{w_{t+1}^1+w_{t+1}^2}$, combined with the numeraire assumption $p_0=1$, solve the price sequence $\left\{\hat{p}_t\right\}^{\infty}_{t=0}$.
        \item[-] \textit{\textbf{Step 3}}: Plug $\left\{\hat{p}_t\right\}^{\infty}_{t=0}$ and $p_{t+1}c_{t+1}^i=\beta p_t c_t^i$ back to budget constraint
        $$\sum^{\infty}_{t=0}p_t c_t^i=\sum^{\infty}_{t=0}\beta^t c_0^i=\frac{c_0^i}{1-\beta}= \sum^{\infty}_{t=0}p_t w_t^i$$
        to solve allocation $\left\{\hat{c}^1_t,\hat{c}^2_t\right\}^{\infty}_{t=0}$
        \end{itemize}
        \end{minipage}
    };
    \end{tikzpicture}
    \end{minipage}
};
\node[iredtitle, right=4pt] at (box.north west) {Equilibrium: Basic Case};
\end{tikzpicture}

\vspace{2pt}
\begin{tikzpicture}
\node [redbox] (box){%
    \begin{minipage}{0.23\textwidth}
    \color{myred}
    \scriptsize
    An allocation $\left(c^1,c^2\right)=\left\{c_t^1,c_t^2\right\}^{\infty}_{t=0}$ is \textbf{Pareto efficient} if:
    \begin{itemize}
        \item[-] it is feasible: $\sum^{\infty}_{t=0}p_t c^i_t\leq \sum^{\infty}_{t=0}p_t w^i_t$
        \item[-] no other feasible allocation $(\tilde{c}^1,\tilde{c}^2)$ such that $\forall i, U(\tilde{c}^i)\geq U(c^i)$ and $\exists i, U(\tilde{c}^i)\geq U(c^i)$
    \end{itemize}
    
    \vspace{2pt}
    \begin{tikzpicture}
    \node [iredbox] (box){%
        \begin{minipage}{0.935\textwidth}
        \color{white}
        \scriptsize
        
        \textbf{An AD competitive equilibrium allocation $\left(c^1,c^2\right)=\left\{c^1_t,c^2_t\right\}^{\infty}_{t=0}$ is Pareto efficient.}
        \end{minipage}
    };
    \end{tikzpicture}
    
    \vspace{2pt}
    \underline{\textit{\textbf{Proof}}}:
    Suppose there is an allocation $\left(\bar{c}^1,\bar{c}^2\right)=\left\{\bar{c}^1_t,\bar{c}^2_t\right\}^{\infty}_{t=0}$, Pareto-dominating AD allocation $\left(\hat{c}^1,\hat{c}^2\right)=\left\{\hat{c}^1_t,\hat{c}^2_t\right\}^{\infty}_{t=0}$.
    
    If the Pareto dominating allocation $\left(\bar{c}^1,\bar{c}^2\right)$ exists, its utility $\bar{U}=U(\bar{c}^1,\bar{c}^2)$ must be bigger than the AD allocation utility $\tilde{U}=U(\tilde{c}^1,\tilde{c}^2)$, therefore, the only reason that it is not chosen as the AD allocation is that it is \textbf{infeasible}.
    
    \vspace{2pt}
    Formally, suppose $\bar{c}^1>\hat{c}^1$ and $\bar{c}^2\geq \hat{c}^2$,
    \begin{itemize}
        \item[-] \textbf{Step 1}: for household 1 ($\hat{c}^1<\bar{c}^1$), if $\sum^{\infty}_{t=0}\hat{p}_t \bar{c}^1_t \leq \sum^{\infty}_{t=0} \hat{p}_t\hat{c}^1_t=\sum^{\infty}_{t=0}\hat{p}_t w_t^1$ (the Pareto-dominating allocation is also feasible), the AD equilibrium $(\hat{c}^1,\hat{c}^2)$ will NOT maximize HH1's utility, hence contradiction.
        \item[-] \textbf{Step 2}: for household 2 ($\hat{c}^2\leq\bar{c}^2$), if $\sum^{\infty}_{t=0}\hat{p}_t \bar{c}^2_t<\sum^{\infty}_{t=0}\hat{p}_t\hat{c}^2_t=\sum^{\infty}_{t=0}\hat{p}_t w_t^2$ (the Pareto-superior allocation cost less for HH2), then $\exists \delta>0$ s.t. $\sum^{\infty}_{t=0}\hat{p}_t\bar{c}^2_t+\delta\leq \sum^{\infty}_{t=0}\hat{p}_t\hat{c}^c_t=\sum^{\infty}_{t=0}\hat{p}_t w_t^2$, then there is always an allocation $\left\{\bar{c}_0^2+\delta,\bar{c}_t^2\right\}$, achieves a strictly higher utility than the AD allocation (which is utility maximizing), hence contradiction.
        \item[-] \textbf{Step 3}: from \textbf{Step 1}, $\sum^\infty_{t=0}\hat{p}_t\bar{c}^1_t>\sum^{\infty}_{t=0}\hat{p}_t\hat{c}^1_t=\sum^{\infty}_{t=0}\hat{p}_t w^1_t$; from \textbf{Step 2}, $\sum^{\infty}_{t=0}\hat{p}_t\bar{c}^2_t\geq \sum^{\infty}_{t=0}\hat{p}_t \hat{c}^2_t=\sum^{\infty}_{t=0}\hat{p}_t w^2_t$, then
        $$\sum_{i=1,2}\sum^{\infty}_{t=0}\hat{p}_t w_t^i<\sum_{i=1,2}\sum^{\infty}_{t=0}\hat{p}_t\bar{c}^i_t$$
    \end{itemize}
    Therefore, this Pareto allocation is actually infeasible.
    
    This proof requires \textbf{the value of the aggregate endowment is finite}, which is quite intuitive.
    \end{minipage}
};
\node[redtitle, right=4pt] at (box.north west) {Pareto efficiency: Basic Case};
\end{tikzpicture}

\vspace{2pt}
\begin{tikzpicture}
\node [iredbox] (box){%
    \begin{minipage}{0.23\textwidth}
    \color{white}
    \scriptsize
    By \textbf{1st welfare theorem}, we can solve the competitive equilibrium allocation by solving Pareto efficient allocation. This is the social planner's problem: a \textbf{weighted} utility maximization problem.
    
    \begin{itemize}
        \item[-] allocation: $\left\{\hat{c}^1_t,\hat{c}^2_t\right\}^{\infty}_{t=0}$
        \item[-] utility weight: $\left\{\hat{\alpha}^1,\hat{\alpha}^2\right\}$
    \end{itemize}
    such that:
    \begin{itemize}
        \item[-] given $\left\{\hat{\alpha}^1,\hat{\alpha}^2\right\}$, $\left\{\hat{c}^1_t,\hat{c}^2_t\right\}^{\infty}_{t=0}$ solves:
        $$\max \alpha^1 \sum^{\infty}_{t=0}\beta^t u(c_t^1)+\alpha^2 \sum^{\infty}_{t=0}\beta^t u(c_t^2)$$
        s.t. $c^1_t+c^2_t\leq w^1_t+w^2_t,\alpha^1+\alpha^2=1,\alpha^i,c^i_t\geq0$
    \item[-] market clearing is the budget constraint.
    \end{itemize}
    
    \vspace{2pt}
    \begin{tikzpicture}
    \node [redbox] (box){%
        \begin{minipage}{0.935\textwidth}
        \color{myred}
        \scriptsize
        \underline{\textit{\textbf{How to solve}}}:
        
        solve the Lagrangian
        $$\alpha^1 \sum^{\infty}_{t=0}\beta^t u(c_t^1)+\alpha^2 \sum^{\infty}_{t=0}\beta^t u(c_t^2)+\sum^{\infty}_{t=0}\mu_t\left(w_t^1+w_t^2-c_t^1-c_t^2\right)$$
        FOC gives $\alpha^1\beta^t u'(c_t^1)=\mu_t=\alpha^2\beta^t u'(c_t^2)\Rightarrow \alpha^1 u'(c_t^1)=\alpha^2 u'(c_t^2)$, plug them back to $c_t^1+c_t^2=w_t^1+w_t^2$, solve $(c_t^1,c_t^2)=\left(c_t^1(\alpha_1,\alpha_2),c_t^2(\alpha_1,\alpha_2)\right)$. 
        
        \textbf{The Lagrangian multiplier $\mu_t$ is the AD equilibrium prices, normalized by the total endowment each period}:
        $\hat{\alpha}^i\beta^t u'(\hat{c}^i_t)=\hat{\mu}_t,\beta^t u'(\hat{c}^i_t)=\hat{\lambda}^i\hat{p}_t\Rightarrow \frac{\hat{\mu}_t}{\hat{\alpha}^i}=\hat{\lambda}^i\hat{p}_t$
        \end{minipage}
    };
    \end{tikzpicture}
    
    \vspace{4pt}
    \textbf{All the PE allocations are Pareto efficient}, but only one is AD equilibrium allocation, that allocation needs to satisfy the AD budget constraint, achieved by \underline{\textit{\textbf{transfer}}}. This procedure, Negishi "trick", follows the second welfare theorem: every Pareto-efficient allocation can be decentralized as an equilibrium with transfers.
    
    \vspace{2pt}
    \begin{tikzpicture}
    \node [redbox] (box){%
        \begin{minipage}{0.935\textwidth}
        \color{myred}
        \scriptsize
        
        The AD equilibrium with transfer is:
        \begin{itemize}
            \item[-] allocation: $\left\{\hat{c}^1_t,\hat{c}^2_t\right\}^{\infty}_{t=0}$
            \item[-] lifetime transfer: $\left\{\hat{t}^1,\hat{t}^2\right\}$
            \item[-] regulating mechanism: $\left\{\hat{p}_t\right\}^{\infty}_{t=0}$, with numeraire $\hat{p}_0=1$
        \end{itemize}
        such that:
        \begin{itemize}
            \item[-] given $\left\{\hat{p}_t\right\}^{\infty}_{t=0}$, $\left\{\hat{c}^1_t,\hat{c}^2_t\right\}^{\infty}_{t=0}$ solves:
            $$\max \sum^{\infty}_{t=0}\beta^t u(c_t^i)\ \text{ s.t. }\sum^{\infty}_{t=0}\hat{p}_t c^i_t\leq \sum^{\infty}_{t=0}\hat{p}_t w^i_t+\hat{t}^i_t,c^i_t\geq0$$
        \item[-] market clearing: $\hat{c}^1_t+\hat{c}^2_t=w^1_t+w^2_t,\forall t$
        \end{itemize}
        \end{minipage}
    };
    \end{tikzpicture}
    
    But the solving is actually much easier, we just solve the zero lifetime transfer condition:
    $$t^i(\alpha)\equiv \sum^{\infty}_{t=0}\mu_t\left(c^i_t(\alpha)-w_t^i\right) =\sum^{\infty}_{t=0}\alpha^i\beta^t u'(c_t^i)\left(c^i_t(\mathbf{\alpha})-w_t^i\right)=0$$
    In general, transfer function $t^i(\mathbf{\alpha})$ satisfies \textbf{zero sum}: $t^1(\mathbf{\alpha})+t^2(\mathbf{\alpha})=0$; and \textbf{homogeneous of degree 1} $t^i(k\mathbf{\alpha})=kt^i(\mathbf{\alpha})$.
    
    \end{minipage}
};
\node[iredtitle, right=4pt] at (box.north west) {PE allocation: Social planner's problem};
\end{tikzpicture}

\subsection*{Sequential Market (SM) approach}

\vspace{2pt}
\begin{tikzpicture}
\node [redbox] (box){%
    \begin{minipage}{0.23\textwidth}
    \color{myred}
    \scriptsize
    Households trade in spot markets for immediate delivery of consumption goods at every $t$, \textbf{bond} is traded, (purchasing at $t$ denoted by $a^i_{t+1}$), they are traded at $t$, representing one unit of consumption at $t+1$. The interest rate of bonds $r_{t+1}$ regulates the market: bond of 1 unit of consumption at $t$ will be compensated by $(1+r_{t+1})$ units of consumption at $t+1$.
    
    Households have perfect information of the entire endowment sequence, all information is public.
    
    \end{minipage}
};
\node[redtitle, right=4pt] at (box.north west) {Market structure};
\end{tikzpicture}

\vspace{2pt}
\begin{tikzpicture}
\node [iredbox] (box){%
    \begin{minipage}{0.23\textwidth}
    \color{white}
    \scriptsize
    \begin{itemize}
        \item[-] allocation: $\left\{\left\{\tilde{c}^1_t,\tilde{c}^2_t\right\},\left\{\tilde{a}^1_{t+1},\tilde{a}^2_{t+1}\right\}\right\}^{\infty}_{t=0}$
        \item[-] regulating mechanism: $\left\{\tilde{r}_{t+1}\right\}^{\infty}_{t=0}$
    \end{itemize}
    such that:
    \begin{itemize}
        \item[-] given $\left\{\tilde{r}_{t+1}\right\}^{\infty}_{t=0}$, $\left\{\left\{\tilde{c}^1_t,\tilde{c}^2_t\right\},\left\{\tilde{a}^1_{t+1},\tilde{a}^2_{t+1}\right\}\right\}^{\infty}_{t=0}$ solves:
        $$\max \sum^{\infty}_{t=0}\beta^t u(c_t^i)$$
        s.t. $c^i_t+\frac{a^i_{t+1}}{1+\tilde{r}_{t+1}}\leq w^i_t+a^i_t,\ c^i_t\geq0,\ a^i_{t+1}\geq-\bar{A}^i>-\infty$
    \item[-] market clearing (disposal of unused goods is costly):
    $$\tilde{c}^1_t+\tilde{c}^2_t=w^1_t+w^2_t,\  \tilde{a}^1_{t+1}+\tilde{a}^2_{t+1}=0,\ \forall t$$
    \end{itemize}
    
    \vspace{2pt}
    \begin{tikzpicture}
    \node [redbox] (box){%
        \begin{minipage}{0.935\textwidth}
        \color{myred}
        \scriptsize
        \underline{\textit{\textbf{How to solve}}}:
        \begin{itemize}
            \item[-] \textbf{\textit{Step 1}}: Take advantage the fact that the SM equilibrium allocation $\left\{\tilde{c}^1_t,\tilde{c}^2_t\right\}^{\infty}_{t=0}$ and the AD equilibrium $\left\{\hat{c}^1_t,\hat{c}^2_t\right\}^{\infty}_{t=0}$ are equivalent.
            \item[-] \textbf{\textit{Step 2}}: Solve the asset holdings $\left\{\tilde{a}^1_t,\tilde{a}^2_t\right\}^{\infty}_{t=0}$ with: $$\tilde{a}^i_{t+1}=\sum^{\infty}_{\tau=1}\frac{\hat{p}_{t+\tau}\left(\hat{c}^i_{t+\tau}-w^i_{t+\tau}\right)}{\hat{p}_{t+1}}$$
            i.e., the asset holding at $t$ is the sum of all future excess demands, discounted by the $t+1$ price.
        \end{itemize}
        \end{minipage}
    };
    \end{tikzpicture}
    \vspace{4pt}
    The existence of a SM equilibrium requires No-Ponzi scheme: $\bar{A}^i<\infty$. 
    
    \vspace{2pt}
    \begin{tikzpicture}
    \node [redbox] (box){%
        \begin{minipage}{0.935\textwidth}
        \color{myred}
        \scriptsize
        \underline{\textit{\textbf{Proof by contradiction}}}: Suppose there is no debt limit.
        
        Without debt limit, agent $i$ can consume more at $t=0$ and keep borrowing to keep the consumption level in the future, formally:
        \begin{align*}
            c_0^i=\tilde{c}_0^i+\frac{\epsilon}{1+\tilde{r}_1},&\ c_t^i=\tilde{c}_t^i\\
            a_1^i=\tilde{a}_1^i+\epsilon,&\ a_{t+1}^i=\tilde{a}_{t+1}^i-\prod^t_{\tau=1}\left(1+\tilde{r}_{\tau+1}\right)\epsilon
        \end{align*}
        This allocation satisfies the budget constraint, and can achieve a strictly higher utility, hence contradicting utility maximization. At the same time, $\prod^t_{\tau=1}\left(1+\tilde{r}_{\tau+1}\epsilon\right)\xrightarrow{t\rightarrow\infty}\infty$, contradicting to the limited resource nature of the economy.
        \end{minipage}
    };
    \end{tikzpicture}
    
    \end{minipage}
};
\node[iredtitle, right=4pt] at (box.north west) {Equilibrium};
\end{tikzpicture}

\subsection*{Link AD and SM equilibrium}
The link between AD equilibrium and SM equilibrium is built on 2 propositions:

\vspace{2pt}
\begin{tikzpicture}
\node [ibluebox] (box){%
    \begin{minipage}{0.23\textwidth}
    \color{white}
    \scriptsize
    \begin{itemize}
        \item[1] For an AD equilibrium allocation $\left\{\hat{c}^1_t,\hat{c}^2_t\right\}^{\infty}_{t=0}$ and prices $\left\{\hat{p}_t\right\}^{\infty}_{t=0}$ with $\frac{\hat{p_{t+1}}}{\hat{p}_{t}}=\xi<1,\forall t$, then there exists debt limits $\left(\bar{A}^1,\bar{A}^2\right)$ and a corresponding SM equilibrium, with allocation $\left\{\tilde{c}^1_t,\tilde{c}^2_t\right\}^{\infty}_{t=0}$ and interest rates $\left\{\tilde{r}_{t+1}\right\}^{\infty}_{t=0}$, such that $$\tilde{c}^i_t=\hat{c}^i_t,\forall i,t$$
        \item[2] For a SM equilibrium allocation $\left\{\left\{\tilde{c}^1_t,\tilde{c}^2_t\right\},\left\{\tilde{a}^1_{t+1},\tilde{a}^2_{t+1}\right\}\right\}^{\infty}_{t=0}$ and interest rates $\left\{\tilde{r}_{t+1}\right\}^{\infty}_{t=0}$ where $\tilde{a}^i_t\geq -\bar{A}^i$, $\tilde{r}_{t+1}>0$, there exists a corresponding AD equilibrium with allocation $\left\{\hat{c}^1_t,\hat{c}^2_t\right\}^{\infty}_{t=0}$ and prices $\left\{\hat{p}_t\right\}^{\infty}_{t=0}$ such that $$\hat{c}^i_t=\tilde{c}^i_t,\forall i,t$$
    \end{itemize}
    
    \end{minipage}
};
\node[ibluetitle, right=4pt] at (box.north west) {Propositions of AD$\equiv$SM};
\end{tikzpicture}

\vspace{2pt}
    \begin{tikzpicture}
    \node [bluebox] (box){%
        \begin{minipage}{0.23\textwidth}
        \color{myblue}
        \scriptsize
        \begin{itemize}
            \item[-] \textbf{Step 1: Construct interest rate}
            
            Define the SM interest rate as $\frac{1}{1+\tilde{r}_{t+1}}=\frac{\hat{p}_{t+1}}{\hat{p}_t}$, then the SM budget will be $\hat{p}_t\tilde{c}_t^i+\hat{p}_{t+1}\tilde{a}^i_{t+1}=\hat{p}_tw_t^i+\hat{p}_t\tilde{a}^i_t$, iterate this, get $\sum^{\infty}_{t=0}\hat{p}_t\tilde{c}^i_t+\lim_{T\rightarrow\infty}\hat{p}_{T+1}\tilde{a}^i_{T+1}=\sum^{\infty}_{t=0}\hat{p}_t w^i_t$, since $\lim_{T\rightarrow\infty}\hat{p}_{T+1}\tilde{a}^i_{T+1}=\lim_{T\rightarrow\infty}\prod^{T+1}_{\tau=1}\frac{\tilde{a}^i_{T+1}}{1+\tilde{r}_{\tau}^i}\geq 0$, \textbf{SM budget satisfaction leads to AD budget satisfaction}.
            
            \item[-] \textbf{Step 2: Derive $\tilde{a}^i_{t+1}$}
            
            By plug in $\frac{1}{1+\tilde{r}_{t+1}}=\frac{\hat{p}_{t+1}}{\hat{p}_t}$ to SM budget constraint, get $\tilde{a}_t^i=c_t^i-w_t^i+\tilde{a}_{t+1}^i\frac{\hat{p}_{t+1}}{\hat{p}_t}$, do an iteration of this equation, derive asset holding as $$\tilde{a}^i_{t+1}=\sum^{\infty}_{\tau=1}\frac{\hat{p}_{t+\tau}\left(\hat{c}^i_{t+\tau}-w^i_{t+\tau}\right)}{\hat{p}_{t+1}}$$
            plug it in the SM equilibrium budget constraint $\hat{c}^i_t+\frac{\tilde{a}_{t+1}^i}{1+\tilde{r}_{t+1}}=w_t+\tilde{a}_t^i$, the constraint is satisfied.
            
            \item[-] \textbf{Step 3: Find debt limit $\bar{A}^i$}
            
            It is easy to show that the asset holding $\tilde{a}^i_{t+1}=\sum^{\infty}_{\tau=1}\frac{\hat{p}_{t+\tau}\left(\hat{c}^i_{t+\tau}-w^i_{t+\tau}\right)}{\hat{p}_{t+1}}\geq -\sum^{\infty}_{\tau=1}\frac{\hat{p}_{t+\tau}w^i_{t+\tau}}{\hat{p}_{t+1}}\geq -\sum^{\infty}_{\tau=1}\xi^{\tau-1}{\hat{p}_{t+1}}>-\infty$, therefore, households will \textbf{never} choose $\tilde{a}^i_{t+1}$ to exceed the debt limit $-\bar{A}^i=-\sup \sum^{\infty}_{\tau=1}\frac{\hat{p}_{t+\tau}}{\hat{p}_{t+1}}w^i_{t+\tau}$. This debt limit can also be derived by setting consumption $c^i_t$ as 0 in SM budget constraint and a backward iteration.
            
            \item[-] \textbf{Step 4: Check utility maximization}
            
            This constructed SM allocation must satisfy AD budget as well, and it is an AD equilibrium allocation, hence it is lifetime utility maximizing allocation.

        \end{itemize}
        \end{minipage}
    };
\node[bluetitle, right=4pt] at (box.north west) {Proof of Position 2: AM$\Leftarrow$SM};
\end{tikzpicture}

\vspace{2pt}
    \begin{tikzpicture}
    \node [bluebox] (box){%
        \begin{minipage}{0.23\textwidth}
        \color{myblue}
        \scriptsize
        \begin{itemize}
            \item[-] \textbf{Step 1: Construct price series}
            
            From the construction before, $$\hat{p}_{t+1}=\frac{\hat{p}_t}{1+\tilde{r}_{t+1}}$$
            SM equilibrium consumption satisfies AD budget constraint, market clearing and non-negative consumption condition.
            
            \item[-] \textbf{Step 2: SM equilibrium allocation is utility maximizing within AD budget}
            
            Prove by contradiction: suppose there exist an alternative satisfying AD budget and yields a higher utility. Optimization within SM budget has one more constraint than optimization within AD budget: \textbf{No Ponzi scheme}.  
            
            Since no Ponzi scheme constraints ever bind in the SM equilibrium, SM equilibrium allocation will be optimizer within AD budget set as well.

        \end{itemize}
        \end{minipage}
    };
\node[bluetitle, right=4pt] at (box.north west) {Proof of Position 2: AM$\Leftarrow$SM};
\end{tikzpicture}

\section*{Overlapping Generation Model}
\vspace{2pt}
\begin{tikzpicture}
\node [ibluebox] (box){%
    \begin{minipage}{0.23\textwidth}
    \color{white}
    \scriptsize
    \begin{itemize}
        \item[-] \textbf{discrete} time, indexed by $t$
        \item[-] For each $t$, a new generation $t$ of identical individuals is born and live for two periods $t,t+1$.
        \item[-] There is an initial old generation (generation 0), they are born before $t=1$, they can be endowed with $m$ units of \textbf{fiat maoney}.
        \item[-] single commodity exogenously produced, indexed by $t$, pure \textbf{exchange/endowment} economy.
        \item[-] There is exogenouly given \textbf{net} asset/liability of the entire private sector
    \end{itemize}
    \end{minipage}
};
\node[ibluetitle, right=4pt] at (box.north west) {Features};
\end{tikzpicture}

\vspace{2pt}
\begin{tikzpicture}
\node [bluebox] (box){%
    \begin{minipage}{0.23\textwidth}
    \color{myblue}
    \scriptsize
    utility of a generation $t$ agent is
    $$U_t(c^t)=u(c_t^t)+\beta u(c_{t+1}^t)$$
    utility of the initial old is $$U_0(c^0)=u(c_1^0)$$
    
    Again, $u(\cdot)$ is assumed to be strictly increasing, strictly concave, twice continuously differentiable, typically satisfying Inada condition and CRRA as well.
    \end{minipage}
};
\node[bluetitle, right=4pt] at (box.north west) {Agents' preferences};
\end{tikzpicture}

\vspace{2pt}
\begin{tikzpicture}
\node [iredbox] (box){%
    \begin{minipage}{0.23\textwidth}
    \color{white}
    \scriptsize
    \begin{itemize}
        \item[-] allocation: $\left\{\hat{c}^0_1,\left\{\hat{c}^t_t,\hat{c}^t_{t+1}\right\}^{\infty}_{t=1}\right\}$
        \item[-] regulating mechanism: $\left\{\hat{p}_t\right\}^{\infty}_{t=1}$, with $m$ or $p_1$ (when $m=0$) as the numeraire.
    \end{itemize}
    such that:
    \begin{itemize}
        \item[-] given $\left\{\hat{p}_t\right\}^{\infty}_{t=1}$, $\left\{\hat{c}^0_1,\left\{\hat{c}^t_t,\hat{c}^t_{t+1}\right\}^{\infty}_{t=1}\right\}$ solves:
        \begin{align*}
            \max_{c^t\geq 0} u(c_t^t)+\beta u(c_{t+1}^t)\ &\text{ s.t. } \hat{p}_t c^t_t+\hat{p}_{t+1}c^t_{t+1}\leq \hat{p}_t w^t_t+\hat{p}_{t+1}w^t_{t+1}\\
            \max_{c_1^0\geq 0} u(c_1^0)\ &\text{ s.t. } \hat{p}_1 c_1^0\leq \hat{p}_1 w_1^0+m
        \end{align*}
        
    \item[-] market clearing (disposal of unused goods is costly):
    $$\hat{c}^{t-1}_t+\hat{c}^t_t=w^{t-1}_t+w^t_t,\forall t\geq 1$$
    \end{itemize}

    \end{minipage}
};
\node[iredtitle, right=4pt] at (box.north west) {AD Equilibrium};
\end{tikzpicture}

\scriptsize
\underline{\textit{\textbf{How to solve}}}:
\begin{itemize}
    \item[-] \textbf{\textit{Step 1}}: solve the Lagrangian
    $$ u_t(c^t_t)+\beta u_{t+1}(c^t_{t+1})+\lambda^t\left( p_t w_t^t+p_{t+1}w_{t+1}^t-p_t c_t^t-p_{t+1}c^t_{t+1} \right)$$
    FOCs: $\frac{u'_t(c^t_t)}{p_t}=\lambda^t =\frac{\beta u'_{t+1}(c^t_{t+1})}{p_{t+1}}\Rightarrow \frac{p_{t+1}}{p_t}=\frac{\beta u'_{t+1}(c^t_{t+1})}{u'_t(c_t^t)}\Rightarrow c_t^t=f(c^t_{t+1},w^t_t,w^t_{t+1})$. FOC gives that price changes $p_t/p_{t+1}$ and subjective discounting $\beta$ determines consumption smoothing.
    
    \item[-] \textit{\textbf{Step 2}}: Solve the initial old's Lagrangian:
    $$u(c^0_1)+\lambda^0\left(w_1^0+m-c_1^0\right)$$ get the initial old's consumption $\hat{c}^0_1=w_1^0+m$.
    
    \item[-] \textbf{\textit{Step 3}}: Plug $\hat{c}_1^0$ into the market clearing condition $c_t^{t-1} +c_t^t=w_t^{t-1}+w_t^t$, get
    $$\hat{c}_1^1=w_1^0+w_1^1-\hat{c}_1^0=w_1^1-m$$
    
    
    \item[-] \textit{\textbf{Step 4}}: Plug $\hat{c}_1^1=w_1^1-m$ and the FOC back to the budget constraint
    $$\hat{p}_1 c^1_1+\hat{p}_2 c^1_2 = \hat{p}_1 w^1_1+\hat{p}_2 w^1_2$$
    get $\hat{c}^1_2$. Iterate this process forward, solve the allocation $\left\{\hat{c}^0_1,\left\{\hat{c}^t_t,\hat{c}^t_{t+1}\right\}^{\infty}_{t=1}\right\}$ and price stream $\left\{\hat{p}_t\right\}^{\infty}_{t=1}$.
\end{itemize}

\vspace{2pt}
\begin{tikzpicture}
\node [iredbox] (box){%
    \begin{minipage}{0.23\textwidth}
    \color{white}
    \scriptsize
    \begin{itemize}
        \item[-] allocation: $\left\{\tilde{c}^0_1,\left\{\tilde{c}^t_t,\tilde{c}^t_{t+1},\tilde{s}_t^t\right\}^{\infty}_{t=1}\right\}$
        \item[-] regulating mechanism: interest rates $\left\{\tilde{r}_{t}\right\}^{\infty}_{t=1}$
    \end{itemize}
    such that:
    \begin{itemize}
        \item[-] given $\left\{\tilde{r}_{t+1}\right\}^{\infty}_{t=1},\forall t>1$, $\left\{\tilde{c}^t_t,\tilde{c}^t_{t+1}\right\}^{\infty}_{t=1}$ solves:
        \begin{align*}
            \max_{c^t\geq 0} u(c_t^t)+\beta u(c_{t+1}^t)\ \text{ s.t. } & c^t_t+s^t_t\leq w_t^t,\\
            & c^t_{t+1}\leq w^t_{t+1}+(1+\tilde{r}_{t+1})s^t_t\\
            \max_{c_1^0\geq 0} u(c_1^0)\ \text{ s.t. } & c_1^0\leq w_1^0+m(1+\tilde{r}_1)
        \end{align*}
    \item[-] good market clearing (disposal of unused goods is costly):
    $$\tilde{c}^{t-1}_t+\tilde{c}^t_t=w^{t-1}_t+w^t_t,\forall t\geq 1$$
    \item[-] asset market clearing: the budget constraint gives $$\tilde{c}^{t+1}_{t+1}+\tilde{c}^t_{t+1}+s^{t+1}_{t+1}=w^{t+1}_{t+1}+w^t_{t+1}+(1+\tilde{r}_{t+1})s^t_t$$
    plug the good market clearing condition, get
    $$s^{t+1}_{t+1}=(1+\tilde{r}_{t+1})s^t_t $$
    iterate this backwards to $s^0_0=m$, get
    $$s^t_t=\prod^t_{\tau=1}(1+\tilde{r}_{\tau})m$$
    \end{itemize}
    \end{minipage}
};
\node[iredtitle, right=4pt] at (box.north west) {SM Equilibrium};
\end{tikzpicture}

\scriptsize
\underline{\textit{\textbf{How to solve}}}:
\begin{itemize}
    \item[-] \textbf{\textit{Step 1}}: solve the Lagrangian
    \begin{align*}
        u_t(c^t_t)+\beta u_{t+1}(c^t_{t+1}) &+ \mu^t_t\left(w^t_t-c^t_t-s^t_t \right)\\
        &+ \mu^t_{t+1}\left(w^t_t+(1+\tilde{r}_{t+1})s^t_t-c^t_{t+1} \right)
    \end{align*}
    FOCs: $u'_t(c^t_t)=\mu^t =\beta u'_{t+1}(c^t_{t+1})(1+\tilde{r}_{t+1})\Rightarrow \tilde{r}_{t+1}=\frac{u'_{t}(c^t_t)}{\beta u'_{t+1}(c_{t+1}^t)}-1\Rightarrow c_t^t=f(c^t_{t+1},w^t_t,w^t_{t+1})$.
    
    \item[-] \textit{\textbf{Step 2}}: Solve the initial old's Lagrangian:
    $$u(c^0_1)+\lambda^0\left(w_1^0+m(1+r_1)-c_1^0\right)$$ get the initial old's consumption $\hat{c}^0_1=w_1^0+m(1+\tilde{r}_{t+1})$.
    
    \item[-] \textbf{\textit{Step 3}}: Plug $\hat{c}_1^0$ into the market clearing condition $c_t^{t-1} +c_t^t=w_t^{t-1}+w_t^t$, get
    $$\hat{c}_1^1=w_1^0+w_1^1-\hat{c}_1^0=w_1^1-m(1+\tilde{r}_{t+1})$$
    
    \item[-] \textit{\textbf{Step 4}}: Plug $\hat{c}_1^1=w_1^1-m(1+\tilde{r}_{t+1})$ and the FOC back to the budget constraint
    $$w_t^t+\frac{1}{1+\tilde{r}_{t+1}}w_{t+1}^t= c_t^t+\frac{1}{1+\tilde{r}_{t+1}}c^t_{t+1}$$
    get $\hat{c}^1_2$. Iterate this process forward, solve the allocation $\left\{\hat{c}^0_1,\left\{\hat{c}^t_t,\hat{c}^t_{t+1}\right\}^{\infty}_{t=1}\right\}$ and price stream $\left\{\tilde{r}_{t+1}\right\}^{\infty}_{t=1}$.
\end{itemize}

\vspace{2pt}
\begin{tikzpicture}
\node [ibluebox] (box){%
    \begin{minipage}{0.23\textwidth}
    \color{white}
    \scriptsize
    \begin{itemize}
        \item[1] For an AD equilibrium allocation $\left\{\hat{c}^0_1,\left\{\hat{c}^t_t,\hat{c}^t_{t+1}\right\}^{\infty}_{t=1}\right\}$ and prices $\left\{\hat{p}_t\right\}^{\infty}_{t=1}$ with $\hat{p}_t>0$, then there exists a corresponding SM equilibrium, with allocation $\left\{\tilde{c}^0_1,\left\{\tilde{c}^t_t,\tilde{c}^t_{t+1}\right\}^{\infty}_{t=1}\right\}$ and interest rates $\left\{\tilde{r}_{t+1}\right\}^{\infty}_{t=0}$, such that $$\tilde{c}^{t-1}_t=\hat{c}^{t-1}_t,\tilde{c}^t_t=\hat{c}^t_t\forall t\geq 1$$
        \item[2] For a SM equilibrium allocation $\left\{\tilde{c}^0_1,\left\{\tilde{c}^t_t,\tilde{c}^t_{t+1}\right\}^{\infty}_{t=1}\right\}$ and interest rates $\left\{\tilde{r}_{t+1}\right\}^{\infty}_{t=0}$ where $\tilde{r}_{t+1}>-1$, there exists a corresponding AD equilibrium with allocation $\left\{\hat{c}^0_1,\left\{\hat{c}^t_t,\hat{c}^t_{t+1}\right\}^{\infty}_{t=1}\right\}$ and prices $\left\{\hat{p}_t\right\}^{\infty}_{t=1}$ such that $$\hat{c}^{t-1}_t=\tilde{c}^{t-1}_t,\hat{c}^t_t=\tilde{c}^t_t\forall t\geq 1$$
    \end{itemize}
    
    The interest rate and price stream are still inter-determined:
    $$\frac{1}{1+\tilde{r}_{t+1}}=\frac{\hat{p}_{t+1}}{\hat{p}_t},\ \frac{1}{1+\tilde{r}_1}=\hat{p}_1$$
    
    The two Euler equations are:
    \begin{align*}
        u'_t(c_t^t)=\beta u'_{t+1}(c_{t+1}^t)(\hat{p}_t/\hat{p}_{t+1}) &\ \cdots \text{ AD}\\
        u'_t(c_t^t)=\beta u'_{t+1}(c_{t+1}^t)(1+\tilde{r}_{t+1}) & \ \cdots\text{ SM}
    \end{align*}
    
    \vspace{2pt}
    \begin{tikzpicture}
    \node [bluebox] (box){%
        \begin{minipage}{0.935\textwidth}
        \color{myblue}
        \scriptsize
        \underline{\textit{\textbf{An easy proof}}}:
        \begin{itemize}
            \item[-] \textbf{\textit{Prop. 1}}: AD equilibrium allocation  satisfies SM FOC and the SM budget constraints: 
            $$\tilde{\mu}_t^t=\hat{\lambda}\hat{p}_t,\tilde{\mu}_{t+1}^t=\hat{\lambda}\hat{p}_{t+1},\forall t\geq 0;\  \tilde{s}^t_t=w^t_t-\hat{c}^t_t,\forall t\geq 1$$
            \item[-] \textbf{\textit{Prop. 2}}: SM equilibrium allocation satisfies AD FOC and the AD budget constraints:  $$\hat{\lambda}^t=\frac{\tilde{\mu}^t_t}{\prod^{t-1}_{\tau=0}(1+\tilde{r}_{t-\tau})}=\frac{\tilde{\mu}^t_{t+1}}{\prod^{t-1}_{\tau=0}(1+\tilde{r}_{t+1-\tau})},\forall t\geq 1$$
            
            $$\hat{\lambda}^0=\tilde{\mu}^0_1(1+\tilde{r}_1)$$
        \end{itemize}
        \end{minipage}
    };
    \end{tikzpicture}
    
    \end{minipage}
};
\node[ibluetitle, right=4pt] at (box.north west) {Propositions of AD$\equiv$SM};
\end{tikzpicture}

\subsection*{Offer curve}
\scriptsize
Assume the economy is stationary, the endowments at each period of life by successive generations are constant:

$$w^t_t=w_1,w^t_{t+1}=w_2,\forall t\geq 1;\ w^0_1=w_2$$

By the Lagrangian of AD equilibrium:
\begin{align*}
    u'(c^t_t) &=\lambda^t p_t\\
    \beta u'(c^t_{t+1}) &= \lambda^t p_{t+1}
\end{align*}

solve $c^t_t,c^t_{t+1}$ as functions of $p_t,p_{t+1}$, get $c^t_{t}(p_t,p_{t+1}),c^t_{t+1}(p_t,p_{t+1})$, then define $y$ as the excess demand at $t$, define $z$ as the excess demand at $t+1$:
\begin{align*}
    y(p_t,p_{t+1})=&c^t_t(p_t,p_{t+1})-w_1\\
    z(p_t,p_{t+1})=&c^t_{t+1}(p_t,p_{t+1})-w_2
\end{align*}

The excess demand at $t=0$ is
$$z(m,p_1)=c^0_1-w_2=\frac{m}{p_1}$$

\vspace{2pt}
\begin{tikzpicture}
\node [ibluebox] (box){%
    \begin{minipage}{0.23\textwidth}
    \color{white}
    \scriptsize
    By the AD budget constraint: $p_t c_t^t+p_{t+1} c^t_{t+1}=p_t w_t+p_{t+1} w^t_{t+1}$, this gives
    $$ p_t y(p_t,p_{t+1}) + p_{t+1} z(p_t,p_{t+1})=0$$
    leading to
    $$\frac{z(p_t,p_{t+1})}{y(p_t,p_{t+1})}=-\frac{p_t}{p_{t+1}},\forall t\geq 1$$
    Notice that $y(p_t,p_{t+1})$ and $z(p_t,p_{t+1})$ are both functions of $\frac{p_t}{p_{t+1}}$, this way, we can replace $\frac{p_t}{p_{t+1}}$ as a function of $y$, plug this $\frac{p_t}{p_{t+1}}=g(y)$ into $z(p_t,p_{t+1})$, we will have the offer curve $$z=f(y)$$

    \vspace{2pt}
    \begin{tikzpicture}
    \node [bluebox] (box){%
        \begin{minipage}{0.935\textwidth}
        \color{myblue}
        \scriptsize
        \underline{\textit{\textbf{Properties of offer curve}}}:
        \begin{itemize}
            \item[-] The offer curve is bounded by endowment:
            $$y(p_t,p_{t+1})\geq -w_1,z_(p_t,p_{t+1})\geq -w_2$$
            \item[-] The curve is in the $2^{nd}$ and $4^{th}$ Quadrant:
            $$y(p_t,p_{t+1})\cdot z(p_t,p_{t+1})<0$$
            \item[-] The origin $(0,0)$ is on the offer curve: $z^*(p_t,p_{t+1})=y^*(p_t,p_{t+1})=0$.
            \item[-] for $y(p_t,p_{t+1})\neq 0$, AD budget constraint is always satisfied:
            $$\frac{z(p_t,p_{t+1})}{y(p_t,p_{t+1})}=-\frac{p_t}{p_{t+1}},\forall t\geq 1$$
            hence the slope of the straight line connecting each point on the offer curve to the origin determines the price ratio $\frac{p_t}{p_{t+1}}$
        \end{itemize}
        \end{minipage}
    };
    \end{tikzpicture}
    
    Besides, we also have a market clearing curve:
    \begin{align*}
        y(p_t,p_{t+1})+z(p_{t-1},p_t)=0,&\ \forall t>1\\
        y(p_1,p_2)+z(m,p_1)=0,&\ t=1
    \end{align*}
    
    this is a 45-degree line through the $2^{nd}$ and $4^{th}$ quadrants.
    \end{minipage}
};
\node[ibluetitle, right=4pt] at (box.north west) {Features of offer curve};
\end{tikzpicture}

With offer curve and market clearing curve, we can determine the \textbf{entire sequence of excess demands} of young and old at every date as following:
\begin{itemize}
    \item[-] \textit{\textbf{Step 1}}: for a given $m$, at $t=1$, the excess demand of the initial old $z^0=z^0(p_1,m)=m/p_1$
    \item[-] \textit{\textbf{Step 2}}: the initial young's excess demand at $t=1$ is determined by the market clearing curve $y^1(p_1,p_2)=-z^0$
    \item[-] \textit{\textbf{Step 3}}: $y^1$ will determine $z^1(p_1,p_2)$ on the offer curve.
    \item[-] \textit{\textbf{Step 4}}: repeat this \textbf{market clearing curve}-\textbf{offer curve} procedure.
\end{itemize}
In this procedure, the initial price $p_1$ must be picked first. If $m\neq 0$, different $p_1$ can index a continuum of equilibria.

\vspace{2pt}
\begin{tikzpicture}
\node [ibluebox] (box){%
    \begin{minipage}{0.23\textwidth}
    \color{white}
    \scriptsize
    First, we need the slope of the offer curve to be convex:
    $$ \frac{\partial c^t_t}{\partial (p_t/p_{t+1})} =\frac{\partial c^t_t}{\partial (1+r_{t+1})} <0 $$
    \begin{itemize}
        \item[-] graphically, $-\frac{p_t}{p_{t+1}}$ is the slope of the line connecting a point on the offer curve to the origin, $c_t^t-w_t^t=y(p_t,p_{t+1})$ is the x-axis, hence $\frac{\partial c^t_t}{\partial (p_t/p_{t+1})}<0$ means that as $c^t_t$ increases, the slope of the connecting line is increasing (less negative, or shallower).
        \item[-]economically, if interest rates increasing, young period saving will increase, consumption will decrease (substitution effect).
    \end{itemize}

    \vspace{2pt}
    \begin{tikzpicture}
    \node [bluebox] (box){%
        \begin{minipage}{0.935\textwidth}
        \color{myblue}
        \scriptsize
        \textit{\textbf{A very important object: \underline{Autarkic interest rate $\bar{r}$}}}:
        
        If we assume stationary endowment: $w^t_t=w1,w^t_{t+1}=w_2,\forall t\geq 1; w^0_1=w_2$, we will the autarkic interest rate as:
        $$1+\bar{r} = \frac{\hat{p}_t}{\hat{p}_{t+1}}=\frac{u_t'(w_1)}{\beta u_{t+1}'(w_2)}$$
        where 
        \begin{itemize}
            \item[-] $\hat{p}_t,\hat{p}_{t+1}$ are the autarky equilibrium price.
            \item[-]$\bar{r}$ is determined by the relative size of endowment $w_1,w_2$
            \item[-] $\bar{r}$ determines the general shape of the offer curve: $1+\bar{r}$ it is the \textbf{\underline{negative} of the slop of the offer curve at the origin}.
        \end{itemize}
        \end{minipage}
    };
    \end{tikzpicture}
    \end{minipage}
};
\node[ibluetitle, right=4pt] at (box.north west) {Slope of offer curve};
\end{tikzpicture}

\vspace{4pt}
\subsection*{Different cases of OLG model}
In the overlapping generation economy:
\begin{itemize}
    \item[-] $m$ and $p_1$
    
    $m$ determines the equilibrium sequences, the initial price level $p_1$ determines the starting point of an equilibrium sequence hence indexes a continuum of equilibria. There is a special initial price $p^*_1$ such that $m/p^*_1$ is exactly \textbf{the intersection point} of the offer curve and the market clearing curve. This will be the \textbf{stationary monetary equilibrium}.
    
    \item[-] $1+\bar{r}=\hat{p}_t/\hat{p}_{t+1}=u_t'(w_1)/\beta u_{t+1}'(w_2)$
    
    relative size of endowments $w_1,w_2$ determines autarkic interest rate $1+\bar{r}$, which determines the shape of the offer curve, and the Pareto efficiency and the stationary of the equilibria.
\end{itemize}
Now we discuss the different cases:

\vspace{2pt}
\begin{tikzpicture}
\node [orangebox] (box){%
    \begin{minipage}{0.23\textwidth}
    \color{myorange}
    \scriptsize
    \begin{itemize}
        \item[-] shape of offer curve: the offer curve is tangent to the market clearing line \textbf{at the origin}
        \item[-] equilibrium: 
        \begin{itemize}
            \item[-] the only stationary equilibrium is the origin allocation, autarky. It is Pareto efficiency.
            \item[-] there is NO stationary monetary equilibria, i.e., money is not valued; But $\forall m<0$, there is a continuum of monetary equilibria, converging to autarky.
        \end{itemize}
    \end{itemize}
    \end{minipage}
};
\node[orangetitle, right=4pt] at (box.north west) {Knife-edge economy $1+\bar{r} = \hat{p}_t/\hat{p}_{t+1}=u_t'(w_1)/\beta u_{t+1}'(w_2)=1$};
\end{tikzpicture}

\vspace{2pt}
\begin{tikzpicture}
\node [orangebox] (box){%
    \begin{minipage}{0.23\textwidth}
    \color{myorange}
    \scriptsize
    In this economy, the \textbf{old} generation is richer: $w_2>w_1$, since $u'>0,u''<0$, $u'(w_1)>u'(w_2)>0$
    \begin{itemize}
        \item[-] shape of offer curve: the slope of the offer curve is \textbf{steeper} than the 45-degree market clearing line \textbf{at the origin}.
        \item[-] two intersections: the origin and a point in the \textbf{fourth} quadrant $m/p^*_1, m<0$
        \item[-] equilibrium: 
        \begin{itemize}
            \item[-] autarky is a stationary equilibrium, and it is \textbf{Pareto efficient}.
            \item[-] there is a stationary monetary equilibria at the other intersection point $m/p^*_1$. It is \textbf{Pareto efficient}. In this equilibrium, $m<0$, the initial old has a debt, the economy has a net debt.
            \item[-] there is a continuum of dynamic monetary equilibria with $p_1>p^*_1$. For all of them, $m<0$. They all converge to the monetary stationary equilibrium.
        \end{itemize}
        \item[-] AD prices: $\hat{p}_t$ decreases over time.
    \end{itemize}
    
    \end{minipage}
};
\node[orangetitle, right=4pt] at (box.north west) {Classical economy $1+\bar{r} = \hat{p}_t/\hat{p}_{t+1}=u_t'(w_1)/\beta u_{t+1}'(w_2)>1$};
\end{tikzpicture}

\vspace{2pt}
\begin{tikzpicture}
\node [orangebox] (box){%
    \begin{minipage}{0.23\textwidth}
    \color{myorange}
    \scriptsize
    In this economy, the \textbf{young} generation is richer: $w_1>w_2$, since $u'>0,u''<0$, $u'(w_2)>u'(w_1)>0$
    \begin{itemize}
        \item[-] shape of offer curve: the slope of the offer curve is \textbf{shallower} than the 45-degree market clearing line \textbf{at the origin}.
        \item[-] two intersections: the origin and a point in the \textbf{second} quadrant $m/p^*_1, m>0$
        \item[-] equilibrium: 
        \begin{itemize}
            \item[-] autarky is a stationary equilibrium, and it is \textbf{NOT} Pareto efficient.
            \item[-] there is a stationary monetary equilibria at the other intersection point $m/p^*_1$. In this equilibrium, $m>0$, the initial old has positive fiat money.
            \item[-] there are two continuum of dynamic monetary equilibria:
            \begin{itemize}
                \item[-] $m>0$: the continuum is between autarky and stationary monetary equilibrium, with $p_1>p^*_1$, they all converge to autarky.
                \item[-] $m<0$: every possible $(m,p_1)$ is a dynamic monetary equilibrium, they all converge to autarky.
            \end{itemize}
            
        \end{itemize}
        \item[-] AD prices: $\hat{p}_t$ increases over time.
    \end{itemize}
    
    \end{minipage}
};
\node[orangetitle, right=4pt] at (box.north west) {Samuelson economy $1+\bar{r} = \hat{p}_t/\hat{p}_{t+1}=u_t'(w_1)/\beta u_{t+1}'(w_2)<1$};
\end{tikzpicture}

\subsection*{Pareto efficiency of the equilibria}
\begin{itemize}
    \item[-] \underline{\textbf{Autarky equilibrium}} 
    \begin{itemize}
        \item[-] In \textit{knife-edge economy}, \textbf{Autarky is PE}: young and old have the same marginal utility. To compensate old by transferring from young, the utility loss of transferring in young cannot be compensated by the compensation in old since utility is concave.
        \item[-] In \textit{classical economy}, \textbf{Autarky is PE}: young are endowed less, have a higher marginal utility. The utility loss of young from transferring to old can not be compensated by receiving that amount when old, again, it is due to the concavity of utility.
        \item[-] In \textit{Samuelson economy}, \textbf{Autarky is NOT PE}: young are endowed more, have a lower marginal utility. The utility loss of young from transferring to old actually can be compensated by receiving that amount when old, it is also due to the concavity of utility.
    \end{itemize}
    
    \item[-] \underline{\textbf{Monetary stationary equilibrium}}
    \begin{itemize}
        \item[-] In \textit{Samuelson economy}, \textbf{MSE is PE} and \textbf{Pareto dominating Autarky}: the initial old, with positive fiat money, are strictly better off, and as well off as they can be in equilibrium for any $m>0$. For generation $t\geq 1$, MSE allocation is identical to the steady-state utility maximizing allocation, hence every generation is at least as good as in autarky.
        
        \item[-] In \textit{classical economy}, \textbf{MSE is PE} but \textbf{NOT Pareto dominating Autarky}, MSE allocation is also steady-state utility maximizing, hence, it is Pareto efficient, but Autarky can not be Pareto improved either.
    \end{itemize}
    \item[-] \underline{\textbf{Monetary non-stationary equilibria}}
    \begin{itemize}
        \item[-] In \textit{Samuelson economy}, \textbf{MNSE is not PE}. All the non-stationary equilibria converge to Autarky, and features \textbf{rising prices}, hence $\lim_{t\rightarrow\infty}p_t=\infty$. All of these non-stationary equilibria are arbitrarily close to each other, they all converge to Autarky, which has a positive, constant price inflation $\hat{p}_t/\hat{p}_{t+1}<1$, hence a negative autarky interest rate $\bar{r}<0$, this means NOT PE.
        
        \item[-] In \textit{classical economy}, \textbf{every MNSE is PE}. All the non-stationary equilibria converge to the monetary stationary equilibrium, which has zero inflation, zero interest rate, hence they are all PE.
        
        \item[-] In \textbf{knife-edge economy}, the conclusion is the same as the classical economy.
    \end{itemize}
\end{itemize}

A general theoretical result is given by Balasko and Shell:

\vspace{2pt}
\begin{tikzpicture}
\node [redbox] (box){%
    \begin{minipage}{0.23\textwidth}
    \color{myred}
    \scriptsize
    Assume:
    \begin{itemize}
        \item[-] stationary endowment $w^t_t=w_1>0,w^t_{t+1}=w^0_1=w_2>0$
        \item[-] allocation is bounded away from zero: $\left(\hat{c}^{t-1}_t,\hat{c}^t_t\right)\geq \delta >0$
    \end{itemize}
    
    Define $$\frac{1}{1+r_{t+1}}=\frac{\beta U'\left(\hat{c}^t_{t+1}\right)}{U'\left(\hat{c}^t_t\right)}=\frac{\hat{p}_{t+1}}{\hat{p}_t}$$
    
    Then the allocation is Pareto efficient if and only if
    $$\sum^{\infty}_{t=1}\prod_{\tau}^t\left(1+r_{\tau+1}\right) =+\infty$$
    
    This \underline{\textbf{includes}} two scenarios (PE):
    \begin{itemize}
        \item[-] AD prices falling, i.e., positive interest rate $r_{t+1}$
        \begin{itemize}
            \item[-] \underline{classic Autarky}
        \end{itemize}
        
        \item[-] AD prices constant, i.e., zero interest rate $r_{t+1}$
        \begin{itemize}
            \item[-] \underline{knife-edge Autarky}, \underline{all knife-edge MNSE} (converge to Autarky) 
            \item[-]\underline{classic MSE}, \underline{all classic MNSE} (converge to classic MSE)
            \item[-] \underline{Samuelson MSE}
        \end{itemize}
    \end{itemize}
    This \underline{\textbf{excludes}} one scenario (not PE):
    \begin{itemize}
        \item[-] AD prices increasing, i.e., negative interest rate $r_{t+1}$ 
        \begin{itemize}
            \item[-] \underline{Samuelson Autarky}, \underline{all Samuelson MNSE} (converge to Samuelson Autarky). 
        \end{itemize}
    \end{itemize}
    \end{minipage}
};
\node[redtitle, right=4pt] at (box.north west) {Balasko and Shell Pareto-efficiency condition};
\end{tikzpicture}

\section*{Neoclassical growth model}

\vspace{2pt}
\begin{tikzpicture}
\node [ibluebox] (box){%
    \begin{minipage}{0.23\textwidth}
    \color{white}
    \scriptsize
    \begin{itemize}
        \item[-] \textbf{discrete} time, indexed by $t$
        \item[-] economy, households, firms live \textbf{infinitely}
        \item[-] single commodity endogenously produced by firms, consumed and invested by households.
        \item[-] there is a continuum of identical \underline{\textbf{households}}, they are \textbf{price takers}, they maximize their lifetime utility, can be represented by a \textbf{representative} household.
        \item[-] there is a continuum of identical \underline{\textbf{firms}}, they are perfectly competitive, maximize their lifetime profits, can be represented by a representative household.
    \end{itemize}
    \end{minipage}
};
\node[ibluetitle, right=4pt] at (box.north west) {Features};
\end{tikzpicture}

\vspace{2pt}
\begin{tikzpicture}
\node [bluebox] (box){%
    \begin{minipage}{0.23\textwidth}
    \color{myblue}
    \scriptsize
    households are endowed with
    \begin{itemize}
        \item[-] 1 unit of time, can be used for labor or leisure $n_t+l_t=1$
        \item[-] initial capital stock $\bar{k}_0>0$
        \item[-] \textbf{NO goods}: all final goods are produced endogenously.
    \end{itemize}
    \vspace{3pt}
    households and firms meet in markets and trade:
    \begin{itemize}
        \item[-] households sell capital and labor, earn rental and wage
        \item[-] firms sell goods, earn \textbf{zero} profit
    \end{itemize}
    \vspace{3pt}
    \textbf{And, Households are assumed to own firms}, hence own profit.
    \end{minipage}
};
\node[bluetitle, right=4pt] at (box.north west) {Agents' endowment};
\end{tikzpicture}

\vspace{2pt}
\begin{tikzpicture}
\node [bluebox] (box){%
    \begin{minipage}{0.23\textwidth}
    \color{myblue}
    \scriptsize
    household: $\max U(c)=\sum^{\infty}_{t=0}\beta^t [u(c_t)+\psi(l_t)]$
    
    firms: $\max F(k_t,n_t)-r_t k_t-w_t n_t$ where $F(k_t,n_t)=y_t =c_t+i_t$
    
    \vspace{-6pt}
\begin{center}
\tiny
       \begin{tabular}{rl}
       \multicolumn{2}{c}{assumptions of $u(\cdot)$ and $F(\cdot)$}\\
        \hline\hline
        $u(c_t):$ & \\
          &continuously differentiable\\
          &strictly increasing and concave \\
          &satisfying Inada condition\\
          &time separable, $\beta\in (0,1)$\\
          \hline
          $\psi(l_t):$ & \\
          &value of leisure\\
          \hline
          $F(k_t,n_t):$ & \\
          &continuously differentiable\\
          &strictly increasing and concave \\
          &both $F_k$ and $F_n$ satisfying Inada condition\\
          & homogeneous of degree 1, constant return to scale\\
          \hline\hline
        \end{tabular} 
\end{center}

Households and firms are perfectly competitive: they are price takers, earn zero economic profit, and achieve a perfectly \textbf{competitive equilibrium environment}.
    \end{minipage}
};
\node[bluetitle, right=4pt] at (box.north west) {Agents' preferences and equilibrium behavior};
\end{tikzpicture}

\vspace{2pt}
\begin{tikzpicture}
\node [bluebox] (box){%
    \begin{minipage}{0.23\textwidth}
    \color{myblue}
    \scriptsize
    \textbf{households}: households are endowed with initial capital $k_0$ and produce new capital via: $$k_{t+1}=(1-\delta)k_t+i_t$$
    where $\delta$ is the depreciation ratio $\delta\in [0,1]$
    
    \rule{\textwidth}{0.4pt}
    
    \vspace{3pt}
    \textbf{firms}: the most common production functions is the family of the constant elasticity of substitution (CES) function:
    $$y_t = F(k_t,n_t)=A\left(\alpha k_t^{1-\frac{1}{\gamma}}+(1-\alpha)n_t^{1-\frac{1}{\gamma}}\right)^\frac{1}{1-\frac{1}{\gamma}}$$
    it has the following properties:
    
    \vspace{2pt}
    \begin{tikzpicture}
    \node [ibluebox] (box){%
        \begin{minipage}{0.935\textwidth}
        \color{white}
        \scriptsize
        \underline{\textit{\textbf{Properties of CES function}}}:
        \begin{itemize}
            \item[-] Convergence:
            \begin{itemize}
                \item[-] $\gamma\rightarrow 1$: converges to Cobb-Douglas function 
                \item[-] $\gamma\rightarrow 0$: converges to Leontief function
            \end{itemize}
            \item[-] constant return to scale
            \item[-] positive and diminishing marginal productivity
            \item[-] inputs are complements (cross partial derivatives $>0$)
        \end{itemize}
        \end{minipage}
    };
    \end{tikzpicture}
    \end{minipage}
};
\node[bluetitle, right=4pt] at (box.north west) {Agents' technology};
\end{tikzpicture}

Here is the proof of the properties of CES functions:
\begin{itemize}
    \item[-] $\gamma\rightarrow 1$ leading to Cobb-Douglas:
    \begin{itemize}
        \item[-] \textbf{L'Hopital rule:} rewrite the function as 
        \begin{align*}
        \ln y_t &= \ln A +\frac{\ln\left[\alpha k_t^{1-\frac{1}{\gamma}}+(1-\alpha)n_t^{1-\frac{1}{\gamma}}\right]{\color{myorange}\xrightarrow{\gamma\rightarrow 1}0}}{1-\frac{1}{\gamma}{\color{myorange}\xrightarrow{\gamma\rightarrow 1}0}}\\
        {\color{myorange}(\text{L'Hopital})}&\rightarrow\ln A+\frac{\alpha \ln k_t\cdot k_t^{1-\frac{1}{\gamma}}+(1-\alpha) \ln n_t\cdot n_t^{1-\frac{1}{\gamma}}}{\left(\alpha k_t^{1-\frac{1}{\gamma}}+(1-\alpha)n_t^{1-\frac{1}{\gamma}}\right)}\\
        \Rightarrow\lim_{\gamma\rightarrow 1}\ln y_t & =\ln A+\alpha \ln k_t +(1-\alpha)\ln n_t\\
        \Rightarrow \lim_{\gamma\rightarrow 1} y_t & =e^{\ln A+\alpha \ln k_t +(1-\alpha)\ln n_t} = Ak_t^{\alpha}n_t^{1-\alpha}
        \end{align*}
        
        \item[-] \textbf{Total differentials:} rewrite the function as 
        $$
            y_t^{1-\frac{1}{\gamma}}=A^{1-\frac{1}{\gamma}}\left[\alpha k_t^{1-\frac{1}{\gamma}}+(1-\alpha)n_t^{1-\frac{1}{\gamma}}\right]
        $$
        take total differentials, get:
        \begin{align*}
            \left(1-\frac{1}{\gamma}\right)y_t^{-\frac{1}{\gamma}}\mathrm{d}y_t =& A^{1-\frac{1}{\gamma}}\left(\left(1-\frac{1}{\gamma}\right)\alpha k_t^{-\frac{1}{\gamma}}\mathrm{d}k_t\right.\\
            &+\left.\left(1-\frac{1}{\gamma}\right)(1-\alpha)n_t^{-\frac{1}{\gamma}}\mathrm{d}n_t\right)\\
            \Rightarrow y_t^{-\frac{1}{\gamma}}\mathrm{d}y_t =&A^{1-\frac{1}{\gamma}}\left(\alpha k_t^{-\frac{1}{\gamma}}\mathrm{d}k_t +(1-\alpha)n_t^{-\frac{1}{\gamma}}\mathrm{d}n_t\right)\\
            \xRightarrow{\gamma\rightarrow 1} \frac{1}{y_t}\mathrm{d}y_t=&\left(\alpha\frac{1}{k_t}\mathrm{d}k_t+(1-\alpha)\frac{1}{n_t}\mathrm{d}n_t \right)\\
            \xRightarrow{\int} \ln y_t +c_y =& \left[\alpha\left(\ln k_t+c_k\right)+(1-\alpha)\left( \ln n_t+c_n\right) \right]\\
            \Rightarrow y_t =& {\color{myorange}e^{\alpha c_k+(1-\alpha)c_n-c_y}}k_t^{\alpha}n_t^{1-\alpha}=\tilde{A}k_t^{\alpha}n_t^{1-\alpha}
        \end{align*}
        
        \item[-] \textbf{Taylor expansion:} expand $Q(k_t,n_t)\equiv\alpha k_t^{1-\frac{1}{\gamma}}+(1-\alpha)n_t^{1-\frac{1}{\gamma}}$ at $\gamma=1$, get:
        \begin{align*}
            Q(k_t,n_t)= & \left.\alpha k_t^{1-\frac{1}{\gamma}}\right\vert_{\gamma=1}+\left.(1-\alpha)n_t^{1-\frac{1}{\gamma}}\right\vert_{\gamma=1}\\
            &+ \left.\alpha\ln k_t\cdot k_t^{1-\frac{1}{\gamma}}\right\vert_{\gamma=1}\cdot(\gamma-1)\\
            &+ \left. (1-\alpha)\ln n_t\cdot n_t^{1-\frac{1}{\gamma}} \right\vert_{\gamma=1}\cdot (\gamma-1) + O\left((\gamma-1)^2\right)\\
            = & 1+(\gamma-1)\ln k_t^{\alpha}n_t^{1-\alpha} +O\left((\gamma-1)^2\right)
        \end{align*}
        plug $Q(k_t,n_t)$ into $y_t$, get
        \begin{align*}
            y_t &= A\cdot Q(k_t,n_t)^{\frac{1}{1-\frac{1}{\gamma}}}\\
             &= A \left[1+(\gamma-1)\ln k_t^{\alpha}n_t^{1-\alpha} +O\left((\gamma-1)^2\right)\right]^{\frac{1}{1-\frac{1}{\gamma}}}\\
            \xRightarrow{r=\frac{1}{\gamma-1}} &= A\left[ 1+\frac{1}{r}\ln k_t^{\alpha}n_t^{1-\alpha} +O(\frac{1}{r^2}) \right]^{\gamma r}
        \end{align*}
        since $\gamma\rightarrow 1\Rightarrow r\rightarrow \infty$, we have 
        \begin{align*}
            \lim_{\gamma\rightarrow 1} y_t &= \lim_{\gamma\rightarrow 1} A\left[ 1+\frac{\ln k_t^{\alpha}n_t^{1-\alpha}}{r}+O(\frac{1}{r^2}) \right]^{\gamma r}\\
            &=\lim_{\gamma\rightarrow 1} A\left[ 1+\frac{\ln k_t^{\alpha}n_t^{1-\alpha}}{r} \right]^{\gamma r}\\
            \xRightarrow{(1+\frac{x}{\infty})^{\infty}\rightarrow e^x} & = \left(e^{\ln k_t^{\alpha}n_t^{1-\alpha}}\right)^{\gamma} = k_t^{\alpha}n_t^{1-\alpha}
        \end{align*}
    \end{itemize}
    
    \item[-] $\gamma\rightarrow 0$ leading to Leontief:
    \begin{itemize}
        \item[-] \textbf{L'Hopital rule:} rewrite the function as 
        \begin{align*}
        \ln y_t &= \ln A +\frac{\ln\left[\alpha k_t^{1-\frac{1}{\gamma}}+(1-\alpha)n_t^{1-\frac{1}{\gamma}}\right]{\color{myorange}\xrightarrow{\gamma\rightarrow 0}-\infty}}{1-\frac{1}{\gamma}{\color{myorange}\xrightarrow{\gamma\rightarrow 0}-\infty}}\\
        {\color{myorange}(\text{L'Hopital})}&\rightarrow\ln A+\frac{\alpha \ln k_t\cdot k_t^{1-\frac{1}{\gamma}}+(1-\alpha) \ln n_t\cdot n_t^{1-\frac{1}{\gamma}}{\color{myorange}\rightarrow 0}}{\left(\alpha k_t^{1-\frac{1}{\gamma}}+(1-\alpha)n_t^{1-\frac{1}{\gamma}}\right){\color{myorange}\rightarrow 0}}
        \end{align*}
        
        Here, we have a $\frac{0}{0}$ limit. Using L'Hopital rule again yields NOTHING. To proceed, define $x_t=\min\left\{k_t,n_t\right\}$, then we have
        \begin{align*}
        \lim_{\gamma\rightarrow 0}\ln y_t &=\ln A+\frac{\alpha \ln k_t\cdot \left(\frac{k_t}{x_t}\right)^{1-\frac{1}{\gamma}}+(1-\alpha) \ln n_t\cdot \left(\frac{n_t}{x_t}\right)^{1-\frac{1}{\gamma}}}{\left(\alpha \left(\frac{k_t}{x_t}\right)^{1-\frac{1}{\gamma}}+(1-\alpha)\left(\frac{n_t}{x_t}\right)^{1-\frac{1}{\gamma}}\right)}\\
        &=\ln A +
        \begin{cases}
        \ln k_t, & k_t<n_t\\
        \ln n_t, & k_t>n_t\\
        \ln \alpha\ln k_t +(1-\alpha)\ln n_t, & k_t=n_t\\
        \end{cases}\\
        &=\ln A + \ln(\min\{n_t,k_t\})\Rightarrow \lim_{\gamma\rightarrow 0}y_t = A\min\{n_t,k_t\}
        \end{align*}
        
        \item[-] \textbf{Sandwich theorem:} without losing generality, assume $k_t\geq n_t>0$, consider this inequality:
        \begin{align*}
            & (1-\alpha)n_t^{1-\frac{1}{\gamma}} \leq \alpha k_t^{1-\frac{1}{\gamma}}+(1-\alpha)n_t^{1-\frac{1}{\gamma}} \leq n_t^{1-\frac{1}{\gamma}}\\
            \Rightarrow & (1-\alpha)^{\frac{1}{1-\frac{1}{\gamma}}}n_t \leq \left[\alpha k_t^{1-\frac{1}{\gamma}}+(1-\alpha)n_t^{1-\frac{1}{\gamma}}\right]^{\frac{1}{1-\frac{1}{\gamma}}} \leq n_t
        \end{align*}
        notice that $$ \lim_{\gamma\rightarrow 0}(1-\alpha)^{\frac{1}{1-\frac{1}{\gamma}}}n_t = n_t$$ then by the Sandwich Theorem, we have
        $$ \lim_{\gamma\rightarrow 0}\left[\alpha k_t^{1-\frac{1}{\gamma}}+(1-\alpha)n_t^{1-\frac{1}{\gamma}}\right]^{\frac{1}{1-\frac{1}{\gamma}}} =n_t$$
        therefore
        \begin{align*}
            \lim_{\gamma\rightarrow 0}y_t &= \lim_{\gamma\rightarrow 0}A\left[\alpha k_t^{1-\frac{1}{\gamma}}+(1-\alpha)n_t^{1-\frac{1}{\gamma}}\right]^{\frac{1}{1-\frac{1}{\gamma}}}\\
            & =A n_t = A\min\left\{k_t,n_t\right\}
        \end{align*}
    \end{itemize}
    
    \item[-] Constant return to scale: $F(\lambda k_t,\lambda n_t)=\lambda F(k_t,n_t)$
    
    It is very easy to verify directly. Another way to verify this is use Euler's theorem, constant return to scale means homogeneity of degree 1, hence Euler's theorem requires $F(k_t,n_t)=F_k(k_t,n_t)k_t+F_n(k_t,n_t)n_t$, that is:
    \begin{align*}
        & k_t F_k(k_t,n_t)+n_t F_n(k_t,n_t) \\
        =& k_t A \frac{1}{1-\frac{1}{\gamma}}\cdot \left(\alpha k_t^{1-\frac{1}{\gamma}}+(1-\alpha)n_t^{1-\frac{1}{\gamma}}\right)^{\frac{1}{\gamma-1}} \alpha\left(1-\frac{1}{\gamma}\right)k_t^{-\frac{1}{\gamma}}\\ &+n_t A \frac{1}{1-\frac{1}{\gamma}}\cdot \left(\alpha k_t^{1-\frac{1}{\gamma}}+(1-\alpha)n_t^{1-\frac{1}{\gamma}}\right)^{\frac{1}{\gamma-1}} (1-\alpha)\left(1-\frac{1}{\gamma}\right)n_t^{-\frac{1}{\gamma}}\\
        =& A\left( \alpha k_t^{1-\frac{1}{\gamma}}+(1-\alpha) n_t^{1-\frac{1}{\gamma}} \right)\cdot\left(\alpha k_t^{1-\frac{1}{\gamma}}+(1-\alpha)n_t^{1-\frac{1}{\gamma}}\right)^{\frac{1}{\gamma-1}}\\
        =& A\left(\alpha k_t^{1-\frac{1}{\gamma}}+(1-\alpha)n_t^{1-\frac{1}{\gamma}}\right)^{\frac{1}{1-\frac{1}{\gamma}}}= F(k_t,n_t)
    \end{align*}
    
    \item[-] positive and diminishing marginal productivity.
    
    Again, fairly straightforward, we verity $F_k(k_t,n_t)>0, F_n(k_t,n_t)>0$; $F_{kk}(k_t,n_t)<0,F_{nn}(k_t,n_t)<0$:
    \begin{itemize}
        \item[-] $F_k$ and $F_n$: take partial derivatives w.r.t. $k_t$, get
        $$
        F_k=A\left(\alpha k_t^{1-\frac{1}{\gamma}}+(1-\alpha)n_t^{1-\frac{1}{\gamma}}\right)^{\frac{1}{1-\frac{1}{\gamma}}-1}\cdot\alpha k_t^{-\frac{1}{\gamma}}\geq 0
        $$
        
        for $n_t$, similarly
        $$
        F_n=A\left(\alpha k_t^{1-\frac{1}{\gamma}}+(1-\alpha)n_t^{1-\frac{1}{\gamma}}\right)^{\frac{1}{1-\frac{1}{\gamma}}-1}\cdot(1-\alpha) n_t^{-\frac{1}{\gamma}}\geq 0
        $$
        
        \item[-] $F_{kk}$ and $F_{nn}$: take twice partial derivatives w.r.t. $k_t$, for simplicity, again let $Q_t\equiv \alpha k_t^{1-\frac{1}{\gamma}}+(1-\alpha)n_t^{1-\frac{1}{\gamma}}$, get
        \begin{align*}
            F_{kk} =& AQ_t^{\frac{1}{1-\frac{1}{\gamma}}-1-1}\cdot\left(\frac{1}{1-\frac{1}{\gamma}}-1\right)\cdot\left(1-\frac{1}{\gamma}\right)\alpha k_t^{-\frac{1}{\gamma}}\cdot \alpha k_t^{-\frac{1}{\gamma}}\\
            &+ \alpha\left(-\frac{1}{\gamma}\right)k_t^{-\frac{1}{\gamma}-1}\cdot AQ_t^{\frac{1}{1-\frac{1}{\gamma}}-1}\\
            =& AQ_t^{\frac{1}{1-\frac{1}{\gamma}}-1-1}\alpha \frac{1}{\gamma}\cdot \alpha k_t^{-\frac{2}{\gamma}}-\alpha\frac{1}{\gamma}k_t^{-\frac{1}{\gamma}-1}AQ_t^{\frac{1}{1-\frac{1}{\gamma}}-1}\\
            =& \alpha \frac{1}{\gamma}k_t^{-\frac{1}{\gamma}-1}AQ_t^{\frac{1}{1-\frac{1}{\gamma}}-1}\cdot \left( Q_t^{-1}\alpha k_t^{1-\frac{1}{\gamma}} -1\right)
        \end{align*}
        since $\alpha k_t^{1-\frac{1}{\gamma}}Q_t^{-1}=\frac{\alpha k_t^{1-\frac{1}{\gamma}}}{\alpha k_t^{1-\frac{1}{\gamma}}+(1-\alpha) n_t^{1-\frac{1}{\gamma}}}<1$, $F_{kk}<0$. Similarly for $F_{nn}$:
        $$
            F_{nn} = (1-\alpha) \frac{1}{\gamma}n_t^{-\frac{1}{\gamma}-1}AQ_t^{\frac{1}{1-\frac{1}{\gamma}}-1}\cdot \left( Q_t^{-1}(1-\alpha) n_t^{1-\frac{1}{\gamma}} -1\right)<0
        $$
        
        \item[-] $F_{kn}=F_{nk}>0$: by the symmetry of Hessian matrix, we only need to verifyt $F_{kn}$, that is
        \begin{align*}
            F_{kn} =& AQ_t^{\frac{1}{1-\frac{1}{\gamma}}-1-1}\left(\frac{1}{1-\frac{1}{\gamma}}-1\right)\alpha k_t^{-\frac{1}{\gamma}}\cdot (1-\alpha)(1-\frac{1}{\gamma})n_t^{-\frac{1}{\gamma}}\\
            =& AQ_t^{\frac{1}{1-\frac{1}{\gamma}}-2}\frac{1}{\gamma}\alpha(1-\alpha)(k_tn_t)^{-\frac{1}{\gamma}}>0
        \end{align*}
    \end{itemize}
\end{itemize}

\subsection*{How to solve the model: social planner's problem}

This model can be analytically solved only when certain assumptions are imposed on utility functions and production functions. However, there exists a solution: capital must be cleared eventually, that is $k_{T+1}=0$; and constraint set for $k_{t+1}$ is compact.

\vspace{2pt}
\begin{tikzpicture}
\node [iredbox] (box){%
    \begin{minipage}{0.23\textwidth}
    \color{white}
    \scriptsize
        $$\max_{\left\{c_t,i_t,k_{t+1},l_t,n_t,y_t\right\}^{\infty}_{t=0}} \sum^{\infty}_{t=0}\beta^t(u(c_t)+\psi(l_t))$$
    s.t.
    \begin{align*}
        y_t&=c_t+i_t &\text{market clearing}\\
        k_{t+1}&=(1-\delta)k_t+i_t &\text{capital accumulation}\\
        1&=n_t+l_t & \text{time endowment}\\
        y_t &= A\left(\alpha k_t^{1-\frac{1}{\gamma}}+(1-\alpha)n_t^{1-\frac{1}{\gamma}}\right)^{\frac{1}{1-\frac{1}{\gamma}}} & \text{production function}\\
        0&\leq c_t,i_t,k_t,n_t,l_t,y_t,\forall t&\text{non-negativity}
    \end{align*}
    \end{minipage}
};
\node[iredtitle, right=4pt] at (box.north west) {Social planner's problem};
\end{tikzpicture}

This problem can be rewritten as:
$$
\max_{\left\{c_t,n_t,k_{t+1}\right\}^{\infty}_{t=0}}\sum^{\infty}_{t=0}\beta^t\left(u(c_t)+\psi(1-n_t)\right)
$$
s.t. 
$$
k_{t+1}=(1-\delta)k_t + A\left(\alpha k_t^{1-\frac{1}{\gamma}}+(1-\alpha)n_t^{1-\frac{1}{\gamma}}\right)^{\frac{1}{1-\frac{1}{\gamma}}} - c_t 
$$
and $c_t,k_t,n_t\geq 0,\forall t$. Then, we can derive the Euler equation with Lagrange method:

\vspace{2pt}
\begin{tikzpicture}
\node [redbox] (box){%
    \begin{minipage}{0.23\textwidth}
    \color{myred}
    \scriptsize
        $$\mathcal{L}= \sum^{\infty}_{t=0}\left[ \beta^t\left(u(c_t)+\psi(1-n_t)\right) +\lambda_t\left( (1-\delta)k_t + F(k_t,n_t)-k_{t+1}-c_t\right)  \right]$$
    FOC gives:
    \begin{align*}
        \frac{\partial \mathcal{L}}{\partial c_t}=0 \Rightarrow & \beta^t u'(c_t)=\lambda_t &(1)\\
        \frac{\partial \mathcal{L}}{\partial n_t}=0 \Rightarrow & \beta^t \psi'(1-n_t)= \lambda_t F_n(k_t,n_t) &(2)\\
        \frac{\partial \mathcal{L}}{\partial k_{t+1}}=0\Rightarrow & \lambda_{t+1}\left(F_k(k_{t+1},n_{t+1}) +1-\delta \right)=\lambda_t &(3)
    \end{align*}
    
    \begin{tikzpicture}
    \node [iredbox] (box){%
        \begin{minipage}{0.935\textwidth}
        \color{white}
        (1) and (3) together, gives the Euler equation:
        $$
        u'(c_t)=\left(F_k(k_{t+1},n_{t+1})+1-\delta\right)\beta u'(c_{t+1})
        $$
        \end{minipage}
    };
    \end{tikzpicture}
    
    Interpretation: today's consumption lost must be compensated by discounted tomorrow's consumption gain.
    \begin{itemize}
        \item[-] $u'(c_t)$: the utility loss of giving up 1 unit of $c_t$ for investment
        \item[-] $\beta u'(c_{t+1})(F_k(k_{t+1},n_{t+1})+1-\delta)$:
        \begin{itemize}
            \item[-] $F_k(k_{t+1},n_{t+1})+1-\delta$: the capital accumulation ($1-\delta$) and production $F_k(k_{t+1},n_{t+1})$ of an extra unit of capital
            \item[-] $\beta u'(c_{t+1})$: per-unit discounted utility compensation at $t+1$
        \end{itemize}
    \end{itemize}
    
    \vspace{2pt}
    \begin{tikzpicture}
    \node [iredbox] (box){%
        \begin{minipage}{0.935\textwidth}
        \color{white}
        (1) and (2) together, gives the consumption-leisure condition:
        $$
        u'(c_t)\cdot F_n(k_t,n_t)=\psi'(1-n_t)
        $$
        \end{minipage}
    };
    \end{tikzpicture}
    
    Interpretation: the utility loss of giving up 1 unit of $l_t$ ($\psi'(1-n_t)$) must be compensated by the consumption utility gain due to the additional output $F_n(k_t,n_t)\cdot u'(c_t)$
    
    \end{minipage}
};
\node[redtitle, right=4pt] at (box.north west) {FOC of social planner's problem};
\end{tikzpicture}

\vspace{2pt}
With the FOCS, we can characterize the equilibrium and steady state of this economy:

\vspace{2pt}
\begin{tikzpicture}
\node [redbox] (box){%
    \begin{minipage}{0.23\textwidth}
    \color{myred}
    \scriptsize
    
    The equilibrium in this economy is characterized by the following equations:
    \begin{align*}
        u'(c_t)&=\left(F_k(k_{t+1},n_{t+1})+1-\delta\right)\beta u'(c_{t+1}) & \text{Euler equation} \\
        u'(c_t)&=\frac{\psi'(1-n_t)}{F_n(k_t,n_t)} & \text{leisure condition}\\
        k_{t+1}&=F(k_t,n_t)+(1-\delta)k_t-c_t & \text{budget constraint}
    \end{align*}
    where $F(k_t,n_t)=A\left(\alpha k_t^{1-\frac{1}{\gamma}}+(1-\alpha) n_t^{1-\frac{1}{\gamma}} \right)^{\frac{1}{1-\frac{1}{\gamma}}}$
    \end{minipage}
};
\node[redtitle, right=4pt] at (box.north west) {Social planner's problem: equilibrium};
\end{tikzpicture}

\vspace{2pt}
\begin{tikzpicture}
\node [redbox] (box){%
    \begin{minipage}{0.23\textwidth}
    \color{myred}
    \scriptsize
    The steady state in this economy is achieved when $c_t\equiv c^*,k_{t+1}\equiv k^*,n_t\equiv n^*$, hence:
    \begin{align*}
        \frac{1}{\beta}-(1-\delta)&=F_k(k^*,n^*) & \text{Euler equation} \\
        u'(c^*)&=\frac{\psi'(1-n^*)}{F_n(k^*,n^*)} & \text{leisure condition}\\
        c^*&=F(k^*,n^*)-\delta k^* & \text{budget constraint}
    \end{align*}
    Plug the output level $y_t=F(k_t,n_t)$ and its partial derivatives back into these three equations, we have the steady state:
    \begin{align*}
        \frac{1}{\beta}-(1-\delta)&=\alpha A^{1-\frac{1}{\gamma}}\left(\frac{y^*}{k^*}\right)^{\frac{1}{\gamma}} & \text{Euler equation} \\
        u'(c^*)&=\frac{\psi'(1-n^*)}{(1-\alpha) A^{1-\frac{1}{\gamma}}\left(\frac{y^*}{n^*}\right)^{\frac{1}{\gamma}}} & \text{leisure condition}\\
        c^*&=y^*-\delta k^* & \text{budget constraint}
    \end{align*}
    
    And we can have the steady state capital-labor ratio:
    $$
    \frac{k^*}{n^*}= \left( \frac{\alpha}{1-\alpha}\cdot\frac{\psi'(1-n^*)/u'(c^*)}{\frac{1}{\beta}-(1-\delta)} \right)
    $$
    but it is generally impossible to solve analytically.
    \end{minipage}
};
\node[redtitle, right=4pt] at (box.north west) {Social planner's problem: steady state};
\end{tikzpicture}

%%%%%%%%%%%%%%%%%%%%%%%%%%%%%%%%%%%%%%%%%
\subsection*{Social planner's problem: a simple example}
In general, social planner's problem of NGM cannot be solved analytically, but for the simplest example, it actually is possible.





%%%%%%%%%%%%%%%%%%%%%%%%%%%%%%%%%%%%%%%%%
% Endogenous growth model:
\newpage
\section*{Endogenous growth models}
To explain \textbf{sustained differences} in growth rates across countries. Instead of assuming exogenous technological changes, these models either alter specifications of the production function and capital accumulation or endogenize technological changes.

\subsection*{AK model}
The basic idea is to \textbf{NOT} let the production function cross the 45-degree line so that it can imply sustained growth.

\begin{tikzpicture}
\node [bluebox] (box){%
    \begin{minipage}{0.23\textwidth}
    \color{myblue}
    \scriptsize
    Start from a simple Solow world: savings and investment is constant fractions of output. The savings and consumption per capita are:
    $$s_t =sAk_t,\ c_t =(1-s)Ak_t$$
    Then the capital accumulation is 
    $$k_{t+1}=(1-\delta)k_t +sAk_t\Rightarrow k_{t+1}=(1-\delta+sA)k_t $$
    Hence, capital growth rate is constant:
    $$ g_k = \frac{k_{t+1}}{k_t}=1-\delta+sA $$
    All endogenous variables grow at this rate:
    $$
    g_y=\frac{Ak_{t+1}}{Ak_t}=g_k = \frac{(1-s)Ak_{t+1}}{(1-s)Ak_t}= g_c = \frac{k_{t+2}-(1-\delta)k_{t+1}}{k_{t+1}-(1-\delta)k_t}=g_i
    $$
    That is, growth rate of output per capita $g_y$ always equals the growth rate of capital per capita $g_k$, \textbf{on or off a balance growth path}. With Solow preferences, there is no transition to BGP: you either on the path, or not.
    \end{minipage}
};
\node[bluetitle, right=4pt] at (box.north west) {A simple example: Solow policy};
\end{tikzpicture}

\vspace{2pt}
Now, we move to a AK model with CES preferences, consider the social planner's problem:

\begin{tikzpicture}
\node [redbox] (box){%
    \begin{minipage}{0.23\textwidth}
    \color{myred}
    \scriptsize
$$\max \sum^{\infty}_{t=0}\beta^t\frac{c_t^{1-\sigma}-1}{1-\sigma}$$
s.t.
$$c_t+k_{t+1} - (1-\delta)k_t =Ak_t, k_0>0$$
Here the production function is $F(K,L)=AK$, the per capita production function is $f\left(\frac{K}{L}\right)=f(k)=Ak$.
    \end{minipage}
};
\node[redtitle, right=4pt] at (box.north west) {Social planner's problem};
\end{tikzpicture}

The Lagrange:
$$ \mathcal{L}\sum^{\infty}_{t=0}\beta^t\frac{c_t^{1-\sigma}-1}{1-\sigma} +\mu_t \left[Ak_t - (c_t+k_{t+1}-(1-\delta)k_t) \right] $$
FOCs are:
\begin{align*}
    \frac{\partial \mathcal{L}}{\partial c_t}=0 &\Rightarrow \beta^t c_t^{-\sigma}=\mu_t\\
    \frac{\partial \mathcal{L}}{\partial k_{t+1}}=0 &\Rightarrow \mu_t = \mu_{t+1}(A+1-\delta)
\end{align*}
which give the Euler equation:

\begin{tikzpicture}
\node [iredbox] (box){%
    \begin{minipage}{0.23\textwidth}
    \color{white}
    \scriptsize
    The Euler equation is
    $$ c_t^{-\sigma} = \beta c_{t+1}^{-\sigma}(A+1-\delta) $$ which gives a constant consumption growth rate $$g_c = \frac{c_{t+1}}{c_t}=\left[\beta(A+1-\delta)\right]^{1/\sigma}$$
    \end{minipage}
};
\node[iredtitle, right=4pt] at (box.north west) {AK model: Euler equation};
\end{tikzpicture}

This optimal, constant consumption growth rate will be equal to the constant growth rate of investment and output.

\begin{tikzpicture}
\node [redbox] (box){%
    \begin{minipage}{0.23\textwidth}
    \color{myred}
    \scriptsize
    Rearrange the constraint:
    $$c_t+k_{t+1} - (1-\delta)k_t =Ak_t \xRightarrow{\div k_t} \frac{c_t}{k_t}+\frac{k_{t+1}}{k_t}=A+1-\delta$$
    this will give that $g_k$ is constant:
    \begin{itemize}
        \item[-] \textbf{if $g_k$ increases}, $c_t/k_t$ must be decreasing over time, but this violates the transversality condition, as marginal utility would fall less quickly than capital grows, hence it is \textbf{NOT an optimal path}. 
        \item[-] \textbf{if $g_k$ decreases}, $c_t/k_t$ must be increasing over time, consumption would exhaust output in finite time and \textbf{violate non-negativity}.
    \end{itemize}
    
    \vspace{2pt}
    \begin{tikzpicture}
    \node [iredbox] (box){%
        \begin{minipage}{0.935\textwidth}
        \color{white}
        Hence, on BGP, capital must grow at a \textbf{constant rate}, the same growth rate of consumption $g_k=g_c$. Rewrite the consumption-capital ratio:
        \begin{align*}
            & \frac{c_t}{k_t} +g_c=A+1-\delta\\
            & \Rightarrow \frac{c_t}{k_t}=A+1-\delta - \left[\beta(A+1-\delta)\right]^{1/\sigma}>0\\
            & \Rightarrow (A+1-\delta)\cdot\left[1-\beta^{1/\sigma}(A+1-\delta)^{\frac{1-\sigma}{\sigma}}\right]>0\\
            & \Rightarrow 1- \beta^{1/\sigma}(A+1-\delta)^{\frac{1-\sigma}{\sigma}}>0
        \end{align*}
        \end{minipage}
    };
    \end{tikzpicture}
    
    \vspace{2pt}
    This condition $1- \beta^{1/\sigma}(A+1-\delta)^{\frac{1-\sigma}{\sigma}}>0$, combined with the positive growth of consumption: $g_c=\frac{c_{t+1}}{c_t}=\left(\beta (A+1-\delta)\right)^{1/\sigma}>1$, gives the conditions of $\beta, A,\delta$:
    $$ \left(\beta (A+1-\delta)\right)^{1/\sigma}>1 >\beta^{1/\sigma}(A+1-\delta)^{\frac{1-\sigma}{\sigma}}$$
    
    \vspace{2pt}
    \begin{tikzpicture}
    \node [iredbox] (box){%
        \begin{minipage}{0.935\textwidth}
        \color{white}
        For a given $k_0$, the unique value of initial consumption $c_0$ that puts the economy on the unique balanced growth path is given by:
        $$ c_0 +k_1 -(1-\delta)k_0 =Ak_0 $$
        rearrange this, get
        $$
        c_0 = Ak_0 - k_1 + (1-\delta)k_0 = (A-g_c +(1-\delta))k_0>0
        $$
        If and only if $c_0 = (A-g_c +(1-\delta))k_0>0$, the economy is on the balanced growth path.
        \end{minipage}
    };
    \end{tikzpicture}
    
    \vspace{2pt}
    The balanced growth rate of this economy is
    $$ g_c = g_k =g_i =g_y = \left(\beta (A+1-\delta)\right)^{1/\sigma}$$
    Finally, we need to check whether lifetime utility is bounded: 
    $$ \max\sum^\infty_{t=0}\beta^t\frac{c_t^{1-\sigma}-1}{1-\sigma} = \sum^\infty_{t=0}\beta^t \frac{\left[\left(\beta (A+1-\delta)\right)^{t/\sigma}c_0\right]^{1-\sigma}-1}{1-\sigma}$$
    For this to be bounded, we must have $\beta^t\cdot \left(\beta (A+1-\delta)\right)^{\frac{1-\sigma}{\sigma}\cdot t} $ is bounded, or 
    $$ \beta \cdot \left(\beta (A+1-\delta)\right)^{\frac{1-\sigma}{\sigma}} = \beta^{\frac{1}{\sigma}} (A+1-\delta)^{\frac{1-\sigma}{\sigma}} <1$$
    this is satisfied by the condition of $c_t/k_t>0$ already.
    \end{minipage}
};
\node[redtitle, right=4pt] at (box.north west) {Growth rate of $c_t$, $k_t$, $i_t$, $y_t$};
\end{tikzpicture}

The core feature of this model is that the balanced growth rate, always attained immediately after selecting the proper initial consumption, is a function of preferences ($\beta,\sigma$) and technology ($A$), which vary across countries.


%%%%%%%%%% lucas two-sector
\newpage
\subsection*{Lucas two-sector model}
\vspace{2pt}
\begin{tikzpicture}
\node [redbox] (box){%
    \begin{minipage}{0.23\textwidth}
    \color{myred}
    \scriptsize
$$\max \sum^{\infty}_{t=0}\beta^t\frac{c_t^{1-\sigma}-1}{1-\sigma}$$
s.t.
\begin{align*}
    c_t+K_{t+1}-(1-\delta_K)K_t& = K_t^{\alpha}\left(\phi_t H_t \right)^{1-\alpha} & {\color{black}{\scriptscriptstyle\phi_tH_t}\textbf{ output}} \\
    H_{t+1}-H_{t}&=A(1-\phi_t)H_t & {\color{black}{\scriptscriptstyle(1-\phi_t)H_t} \textbf{ accumulation}}\\
    c_t, \phi_t, H_{t+1},K_{t+1} & \geq 0
\end{align*}
    \end{minipage}
};
\node[redtitle, right=4pt] at (box.north west) {Social planner's problem};
\end{tikzpicture}

\vspace{2pt}
The \textbf{Lagrangean} (choosing ${\color{myred}\left\{c_t,\phi_t,H_{t+1},K_{t+1}\right\}}$) is
\begin{align*}
    \mathcal{L} = &\sum^{\infty}_{t=0}\beta^t\frac{c_t^{1-\sigma}-1}{1-\sigma} \\
    &+\mu_{K,t}\left[ K_t^{\alpha}\left(\phi_t H_t \right)^{1-\alpha}-\left(c_t+K_{t+1}-\left(1-\delta_K\right)K_t\right) \right]\\
    & + \mu_{H,t}\left[A\left(1-\phi_t\right)H_t-\left(H_{t+1}-H_t\right) \right]
\end{align*}

We have two sets of \textbf{FOC}:

\begin{itemize}
    \item[-] w.r.t. control variables $\{c_t,\phi_t\}$:
    \begin{align*}
        \frac{\partial \mathcal{L}}{\partial c_t}=0 \Rightarrow & \beta^t c_t^{-\sigma}=\mu_{K,t}\\
        \frac{\partial \mathcal{L}}{\partial \phi_t}=0 \Rightarrow& (1-\alpha)\left(\frac{\phi_t H_t}{K_t}\right)^{-\alpha}\mu_{K,t}=A\mu_{H,t}
    \end{align*}
    
    \item[-] w.r.t. state variables $\{K_{t+1},H_{t+1}\}$:
    \begin{align*}
        \frac{\partial \mathcal{L}}{\partial K_{t+1}} =0\Rightarrow  \mu_{K,t}=&\mu_{K,t+1}\left[\alpha\left( \frac{\phi_{t+1}H_{t+1}}{K_{t+1}}\right)^{1-\alpha} + (1-\delta_K)  \right]\\
        \frac{\partial \mathcal{L}}{\partial H_{t+1}}=0\Rightarrow \mu_{H,t}=&\mu_{H,t+1}\left[A(1-\phi_{t+1})+1\right]+\\
         &\mu_{K,t+1}(1-\alpha)\phi_{t+1}\left(\frac{\phi_{t+1}H_{t+1}}{K_{t+1}}\right)^{-\alpha}\\
    \end{align*}
\end{itemize}

To solve this system of equations, first plug $(1-\alpha)\left(\frac{\phi_t H_t}{K_t}\right)^{-\alpha}\mu_{K,t}=A\mu_{H,t}$ into the second set of FOCs, to get the relation between $\mu_{K,t}$ and $\mu_{K,t+1}$:
\begin{align*}
    &(1-\alpha)\left(\frac{\phi_t H_t}{K_t}\right)^{-\alpha}\mu_{K,t}\\
    = & A\left\{ \mu_{H,t+1}\left[A(1-\phi_{t+1})+1\right]+\mu_{K,t+1}(1-\alpha)\phi_{t+1}\left(\frac{\phi_{t+1}H_{t+1}}{K_{t+1}}\right)^{-\alpha} \right\} \\
    = & (1-\alpha)\left(\frac{\phi_{t+1}H_{t+1}}{K_{t+1}}\right)^{-\alpha}\mu_{K,t+1}\left[A(1-\phi_{t+1})+1\right]\\
    &+ A \mu_{K,t+1}(1-\alpha)\phi_{t+1}\left(\frac{\phi_{t+1}H_{t+1}}{K_{t+1}}\right)^{-\alpha}\\
    \Rightarrow & \left(\frac{\phi_t H_t}{K_t}\right)^{-\alpha}\mu_{K,t} = \mu_{K,t+1}\left(\frac{\phi_{t+1}H_{t+1}}{K_{t+1}}\right)^{-\alpha}(1+A)
\end{align*}

Plug this equation back to the capital accumulation FOC ($\partial \mathcal{L}/\partial K_{t+1}=0$), ${\color{myred}\mu_{K,t}=\mu_{K,t+1}\left[\alpha\left( \frac{\phi_{t+1}H_{t+1}}{K_{t+1}}\right)^{1-\alpha} + (1-\delta_K)  \right]}$, get: 
\begin{align*}
    &\left(\frac{(\phi_{t+1}/\phi_t)(H_{t+1}/H_t)}{(K_{t+1}/K_t)}\right)^{-\alpha}(1+A) = \left[\alpha\left( \frac{\phi_{t+1}H_{t+1}}{K_{t+1}}\right)^{1-\alpha} + (1-\delta_K)  \right]\\
    \Rightarrow & {\color{myred}\left(\frac{(\phi_{t+1}/\phi_t)(H_{t+1}/H_t)}{(K_{t+1}/K_t)}\right)^{-\alpha}=\frac{\left[\alpha\left( \frac{\phi_{t+1}H_{t+1}}{K_{t+1}}\right)^{1-\alpha} + (1-\delta_K)  \right]}{1+A}}
\end{align*}

Combine this with FOC w.r.t. $c_t$, ${\color{myred}\mu_{K,t}=\beta^t c_t^{-\sigma}}$, get the Euler equation:

\vspace{2pt}
\begin{tikzpicture}
\node [iredbox] (box){%
    \begin{minipage}{0.23\textwidth}
    \color{white}
    \scriptsize
    The Euler equation of this problem is
    $$
    \left(\frac{\phi_t H_t}{K_t}\right)^{-\alpha}c_t^{-\sigma} = \beta c_{t+1}^{-\sigma}\left(\frac{\phi_{t+1}H_{t+1}}{K_{t+1}}\right)^{-\alpha}(1+A)
    $$
    This equation gives the expression of the consumption growth rate:
\begin{align*}
    &\left(\frac{\phi_t H_t}{K_t}\right)^{-\alpha}c_t^{-\sigma} = \beta c_{t+1}^{-\sigma}\left(\frac{\phi_{t+1}H_{t+1}}{K_{t+1}}\right)^{-\alpha}(1+A)\\
    \Rightarrow & g_c=\frac{c_{t+1}}{c_t}=\left\{\beta \cdot \frac{\left[\alpha\left( \frac{\phi_{t+1}H_{t+1}}{K_{t+1}}\right)^{1-\alpha} + (1-\delta_K)  \right]}{1+A}\cdot(1+A) \right\}^{1/\sigma}\\
    \Rightarrow & g_c = \left\{\beta \cdot \left[\alpha\left( \frac{\phi_{t+1}H_{t+1}}{K_{t+1}}\right)^{1-\alpha} + (1-\delta_K)  \right] \right\}^{1/\sigma}
\end{align*}
    \end{minipage}
};
\node[iredtitle, right=4pt] at (box.north west) {Euler equation};
\end{tikzpicture}

\vspace{2pt}
For a \textbf{balanced growth path} to satisfy this equation, that is, $g_c$ being constant, $H_{t+1}$ and $K_{t+1}$ must have the same growth rate, and this growth rate is also the growth rate of output ($\phi_t$ constant): 
$$ g_K = g_H\Rightarrow g_y =g_K^{\alpha}g_H^{1-\alpha}=g_K=g_H$$

Next, calculate the growth rates:

\vspace{2pt}
\begin{tikzpicture}
\node [redbox] (box){%
    \begin{minipage}{0.23\textwidth}
    \color{myred}
    \scriptsize
    On BGP, the growth rate of all variables are constant:
    \begin{itemize}
        \item[-] $g_H$: growth rate of human capital
        $$g_H = \frac{H_{t+1}}{H_t}=A\left(1-\phi_t\right)+1 \xRightarrow{\text{constant }g_H} g_H=A(1-\phi)+1 $$
        \item[-] $g_K$: growth rate of physical capital
        $$ g_K =g_H =A(1-\phi)+1$$
        \item[-] $g_y$: growth rate of output
        $$g_y = g_K = g_H = A(1-\phi)+1$$ 
    \end{itemize}
    \rule{\textwidth}{0.4pt}
    \vspace{1pt}
    
    What about the consumption rate? 
    
    By the Euler equation: 
    $$
    \left(\frac{\phi_t H_t}{K_t}\right)^{-\alpha}c_t^{-\sigma} = \beta c_{t+1}^{-\sigma}\left(\frac{\phi_{t+1}H_{t+1}}{K_{t+1}}\right)^{-\alpha}(1+A)
    $$
    the consumption growth rate is
    \begin{align*}
        g_c & =\frac{c_{t+1}}{c_t}=\left\{\beta \cdot \left(\frac{\frac{\phi_{t+1}}{\phi_t}\frac{H_{t+1}}{H_t}}{\frac{K_{t+1}}{K_t}}\right)^{-\alpha}\cdot(1+A) \right\}^{1/\sigma}\\
        \xRightarrow[\text{constant }\phi]{g_H=g_K} g_c & =\left(\beta(1+A) \right)^{1/\sigma} = g_H=g_K=A(1-\phi)+1
    \end{align*}
    In return, $\phi = 1-\frac{\left(\beta(1+A) \right)^{1/\sigma}-1}{A}$ can be used to determine the optimal, BGP allocation of human capital to human capital accumulation $\phi$
    \end{minipage}
};
\node[redtitle, right=4pt] at (box.north west) {Growth rate on balanced growth path (BGP)};
\end{tikzpicture}

How do we know that on a balanced growth path, consumption growth rate $g_c$ is the same as $g_K$ and $g_H$? By looking at the resource constraint:
\begin{align*}
    & c_t+K_{t+1}-(1-\delta_K)K_t =Y_t\\
    \xRightarrow{\div K_t} &  \frac{c_t}{K_t}+\frac{K_{t+1}}{K_t}=\frac{Y_t}{K_t}+1-\delta_K
\end{align*}
where $K_{t+1}/K_t=g_K =A(1-\phi)+1 $ is constant, $Y_t/K_t$ is also constant since $Y_{t+1}/Y_t=K_{t+1}/K_t$, hence $c_t/K_t$ must also be constant, therefore, $c_{t+1}/c_{t}=K_{t+1}/K_t$.


%%%%%%%%%%%%%%%%%%%%%%%%%%%%%%%%%%%%%%%%%%%%%%%%
\newpage
%Romer's endogeneous imperfect competitive market

\section*{Romer's endogenous growth model}
\subsection*{Love of variety: Dixit and Stiglitz (1977)}
The \textbf{love of variety} (here in final goods production) rationalizes the growth of trade in similar intermediate inputs among similar countries: 

\begin{itemize}
    \item[-]\textbf{{\color{myblue}demand side}}: different varieties of the same product are produced in each country and very similar varieties are traded among countries because consumers or firms value variety.
    \item[-]\textbf{{\color{myblue}supply side}}: international trade in varieties of the same intermediate good is due to a search for larger markets by firms facing increasing returns to scale/falling average costs.
\end{itemize}

\vspace{2pt}
\begin{tikzpicture}
\node [bluebox] (box){%
    \begin{minipage}{0.23\textwidth}
    \color{myblue}
    \scriptsize
    Consumers value diversity, a mass of varieties of consumer goods, branded by $i$:
    $$
    U=\left(\int^n_0 q(i)^{1-\frac{1}{\sigma}}\mathrm{d}i\right)^{\frac{\sigma}{\sigma-1}}
    $$
    here $i$ is the index, $q(i)$ is quantity of demand, $\sigma$ is the elasticity of substitution among products, assumed to be $\sigma > 1$.
    
    A consumer's problem is then:
    $$
    \max U \text{ s.t. }\int^n_0q(i)p(i)\mathrm{d}i\leq I
    $$
    where $I$ is the income, $p(i)$ is the $i$th product's price.
    \end{minipage}
};
\node[bluetitle, right=4pt] at (box.north west) {Love of variety: model setting};
\end{tikzpicture}

\vspace{2pt}
To solve this model, write Lagrange:
$$
\mathcal{L} = \left(\int^n_0 q(i)^{1-\frac{1}{\sigma}}\mathrm{d}i\right)^{\frac{\sigma}{\sigma-1}}+\mu\left( I-\int^n_0q(i)p(i)\mathrm{d}i\right)
$$
FOC w.r.t. $q(i)$ is:
\begin{align*}
    \frac{\partial\mathcal{L}}{\partial q(i)}=0\Rightarrow & \frac{\sigma}{\sigma-1} \left(\int^n_0 q(i)^{1-\frac{1}{\sigma}}\mathrm{d}i\right)^{\frac{1}{\sigma-1}}\cdot\frac{\sigma-1}{\sigma}q(i)^{-\frac{1}{\sigma}}=\mu p(i)\\
    \Rightarrow & U^{\frac{1}{\sigma}}q(i)^{-\frac{1}{\sigma}}=\mu p(i)
\end{align*}

Two objects can be derived from the FOC:
\begin{itemize}
    \item[-] demand for variety zero $q(0)$: 
    
    for $i$th and $j$th good, the relative demands $q(i)/q(j)$ is:
$$ \frac{q(i)}{q(j)} = \left( \frac{p(i)}{p(j)} \right)^{-\sigma} $$

    Then if we take an arbitrary good, good 0, as reference, the demand for every variety $i$ can be written as:
    $$
    q(i) = q(0)\left(\frac{p(i)}{p(0)} \right)^{-\sigma}
    $$
    plug this expression of $q(i)$ back into the budget constraint, get
    \begin{align*}
        & I=\int^n_0 q(0)p(i)\left(\frac{p(i)}{p(0)} \right)^{-\sigma} \mathrm{d}i 
    =q(0)p(0)^{\sigma} \int^n_0 p(i)^{1-\sigma}\mathrm{d}i\\
    \Rightarrow & q(0)=\frac{I}{p(0)^{\sigma} \int^n_0 p(i)^{1-\sigma}\mathrm{d}i}\ {\color{myblue}\cdots \text{demand for variety zero}}
    \end{align*}

    \item[-] price index for varieties $P$:
    
    Transform FOC, we can get:
    \begin{align*}
        &U^{\frac{1}{\sigma}}q(i)^{-\frac{1}{\sigma}}=\mu p(i)\ {\color{myblue}\cdots {\text{ FOC w.r.t. }q(i)}}\\
        \xRightarrow{{\color{myblue}(\cdot)^{1-\sigma}}}& U^{\frac{1-\sigma}{\sigma}}q(i)^{-\frac{1-\sigma}{\sigma}}=\mu ^{1-\sigma}p(i)^{1-\sigma}\\
        \xRightarrow{{\color{myblue}\int\mathrm{d}i}} & U^{\frac{1-\sigma}{\sigma}} \int^n_0 q(i)^{-\frac{1-\sigma}{\sigma}}\mathrm{d}i=\mu ^{1-\sigma} \int^n_0 p(i)^{1-\sigma}\mathrm{d}i
    \end{align*}
    and we know $U=\left(\int^n_0 q(i)^{\frac{\sigma-1}{\sigma}}\mathrm{d}i\right)^{\frac{\sigma}{\sigma-1}}$, hence
    $$
    \mu^{1-\sigma}\int^n_0 p(i)^{1-\sigma}\mathrm{d}i=1\Rightarrow \frac{1}{\mu}=P=\left( \int^n_0 p(i)^{1-\sigma}\mathrm{d}i \right)^{\frac{1}{1-\sigma}}
    $$
    here, $\mu$ can be interpreted as {\color{myblue}\textbf{the shadow/utility value}} of an additional unit of expenditure on the {\color{myblue}\textbf{consumption index}}; $\frac{1}{\mu}$ can be interpreted as the {\color{myblue}\textbf{expenditure}} required for one unit of the {\color{myblue}\textbf{consumption index}}.
\end{itemize}

Now with $q(0)=\frac{I}{p(0)^{\sigma} \int^n_0 p(i)^{1-\sigma}\mathrm{d}i}$ and $P=\left( \int^n_0 p(i)^{1-\sigma}\mathrm{d}i \right)^{\frac{1}{1-\sigma}}$, we have demand for good 0:
$$
q(0)=\frac{I}{p(0)^{\sigma}P^{1-\sigma}}=\frac{I}{P}\left( \frac{p(0)}{P} \right)^{-\sigma}
$$
and since $q(i) = q(0)\left(\frac{p(i)}{p(0)} \right)^{-\sigma}$, this will give a nice result:

\vspace{2pt}
\begin{tikzpicture}
\node [bluebox] (box){%
    \begin{minipage}{0.23\textwidth}
    \color{myblue}
    \scriptsize
    The demand for good $i$ is:
    $$
    q(i)=\frac{I}{P}\left( \frac{p(0)}{P} \right)^{-\sigma}\cdot \left(\frac{p(i)}{p(0)} \right)^{-\sigma}= \frac{I}{P}\left( \frac{p(i)}{P} \right)^{-\sigma}
    $$
    it is clear that $\frac{\partial q(i)}{\partial p(i)}=\frac{I}{P^{1-\sigma}}(-\sigma)p(i)^{-\sigma-1}<0$: each variety has a \textbf{downward sloping} demand function.
    
    \rule{\textwidth}{0.4pt}
    \vspace{1pt}
    
    put the demand function back to utility function, get consumers \textbf{indirect utility}:
    \begin{align*}
       U & =\left(\int^n_0 q(i)^{1-\frac{1}{\sigma}}\mathrm{d}i\right)^{\frac{\sigma}{\sigma-1}} = \left(\int^n_0 \left( \frac{I}{P}\left( \frac{p(i)}{P} \right)^{-\sigma} \right)^{1-\frac{1}{\sigma}}\mathrm{d}i\right)^{\frac{\sigma}{\sigma-1}}\\
       & = \frac{I}{P}\left(\int^n_0 \left( \frac{p(i)}{P} \right)^{1-\sigma}\mathrm{d}i\right)^{\frac{\sigma}{\sigma-1}}=\frac{I}{P}\cdot P^\sigma \cdot \left(\int^n_0 p(i)^{1-\sigma}\mathrm{d}i\right)^{\frac{\sigma}{\sigma-1}}
    \end{align*}
    since $P=\left( \int^n_0 p(i)^{1-\sigma}\mathrm{d}i \right)^{\frac{1}{1-\sigma}}$, the indirect utility is
    $$
    U = \frac{I}{P}P^{\sigma}P^{-\sigma} =\frac{I}{P}
    $$
    
    If all goods have the same price, all goods are consumed in equal amounts i.e. $q(i)=q(j)= q,\forall i,j$. The by the \textbf{symmetry} of the market, the budget constraint can be rewritten as $I=\int ^n_0 p(i)q(i)\mathrm{d}i=nqp$, which gives the quality consumed for each variety:
    $$ q=\frac{I}{np} $$
    and the utility is 
    \begin{align*}
        U & =\left(\int^n_0 q(i)^{1-\frac{1}{\sigma}}\mathrm{d}i\right)^{\frac{\sigma}{\sigma-1}} = \left(\int^n_0 \left(\frac{I}{np}\right)^{1-\frac{1}{\sigma}}\mathrm{d}i\right)^{\frac{\sigma}{\sigma-1}}\\
        & = \left[\left(\frac{I}{np}\right)^{\frac{\sigma-1}{\sigma}}\cdot n\right]^{\frac{\sigma}{\sigma-1}} = n^{\frac{1}{\sigma-1}}\frac{I}{P}
    \end{align*}
    here, $\partial U/\partial n = \frac{1}{\sigma-1}n^{\frac{-\sigma}{\sigma-1}}\cdot \frac{I}{P}>0$, the love for variety is there.
    \end{minipage}
};
\node[bluetitle, right=4pt] at (box.north west) {Love of variety: demand and indirect utility};
\end{tikzpicture}

\textbf{The role of $\sigma$}: $\sigma$ is the willingness to substitute among varieties, when $n\geq 1$ (at least one type) and $\sigma>1$ (assumed) higher $\sigma$ leads to lower $\partial U/\partial n$, or a more \textit{mildly} love for variety, as shown in the following picture:

\begin{center}
\begin{tikzpicture}
\draw[->] (-0.2, 0) -- (4, 0) node[below right] {$n$};
\draw[->] (0, -0.2) -- (0, 4) node[above] {$U$};
\draw[scale=0.5, domain=1:2.7, smooth, variable=\x, black] plot ({\x}, {(\x)^2})
node[right] {$\sigma=1.5$};
\draw[scale=0.5, domain=1:7, smooth, variable=\x, black] plot ({\x}, {\x})
node[above] {$\sigma=2$};
\draw[scale=0.5, domain=1:8, smooth, variable=\x, black] plot ({\x}, {(\x)^(2/3)})
node[above] {$\sigma=2.5$};
\draw[scale=0.5, domain=1:8, smooth, variable=\x, black] plot ({\x}, {(\x)^(1/2)})
node[above] {$\sigma=3$};
\draw[scale=0.5, domain=1:8, smooth, variable=\x, black] plot ({\x}, {(\x)^(2/5)})
node[below] {$\sigma=3.5$};
\draw[scale=0.5, domain=1:8, smooth, variable=\x, black] plot ({\x}, {1})
node[below] {$\sigma=\infty$};
\end{tikzpicture}
\end{center}

But the 1st order increasing, i.e., the love for variety, persists.

Now we analysis the cost and prices in this model:
\vspace{2pt}

\vspace{2pt}
\begin{tikzpicture}
\node [bluebox] (box){%
    \begin{minipage}{0.23\textwidth}
    \color{myblue}
    \scriptsize
    Assume the profit is
    $$ \pi(i)=p(i)q(i)- wcq(i)-c_f $$
    where
    \begin{itemize}
        \item[-] $w$: wage of one unit of labor
        \item[-] $c$: number of units of labor required to produce one unit of output
        \item[-] $c_f$: fixed labor cost
    \end{itemize}
    
    Since each firm produces a differentiated (but very similar) good, they have a monopoly power (\textbf{monopolistic} competition), they can either be \textbf{price-setting} or \textbf{quantity-setting}. Here, assume price setting, the FOC is:
    $$
    \frac{\partial \pi(i)}{p(i)}=q(i)+p(i)\frac{\partial q(i)}{\partial p(i)}-wc\frac{\partial q(i)}{\partial p(i)}=0
    \Rightarrow p(i)=wc - \frac{q(i)}{\partial q(i)/\partial p(i)}$$
    here, $\partial q(i)/\partial p(i)<0$ is a market-up over marginal cost, i.e., the price is set to be higher than the perfect competition price (marginal cost).
    
    
    \rule{\textwidth}{0.4pt}
    \vspace{1pt}
    
    From the demand side, demand for good $i$ is 
    $$
    q(i)=\frac{I}{P}\left(\frac{p(i)}{P}\right)^{-\sigma}
    $$
    which gives
    $$
    \frac{\partial q(i)}{\partial p(i)}=-\sigma \frac{I}{P}\left(\frac{p(i)}{P}\right)^{-\sigma-1}\frac{1}{P}
    $$
    leading to 
    $$
    \frac{q(i)}{\partial q(i)/\partial p(i)} = \frac{1}{-\sigma \left(\frac{p(i)}{P}\right)^{-1}\frac{1}{P}}=-\frac{p(i)}{\sigma}<0
    $$
    plug this back in the pricing function, get:
    $$
    p(i)=wc-\frac{q(i)}{\partial q(i)/\partial p(i)}=wc+\frac{p(i)}{\sigma}\Rightarrow p(i)=\frac{wc\sigma}{\sigma-1}
    $$
    Hence, the optimal pricing strategy is a \textbf{proportional markup} over marginal cost.
    
    \end{minipage}
};
\node[bluetitle, right=4pt] at (box.north west) {Love of variety: supply profit, cost, prices};
\end{tikzpicture}

A typical variant of this model, production of \textbf{each variety} is free-entry, i.e., for each variety $i$, there is a level of $q(i)$ leading to zero-profit, that is 
$$
\pi(i)=p(i)q(i)- wcq(i)-c_f=0\Rightarrow \left(\frac{\sigma}{\sigma-1}-1\right) q(i) = c_f\Rightarrow q(i)=\sigma c_f
$$

\subsection*{Romer's 3 sector model}
To incorporate the imperfect competition in real economy, Romer model constructed a \textbf{3-technology/sector} model:
\begin{itemize}
    \item[1.] Final good sector: \textbf{\underline{perfect}} competition
    \item[2.] Variety/intermediate sector: \textbf{\underline{monopolistic}} competition
    \item[3.] R\&D sector: \textbf{\underline{perfect}} competition.
\end{itemize}

\vspace{2pt}
\begin{tikzpicture}
\node [redbox] (box){%
    \begin{minipage}{0.23\textwidth}
    \color{myred}
    \scriptsize
    Market structure:
    \begin{itemize}
        \item[-] production: constant-return-to-scale technology with intermediate goods and labor:
        $$
        Y_t = L_{1,t}^{1-\alpha}\int^{A_t}_0 x_t^{\alpha}(i)\mathrm{d}i
        $$
        where
        \begin{itemize}
            \item[-] $A_t$ is the measure of varieties, final goods production take $A_t$ as exogenously given, $Y_t \propto A_t$.
            \item[-] labor $L_{1,t}$ is purchased from HHs, assume to be a fraction of the total labor time $L$ (assumed to be constant): 
            $$L_{1,t}=\phi_t L$$
            \item[-] intermediate goods $x_t(i)$ are from variety sector
        \end{itemize}
        
        \item[-] Perfectly competitive, \textbf{zero profit}.
    \end{itemize}
    
    \end{minipage}
};
\node[redtitle, right=4pt] at (box.north west) {Final goods market};
\end{tikzpicture}

\vspace{2pt}
\begin{tikzpicture}
\node [redbox] (box){%
    \begin{minipage}{0.23\textwidth}
    \color{myred}
    \scriptsize
    Market structure:
    \begin{itemize}
        \item[-] production: linear in raw physical capital, requires a \textbf{fixed cost} (R\&D patent):
        $$
        \int^{A_t}_0 x_t \mathrm{d}i = K_t
        $$
        where
        \begin{itemize}
            \item[-] $x_t(i)=K_t(i)$: one unit of $K_t$ produces one unit of $x_t(i)$, \underline{$\forall i$}
            \item[-] capital is purchased from HHs, the accumulation follows: 
            $$K_{t+1}=(1-\delta)K_t+i_t$$
            
        \end{itemize}
        
        \item[-] Monopolistically competitive, with pricing power, prices \textbf{above marginal cost}, makes \textbf{positive profits} to cover the fixed cost.
    \end{itemize}
    
    \end{minipage}
};
\node[redtitle, right=4pt] at (box.north west) {Intermediate goods market};
\end{tikzpicture}

\vspace{2pt}
\begin{tikzpicture}
\node [redbox] (box){%
    \begin{minipage}{0.23\textwidth}
    \color{myred}
    \scriptsize
    Market structure:
    \begin{itemize}
        \item[-] production: using labor and \textbf{existing ideas}, generating more varieties:
        $$
        A_{t+1}-A_t = \kappa (1-\phi_t)L_t A_t
        $$
        where
        \begin{itemize}
            \item[-] $L_{2,t}=(1-\phi_t)L$: labor used to produce new varieties
            \item[-] $A_t$ are the existing ideas: non-rivalry, hence \textbf{externality}, representing ''standing on shoulders'' effect
            
        \end{itemize}
        
        \item[-] Perfectly competitive, free entry and \textbf{zero profit}.
    \end{itemize}
    
    \end{minipage}
};
\node[redtitle, right=4pt] at (box.north west) {Patent market};
\end{tikzpicture}

\vspace{2pt}
This model has:
\begin{itemize}
    \item[-] appeals: incorporating the imperfect competition and externality of ideas

    \item[-] shortfalls: the patent production assumption doesn't reflect empirical observations:
    $$ A_{t+1} = A_t+\kappa (1-\phi_t)L A_t\Rightarrow \frac{A_{t+1}}{A_t}= \kappa (1-\phi_t)L+1$$
    the growth rate of new ideas is linearly increasing in labor, which doesn't reflect
    \begin{itemize}
        \item[-] fishing-out effect: the easiest-to-discover ideas have already been discovered
        \item[-] stepping-on-toes effect: a bigger workforce does NOT simply imply more ideas
    \end{itemize}
\end{itemize}

\vspace{2pt}
Next, we solve the social planner's problem of this model in 3 steps.

\vspace{2pt}
\begin{tikzpicture}
\node [redbox] (box){%
    \begin{minipage}{0.23\textwidth}
    \color{myred}
    \scriptsize
    The first step is to solve the \textbf{static choice of production inputs}:
    $$\max L_{1,t}^{1-\alpha}\int^{A_t}_0 x_t^{\alpha}(i)\mathrm{d}i
    $$
    s.t.
    $$
    \int^{A_t}_0 x_t(i)\mathrm{d}i=K_t,\ L_{1,t}=\phi_tL
    $$
    and the solution is symmetric: $x_t(i)=x_t,\forall i$, that is, there is a preference for variety, the more the better, all varieties are \textbf{equally costly} in terms of capital, hence \textbf{equally valued}:
    $$
    \int^{A_t}_0 x_t\mathrm{d}i=K_t\Rightarrow x_t = \frac{K_t}{A_t}
    $$
    plug this back into the production inputs, get:
    $$
    Y_t = L_{1,t}^{1-\alpha}\int^{A_t}_0 x_t^{\alpha}(i)\mathrm{d}i = \left(\phi_t L\right)^{1-\alpha} A_t\left(\frac{K_t}{A_t} \right)^{\alpha}=K_t^{\alpha}\left(\phi_t L A_t\right)^{1-\alpha}
    $$
    this function is still CRS, with $A_t$ given exogenously.
    
    \end{minipage}
};
\node[redtitle, right=4pt] at (box.north west) {Solve social planner's problem: Step 1};
\end{tikzpicture}

\vspace{2pt}
\begin{tikzpicture}
\node [redbox] (box){%
    \begin{minipage}{0.23\textwidth}
    \color{myred}
    \scriptsize
    The second step is to solve the familiar \textbf{intertemporal planning problem}:
    $$\max\sum^{\infty}_{t=0}\beta^t \frac{c_t^{1-\sigma}-1}{1-\sigma}
    $$
    s.t.
    \begin{align*}
        c_t+K_{t+1}-(1-\delta)K_t &= K_t^{\alpha}\left(\phi_t L A_t\right)^{1-\alpha}\\
        A_{t+1}-A_t &= \kappa(1-\phi_t)LA_t
    \end{align*}
    Lagrange:
    \begin{align*}
        \mathcal{L}=&\sum^{\infty}_{t=0}\beta^t\frac{c_t^{1-\sigma}-1}{1-\sigma} +\lambda_{A,t}\left[\kappa(1-\phi_t)LA_t-\left(A_{t+1}-A_t\right)\right]\\
        &+ \lambda_{K,t}\left[K_t^{\alpha}\left(\phi_t L A_t\right)^{1-\alpha}-\left(c_t+K_{t+1}-(1-\delta)K_t\right)\right]
    \end{align*}
    FOC gives
    \begin{align*}
        \frac{\partial \mathcal{L}}{\partial c_t}=0 \Rightarrow & \beta^t c^{-\sigma}=\lambda_{K,t}\\
        \frac{\partial \mathcal{L}}{\partial \phi_t} =0\Rightarrow & \lambda_{A,t}\kappa LA_t = \lambda_{K,t}K_t^{\alpha}(1-\alpha)\left(\phi_tLA_t\right)^{-\alpha}LA_t \\
        \frac{\partial \mathcal{L}}{\partial K_{t+1}}=0\Rightarrow & \left(\right)\\
        \frac{\partial \mathcal{L}}{\partial A_{t+1}}=0\Rightarrow & 
    \end{align*}
    
    \end{minipage}
};
\node[redtitle, right=4pt] at (box.north west) {Solve social planner's problem: Step 2};
\end{tikzpicture}

\end{multicols*}
\end{document}