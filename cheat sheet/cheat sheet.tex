\documentclass[10pt,landscape,a4paper]{article}
\usepackage[utf8]{inputenc}
\usepackage[ngerman]{babel}
\usepackage{tikz}
\usetikzlibrary{shapes,positioning,arrows,fit,calc,graphs,graphs.standard}
\usepackage[nosf]{kpfonts}
\usepackage{graphicx}
\usepackage[t1]{sourcesanspro}
%\usepackage[lf]{MyriadPro}
%\usepackage[lf,minionint]{MinionPro}
\usepackage{multicol}
\usepackage{wrapfig}
\usepackage[top=1mm,bottom=2mm,left=1mm,right=2mm]{geometry}
\usepackage[framemethod=tikz]{mdframed}
\usepackage{microtype}
\usepackage{hyperref}

\usepackage{url}
\usepackage{multirow}
\usepackage{esint}
\usepackage{amsfonts}
\usetikzlibrary{decorations.pathmorphing}

\usepackage{colortbl}
\usepackage{xcolor}
\usepackage{mathtools}
\usepackage{amsmath,amssymb}
\usepackage{enumitem}
\makeatletter

\let\bar\overline

\setlist[itemize]{topsep=0pt,leftmargin=10pt,itemsep=-0.2em}
\definecolor{myblue}{cmyk}{1,.72,0,.38}
\definecolor{mypurple}{cmyk}{.57,1,0,.58}
\definecolor{myred}{cmyk}{0,.88,.88,.58}
\definecolor{mygreen}{cmyk}{1,0,.69,.66}
\definecolor{myorange}{cmyk}{0,.58,100,.20}

\def\firstcircle{(0,0) circle (1.5cm)}
\def\secondcircle{(0:2cm) circle (1.5cm)}

\pgfdeclarelayer{background}
\pgfsetlayers{background,main}

\renewcommand{\baselinestretch}{.8}
\pagestyle{empty}

\global\mdfdefinestyle{header}{%
linecolor=gray,linewidth=1pt,%
leftmargin=0mm,rightmargin=0mm,skipbelow=0mm,skipabove=0mm,
}

%\newcommand{\header}{
%\begin{mdframed}[style=header]
%\scriptsize
%\sffamily
%Cheat sheet\\
%by~Your~Name,~page~\thepage~of~2
%\end{mdframed}
%}

\makeatletter
\renewcommand{\section}{\@startsection{section}{1}{0mm}{1ex}{.2ex}{\normalsize\bfseries}}
\renewcommand{\subsection}{\@startsection{subsection}{1}{0mm}{.2ex}{.2ex}{\bfseries}}

\newcommand*\bigcdot{\mathpalette\bigcdot@{.5}}
\newcommand*\bigcdot@[2]{\mathbin{\vcenter{\hbox{\scalebox{#2}{$\m@th#1\bullet$}}}}}
\makeatother

\def\multi@column@out{%
   \ifnum\outputpenalty <-\@M
   \speci@ls \else
   \ifvoid\colbreak@box\else
     \mult@info\@ne{Re-adding forced
               break(s) for splitting}%
     \setbox\@cclv\vbox{%
        \unvbox\colbreak@box
        \penalty-\@Mv\unvbox\@cclv}%
   \fi
   \splittopskip\topskip
   \splitmaxdepth\maxdepth
   \dimen@\@colroom
   \divide\skip\footins\col@number
   \ifvoid\footins \else
      \leave@mult@footins
   \fi
   \let\ifshr@kingsaved\ifshr@king
   \ifvbox \@kludgeins
     \advance \dimen@ -\ht\@kludgeins
     \ifdim \wd\@kludgeins>\z@
        \shr@nkingtrue
     \fi
   \fi
   \process@cols\mult@gfirstbox{%
%%%%% START CHANGE
\ifnum\count@=\numexpr\mult@rightbox+2\relax
          \setbox\count@\vsplit\@cclv to \dimexpr \dimen@-1cm\relax
\setbox\count@\vbox to \dimen@{\vbox to 1cm{\header}\unvbox\count@\vss}%
\else
      \setbox\count@\vsplit\@cclv to \dimen@
\fi
%%%%% END CHANGE
            \set@keptmarks
            \setbox\count@
                 \vbox to\dimen@
                  {\unvbox\count@
                   \remove@discardable@items
                   \ifshr@nking\vfill\fi}%
           }%
   \setbox\mult@rightbox
       \vsplit\@cclv to\dimen@
   \set@keptmarks
   \setbox\mult@rightbox\vbox to\dimen@
          {\unvbox\mult@rightbox
           \remove@discardable@items
           \ifshr@nking\vfill\fi}%
   \let\ifshr@king\ifshr@kingsaved
   \ifvoid\@cclv \else
       \unvbox\@cclv
       \ifnum\outputpenalty=\@M
       \else
          \penalty\outputpenalty
       \fi
       \ifvoid\footins\else
         \PackageWarning{multicol}%
          {I moved some lines to
           the next page.\MessageBreak
           Footnotes on page
           \thepage\space might be wrong}%
       \fi
       \ifnum \c@tracingmulticols>\thr@@
                    \hrule\allowbreak \fi
   \fi
   \ifx\@empty\kept@firstmark
      \let\firstmark\kept@topmark
      \let\botmark\kept@topmark
   \else
      \let\firstmark\kept@firstmark
      \let\botmark\kept@botmark
   \fi
   \let\topmark\kept@topmark
   \mult@info\tw@
        {Use kept top mark:\MessageBreak
          \meaning\kept@topmark
         \MessageBreak
         Use kept first mark:\MessageBreak
          \meaning\kept@firstmark
        \MessageBreak
         Use kept bot mark:\MessageBreak
          \meaning\kept@botmark
        \MessageBreak
         Produce first mark:\MessageBreak
          \meaning\firstmark
        \MessageBreak
        Produce bot mark:\MessageBreak
          \meaning\botmark
         \@gobbletwo}%
   \setbox\@cclv\vbox{\unvbox\partial@page
                      \page@sofar}%
   \@makecol\@outputpage
     \global\let\kept@topmark\botmark
     \global\let\kept@firstmark\@empty
     \global\let\kept@botmark\@empty
     \mult@info\tw@
        {(Re)Init top mark:\MessageBreak
         \meaning\kept@topmark
         \@gobbletwo}%
   \global\@colroom\@colht
   \global \@mparbottom \z@
   \process@deferreds
   \@whilesw\if@fcolmade\fi{\@outputpage
      \global\@colroom\@colht
      \process@deferreds}%
   \mult@info\@ne
     {Colroom:\MessageBreak
      \the\@colht\space
              after float space removed
              = \the\@colroom \@gobble}%
    \set@mult@vsize \global
  \fi}

\hypersetup{
    colorlinks=true,
    linkcolor=myblue,
    filecolor=magenta,      
    urlcolor=myblue,
    pdfpagemode=FullScreen,
    }

\urlstyle{same}

\makeatother
\setlength{\parindent}{0pt}

% material references: Prof. Caroline Betts' lecture notes
% latex coding references: https://github.com/tim-st/latex-cheatsheet, https://www.overleaf.com/latex/templates/hoja-de-ecuaciones-electricidad-y-magnetismo/xwgjqkrjjgcb

\begin{document}
\begin{center}{\large{\textbf{Microeconomic Model Cheat Sheet}}}\\
Author: Sai Zhang (\href{mailto:saizhang.econ@gmail.com}{email} me or check my \href{https://github.com/SaiChrisZHANG}{Github} page)
\end{center}

\small
\begin{multicols*}{4}

% set box styles: in this file, I only use red blue and purple boxes
%% blue boxes
\tikzstyle{bluebox} = [draw=myblue, fill=white, thick, rectangle, rounded corners, inner sep=5pt, inner ysep=10pt, text=myblue]
\tikzstyle{bluetitle} =[fill=myblue, text=white, font=\bfseries]
\tikzstyle{ibluebox} = [draw=myblue, fill=myblue, thick, rectangle, rounded corners, inner sep=5pt, inner ysep=10pt, text=white]
\tikzstyle{ibluetitle} =[draw=myblue, fill=white, text=myblue, font=\bfseries]
%% red boxes
\tikzstyle{redbox} = [draw=myred, fill=white, thick, rectangle, rounded corners, inner sep=5pt, inner ysep=10pt, text=myred]
\tikzstyle{redtitle} =[fill=myred, text=white, font=\bfseries]
\tikzstyle{iredbox} = [draw=myred, fill=myred, thick, rectangle, rounded corners, inner sep=5pt, inner ysep=10pt, text=white]
\tikzstyle{iredtitle} =[draw=myred, fill=white, text=myred, font=\bfseries]
%% purple boxes
\tikzstyle{purplebox} = [draw=mypurple, fill=white, thick, rectangle, rounded corners, inner sep=5pt, inner ysep=10pt, text=mypurple]
\tikzstyle{purpletitle} =[fill=mypurple, text=white, font=\bfseries]
\tikzstyle{ipurplebox} = [draw=mypurple, fill=mypurple, thick, rectangle, rounded corners, inner sep=5pt, inner ysep=10pt, text=white]
\tikzstyle{ipurpletitle} =[draw=mypurple, fill=white, text=mypurple, font=\bfseries]
%% orange boxes
\tikzstyle{orangebox} = [draw=myorange, fill=white, thick, rectangle, rounded corners, inner sep=5pt, inner ysep=10pt, text=myorange]
\tikzstyle{orangetitle} =[fill=myorange, text=white, font=\bfseries]
\tikzstyle{iorangebox} = [draw=myorange, fill=myorange, thick, rectangle, rounded corners, inner sep=5pt, inner ysep=10pt, text=white]
\tikzstyle{iorangetitle} =[draw=myorange, fill=white, text=myorange, font=\bfseries]

\vspace{2pt}
\begin{tikzpicture}
\node [orangebox] (box){%
    \begin{minipage}{0.235\textwidth}
    \color{myorange}
    \scriptsize
    \begin{itemize}
        \item[-] \textbf{Primitive assumptions}:
        \begin{itemize}
            \item[-] Who the \textbf{agents} are, what are their \textbf{preferences} and objective functions
            \item[-] What \textbf{technology} agents can access
            \item[-] What \textbf{endowment} agents have
        \end{itemize}
        \item[-] \textbf{Decision problems}: resource allocation problem (among agents, over time, etc.).
        \item[-] \textbf{Information sets}: what do agents know, how will their knowledge change, what is their \textbf{expectation}.
        \item[-] \textbf{Allocation mechanism}: how agents interact and achieve equilibrium. 2 main mechanisms are:
        \begin{itemize}
            \item[-] \textbf{price system} in competitive equilibrium
            \item[-] benevolent \textbf{central planner} maximizes a social welfare function.
        \end{itemize}
    \end{itemize}
    \end{minipage}
};
\node[orangetitle, right=4pt] at (box.north west) {Modelling essential: what we need to decide};
\end{tikzpicture}

\section*{Infinitely Lived Agent Model}

\vspace{2pt}
\begin{tikzpicture}
\node [ibluebox] (box){%
    \begin{minipage}{0.235\textwidth}
    \color{white}
    \scriptsize
    \begin{itemize}
        \item[-] \textbf{discrete} time, indexed by $t$
        \item[-] economy lives \textbf{infinitely}, $t=0,1,2,\cdots$
        \item[-] single commodity exogenously produced, indexed by $t$, pure \textbf{exchange/endowment} economy.
        \item[-] no firms/government, only \textbf{two types of households}.
        \item[-] each type of households is continuum of \textbf{identical} households of that type, they are \textbf{price takers}, can be represented by a \textbf{representative} household
    \end{itemize}
    \end{minipage}
};
\node[ibluetitle, right=4pt] at (box.north west) {Features};
\end{tikzpicture}

\vspace{2pt}
\begin{tikzpicture}
\node [bluebox] (box){%
    \begin{minipage}{0.235\textwidth}
    \color{myblue}
    \scriptsize
    Utility of type $i$ household is
    $$U(c^i)=\sum^{\infty}_{t=0}\beta^t_i u(c_t^i)$$
    where $\left(c^i\right)=\left\{c_t^i\right\}^{\infty}_{t=0}$, $\beta_i\in(0,1)$,
    
    The utility function $u(c_t^i)$ is assumed to be:
    \begin{itemize}
        \item[-] \textbf{continuously differentiable} of the second order
        \item[-] \textbf{monotonically increasing, strictly concave}: $u'(c_t^i)>0,u''(c_t^i)<0$
        \item[-] \textbf{satisfies Inada conditions} (never 0 or infinity consumption): $\lim_{c^i_t\rightarrow \infty}u'(c_t^i)=0, \lim_{c^i_t\rightarrow 0}u'(c_t^i)=\infty$
        \item[-] \textbf{time additivity}: $u(c_t^i)$ is independent of $c_{t+j}^i$, $c_{t-j}^i$.
        \item[-] \textbf{impatient discounting} $\beta_i<1$: households value today's consumption more than future's.
        \item[-] Constant relative risk aversion (\textbf{CRRA}): $u(c_t^i)=\frac{c^{1-\sigma}-1}{1-\sigma}$    
        \tiny
        $\left(\lim_{\sigma\rightarrow1}\frac{c^{1-\sigma}-1}{1-\sigma}=\lim_{\sigma\rightarrow1}\frac{e^{(1-\sigma)\ln(c)}-1}{1-\sigma}=\ln(c)\right)$, 
        \scriptsize
        \textbf{The RRA coefficient} $\sigma(c)=$
        \tiny
        $\frac{-u''(c_t^i)c}{u'(c_t^i)}=\frac{-\left(-\sigma c^{-(1+\sigma)}c\right)}{c^{-\sigma}}$
        \scriptsize
        $=\sigma$. Higher RRA means higher risk aversion. 
        \item[-] Constant intertemporal elasticity of substitution \textbf{(IES)}: $$IES=-\frac{\mathrm{d}\ln\left(c_{t+1}/c_t\right)}{\mathrm{d}\ln\left(u'(c_{t+1}/u'(c_t)\right)}=\frac{1}{\sigma}$$
        hence higher RRA (more risk-averse), lower IES (consumption variation over time).
    \end{itemize}
    \end{minipage}
};
\node[bluetitle, right=4pt] at (box.north west) {Agents' preferences};
\end{tikzpicture}

\vspace{2pt}
\begin{tikzpicture}
\node [bluebox] (box){%
    \begin{minipage}{0.235\textwidth}
    \color{myblue}
    \scriptsize
    A deterministic endowment stream of the consumption good for type $i$ household is
    $$w^i=\left(w^i_0,w_1^i,\cdots\right)=\left\{ w^i_t \right\}^{\infty}_{t=0}$$
    
    \end{minipage}
};
\node[bluetitle, right=4pt] at (box.north west) {Agents' endowment};
\end{tikzpicture}



\end{multicols*}

\end{document}